% !TEX TS-program = pdflatex
% !TEX encoding = UTF-8 Unicode
\documentclass[a4paper, oneside, justified=true, sfsidenotes]{tufte-book} 
\usepackage[utf8]{inputenc} % set input encoding to utf8
\usepackage[greek, french, english]{babel}[2005/11/23]
\usepackage{phd}


\usepackage{listings}
\usepackage{siunitx}
%% Bibliography
\usepackage{natbib}


\bibliographystyle{plainnat} % note the change here
\bibpunct{(}{)}{;}{a}{,}{,}




\usepackage{alltt}[1997/06/16]   
\usepackage{epigraph}



\RequirePackage{caption}
\DeclareCaptionFont{blue}{\color{black}}
\captionsetup{justification=raggedright, singlelinecheck=false,font={blue,sf,small}}
\usepackage{pgfplots}





%overwrite tufte and have numbering
\setcounter{secnumdepth}{5}


%% Change some of the looks
%%% CHAPTER STYLES AND SECTION STYLES
\pagestyle{fancy}
\renewcommand{\chaptermark}[1]%
{\markboth{\MakeUppercase{\chaptertitlename\ \thechapter\ #1}}{}} 
\renewcommand{\sectionmark}[1]%
{\markright{\MakeUppercase{\thesection~\ #1}}}
\renewcommand{\headrulewidth}{0.5pt}
\renewcommand{\footrulewidth}{0pt}
\newcommand{\helv}{%
\fontfamily{phv}\fontseries{b}\fontsize{9}{11}\selectfont}
\fancyhf{}
\fancyhead[LE,RO]{\helv \thepage}
\fancyhead[LO]{\helv \rightmark}
\fancyhead[RE]{\helv \leftmark}

%%hyperref
\hypersetup{urlcolor=green, colorlinks=true}  % Colours hyperlinks in blue, but this can be 
                                              % distracting if there are many links
\urlstyle{sf}  %set url as sans-serif font 


% Typesets the font size, leading, and measure in the form of 10/12x26 pc.
\newcommand{\measure}[3]{#1/#2$\times$\unit[#3]{pc}}

% Macros for typesetting the documentation


\newcommand{\hlred}[1]{\textcolor{Maroon}{#1}}% prints in red

\newcommand{\hlmaroon}[1]{\textcolor{Maroon}{#1}}%print maroon

\newcommand{\hangleft}[1]{\makebox[0pt][r]{#1}}


\newcommand{\sourceatright}[2]{{%
  \unskip\nobreak\hfil\penalty100
  \hskip#1\hbox{}\nobreak\hfil{#2}
  \parfillskip 15pt \par}}


\newcommand{\hairsp}{\hspace{1pt}}% hair space
\newcommand{\hquad}{\hskip0.5em\relax}% half quad space
\newcommand{\TODO}{\textcolor{red}{\bf TODO!}\xspace}

\newcommand{\ie}{\textit{i.\hairsp{}e.}\xspace}
\newcommand{\eg}{\textit{e.\hairsp{}g.}\xspace}



%% Set up the epigraph to be a bit wider
\setlength{\epigraphwidth}{8cm} 
\setlength{\epigraphrule}{0pt}
\newcommand{\theepigraph}[2]{\epigraphhead[30]{\epigraph{#1}{\textit{#2}}}}

% Generates the index we use before the
% out
\usepackage{makeidx}
\makeindex



%%% ACTIVITY DATES
%% Define a new command for activity key-dates
%% this can be saved for shipout later
\newcommand{\keydate}[2]{#1  #2 \\}
\newcommand{\out}[2]{%
  {\index{key dates!#1}% add to index
\immediate\write\tempfile{\noexpand\keydate{ #1}{#2}}}}


%% We open the file to  to write the key-dates
%% we will use it later to import
\newwrite\tempfile
\immediate\openout\tempfile=keydates.tex


\usepackage{fancyvrb}



% We use for highlighting text


\makeatletter
\def\chapter{\clearpage\thispagestyle{plain}\global\@topnum
  \z@\@afterindentfalse
  \secdef\@chapter\@schapter
}
\makeatother


\title{CONSTRUCTION \\ STRATEGY FOR MEP}
\author{City Center Phase 3a \&3b}
\publisher{Habtoor Leighton Specon llc}
\date{January 2011} % Delete this line to display the current date



% shorthands for typing equipment numbers
% that make sense.
\def\panel#1{{\inc\space\space\small\texttt{#1}}}
\begin{document}

\input{./sections/definitions}
\setcounter{step}{0}


\begin{longtable}{p{2cm}lllllllp{3.8cm}}

\toprule
Floor/Activity& &\textcircled{1}&\textcircled{2}
               &\textcircled{3}&\textcircled{4}

               &\textcircled{5}&\textcircled{6}&Remarks\\

\midrule
Level 6       &\panel{SMDB-MW6-LP1}&\checkmark&\checkmark&\checkmark&\checkmark
    &&&Live Test, 9 Mar 13\\



\midrule
Level 5
   &\panel{SMDB-MW5LP1}&\checkmark&\checkmark&\checkmark&\checkmark
   &&&Live Test, 8 Mar 13\\

   &\panel{SMDB-MW5-EPP1}&\checkmark&\checkmark&\checkmark&\checkmark
   &&&Live Test, 8 Mar 13\\

   &\panel{MCC-MW5-PP1}&\checkmark&\checkmark&\checkmark&\checkmark
   &&&Live Test, 8 Mar 13\\

   &\panel{MCC-MW-AC2}&\checkmark&\checkmark&\checkmark&\checkmark
   &&&Live Test, 8 Mar 13\\

   &\panel{MCC-MW-F1}&\checkmark&\checkmark&\checkmark&\checkmark
   &&&Live Test, 8 Mar 13 \\


\midrule
Level 4
   &\panel{SMDB-MW4-EPP1}&\checkmark&\checkmark&\checkmark&\checkmark
   &&&Live Test, 7 Mar 13\\

   &\panel{SMDB-MW4-EPP2}&\checkmark&\checkmark&\checkmark&\checkmark
   &&&Live Test, 7 Mar 13\\

   &\panel{SMDB-MW4-LP1}&\checkmark&\checkmark&\checkmark&\checkmark
   &&&Live Test, 7 Mar 13\\

   &\panel{MCC-MW4-PL11}&\checkmark&\checkmark&\checkmark&\checkmark
   &&&Live Test, 7 Mar 13\\

   &\panel{MCC-MW4-PL12}&\checkmark&\checkmark&\checkmark&\checkmark
   &&&Live Test, 7 Mar 13\\

  &\panel{SMDB-MW4-PL13}&\checkmark&\checkmark&\checkmark&\checkmark
   && &Live Test, 7 Mar 13\\



\midrule
Level 3
   &\panel{SMDB-MW3-LP1}&\checkmark&\checkmark&\checkmark&\checkmark
   &&&Live Test, 6 Mar 13\\

   &\panel{SMDB-MW3-LP2}&\checkmark&\checkmark&\checkmark&\checkmark
   &&&Live Test, 6 Mar 13 \\

   &\panel{SMDB-MW3-ELP1}&\checkmark&\checkmark&\checkmark&\checkmark
   && &Live Test, 6 Mar 13\\

   &\panel{MCC-L3-MW-EPP1}&\checkmark&\checkmark&\checkmark&\checkmark
   && &Live Test, 6 Mar 13\\

\midrule
Level2   &\panel{SMDB-MW2-AV1}&\checkmark&\checkmark&\checkmark&\checkmark
   && & Live Test, 5 Mar 13 \\

   &\panel{SMDB-MW2-LP1}&\checkmark&\checkmark&\checkmark&\checkmark
   &&&Live Test, 5 Mar 13\\

   &\panel{SMDB-MW2-ELP1}&\checkmark&\checkmark&\checkmark&\checkmark
   &&&Live Test, 5 Mar 13 \\

   &\panel{SMDB-MW2-UP1}&\checkmark&\checkmark&\checkmark&\checkmark
   &&&Live Test, 5 Mar 13\\

   &\panel{SMDB-MW2-EPP1}&\checkmark&\checkmark&\checkmark&\checkmark
   &&&Live Test, 5 Mar 13\\

\midrule
Level 1    &\panel{SMDB-MW1-EPP1}&\checkmark&\checkmark&\checkmark&\checkmark
   &&& Live Test, 4 Mar 13\\

 &\panel{SMDB-MW1-EPP1}&\checkmark&\checkmark&\checkmark&
   &&&Live Test, 4 Mar 13\\

 &\panel{SMDB-MW1-PP1}&\checkmark&\checkmark&\checkmark&\checkmark
   &&& Live Test, 4 Mar 13\\


 &\panel{SMDB-MW1-LP1}&\checkmark&\checkmark&\checkmark&\checkmark
   &&&Live Test, 4 Mar 13 \\



\midrule
Level GF    &\panel{SMDB-MWG-LP1}&\checkmark&\checkmark&\checkmark&\checkmark
   &&& Live Test, 3 Mar 13\\

&\panel{SMDB-MWG-ELP1}&\checkmark&\checkmark&\checkmark&
   &&&Live Test, 3 Mar 13\\

&\panel{SMDB-MWG-EPP2}&\checkmark&\checkmark&\checkmark&\checkmark
   &&&Live Test, 3 Mar 13\\

&\panel{SMDB-MWG-EPP2*}&\checkmark&\checkmark&\checkmark&\checkmark
   && & Live Test, 3 Mar 13\\


\midrule
Level B1   &\panel{SMDB-MW-B1-UP1}&\checkmark&\checkmark&\checkmark&\checkmark
   && & Live Test, 2 Mar 13\\

&\panel{SMDB-MW-B1-UPS}&\checkmark&\checkmark&\checkmark&\checkmark
   && &Live Test, 2 Mar 13\\


&\panel{SMDB-MW-B1-LP1}&\checkmark&\checkmark&\checkmark&\checkmark
   && &Live Test, 2 Mar 13\\

&\panel{SMDB-MW-B1-EPP1}&\checkmark&\checkmark&\checkmark&\checkmark
   &&&Live Test, 2 Mar 13\\


\midrule
level B2      &\panel{SMDB-MW-B2-LP1}&\checkmark&\checkmark&\checkmark&\checkmark
   && &Live test, 1 Mar 13\\
 &\panel{SMDB-MW-B2-EPP1}&\checkmark&\checkmark&\checkmark&\checkmark
   &&& Live Test, 1 Mar 13\\
 
\midrule
Level B3    &\panel{MCC-MW-AC1}&\checkmark&\checkmark&\checkmark&\checkmark
   &\checkmark&\checkmark &\\
&\panel{MCC-MW-B3-FP1}&\checkmark&\checkmark&\checkmark&\checkmark
   &\checkmark&\checkmark &  \\
&\panel{MCC-MW-B3-PL1}&\checkmark&\checkmark&\checkmark&\checkmark
   &\checkmark&\checkmark &\\
&\panel{MCC-MW-B3-PL3}&\checkmark&\checkmark&\checkmark&\checkmark
   &\checkmark&\checkmark &\\

\bottomrule
\end{longtable}
\captionof{table}[MWtana SMD T\&C]{Testing and Commissioning of Merweb SMDB \& MCC Panels, Podium. \textcircled{1} physical installation, \textcircled{2} WIR for physical installation, \textcircled{3} cabling physical installation, \textcircled{4} cable testing, \textcircled{5} panel pre-commissioning, \textcircled{6} Live Test and Inspection.
}
\label{tbl:MWpanels}

\end{document}
%\SetBgContents{Update $26^{th}$ Jan}
%\BgThispage
% Front matter
\frontmatter

\maketitle
\tableofcontents
\listoffigures
\listoftables


\cleardoublepage

\chapter*{Update}

Given the current constraints of the Project, we are satisfied that the general direction of the Project is doing well and we will be in a position to offer the works for Civil Defence Inspection at end of March and shortly thereafter to enable the Operators to take possession of the Hotels.

Physical installations in Towers and Podia have been completed in their majority and final fix activities are following the pace of both the Main Contractor as well as Fit-out activities. 

EIs, RFIs and other constraints have slowed down the works and obstructed a reasonable implementation of our original Closure and Commissioning Plan.

\section*{Objectives}

The aim of this report is to summarize the status of the works
and propose a reasonable plan for Closure of the Works and the implementation of a Commissioning Plan.

I general we are proposing a focus where areas can be fully completed, as well as the enabling of commissioning of all systems to take place. We suggest to start focusing on the
Shangri-la Hotel first, as this presents more difficulties.

\begin{longtable}{lll}
\toprule
Item  &Area   & Milestone \\
\midrule
\inc     & Podium Rotana    & 16 Feb 2013\\
\inc     & Tower Rotana     & 16 Mar 2013\\
\inc     & Podium Shangrila & 30 Jan 2013\\
\inc     & Tower Shangrila  & 15 Feb 2013\\
\inc     & Merweb Podium    & 17 Mar 2013\\
\inc     & Tower Merweb     & 30 Apr 2013\\
\bottomrule
\end{longtable}

We suggest a 4-7 week cycle for each Hotel at which point we expect that all physical works will close (including comments, FORs and NCRs) and the enabling of all systems to be commissioned. A small Team can remain to continue closing any issues that arise out of these activities and to follow-up on activities delayed by the Fit-out subcontractor.

Within Podia the following sequence is preferred, whereas in the Towers a top to bottom sequence will be followed.

\begin{table}[htbp]
\centering
\begin{tabular}{ccc}
\toprule
Item  &Area   & Milestone \\
\midrule
1     & B1 & 20 Jan\\
2     & B3 & 20 Jan \\
3     & B2 & 20 Jan \\
4     & L6 & 20 Jan\\
5     & L4 & 27 Jan\\
6     & L3 & 27 Jan\\
7     & L7 &  5 Feb\\
8     & L2 & 10 Feb\\
9     & L1 & 10 Feb\\
10    & L5 & 10 Feb\\
11    & GF & 20 Feb\\
\bottomrule
\end{tabular}
\caption{Floor sequence of works.}
\end{table}

Sequence as shown is not that important and the logic behind it, is to enable other trades, such as Dragoni to complete the
works before concentrating in the area. Also we expect the CCTV
changes to affect these areas and we prefer to work in them
once all the other areas are completed.

More details can be found in the relevant area and system sections of the report.


\section{Personnel}


 Table~\ref{tbl:manpower} summarizes the manpower currently on the Project. 

\def\Z{\phantom{Z}}
\begin{table}[htbp]
\begin{center}
\begin{tabular}{l r r}
\toprule
Company           &Tradesmen   & Tradesmen\\
~                 &30 January  & 30 March \\
\midrule
Specon            &503         & 681\\
Al-Jaber          &181         &182\\
Crompton          &0           &59\\
QMMC              &112         &113\\
Nafco             &5           &16\\
ERE               &10          & 0\\
                  &\underline{\phantom{1075}}
                  &\underline{\phantom{1075}}\\
Subtotal          &912         &1201\\

Hired manpower    &            &\\
\Z ERE            &10          & nil\\

                  & \underline{\phantom{1075}}
                  & \underline{\phantom{1075}}\\
\Z Total hire manpower &127     &137\\
~&&\\
Total                      & \underline{\underline{1039}}       &\underline{\underline{1338}}\\
                             & &\\
\bottomrule
\end{tabular}
\caption{Average Manpower as of January 2013}
\label{tbl:manpower}
\end{center}
\end{table}



\section*{Current Constraints affecting completion}

\begin{enumerate}
\item Disruption due to Commercial issues.
\item Continued disruption with EIs and other changes.
\item Premature demobiliziation and large number of personnel resigning.
\item Demobilization of Al-Jaber due to Commercial issues.
\item Slow down by all sub-contractors due to commercial issues.
\item Fire Protection
\item Imbalances in resources between the Main Contractor, MEP Subcontractors and Finishing Contractor. This needs to be addressed in order to be able to improve on our completion projections.
\end{enumerate}

\section*{Demobilization}

We have experienced higher than normal attrition of staff and technicians. 

\section*{Recommendations}
Our recommendation is to focus our attention and resources on a Tower by Tower rather than spreading technicians over the full extend of the works.




\mainmatter
\chapter{Electrical Installation}
\label{ch:electrical}\index{Electrical Installation}

\newthought{The majority of the electrical systems} have been completed. However, certain activities such as final fix of lights are still being carried out. \sidenote{These are dependent on Fit-out Contractor completing wall cladding and ceilings on time.} 

\section{List of Services}
Under this section of the report the following systems are discussed.

\bgroup
\small
\label{chap:listofservices}

\begin{longtable}{llllp{3.9cm}}
%\begin{tabular}{llll}
\toprule
Ref	&Package	&Commiss.	&Perc. & Commissioning Status\\
     &         &Entity    &Compl.\\
\midrule
\midrule
2.00	&\textbf{Electrical}		&& 70\% complete\\
2.01	&Standby Generators	&Specialist&100\% & Started on flushing for cooling towers. Commercial issues to be sorted out. (See section for details on Generators.)\\	
2.02	&Medium Voltage System	&HS&25\% & Good progress all over. Panels missing and on order.\\	
2.03	&Low Voltage System		&HS&25\% & Started and being commissioning progressively.\\
2.04	&Earthing \& Lightning Protection &HS&100\% &earthing mostly completed. Lightning protection, missing parts arrived and installed. \\		
2.05	&Lighting \& Emergency Lighting  &HS&100\% & Where possible started. Fit-out needs to be completed in a few areas.\\		
2.06	&Aircraft Warning System &HS&100\% & Ready. Supplier requested to visit and commission. \\		
2.07	&UPS System	&Specialist&100\% & Not started.\\	
2.08  &Power Factor  &Specialist &100\% & Completed.\\
\midrule
\end{longtable}
\egroup



\section{Current Status}
\subsection{MCB SMDBs MCC Panels}


\begin{table}[htbp]
\begin{tabular}{lllp{4.5cm}}
\toprule
Item &Detail &Floor  & Remarks\\

\bottomrule
\end{tabular}
\caption{Final Snag List Electrical, Rotana}
\end{table}




\subsection{Earthing \& Lightning Protection}

\begin{table}[h]
{\RaggedRight
\begin{tabular}{lp{3cm}lp{3cm}}
\toprule
Item &Detail &T\&C & Remarks\\
\midrule
1    & Rotana & \checkmark  & \\
2    & Shangrila & \checkmark & \\
3    & Merweb & \checkmark & \\ 
4    & Overall earthing resistance measurement & x & 13.0.2.13\\  
\bottomrule
\end{tabular}
\caption{Earthing Testing and Commissioning}
}
\end{table}

\begin{table}[h]
{\RaggedRight
\begin{tabular}{lllll}
\toprule
Item &Detail &Installation &T\&C &Milestone\\
\midrule
1    & Rotana         & 90\%  & & \\
2    & Shangrila      & 100\% & &\\
3    & Merweb         & 100\% & &\\   
\bottomrule
\end{tabular}
\caption{Lightning Protection}
}
\end{table}

\subsection{Exit Signage}
Exit signage first and second fix installations have been
completed and inspected. 

\setcounter{step}{0}
\begin{longtable}{llllp{3.8cm}}
\toprule
S/No &Level &Final Fix  & T\&C & Remarks\\
\midrule
\inc &7     &\checkmark       &      &        \\ 
\inc &6     &\checkmark       &      &       \\
\inc &5     &\checkmark       &      & Remaining areas are in current contractor offices.       \\
\inc &4     &\checkmark       &      &Remaining works are in stores and offices.       \\
\inc &3     &\checkmark       &      &Remaining works in stores and offices       \\
\inc &2     &\checkmark       &      &Teatro restaurant, Teatro kitchen, Meeting rooms (Dragoni constraints).       \\
\inc &1     &       &      &       \\
\inc &GR    &partial       &      & Dragoni       \\
\inc &B1    &partial       &      & 7.02.2013       \\
\inc &B2    &\checkmark       &      &       \\
\inc &B3    &\checkmark      &      &       \\
\bottomrule
\caption{Exit Signage status.}
\end{longtable}

\subsection{Constraints}





\begin{table}[h]
{\RaggedRight
\begin{tabular}{lllll}
\toprule
Item &Detail &Installation &T\&C &Milestone\\
\midrule
1    & Rotana         & 90\%  & & \\
2    & Shangrila      & 90\% & &\\
3    & Merweb         & 90\% & &\\   
\bottomrule
\end{tabular}
\caption{Emergency Lighting}
}
\end{table}


\def\panel#1{{\small\texttt{#1}}}

Table\ref{tbl:ROpanels} shows the current and projected dates for completion of T\&C and paperwork for this service.


\begin{longtable}{p{2cm}lllllllp{3.8cm}}

\toprule
Floor/Activity& &\textcircled{1}&\textcircled{2}
               &\textcircled{3}&\textcircled{4}
             
               &\textcircled{5}&\textcircled{6}&Remarks\\
\midrule
Level 7       &\panel{SMDB-RO7EPP1}&\checkmark&\checkmark&\checkmark&\checkmark
&\checkmark&\checkmark&\\

             &\panel{SMDB-RO7EPP1}&\checkmark&\checkmark&\checkmark&\checkmark
&\checkmark&\checkmark&\\

&\panel{SMDB-RO7LP1}&\checkmark&\checkmark&\checkmark&\checkmark
&\checkmark&&\\

&\panel{SMDB-RO7UP1}&\checkmark&\checkmark&\checkmark&\checkmark
&\checkmark& &\\



Level 6        &\panel{SMDB-RO6LP1}&\checkmark&\checkmark&\checkmark&\checkmark
&\checkmark&\checkmark&\\

&\panel{SMDB-RO6LP2}&\checkmark&\checkmark&\checkmark&\checkmark
&\checkmark&\checkmark&\\

&\panel{SMDB-RO6ELP1}&\checkmark&\checkmark&\checkmark&\checkmark
&\checkmark& &\\

&\panel{SMDB-RO6EEP1}&\checkmark&\checkmark&\checkmark&\checkmark
&\checkmark& &\\

&\panel{SMDB-ROPL6}&\checkmark&\checkmark&\checkmark&\checkmark
&\checkmark& &\\

&\panel{SMDB-ROPL13}&\checkmark&\checkmark&\checkmark&\checkmark
&\checkmark& &\\

Level 5        
   &\panel{SMDB-RO5LP1}&\checkmark&\checkmark&\checkmark&\checkmark
   &\checkmark&\checkmark &\\
   &\panel{SMDB-RO5PP1}&\checkmark&\checkmark&\checkmark&\checkmark
   &\checkmark&\checkmark &\\  

   &\panel{MCC-RO5PP1}&\checkmark&\checkmark&\checkmark&\checkmark
   &\checkmark&\checkmark &\\ 
   &\panel{MCC-ROAC2}&\checkmark&\checkmark&\checkmark&\checkmark
   &\checkmark&\checkmark &\\
   &\panel{MCC-RO-F1}&\checkmark&\checkmark&\checkmark&\checkmark
   &\checkmark&\checkmark &\\



Level 4         &\panel{SMDB-RO4EPP1}&\checkmark&\checkmark&\checkmark&\checkmark
   &\checkmark&\checkmark &\\

    &\texttt{SMDB-RO4EPP2}&\checkmark&\checkmark&\checkmark&\checkmark
   &\checkmark&\checkmark &\\

level 3        

level 2        &&&&&&&&\\

level 1        &&&&&&&&\\

level GF        &&&&&&&&\\

level B1        &&&&&&&&\\

level B2        &&&&&&&&\\

level B3        &&&&&&&&\\

\bottomrule
\end{longtable}
\captionof{table}[Rotana SMD T\&C]{Testing and Commissioning of Rotana SMDB \& MCC Panels, Podium. \textcircled{1} physical installation, \textcircled{2} WIR for physical installation, \textcircled{3} cabling physical installation, \textcircled{4} cable testing, \textcircled{5} panel pre-commissioning, \textcircled{5} Live Test and Inspection.
}
\label{tbl:ROpanels}

















\chapter{Access Control and CCTV}

The CCTV has been subjected to a major EI, in order for the
works to comply to new MOI requirements.

The Client has appointed Johnson Controls to carry out these works.
Letter \texttt{HEE 901B/1.3/ZH/jv/16148} dated 22 January 2013 advised the Client about implications on completion and other issues. An extract from the letter, advising the award of containment to the same subcontractor is shown below:

\begin{quotation}
We note that we have over 95\% of the ceilings works completed, inspected and approved. We request that the containment works  be also given to Johnson Controls to avoid unecessary delay claims and excuses by either party.
\end{quotation}

As the works involve the re-opening of ceilings and the installation of over 85,000\si{\meter} of new conduiting and cabling, possibly relocation or major changes to security rooms and given the uncertainty over approvals of the works and the time-frame, no firm commitments can be made as such stage, until such time as we receive a fully resourced programme from Johnson Controls.

\textcircled{1}
\end{document}
\chapter{HVAC Systems}
\section{Introduction  \label{HVAC}}
\begin{marginfigure}%
  \includegraphics[width=\linewidth]{TC}
  \caption{Some of the AHU units need to be dismantled and then assembled before they are located in plant rooms. Logistics for installing missing units in Merweb are expected to delay installation works.}
  \label{fig:marginfig1}
\end{marginfigure}
\newthought{A substantial portion of the HVAC} Systems are expected to be completed by the \deadline.
The Chilled Water parts of the HVAC Systems as well as some critical ventilation systems are discussed under separate headings. This section discusses the balance of the HVAC Systems especially the AHU Installations and associated ductwork (i.e, the air side of the installations). These systems are currently being installed in a number of  areas and do not render themselves easy to change from an area point of view to a systems approach. 
Delays are expected in the installation of Air Handling Units in Podia due to late deliveries. It is also likely that problems will arise with some of the BCU units and fan-coil-units, as continuing revisions of drawings and EIs may result in additional units.

\ramadaneffect The overtime issue needs to be discussed and agreed with HEE.
\section{Constraints}
\subsection{Engineering and Materials}
For all Towers there are still Air handling Units that have not been delivered\sidenote{Last ones are expected to be delivered in September. However, this Supplier has let us down in the past and some delays are to be expected. Delays are also expected due to the congestion in the Mumbai port due to a shipping accident. Our Procurement Manager has visited the factory in India in order to negotiate improvements in deliveries. }. Various areas still need to be co-ordinated fully. Control valves and BMS DDC panels have still not been delivered and some are still under review by the Design Team and or HOK. These are expected to be delivered late.\sidenote{No illusions that they are going to be completed earlier}

Ductwork deliveries for Towers were and are still problematic. Action has been taken to mitigate these issues by starting a Ductwork Fabrication Workshop on Site, which currently is being resourced to operate in two shifts. However, most of the benefits of the workshop will only be realized once the need for site measured pieces are produced for final fix activities.

Grilles and diffusers for Podia are still to be scheduled and ordered. \sidenote{Expected to be ordered early August.}

\subsection{Labour Resource}

The Site currently does not have an adequate number of Duct Erectors and Duct Fabricators.  (see page \pageref{sec:logistics} for a discussion as to how logistics can help alleviate some of the shortfalls in personnel by increasing productivity.)

We are expecting more Duct Erectors and Fabricators to be sourced from ERE and other Suppliers. We are also investigating outsourcing to more sub-contractors. As some of our own technicians had to return back to Dubai due to visa problems, new visas were requested from HEE and we are now expecting to supplement the work-force with an additional 200 technicians from our UAE resources.

A detailed list of target dates is shown in Tables \ref{tbl:AHUrotana}, \ref{tbl:AHUmerweb} and \ref{tbl:AHUSL} for Rotana, Shangri-la and Merweb respectively. 

We are also in discussions to supplement the Podium workforce with the Tower subcontractors, once the Towers are completed and hence have an increased labour force in September and October.

The fact is that the Podia have suffered as a result of the shifting of our personnel to the Towers to compensate for the failures of the subcontractors. 

\begin{table}[htbp]
\setcounter{step}{0}
\label{tbl:AHUrotana}
\footnotesize
\caption{Rotana Air Handling Unit installation target dates}
\begin{tabular}{lll c c p{1.8cm}}
\toprule
Ref.	  &AHU tag 	 &Area	 &Installed	  &T\&C   & Remarks\\
\midrule
 \inc	 &B2-RO-AH-001   &Main Kitchen TFA    &\checkmark  & &\\	
 
\midrule 
 \inc	 &B1-RO-AH-001	 &Mech. Room	 & \checkmark	  &\checkmark & closed\\
 \inc	 &B1-RO-AH-002	 &BMS/ENG	      &\checkmark	  &\checkmark & closed\\
 \inc	 &B1-RO-AH-003	 &Staff Canteen	 &\checkmark	 	 
      &\checkmark 
      &\\
 \inc	  &B1-RO-AH-004	 &TFA Locakers	 &\checkmark	 	 &&\\
 \inc	  &B1-RO-AH-005  &Ballroom Prefunction     &\checkmark	 	 &&\\


\midrule
 \inc	  &GR-RO-AH-001	 &Electrical Rooms	 &\checkmark	 	 &&\\

\inc	  &GR-RO-AH-002	 &Banquet Kitchen	 &\checkmark	 	 &&\\
\inc	  &GR-RO-AH-003	 &Cafe/Lobby	        &\checkmark	    &&\\
\inc	 	 &GR-RO-AH-004	 &Rotana Lobby	 &\checkmark	 	 &&\\

\midrule
\inc	 	 &L1-RO-AH-001    &Lebanese Kitchen	    &\checkmark	 && \\
\inc	 	 &L1-RO-AH-002	  &Boston Bar	       &\checkmark	 &&\\
\inc	 	 &L1-RO-AH-003	  &Ballroom 2	       &\checkmark	 &&\\
\inc	 	 &L1-RO-AH-004	  &All Day Restaurant	 &\checkmark	 &&\\



\midrule
\inc 	 	 &L2-RO-AH-001	 &Teatro Lobby/Corridor &\checkmark	   &&\\	 
\inc	 	 &L2-RO-AH-002	 &Meeting Room TFA	     &\checkmark	&&\\

\midrule
\inc	 &L4-RO-AH-001	 &Teatro Restaurant	 &\checkmark	    &&\\
\inc	 &L4-RO-AH-002	 &MV Panel Room	    &\checkmark	 	 &&        \\

\midrule
\inc	 &L5-RO-AH-001	 &Guestroom TFA	    &\checkmark	 	   && \\
\inc	 &L5-RO-AH-002	 &HX Tertiary Plantroom	 &\checkmark	 && \\
\inc	 &L5-RO-AH-003	 &HX-Pant	 &\checkmark	 	               & & \\
\inc	 &L5-RO-AH-004	 &Admin/Reception	&\checkmark  &&  	 \\
\inc	 &L5-RO-AH-005	 &Fitness Area	 &\checkmark	 	 &&\\
\inc	 &L5-RO-AH-006	 &Change Room	&\checkmark	 	 &\\
\inc	 &L5-RO-AH-007	 &Teatro Restaurant	 &\checkmark	 &&\\	 
\inc	 &L5-RO-AH-008    & Guestroom TFA	 &\checkmark  &&\\	 	 

\midrule
\inc	 &L6-RO-AH-001	 &Ballroom 1	           &\checkmark	 	         &&\\
\inc	 &L6-RO-AH-002	 &Ballroom 3	           &\checkmark	 	         &&\\
\inc	 &L6-RO-AH-003	 &Lebanese Restaurant	 &\checkmark	 	         &&\\
\inc	 &L6-RO-AH-004	 &Teatro Kitchen	    &\checkmark	 	         &&\\
\inc	 &L6-RO-AH-005	 &Swimming Pool Plant	 &\checkmark	 	         &&\\

\midrule
\inc	 &L7-RO-AH-001	 &Pool Kitchen	    &\checkmark     &&\\
\midrule
\inc	    &L46-RO-AH-001 &Guestroom TFA	 &\checkmark	 	 &&\\
\inc	 	 &L46-RO-AH-002 &Guestroom TFA	 &\checkmark	 	 &&\\
\midrule
\inc	    &L47-RO-AH-001 &Mechanical Room	 &\checkmark	 	 &&\\

\bottomrule
\end{tabular}

\end{table}




\begin{table}[htbp]
\setcounter{step}{0}
\label{tbl:AHUSL}
\footnotesize
\caption{Shangri-la Air Handling Unit installation target dates}
\begin{tabular}{llp{3.2cm} c c l}
\toprule
 Ser.	 	 &AHU tag 	 &Area	 			&Installed	&T\&C &Remarks \\
\midrule

 
 \inc	 	 &B3-SL-AH2	 &Plumbing Plantroom	 	 &	 && \\
\midrule
 \inc	 	 &B2-SL-AH1     &Boiler	 	  	 & &&\\
 
\midrule

 \inc	 	 &B1-SL-AH1	 &Staff Canteen		  &   && \\
 \inc	 	 &B1-SL-AH2	 &Housekeeping Uniform	  &   && \\
 \inc	 	 &B1-SL-AH3	 &Kitchen		 	  &   && \\
 \inc	 	 &B1-SL-AH4	 &Main Kitchen	 		  &&  & \\
 \inc	 	 &B1-SL-AH5	 &Laundry	 	 	  &  & &\\

\inc	 	 &B1-SL-AH7    &TFA for EPABX		  &&  & \\
\inc	 	 &B1-SL-AH8	 &TFA Offices	            	  &&  & \\
\inc	 	 &B1-SL-AH9	 &HR	 	 		  &  & &\\
\inc	 	 &B1-SL-AH10	 &Engineering	 	 	  &&  & \\
 
\inc	 	 &B1-SL-AH13	 &Lift Lobby/Corridor	 	& &  & \\
\inc	 	 &B1-SL-AH14	 &Purchase	 	 	 & & &\\
\inc	 	 &B1-SL-AH15	 &TFA for Lockers	 	 & & & \\

\midrule

 
\inc	 	 &GR-SL-AH2	 &Ballroom Kitchen TFA	 &&&\\

\inc	 	 &GR-SL-AH4	 &Cafe/Lounge	 	& &&\\
 
\inc	 	 &GR-SL-AH7	 &Shangri-la Lobby	 	&& &\\
\inc	 	 &GR-SL-AH8	 &	 	 	 	& & &\\
\midrule


\inc	 	 &L1-SL-AH2	 &Ballroom No. 2	 	&  & &\\
\inc	 	 &L1-SL-AH3	 &Ballroom No.3	 	&  & &\\
\inc	 	 &L1-SL-AH4	 &Prefunction	 	 	& && \\
 
\inc	 	 &L1-SL-AH7	 &Sea Food + Kitchen	 	&&  &\\
 
\inc	 	 &L1-SL-AH10	 &Argentina Kitchen	 	& & &\\
 
\inc	 	 &L1-SL-AH12	 &Corridor	 	 	&  & &\\
\midrule

\inc	 	 &L2-SL-AH1	 &TFA Meeting + Retail	& && \\
\inc	 	 &L2-SL-AH2	 &Meeting Room (244)*	& 	 && \\
\inc	 	 &L2-SL-AH3	 &Meeting Room (245 A)*	 &&	 & \\
\inc	 	 &L2-SL-AH4	 &Meeting Room (245 B)*	 & && \\
\inc	 	 &L2-SL-AH5	 &Meeting Room Corridor	 &  && \\

\inc	 	 &L2-Sl-AH10	 &Storage L5 or L2	 	& & &\\
\inc  	 &L2-SL-AH11	 &Kitchen	 	 	&  && \\
\midrule 

\inc  	 &L4-SL-AH1	 &All Day Restaurant Office	& &    &\\
\inc	 	 &L4-SL-AH2	 &Plumbing Plantroom	 	&  & &\\
\inc	 	 &L4-SL-AH3	 &Admin Office Reception	 &  &&\\
\inc	 	 &L4-SL-AH4	 &Argentinian	 	 	 & & &\\
\midrule 

\inc 	 &L5-SL-AH1	 &Change Room 107	 	 & & &\\
\inc	 	 &L5-SL-AH2	 &Change Room 107	 	 &	& & \\
\inc	 	 &L5-SL-AH3	 &Admin Office + Recap	 &	 &&\\
\inc	 	 &L5-SL-AH4	 &Fitness Area	  		 &	 &&\\
\inc	 	 &L5-SL-AH5	 &All Day Rest. Office L02	& &	 &\\
\inc	 	 &L5-SL-AH6	 &Passage Massage	 	 &	 & &\\
\inc	 	 &L5-SL-AH7	 &HX Tertiary Plantroom	 &	 &&\\
\inc	 	 &L5-SL-AH8	 &HX Tertiary Plantroom	 &	 & &\\
\inc	 	 &L5-SL-AH9	 &Shangri-la (TFA)	 	 &	 & &\\
\inc	 	 &L5-SL-AH10	 &Shangri-la (TFA)	 	 &	 & &\\


 
\midrule
\inc	 	 &L6-SL-AH3	 &Pool Chemical Store 612	 &  && \\
\inc	 	 &L6-SL-AH4	 &Corridor 621	 	 	 & & &\\
\midrule



\inc	 	 &L7-SL-AH1	 &Pool Kitchen	 	 	  &  & &\\
 
\inc	 	 &L7-SL-AH3	 &Corridor	 	 	 &  & &\\
\inc	 &L41-SL-AH1	 &Gym	 	 	  	 &	 & &\\
\inc	 &L44-SL-AH1	 &TFA Kitchen	 	 	 && 	 & \\	 
\inc	 	 &L44-SL-AH2	 &Chinese Rest \& TFA Lounge	 &	&  & \\ 	  	 	 
\inc	 	 &L44-SL-AH3	 &Horizontal Lounge	 	 	&  & &\\	 	 
\inc	 	 &L44-SL-AH4	 &Chinese Restaurant + Private	&&  & \\	 	 
\inc	 	 &L44-SL-AH5	 &Kitchen	 	 	  	& 	& & \\ 
\midrule

\inc	 	 &L47-SL-AH1	 &Shangri-la Tower (TFA)	 	  	&&	 &\\
\inc	 	 &L47-SL-AH2	 &Shangri-la Tower (TFA)	 		&&	 &\\
\bottomrule
\end{tabular}

\end{table}



\begin{table}[htbp]
\setcounter{step}{0}
\label{tbl:AHUmerweb}
\footnotesize
\caption{Merweb Air Handling Unit installation target dates}
\begin{tabular}{l l p{2.7cm} c c l}
\toprule
Ref.	    &AHU tag 	    & Area	                 &Installed	&T\&C &Remarks\\
\midrule
 \inc	 	 &B3-MW-AH-001	 & Express Laundry	 	 	 &\checkmark &&\\	 

\midrule
 \inc	 	 &B2-MW-AH-001	 & Plantroom Lobby	    &\checkmark &&\\
\midrule  
 \inc	 	 &B1-MW-AH-001	 & TFA AHU	 	 	       &\checkmark &&\\
\midrule

 \inc		 &GR-MW-AH-001	& Lobby+Atrium+Fire Control	 &\checkmark && \\
 \inc	 	 &GR-MW-AH-002  &Cafe + Pantry	 	           &\checkmark && \\

\midrule
 \inc		 &L1-MW-AH-001	 &All Day Restaurant + Lobby	 &\checkmark &&\\
 \inc	 	 &L1-MW-AH-002	 &Main Kitchen	 	 	 &\checkmark &&\\
 \inc	 	 &L1-MW-AH-003	 &Kitchen, TFA	 	 	 &\checkmark &&\\	
 \inc	 	 &L1-MW-AH-004	 &Substation	 	 	    &\checkmark &&\\
 
 \inc	 	 &L1-MW-AH-005	 & Atrium	 	   	&\checkmark &&\\
 \inc	 	 &L1-MW-AH-006	 & Staff Dining	 	   &\checkmark &&\\
 \inc	 	 &L1-MW-AH-007	 & Change Room	 	   &\checkmark &&\\

\midrule
 \inc	 	 &L2-MW-AH-001   &Meeting Room 1	 	 &\checkmark &&\\
\inc	 	 &L2-MW-AH-002	 &Pre-function Lobby	 	 &\checkmark &&\\
 \inc	 	 &L2-MW-AH-003   & Lift Lobby/Corridor	 	 &\checkmark &&\\
 \inc	 	 &L2-MW-AH-004	 &Training Room	 	       &\checkmark &&\\
 \inc	 	 &L2-MW-AH-005	 &Open Office/Filing	  	 &\checkmark &&\\
\inc	 	 &L2-MW-AH-006	 &Meeting Room 2	 	    &\checkmark &&\\

\midrule
 \inc		& L4-MW-AH-001	 &Reception/Corridor	 	 &\checkmark &&\\
 \inc	 	& L4-MW-AH-002	 &Changing Rooms/Lockers	 &\checkmark &&\\
 \inc	   & L4-MW-AH-003   &Kitchen	 	  	 &\checkmark &&\\
 \inc	 	& L4-MW-AH-004	 &Women's Lounge	 	 &\checkmark &&\\
 \inc	 	& L4-MW-AH-005	 &Children's Area	 	 &\checkmark &&\\

\midrule
 \inc		 &L5-MW-AH-001	 &Corridor and Office	 	 &\checkmark &&\\
 \inc	 	 &L5-MW-AH-002	 &TFA Tower	 	  	 &\checkmark &&\\
 
 \inc	 	 &L5-MW-AH-003	 &Fitness Room	 	 &\checkmark &&\\
 \inc	 	 &L5-MW-AH0-004	 &TFA Tower 	 	  	 &\checkmark &&\\
 
\midrule

 \inc		& L24-MW-AH-001	& Jazz Bar	 	  &\checkmark &&\\
 \inc	 	 &L24-MW-AH-002	& Kitchen	 	  &\checkmark &&\\
 \inc	 	 &L24-MW-AH-003	& Lounge	 	  &\checkmark &&\\
\midrule
	 	 
 \inc		 &L43-MW-AH-001	 &Party Room	 	 &\checkmark &&\\
 \inc	 	 &L43-MW-AH-002	 &Indoor Pool	 	 &\checkmark &&\\	
\midrule

 \inc		 &L44-MW-AH-001	 &TFA Gym/Party Room &\checkmark &&\\
 \inc	 	 &L44-MW-AH-002	 &TFA Pool	 	  &\checkmark &&\\
\midrule
 \inc		 &L46-MW-AH-001	 &TFA Tower	 	 &\checkmark &&\\
 \inc	 	 &L46-MW-AH-002	 &TFA Tower	 	 &\checkmark &&\\
\bottomrule
\end{tabular}

\end{table}









%%
% ROTANA TOWER PROGRESS REPORT
\chapter{Rotana Tower}

\newthought{The physical completion} of the works is currently estimated at 98.5\%. Works remaining are mostly final fix activities\sidenote{These follow the progress of Dragoni.} and snags arising out of commissioning reports. There works will be completed when the Owner's Direct Contractor completes their portion of the works.\sidenote{This mostly affects final fix of wiring accessories, sprinklers and grilles.} 

Closure of the works was earlier hampered due to  Crompton's poor performance. Action has been taken to supplement their labour with HLS technicians, where required.

Commissioning of Electrical systems, fan-coil-units and BMS has started and is mostly completed.  CCTV works are on hold until such time as the Owner's Direct Contractor has mobilized. Target completion final completion and ready for hand-over\sidenote{Except IPTV, CCTV, telephones which were subjected to new EIs.} and Civil Defence Inspection readiness for this Tower is estimated at 30th March 2013.\sidenote{Subject to Dragoni completing their works and allowing adequate time for follow-up activities.} 


\section{Current Status of Works}

The current status of works is summarized in Table~\ref{tbl:RObalance}

\bigskip

%\begin{fullwidth} 
\small
\captionof{table}{Outstanding Works by System. This is a summary of main points. The person responsible to close the issue and report weekly is shown in colour.}
\label{SLsystems}
\RaggedRight
\begin{longtable}{@{}cp{3.5cm}@{}cp{.7cm}p{3cm}}
%\begin{tabular}{llll}
\toprule
Ref	&Package	&Phys. Works & T\&C & Remarks\\
\midrule
1.00	&\textbf{Mechanical}	&	&&\\
1.01	&Chilled Water Systems	
      &\checkmark	
      &\checkmark 
      & GG to close. \\
1.02	&Ductwork (Air side)  
      &\checkmark		
      &\checkmark 
      & Final fix in certain areas is incomplete 
        due to Dragoni works. T\&C measurements taken. \\
\midrule
2.00	&\textbf{Electrical}	& &\\
2.01	&Medium Voltage System	
      &\checkmark & \checkmark 
      & Summary WIRs.\\	

2.03  &Low Voltage System		
      &\checkmark & \checkmark.
      & Final accessories in some areas 
        incomplete due to Dragoni works.\\

2.04	&Earthing \& Lightning Protection   
      &\checkmark
      &X
      & Nidhal (Summary WIR). Commissioning to start.\\
		
2.05	&Lighting 
      &\checkmark
      &\checkmark& Nidhal \\

2.06	&Aircraft Warning System 
      &\checkmark
      &X &Nidhal arrange commissioning.\\	
	
2.08   &Signage      &&.& Nidhal\\
\midrule
3.00	&\textbf{Public Health}		&&&\\
3.01	&Potable Water	
      &\checkmark
      &X 
      & Sripathy \\	
3.02	& Sanitary Fixtures and Fittings 
      & X
      & X
      & Sripathy. Missing materials.\\		
\midrule		

4.00	&\textbf{Fire Defense} 
      &
      &
      &\\	

4.01	&FHC
      &\checkmark
      &\checkmark
      &Hose-rack nozzles to be replaced.\\ \\
	
4.02	&Smoke Exhaust		
      &\checkmark
      &\checkmark
      &\\

4.03	&Staircase Pressurization          
      &\checkmark
      &\checkmark
      & Retesting 5 Mar 2013.\\	
	
4.04	&Fire Alarm	
      &\checkmark
      &IP 
      &Nidhal\\

4.05   & Exit lighting 
       & \checkmark
       &  
       & Nidhal\\	

\midrule
5.00	&\textbf{Specialist Services}&&\\
5.01	&Building Management System	
       &\checkmark
       &\checkmark &Jabir (summary) \\	
5.02	&Security Access Control \& CCTV	
       & X
       & X.
       & EI works\\	
5.05	&Lighting Control		&&N/A&\\
5.06	&Structured Cabling		
      &X
      &X
      & Nidhal\\
5.07	&PA \& Background Music System &&N/A&\\		
5.08	&IPTV \& Satellite System	
      & \checkmark
      & X
      & EI integration, balance works\\	

5.09  &Telephone \& Data 
      &X
      &X
      &report available\\

\bottomrule
%\end{tabular}	
\caption{Summary of pending works}
\label{tbl:RObalance}
\end{longtable}
%\end{fullwidth}






\section{Final Snag List}

\begin{table}[htbp]
\begin{tabular}{lllp{4.5cm}}
\toprule
Item &Detail &Floor  & Remarks\\
\midrule
1 &Final Wiring Accessories &L8-25,42,43 & Subject to Dragoni \\
2 &Vestibule Lighting       & ditto      &      ditto\\
3 &UNC installation         & all        & will be installed during commissioning\\
4  &Floor boxes              &all         &material issue\\
5  &BMS ferrules             &all         & ditto\\
\bottomrule
\end{tabular}
\caption{Final Snag List Electrical, Rotana}
\end{table}

\begin{table}[htbp]
\begin{tabular}{lllp{4.5cm}}
\toprule
Item &Detail &Floor  & Remarks\\
\midrule
1 &FF sanitary &various & Material issue as per email to HLG, 9.01.2013. \\
\bottomrule
\end{tabular}
\caption{Final Snag List mechanical, Rotana}

\end{table}


\section{Paperwork}

\subsection{NCR\&FORs}

These have now been closed. 

\begin{comment}
\begin{table}[htbp]
\small
\begin{tabular}{lp{3.5cm}llp{2.5cm}}
\toprule
Item &NCR & Levels & Status &Remarks\\ 
\midrule
1    &Fire alarm cables and belden cables pulled in same cable tray and CCTV without containment inside ELV room. 
& all & &superceded by events, due to new EI.\\
\bottomrule
\end{tabular}
\caption{List of outstanding electrical NCRs, Rotana Tower}
\end{table}
\end{comment}

We expecting all documentation for closure of outstanding NCRs and FORs to be completed by 31.01.2013.

















\chapter{SHANGRILA TOWER}

\newthought{The physical installation} of the Shangrila Tower works are currently estimated at 97\% completion. Remaining items are described in the sections below.  Most of the remaining items  are constrained on non-personnel related issues. Commissioning has commenced for some of the mechanical systems (water, fan-coil units). Issues that affect Testing and Commissioning are described in Table~\ref{SLsystems}. Currently QMMC averages 63 personnel on site. 

\section{Issues with HOK}

\begin{enumerate}
\item EI No. 358 additional funnel drain for Ice Maker, affecting floors L41-L22 HK and Chute area.
\item EI No. 228 additional funnel drain for Ice Maker, near shaft 2. Instead of funnel drain normal drain was installed. Floor 9-21.
\end{enumerate}


\begin{table}[htbp]
\small\RaggedRight
\begin{tabular}{@{}llp{3cm}p{3cm}}
\toprule
Date &EI No. &Description &Remarks\\
\midrule
5/1/2012 &EI 709  &Additional smoke extract system &EI-709 affcecting LV 42 \&43. Correspondence HOK/225/L3239, HEE 14393.\\
\bottomrule
\end{tabular}
\caption{List of recently received RFI replies for Rotana Tower. The number of RFIs have now been reduced.}
\end{table}
\section{ID}
Level 48 Powder room SL4803 to be completed. Awaiting civil work tiling prior to installation.

%%
% SHANGRILA TOWER PROGRESS REPORT

\begin{table}[htbp]
\small
\caption{Shangri-la Ceiling and Wall Closure Plan}
\begin{tabular}{lllllp{3.5cm}}
\toprule 
\multicolumn{6}{c}{\bf Shangri-la Ceiling and Wall Closure}\\
\midrule
~        & Corridor & Pass. lift  & Serv. lift  & Rooms   & Dry walls \\
Level   &            & lobby             &lobby    &    &              \\ 
\midrule
Lvl 44  &             &                &                &             &              \\
Lvl 43  &             &                &                &             &              \\
Lvl 41  &             &              &         &     &\done        \\
Lvl 40  & \done     &\done&\done&\done&\done\\
Lvl 39  & \done     &\done&\done&\done&\done\\
Lvl 38  & \done     &\done&\done&\done&\done\\
Lvl 37  & \done     &\done&\done&\done&\done\\
Lvl 36  & \done     &\done&\done&\done&\done\\
Lvl 35  & \done     &\done&\done&\done&\done\\
Lvl 34  & \done     &\done&\done&\done&\done\\
Lvl 33  & \done     &\done&\done&\done&\done\\
Lvl 32  & \done     &\done&\done&\done&\done\\
Lvl 31  & \done     &\done&\done&\done&\done\\
Lvl 30  & \done     &\done&\done&\done&\done\\
Lvl 29  & \done     &\done&\done&\done  &\done\\
Lvl 28  & \done     & 27 Jul          &27 Jul         &\done &\done\\
Lvl 27  & \done    & 29 Jul   &29 Jul         &29 Jul         &\done \\
Lvl 26  & \done   & 2 Aug  & 2 Aug        &2 Aug         &\done \\
Lvl 25  & 22 Jul    & 4 Aug  & 4 Aug        &4 Aug         &\done \\
Lvl 24  & 27 Jul    & 7 Aug  & 7 Aug        &7 Aug         &\done \\
Lvl 23  & 27 Jul    & 9 Aug  & 9 Aug        &9 Aug         & \done \\
Lvl 22  & 03 Aug   &04 Aug & 11 Aug        &11 Aug         &\done \\
Lvl 21  & \done   &14 Aug  & 14 Aug        &14 Aug         &\done \\
Lvl 20  & 04 Aug   &09 Aug          &16 Aug         &16 Aug         &\done \\
Lvl 19  & 09 Aug   &28 Jul           &18 Aug         &18 Aug         &1 Aug\\
Lvl 18  & 28 Jul   &29 Jul & 21 Aug                  &21 Aug         &3 Aug\\
Lvl 17  & 29 Jul   &23 Aug  &23 Aug         &23 Aug         &\done \\
Lvl 16  & 01 Aug   &01 Aug  &25 Aug         &25 Aug         &\done \\
Lvl 15  & 01 Aug   &01 Aug  &27 Aug         &27 Aug         &9 Aug\\
Lvl 14  & 30 Aug   &03 Aug  &30 Aug         &30 Aug         &11 Aug\\
Lvl 13  & 03 Aug     &03 Aug    &1 Sep         &1 Sep         &13 Aug\\
Lvl 12  & 03 Aug     &06 Aug    & 4 Sep        &4 Sep         &\done \\
Lvl 11  & 06 Aug     &06 Aug    & 6 Sep        &6 Sep         &\done \\
Lvl 10  & 06 Aug     &06 Aug    & 8 Sep        &8 Sep         &19 Aug\\
Lvl 09  & 09 Aug   &09 Aug   & 11 Sep        &11 Sep         &21 Aug\\
Lvl 08  & 12 Aug   &12 Aug   & 13 Sep        &13 Sep         &23 Aug\\
\bottomrule
\end{tabular}
\normalsize
\end{table}


\section{By System}
This part of outstanding activities is sorted by service. Oustanding activities and constraints are outlined under remarks. 
\bigskip

\begin{fullwidth} 
\captionof{table}{Outstanding Works by System. This is a summary of main points. The person responsible to close the issue and report weekly is shown in colour.}
\label{SLsystems}
\RaggedRight
\begin{longtable}{@{}lp{3.5cm}@{}lp{5cm}p{1.5cm}}
%\begin{tabular}{llll}
\toprule
Ref	&Package	& &Outstanding Works& Responsibility\\
\midrule
1.00	&\textbf{Mechanical}	&	&&\\
1.01	&Chilled Water Systems	&	&Quick fill, AAV, GV, identification labels. None affecting commissioning. System is operational. &\\
1.02	&Ductwork (Air side)  &		& L48, minor works. L8-L45, fire dampers missed. Work on-going not affecting commissioning.3 No. VFD panels level 44-47 pending affecting system commissioning. AHU heater terminations, incomplete.&\\
1.03    &Kitchen Extract System && ECUs installed, black steel ducting completed. Insulation and inspections to be completed.&\\


\midrule
2.00	&\textbf{Electrical}		&&\\
2.02	&Medium Voltage System	&& Bus duct commissioning completed. Level 42-43, 25 mm armoured cable glands.&\\	
2.03	&Low Voltage System		&& 45A DP EI-632E at levels 18, 19 and 41. Materials unavailability.&\\
2.04	&Earthing \& Lightning Protection &&& \\		
2.05	&Lighting && Core shafts bulk-heads EI-179. Dimmer DBs L42-43.& \\
2.06	&Aircraft Warning System && Completed. To be commissioned.& \\		
2.07	&UPS System	&&N/A Level 42 5kVA UPS not installed. Material not available.& \\	
2.08   &Signage      &&Signage lights for 63A (water proof) isolator not done (Level 45). Material issue. Sanitary final fix to be confirmed.& \\
\midrule
3.00	&\textbf{Public Health}		&&&\\
3.01	&Potable Water	&& L47, L48 Water hammer arrestor. Service shafts remedial works, affecting L8-34. Identification, completeness report.&  \\	
3.02	&Above Ground Drainage && Kitchen waste, CI from levels 8-44. Vent pipes 48-49, see also level 40 and level 48 (awaiting tiling layout).& \\		
3.03	&Below Ground Drainage  && N/A&\\		
3.04	&Water Features	     && N/A&\\	
3.05	&Gas Supply Installations	&& Gas detection system outstanding. Works by Gasco.&\\	
\midrule		
4.00	&\textbf{Fire Defense} &&&\\	
4.01	&FHC, FH and Sprinkler Installations	&&Nozzles to be installed. Material issue. Testing of risers. Removal of isolating valve to electrical rooms.&\\	
4.02	&Smoke Exhaust		&&&\\
4.03	&Staircase Pressurization          &&&George to start T\&C\\		
4.04	&Fire Alarm		&& Heat detectors missing. EIs incomplete. Some final fix not complete due to wall papers.&\\
4.05	&FM-200		&&N/A&\\
4.06	&Heliport Foam System	 &&DB heliport not done. Foam installation. Deliveries from Naffco are pending.&\\
4.07   & Emergency lighting &&EI-655 Lvl. 8-48 material not availble HEE responsibility. &Jeffrey\\	
\midrule
5.00	&\textbf{Specialist Services}&&\\

5.01	&Building Management System	&& Commissioning report pending. See chart for BMS per floor.& \\	


5.02	&Security Access Control \& CCTV	&&Installation of battery back-up for access control syste. Material issue. Card readers and push buttons at corridor areas incomplete due to wall paper not completed.&\\	

5.03	&CO Monitoring		&&N/A&\\

5.04	&Car Calling System		&&N/A&\\

5.05	&Lighting Control		&&N/A&\\

5.06	&Structured Cabling		&&To start after podium is ready to avoid damages.& Nidhal\\

5.07	&PA \& Background Music System &&N/A&\\
	
5.08	&IPTV \& Satellite System	&&IPTV System delayed due to late instructions and ordering. Works not started.&\\	

5.09	&Room Management System	&&Status for drawings. RCU terminations not done LVL 18, 19 and 41 VVIP rooms and Presidential suite. No approved termination drawings and RCU components.&Rahul\\

5.10 &Media hub && Not available. Affects all floors.&Nidhal \\

\bottomrule
%\end{tabular}	
\end{longtable}
\end{fullwidth}

\section{EIs}

EI-0012 affecting corridors on Levels 08-17 received 18 Feb 2012. Serious implications.



















%%
% MERWEB TOWER PROGRESS REPORT
\chapter{Merweb Tower}

\newthought{The Merweb physical} installations are estimated as 92\% completed. Progress on the Tower has been suffering due to JEM having reduced personnel a number of times due to commercial issues. The current work force on site is 26 personnel, which is totally inadequate for the works.
Of concern are works still in shafts and works affecting follow-up trades. Our recommendation is to add HS personnel to assist the works.

\index{Towers!Merweb Progress}

\section{Additional Works}

List of recent EIs.


%\begin{table}[htbp]
%\small\RaggedRight
%\begin{tabular}{@{}llp{3cm}p{3cm}}
%\toprule
%Date &RFI No. &Description &Remarks\\
%\midrule
%5/1/2012 &RFI 901-E-1424  &Level 24 add four lights &EI-724 to follow.\\
%\bottomrule
%\end{tabular}
%\caption{List of recently received RFI replies for Rotana Tower. The number of RFIs have now been reduced.}
%\end{table}


\section{By System}
This part of outstanding activities is sorted by service. Oustanding activities and constraints are outlined under remarks. 
\bigskip

\newcounter{sectionstep}
\setcounter{sectionstep}{0}

\def\tablesection{%
  \midrule
  \stepcounter{step}%
  \setcounter{sectionstep}{-1}%
  \setcounter{sectionstep}{-1}%
}

\def\Inc{%
  \stepcounter{sectionstep}%
\thestep.0\thesectionstep%
}


\begin{fullwidth} 
\captionof{table}{Outstanding Works by System. This is a summary of main points. }
\label{SLsystems}
\RaggedRight\small
\setcounter{step}{0}
\begin{longtable}{@{}lp{2.8cm} c l p{3.2cm} }
\toprule
Ref	&Package	 &Inst. & T\&C & Remarks \\
\tablesection

\Inc	&\textbf{Mechanical}	&	&&\\
\Inc	&Chilled Water Systems	&\checkmark	&&  \\
\Inc	&Ductwork (Air side)  &		& & \\
\Inc  &Kitchen Extract System & x & N/A&\\

\tablesection
\Inc	&\textbf{Electrical}		&&&\\
\Inc	&DBs energized	&\checkmark && \\	
\Inc	&Small Power	&x&& \\
\Inc	&Earthing \& Lightning Protection &\checkmark&&  \\		
\Inc	&Lighting &&&  \\
\Inc	&Aircraft Warning System \&Obstruction lights. &&& \\		
\Inc	&UPS System	&&& \\	
\Inc   &Signage      &&& \\
\Inc   &Facade lighting &\checkmark &  &\\

\tablesection
\Inc	&\textbf{Public Health}		&&&\\
\Inc	&Potable Water	&\checkmark&& \\	
       &Sanitary Final Fix &x&&\\
\Inc	&Gas Supply Installations	&\checkmark & &\\
\Inc  &Gas Supply final hook-up & x         &x \\	

\tablesection		
4.00	&\textbf{Fire Defense} &&&\\	
4.01	&FHC, FH and Sprinkler Installations	&&&  \\	

4.02	&Smoke Exhaust		&&&\\

4.03	&Staircase Pressurization          && &\\	
	
4.04	&Fire Alarm		&&& \\

4.05	&FM-200		&&&\\


4.07   & Emergency lighting &&& \\	
\midrule
5.00	&\textbf{Specialist Services}&&\\

5.01	&Building Management System	&&& \\	

5.02	&Security Access Control \& CCTV	&& &\\	

5.03	&CO Monitoring		&&&\\

5.04	&Car Calling System		&&&\\

5.05	&Lighting Control		&&&\\

5.06	&Structured Cabling		&&&\\

5.07	&PA \& Background Music System &&&\\		

5.08	&IPTV \& Satellite System	&&Status incomplete& \\	
5.09	&Room Management System	&&& \\
5.10 &Media hub && & \\
5.11 &Telephone \& Data &&&\\

\bottomrule
%\end{tabular}	
\end{longtable}
\end{fullwidth}

\small
\begin{table}[htbp]
\caption{Merweb Ceiling and Wall Closure Plan}
\begin{tabular}{llllll}
\toprule 
\multicolumn{6}{c}{\bf Merweb Ceiling and Wall Closure}\\
\midrule
~        & Corridor & Pass. lift  & Serv. lift  & Rooms   & Dry walls \\
Level   &            & lobby             &lobby    &    &              \\ 
\midrule
Lvl 43  &             &                &                &             &              \\
Lvl 42  &	  &		&	&	&	\\
Lvl 41  &             &              &         &     &        \\
Lvl 40  & \done     &\done&\done&\done&\done\\
Lvl 39  & \done     &\done&\done&\done&\done\\
Lvl 38  & \done     &\done&\done&\done&\done\\
Lvl 37  & \done     &\done&\done&\done&\done\\
Lvl 36  & \done     &\done&\done&\done&\done\\
Lvl 35  & \done     &\done&\done&\done&\done\\
Lvl 34  & \done     &\done&\done&\done&\done\\
Lvl 33  & \done     &\done&\done&\done&\done\\
Lvl 32  & \done     &\done&\done&\done&\done\\
Lvl 31  & \done     &\done&\done&\done&\done\\
Lvl 30  & \done     &\done&\done&\done&\done\\
Lvl 29  & \done     &\done&\done&\done&\done\\
Lvl 28  & \done     & \done          &\done         &\done        &\done\\
Lvl 27  & \done    & \done   &\done         &\done         &\done \\
Lvl 26  & \done    & \done  & \done        &\done         &\done \\
Lvl 25  & 4 Aug    & \done  & \done        &done         &\done \\
Lvl 24  & 7 Aug    & 7 Aug  & 7 Aug        &7 Aug         &\done \\
Lvl 23  & 9 Aug    & 9 Aug  & 9 Aug        &9 Aug         & 24 Jul\\
Lvl 22  & 11 Aug   &11 Aug & 11 Aug        &11 Aug         &26 Jul\\
Lvl 21  & 14 Aug   &14 Aug  & 14 Aug        &14 Aug         &28 Jul\\
Lvl 20  & 16 Aug   &16 Aug          &16 Aug         &16 Aug         &30 Jul\\
Lvl 19  & 18 Aug   &18 Aug           &18 Aug         &18 Aug         &1 Aug\\
Lvl 18  & 21 Aug   &21 Aug & 21 Aug                  &21 Aug         &3 Aug\\
Lvl 17  & 23 Aug   &23 Aug  &23 Aug         &23 Aug         &5 Aug\\
Lvl 16  & 25 Aug   &25 Aug  &25 Aug         &25 Aug         &7 Aug\\
Lvl 15  & 27 Aug   &27 Aug  &27 Aug         &27 Aug         &9 Aug\\
Lvl 14  & 30 Aug   &30 Aug  &30 Aug         &30 Aug         &11 Aug\\
Lvl 13  & 1 Sep     &1 Sep    &1 Sep         &1 Sep         &13 Aug\\
Lvl 12  & 4 Sep     &4 Sep    & 4 Sep        &4 Sep         &15 Aug\\
Lvl 11  & 6 Sep     &6 Sep    & 6 Sep        &6 Sep         &17 Aug\\
Lvl 10  & 8 Sep     &8 Sep    & 8 Sep        &8 Sep         &19 Aug\\
Lvl 09  & 10 Sep   &10 Sep   & 11 Sep        &11 Sep         &21 Aug\\
Lvl 08  & 12 Sep   &12 Sep   & 13 Sep        &13 Sep         &23 Aug\\
Lvl 07  & 15 Sep   &15 Sep   & 15 Sep        &15 Sep         &25 Aug\\
\bottomrule
\end{tabular}
\normalsize
\end{table}

\out{Rotana Tower Completion : }{15 Sep 2010}



\chapter{Staircase Pressurization}
\label{pressurization}


\newthought{The Staircase Pressurization} systems are expected to be completed by the 25 February 2012. Physical works have been completed and commissioning has been completed for Rotana and Shangrila. The Merweb has been delayed due to the installation of sensor wire. 

          

\section{Constraints}


\begin{marginfigure}
\includegraphics[width=\textwidth]{staircase-pressurization}
\caption{Staircase pressurization systems for Rotana and Shangrila have been commissioned, however the inspections have been returned status 'C' due to being 4Pa less than design pressure at bottom of staircase.}
\end{marginfigure}

\begin{marginfigure}
\includegraphics[width=\textwidth]{staircase-pressurization-01}
\caption{Staircase pressurization systems for Rotana and Shangrila have been commissioned, however the inspections have been returned status 'C' due to being 4Pa less than design pressure at bottom of staircase.}
\end{marginfigure}

No further constraints exist in regards to the commissioning of this service, however temporary doors and closures need to be changed to permanent by HLG before the final inspection by Civil Defence. An issue with the Engineer regarding performance tests is expected to be resolved once full integration tests are completed or earlier. (See Section on Life Safety Integration Acceptance Tests).












%\newcommand{\genon}{25.10.2010}
%\newcommand{\flueon}{25.10.2010}           % ecological units on
%\renewcommand{\toweron}{26.10.2010}



\chapter{Standby Generators }
\begin{marginfigure}
  \includegraphics[width=\linewidth]{463}
  \caption{Cooling tower water pipes in Basement 2.}
  \label{fig:generators}
\end{marginfigure}

\newthought{The Standby Generators} physical works within the Generator Plantrooms have been completed. The cooling tower cooling systems serving the Generators have also been completed and tested.

           \begin{margintable} 
	    \begin{tabular}{lcl}
	      \toprule
	      System    & Key Milestone Date  \\
	      \midrule
                  Rotana Kitchen Extract       &  \ecolon  \\   
                 Shangri-la Kitchen Extract   & \ecolon\\
	     Merweb Kitchen Extract       &  \ecolon  \\
	      
	      \bottomrule
	    \end{tabular}
           \caption{Kitchen Extract Ventilation Key Dates}
           \label{tbl:KEkeydates}
            \end{margintable}
 

\section{Current Status}

The generators are in place, as well as most of the electrical support installations, such as cable trays and cable ladders. Mechanical piping works for the Cooling Tower circuits have been completed. Major work left to carry out includes all Panel installations, completing the Cooling Tower circuits\sidenote{\hl{Considerable progress has been achieved}.}, the generator exhaust piping and the fuel oil system. \sidenote{The fuel oil system has still not be designed, although an EI has been outstanding for more than a year. This is expected to be a three month cycle to approve, procure and install. \hl{Drawings have now been completed and issued to HOK}}

\begin{tabular}{llll}
\toprule
Item & Detail & Status & Remarks\\
\midrule
1    & Cooling Tower piping & 100\% &\\
2    & Coooling Towers      & 100\% &\\
3    & Chemical Treatment   & 100\% &\\
4    & Pumps                & 100\% &\\
5    & Fuel system          & 100\% &\\
\midrule
6    & BMS                  & 100\% &\\
7    & FM-200               &  90\% &\\
8    & Lights               & 100\% &\\
9    & Fire Alarm           & 100\% &\\
\midrule
10   & Panel Testing        &       &\\
11   & ATS switch           &       &\\
\midrule
12   & Manufacturer's Test  &       &\\
13   & T\&C full system     &       &\\
\bottomrule

\end{tabular}



\section{Constraints}



\section{Sequence for Switch-on}
\normalsize
In order to enable progressive commissioning early switch-on will be targetted. Although it may not be possible to achieve overall switch-on within the time frame envsisaged, sub-systems can be tested earlier on for example, the Cooling Tower circuit, the Chemical Treatment for same and the Fuel system. On the Electrical side, it will only be possible to carry full tests, once a substantial electrical load can be ready for switch-on. However, we expect all cable testing and inspections to be completed by the 30th October.

\section{Constraints}
\subsection{Fuel system}
\begin{enumerate}
\item Woquod tanks are not in place.\TODO{Sripathy action list}
\item Day tanks have not been delivered.
\end{enumerate}
\subsection{Commercial Issues}

The appointment of a subcontractor for commissioning has been issued to HLG on the . We have not had a response as yet. This is expected to delay commissioning with a knock-on effect on the Civil Defence Inspections.
\index{Civil Defence Inspection! Constraints! Generator subcontractor.}

\section{Documentation}

\begin{tabular}{lll}
\toprule
Item & Details & Status\\
\midrule
1    & T\&C Method Statement & "B"\\
2    & O\&M Manual           & "FA since 4/11/2012" \\
\bottomrule
\end{tabular}


















\chapter{Telephone and Data}

\newthought{The Telephone and Data Systems} are nearing completion with the PABX rooms in a state where telephones
can be connected in the near future. The system for practical
purposes is divided into the PABX rooms and the network. The network 
installation is still work in progress, mainly for terminations, testing 
labeling and replacement of cabinets to acommodate Engineer's Instuctions on convergent networks.\sidenote{This refers to EI-266 \& EI-815-E, scheme approved by Owner via letter dated 18 Nov 2012. Currently the Owner is disputing time implications.}

\begin{marginfigure}
\includegraphics[width=\linewidth]{PABX-001}
\caption{PABX rooms have been delayed due to the introduction of convergent networks and other issues. We are still awaiting for Supplir to deliver active components.}
\end{marginfigure}

\section{Current Status PABX Rooms}


\begin{table}[h]
\begin{minipage}{\textwidth}
\RaggedRight\small
\begin{tabular}{lp{3.5cm}c>{\small}p{3.5cm}}
\toprule
Item & Detail & Status & Remarks\\
\midrule
\inc    & Physical works, cable pulling, raised floor etc.
     & 100\% & PAC to be commissioned. Supplier pre-inspection took place on 9.01.2012.\textsuperscript{a} \\
\inc    & Final Terminations and testing & \checkmark & 15 Jan 2013\\
\inc    & Intercon Installation PABX      & \checkmark & T\&C balance\textsuperscript{b}\\ 
\bottomrule
\end{tabular}
\protect\def\footnoterule{}
\footnotetext[1]{Awaiting for supplier to confirm date.}
\footnotetext[2]{These works are in the scope of Owner's Subcontractor (Intercon).}
\end{minipage}

\end{table}

\section{Constraints}

The downstream installations for final accessory fixing, is currently being completed after Dragoni finishes. We are unsure as to when Dragoni is to complete these works in Towers and all key dates mentioned here are subject to these works be completed on time.

The EI for Convergent networks is still to be completed.\sidenote{Contract recommendation is with HLG for official appointment.} Zerone has been instructed to proceed with the works.

\section{Target Completion Dates}

\begin{enumerate}
\item  Completion of PABX Rooms 30 Jan 2013.
\item  Telephone activation 17 February 2013.
\item  Balance Completion T\&C  30 Mar 2013.
\item  PAC T\&C  13 February 2013.          
\end{enumerate}


 









\chapter{Chilled Water Plantrooms}
\index{Chilled Water Plantrooms}
\normalsize
%\begin{comment}
%\begin{marginfigure}%
%  \includegraphics[width=\linewidth]{rusting}
%  \caption{Equipment show eveidence of corrosion and neglect. Starting these equipment can present problems. Spares %need to be sourced as early as possible.(\url{http://asymptote.sf.net/}).}
 % \label{fig:marginfig1}
%\end{marginfigure}
%\end{comment}

\newthought{The Chilled Water Plantrooms are expected}  to be completed by late September. The Level 5 Plantrooms have been substantially completed during June,  in order to provide wild air to the Towers. The Qatar Cool Plantrooms in basements  are at an advanced stage (currently hydro-testing).  We expect these systems to be fully running and the Chilled Water System balanced in the Towers by the end of October. \sidenote{Subject to BMS and other materials being available. Control valve deliveries for Qatar Cool Plantrooms may extend these dates, seriously.}

\section{Basement Qatar Cool Plantrooms}
\index{Qatar Cool Plantrooms}
The following are plantroom milestone dates \sidenote{Milestones are for Mechanical works, electrical works and BMS works are depended on deliveries.}:
 

  \begin{table}[htbp]
   \begin{center} 
    \begin{tabular}{lccl}
      
      \toprule
      Plantroom   &  Details & Milestone Date  \\
      \midrule
      Rotana       &  8   &  $26^{th}$ August 2010  \\
      Merweb      &  12 &  1$^st$ September 2010  \\
      Shangrila    &  12 &  30$^th$ August 2010  \\
      \bottomrule
    \end{tabular}
    
  \end{center}
  \caption{Basement 3, Qatar Cool Plantrooms}
  \end{table}
 
  
  



Outstanding works include the installation of balance pumps and seals. \sidenote{These were expected late July, early August, but have been delayed due to Commercial issues. These are now expected next week.}, electrical works, installation of controls, testing and inspections. The most time consuming part are the inspections, testing and insulation that are all consequential.\sidenote{During the Wild Air operation some inspections were omitted, these need to be closed. A detailed Method Statement has been submitted and we are expecting to close the issues soon.}

For electrical works, we are constrained by deliveries of equipment by the Verger. Mostly due to changes in equipment instructed by EIs, although situation is improving.
Another item that is delaying completion is the control valves which have not as yet been finalized. \sidenote{They have now been approved, but quoted deliveries are too long. Procurement is trying to better these.}


\section{Interconnection Piping to Level 4}

This interconnection risers from Basements to Level 5 need to be  completed in order to switch-over the temporary wild air installations to Qatar Cool chilled water.\sidenote{For Qatar Cool Connection targets, see Appendix B}. Large sections of these works have been completed, but finalization is pending the delivery of branch valves. In many instances branch valves have not been indicated on schematics, but these have to be installed to enable progressive testing and commissioning.
\begin{center} 
    \begin{tabular}{lccl}
      \toprule
      Plantroom   &  Details & Milestone Date  \\
      \midrule
      Rotana       &  8   &  $26^{th}$ August 2010  \\
      Merweb      &  12 &  30th August 2010  \\
      Shangrila    &  12 &  30th August 2010  \\
      \bottomrule
    \end{tabular}
 
  \end{center}
  

These are works handled directly by Specon, but in order to speed-up the works portions of this piping will be sub-contracted to Nidhi and possibly MEPE. Dates are subject to commercial issues being expedited for extending Nidhi's work.


\section{Level 5 ETS Plantrooms}
These plant rooms have been substantially completed in order to provide \emph{wild air}, however, they still require the completion of the balance of pumps and electrical works, as well as snagging etc.

\begin{center} 
    \begin{tabular}{lccl}
      \toprule
      Plantroom   &  Details & Milestone Date  \\
      \midrule
      Rotana       &  8   &  $26^{th}$ August 2010  \\
      Merweb      &  12 &  20th August 2010  \\
      Shangrila    &  12 &  30th August 2010  \\
      \bottomrule
    \end{tabular}
 
  \end{center}

Once again we envisage problems with the delivery of Control Valves. Numerous other issues remain to fully close issues with the Consultants.

\section{Qatar Cool Connection}

The major constraints in obtaining Qatar Cool Connections are the following:

\begin{enumerate}
\item Kahraama Power\index{Kahraama Power On}  on
\item Completion of all Chilled Water piping in Podia
\item Flushing in Podia
\item Balancing of Tower Chilled Water Systems
\item Delivery of Control Valves
\item Commissioning of Tower Chilled Water, VFDs etc
\end{enumerate}

It is estimated that flushing - will take between 15-30 days - as piping is mostly horizontal and more difficult to clean debri. However, based with discussions with Qatar Cool, we might be able to get a connection earlier, subject to identifying circuits that can be isolated and flushed individually. However, it is unlikely to do so without all controls working and Kahraama permanent power being available. Dates for the most optimistic scenario are shown below.

\begin{table}[htbp]
\small
\begin{tabular}{p{4.3cm}p{3.5cm}l }
\toprule
\multicolumn{3}{c}{{Rotana Qatar Cool  Milestone Dates}}\\
\midrule
Activity	                                       &Details/Remarks	&Target Date                \\
\midrule
Power On		                           &      			& 25th Aug      \\
~~\em{Power to QC Plantroom}	   &                              &                           \\	
~~\em{Permanent Power to L5 HX}     &                               &                           \\
~~\em{Plantroom}		               &                              &                           \\
Completion Podium Piping	               &		           & 25th Aug     \\
~~\em{Flushing}                                   &                              & 9th Sep         \\
Qatar Cool Inspection 	               &      		           &11th Sep\\
Qatar Cool Connection	               &		           &17th Sep\\
 \bottomrule
\end{tabular}
\caption{Qatar Cool Connection Rotana}
\end{table} 


\begin{table}[htbp]
\small
\begin{tabular}{p{4.3cm}p{3.5cm}l }
\toprule
\multicolumn{3}{c}{{Shangri-la Qatar Cool  Milestone Dates}}\\
\midrule
Activity	                                       &Details/Remarks	&Target Date                \\
\midrule
Power On		                           &      			& 30th Aug      \\
~~\em{Power to QC Plantroom}	   &                              &                           \\	
~~\em{Permanent Power to L5 HX}     &                               &                           \\
~~\em{Plantroom}		               &                              &                           \\
Completion Podium Piping	               &		           & 30th Aug     \\
~~\em{Flushing}                                   &                              & 20th Sep       \\
Qatar Cool Inspection 	               &      		           &21st Sep\\
Qatar Cool Connection	               &		           &27th Sep\\
 \bottomrule
\end{tabular}
\caption{Qatar Cool Connection Shangrila}
\end{table} 

\begin{table}[htbp]
\small
\begin{tabular}{p{4.3cm}p{3.5cm}l }
\toprule
\multicolumn{3}{c}{{Merweb Qatar Cool  Milestone Dates}}\\
\midrule
Activity	                                       &Details/Remarks	&Target Date                \\
\midrule
Power On		                           &      			& 7th Sep      \\
~~\em{Power to QC Plantroom}	   &                              &                           \\	
~~\em{Permanent Power to L5 HX}     &                               &                           \\
~~\em{Plantroom}		               &                              &                           \\
Completion Podium Piping	               &		           & 7th sep    \\
~~\em{Flushing}                                   &                              & 26th Sep      \\
Qatar Cool Inspection 	               &      		           &28th Sep\\
Qatar Cool Connection	               &		           &30th Sep\\
 \bottomrule
\end{tabular}
\caption{Qatar Cool Connection Merweb}
\end{table} 

















\end{document}
%Main files start here
\mainmatter %% Very important to include
%\input{./Sections/CTC}
%\input{./Sections/Materials}
\chapter{Introduction}
\begin{marginfigure}%
  \includegraphics[width=\linewidth]{./graphics/knuth-drofina}
   {\scriptsize \hfill Credit:\thinspace{\small\itshape Florina Yales}}
  \caption{During the eraly days of typography fonts were designed to emulate the looks of calligraphic texts.}
  \label{fig:marginfig1}
\end{marginfigure}

\epigraph{When I put in the calculation of prime numbers  into the TeX manual I was
not thinking of this as the way to use TeX. I was thinking, Oh, by the way,
look at this: dogs can stand on their hind legs and TeX can calculate prime
numbers.}{--- Donald Knuth}

      
Ever since \TeX was invented, the publishing world has never been the same. This book can help you typeset beautiful publications but it will also help you to learn how to program \TeX\ and \latex\. The word macro is a bit of a misnomer, \TeX is perfectly capable to get you writing a compiler, a chess game, make presentations that are an order of magnitude greater than those you can do in power point. As a Turing complete engine.\citep{Bringhurst2005}, \citet{Bringhurst2005}, \citeauthor*{Bringhurst2005}

\TeX is not all that difficult, but it is different. As a special purpose language. Someone has posted a question in 
 stackoverflow.com, if it is still worth learning \TeX\ or its macro \LaTeXe\. The enthusiasm of the answers, will probably mean that Knuth's prediction that \TeX, will last for 100 years will probably come true. One can imagine that the superhuaman race that is to arrive at the Omega point will be using it!\cite{Knuth1990}

Many people swear by Plain TeX, and produce highly respectable documents using it (Knuth is an example of this, of course). But equally, many people are happy to let someone else take the design decisions for them, accepting a small loss of flexibility in exchange for a saving of (mental) effort.

The arguments around this topic can provoke huge amounts of noise and heat, without offering much by way of light; your best bet may be to find out what those around you are using, and to follow them in the hope of some support. Later on, you can always switch your allegiance; don’t bother about it.

If you are preparing a manuscript for a publisher or journal, ask them what markup they want before you develop your own; many big publishers have developed their own (La)TeX styles for journals and books, and insist that authors stick closely to their markup.


\section{What is Latex?}

In 1978, Donald Knuth embarked on a project to create a typesetting system, called Tex (pronounced 'tech'), after being disappointed with the quality of his acclaimed {\em The Art of Programming series}. Around 10 years later, he froze the language after originally anticipating spending a single year! Tex gave extremely fine-grained control of document layout. However, the vast flexibility meant it was complex, so by the mid-80s Leslie Lamport created a set of macros that abstracted away many of the complexities. This allowed for a simpler approach for creating documents, where content and style were separate. This extension became Latex (pronounced 'lay-tech').\citet{Bringhurst2005}

Latex is essentially a markup language. Content is written in plain text and can be annotated with various 'commands' that describe how certain elements should be displayed. The Latex interpreter reads in a Latex marked-up file, renders the content into a document and dumps it a new file. Therefore, it's not an interactive system that is the de-facto method for document creation nowadays.

\section{Separation of content and style}

Not the most obvious advantage, possibly because a lot of Word users don't understand why this so beneficial. When producing your Latex document, you are concentrating on the content itself. You introduce structure explicitly by telling Latex when a new section begins, for example, but you don't then faff around trying to decide how the section headers should look. That's done later.

This is opposed to the average Word user, who will immediately highlight a given section header and apply formatting to it: maybe a larger font, maybe underline, etc. The point is that this will then have to be applied to every header manually. Latex is better as it uses a document style. This defines how different elements within your document should look (like Cascading Style Sheets defining styles in HTML pages). If you fancy a change, you only change the style definitions once, then the presentation of the document will be updated automatically. This also ensures a consistent looking document (you wouldn't believe how many stylistically inconsistent Word docs I've read!)

Word does in fact have a similar Styles feature. However, because it's optional, people don't often know it exists. Latex forces you to declare the document semantics (this is a Good Thing!), which is why you can rely on it to produce a consistent looking document.

\section{Portability}

Latex portability comes in multiple ways:

An actual Latex file is merely a text file, which is just about the most portable format in computing.
The Latex system that processes the text file and produces the finished document has been implemented on just about every mainstream platform you care to mention.
The default output file format for Latex is DVI (which stands for device independent). This was around well before PDF was dreamed up and the high quality files can be viewed via software viewers or printed out. DVI is an open standard, so once again, readers are extremely portable and exist on most operating systems. Admittedly, DVI is hardly ubiquitous and nowadays it's often bypassed in favour for PDF (or it's very simple to convert to other formats like PS or HTML)

\section{Flexibility}
You can get Latex to do just about anything you can think of! Over the years, an overwhelming selection of packages to extend its potential and macros that can simplify complex tasks have come into being, most of which are freely available on CTAN. For example, Latex's main users are within academia and research institutions and they benefit hugely thanks to the Bibtex package that provides bibliography management - I pity my Word-using colleagues who suffer by actually manually word-processing their bibliographies (unless they've shelled out for a program like Endnote). There are other crazy packages that you can install which allow you to typeset music scores, chessboards and cross-words! CTAN is the main repository of these resources. Most are well documented and as you can imagine, with Latex being around for so long, the number of extensions is vast. The chances are, if you're struggling to do a task, someone will have undoubtedly written a package to solve it easily!

\section{Control}
Even with simple documents, you can quickly become frustrated by Word's rather unintelligent interference. The hours that are wasted trying to position that image which you know will fit at the bottom of the page, but Word refuses to put it there! How many can relate to this experience? You have your 30 page document with text, tables and images. You just spent the evening getting it formatted nicely - all your figures in the right place and then you notice that one of your paragraphs isn't clear enough. You add one sentence, which then pushes an image on to the next page, leaving a massive gap at the bottom of that page where your image once was. This then daisy-chains down, knocking other tables and images out of place all the way to the end of your document! It's a real laugh. Fortunately, Latex is much more clever in this respect and positions your images and tables with a lot of common sense. So, if you want your image to appear at the bottom of a given page, it'll stay there!

Whilst Latex makes decent typesetting decisions for you, if you want to, you can have total control over the presentation of your document.

\section{Quality}
It's difficult to disagree that the output from Latex is far superior to what Word can produce. This is emphasised greatest when it comes to documents with high mathematical content, which is a major strength for Latex. It also has much better kerning, hyphenation and justification algorithms that simply make the output far more professional than what any word processor. Its algorithms for laying out text are more sophisticated and extremely fine-grained. For example, the accuracy is so high because it uses a measurement known as a scaled point which translates as 100th of the wavelength of natural light!

Latex works with the concept of niceness (well, I suppose technically it's badness - which it works to minimise). Latex has a large set of metrics that it evaluates against when generating your document. It experiments with various permutations of parameters and determines the one which gives the "nicest" output. It can take the time to do this because it isn't interactive. Word processors don't have the computational resources available (yet) to carry out the equivalent calculations and still remain interactive. Also, many people forget that typesetting is actually a professional skill - people train for years to learn how to layout publications. Yet, as soon as you open a word processor, you go about committing typesetting sins all the way. Typesetters know for example that its easier to read sentences that are approximately 66 characters wide. Have a look in your books and count the letters! Also, why do newspapers and magazines have narrow columns? But, the default layout of a word processor gives an average of 100 words per line. I suppose many people don't mind, but you would notice if you read a lot of large documents.

A quick example. I took a document that I had used previously to demonstrate document structure in Latex. I used the same text and loaded it into Word and applied the equivalent styles. I've used default settings throughout. Word didn't have a style for abstracts, so I put the title in bold. View the Latex output to the Word output. The styles that Word uses aren't great. You could manipulate the default styles in Word to make it look more reasonable, but I've never been bothered because even if I could get it to match Latex stylistically, I still have to use Word, which I'd rather avoid!

Latex has been used regularly typeset entire books. Word processors simply aren't good enough for that job - they are used by the authors to write the content and these files are then imported into professional typesetting software. Ok, that's not strictly true - you could typeset a book in Word, just like you could drive a car with your feet - it's not a good idea though!

\section{Output}
As mentioned, the default output is a |DVI| file. |DVI| was a clever little standard but unfortunately didn't take-off. It takes little effort to convert your document into a Postscript or PDF file (in fact, you can just use the 'pdflatex' command instead of normal 'latex' if you only ever want to create PDFs). There's no need to buy additional software such as Adobe Acrobat like you need to do to convert a Word document into PDF. (At least OpenOffice has its 'Export to PDF' functionality!)

In my personal experience, using Word for documents with more than 20 pages has not been a pleasant experience. Obviously, that could be my own bad luck, but that is also the impression I've got from other users too.

With Latex, I've never found such problems. Additionally, you are free to split up large documents into smaller chunks and then let Latex combine them altogether later (like one chapter per file). It can also create tables of content, indexes and bibliographies easily, even on multi-file projects.

\section{Stability}
One of the reasons why perhaps so many people struggle with Word when creating large documents, is because it is prone to crashes. 'Document recovery' is now a high ranking feature of Word. I'm sure people would prefer if MS would just make their software more stable! (NB stability issues are not necessarily generalisable, so I'm speaking from personal experience, and of my friends and colleagues - I do not know of a single user who hasn't lost work to Word, but that's not to say that such people don't exist.)
\begin{marginfigure}%
  \includegraphics[width=\linewidth]{./graphics/knuth-check}
  \caption{During the eraly days of typography fonts were designed to emulate the looks of calligraphic texts.}
  \label{fig:knuth}
\end{marginfigure}


Because Latex is so mature - and developed by extremely clever programmers - bugs are negligible. And even if it were buggy, then there is no risk of you ever losing your original source text. Where as with Word, almost any tool within its integrated environment is capable of corrupting your file, if it causes a crash.
\sidenote{Knuth still offers a reward for people finding bugs in \TeX .} 

According to an article in the Massachusetts Institute of Technology's Technology Review, these rewards have been described as ``among computerdom's most prized trophies''.[4] As of October 2001, Knuth reports having written more than 2,000 checks, with an average value exceeding \$8 per check.[5] As of March 2005, the total value of the checks signed by Knuth was over \$20,000 (see NPR interview below). Very few of these checks are actually cashed, however, even the largest ones; more often, they are framed, or kept as "bragging rights".[6][7]

Neither \TeX\  nor \LaTeX\   macros can contain any viruses\sidenote{Although they can be written!}.

\begin{quotation}
I have put these systems into the public domain so that
people everywhere can use the ideas freely if they wish.
I have also spent thousands of hours trying to ensure
that the systems produce essentially identical results on
all computers. I strongly believe that an unchanging
system has great value, even though it is axiomatic that
any complex system can be improved. Therefore I believe
that it is unwise to make further “improvements”
to the systems called \tex and METAFONT. Let us regard
these systems as fixed points, which should give
the same results 100 years from now that they produce
today.
\end{quotation}

\section{Cost}

Well, this is one area where Latex wins hands down, since it is free! As with most open source software, the phrase "you get what you pay for" doesn't hold true. You get an extremely mature system, that is still years ahead of its competition.

\section{What about spell checking?}

It's a good point. This is not a deficiency of Latex, because it just processes the words you give it. However, within your text-editor, you do not get fancy lines highlighting your spelling errors or bad grammar as you type, like you get with Word, yet it's a feature users have come to expect when writing documents.

For starters, I do not really care for a grammar checker and anyone who actually relies on it when using Word would be better off buying a book (or looking at writing style guides) than taking the useless advice it provides.

Secondly, the 'auto-correct' feature - whilst looking like a good idea - is not beneficial in the long run. Sure, it corrects the common typos that we all make. However, the problem in my opinion is that it means we don't learn from our mistakes, e.g., you will continue to type 'teh' instead of 'the' because Word will sort it out for you. Having said that, if that's your thing, then you can easily configure any decent text editor to perform the same task. (You could, if you really wanted to, use your favourite word processor as your text editor - but then you back to square one on the stability issue.)

And so on to spelling. The great thing here is that you have a choice! Aspell and Ispell are the most popular spell checkers I know of (both open source). These will check any text file you care to feed it and you can easily configure a decent editor to integrate its functionality from within the editor itself. How to get your text editor to utilise these programs is obviously dependent on your editor of choice. Some, like Kate, interface external spell-checking programs without any effort. I personally use (g)vim which can be configured to use spell-checkers like Ispell.







%\input{./Sections/towers}

%
%
\chapter{Podia}
\newcommand{\sm}{~m$^2$}

\section{General}
\normalsize
\newthought{By and large, Podia are be completed} by with the exception of areas delayed by the Owner's Finishes Subcontractor. In addition disruption is expected in completed areas in order to implement the latest EI for the Security system. This report analyzes the remaining works based on
ceiling areas and final fix activities for walls.

\section{Current Status}

First and second fix MEP  is approximately 97.4\% complete. In many areas work is almost ready for ceiling closures. The Podia have been delayed with substantial changes by Engineer's Instructions. 

\begin{table}[htbp]
\begin{tabular}{lrrrr}
\toprule
Level		&Shangrila		&Rotana		&Merweb		&Total\\
~                      &(\sm)			&\hfil(\sm)\hfil                            &\hfil(\sm)\hfil                       &\hfil(\sm)\hfil\\
\midrule
Basement-3	&32 	&32 	&50 	&114\\
Basement-2	&30 	&35	&32	&97\\
Basement-1	&3806	&3509	&685 	&8000\\
Ground Floor &2968	&2296	&1145	&6409\\
Level-1		&3364	&3804	&1780	&8949\\
Level-2  	&4092	&3432	&1851	&9375\\
Level-3	 	&862	&854	&1195	&2912\\
Level-4		&55	&55	&26	&136\\
Level-5		&87	&169	&26	&282\\
Level-6		&22	&22	&26	&70\\
Level-7		&2079	&1105   	&---		&3184\\ \cline{5-5}
~		&	&		&		&39 528\\
\bottomrule
\end{tabular}								
\caption{Ceiling Areas Podia}
\label{tbl:ceilings}
\end{table}


\section{Completion Strategy}

We propose that a strategy for completion is implemented by tackling the floors that are \textit{easier} to complete first and then focusing on to the more difficult areas. Adopting such an approach, will enable the workforce to be more easily managed, and imporatntly free large areas for complete de-snagging and commissioning.



\begin{tabular}{lll}
Floor   &Detail  &Remark\\
L3      &        &  \\
L4      &        &  \\
L5      &        &  \\
L6      &        &  \\
B3      &        &  \\
B2      &        &  \\
B1      &        &  \\
\end{tabular}





\section{Constraints}\index{Podia! constraints}
\label{sec:podium}
Current resources are inadequate to fully re-cover the delays caused by numerous Engineer's Instructions, late material deliveries and the deployment of Specon technicians to the Towers to assist the sub-contractors. However, once the Tower targets are achieved, personnel currently deployed in Towers can be moved to the Podia to boost production. This should include the re-deployment of sub-contractors technicians and is expected to happen shortly after the 15$^{th}$ of September.



\out{Podium - Basement 3 }{20th Oct}



%%B1 SEQUENCE PLAN
%% 
\begin{figure*}[htbp]
 \includegraphics[width=\linewidth]{B1}
  \caption{B1 sequence of works.}
  \label{fig:B1sequence}
\end{figure*}


\begin{table}[htbp]
\begin{center}
\small
\begin{tabular}{cllcl}
\toprule
Ref  & Area &  \sm & Target Date &Remark \\
\midrule
A       &Main Kitchen &  &10 Sep  &\\
B       &Main Kitchen & &14 Sep &\\
C      &Change Room & &22 Sep &\\
D      & Back-of-house && 26 Sep&\\
E      &Corridor && 9 Oct &\\
F     &Staff Canteen & & 10 Oct &\\
G    &Offices&  & 24 Oct &\\
H    &House Keeping&  & 17 Oct &\\
I     &Engineering & & 24 Oct &\\
J    & BOH tower \&podium &  & 11 Nov &RCP drawing for update\\
K    &Lift lobby   & &\\
\bottomrule
\end{tabular}
\caption{B1  target completion dates}
\end{center}
\end{table}

\pagebreak

%%  END B1

%%B2 SEQUENCE PLAN
%% 
\begin{figure*}[htbp]
 \includegraphics[width=\linewidth]{B2}
  \caption{B2 sequence of works.}
  \label{fig:B2sequence}
\end{figure*}
\vfill
\begin{table}[htbp]
\begin{center}
\small
\begin{tabular}{clllcl}
\toprule
Ref  & Area &  \sm & Target Date &Remark \\
\midrule
A       &TRO Lift lobby   & 33.5    &30 Sep  & RCP available\\
B       &TSL Lift lobby      & 33.5    &30 Sep & RCP available\\
C      &TMW Lift lobby    &30        &30 Sep &RCP available\\
D      &TMW BOH           &30.4     & 10 Oct &RCP available\\
\midrule
~     &Total                    &  127.4 &             &   \\
\bottomrule
\end{tabular}
\caption{B2  target completion dates}
\end{center}
\end{table}
\vfill
%%  END B2
\pagebreak




%% Ground
%% 
\begin{figure*}[htbp]
 \includegraphics[width=\linewidth]{groundfloor}
  \caption{Ground Floor sequence of works.}
  \label{fig:GRsequence}
\end{figure*}
\newcommand{\rcp}{RCP/ID required }
\begin{table}[htbp]
\begin{center}
\begin{tabular}{clcl}
\toprule
Ref &Area  & Target Date & Remark\\
\midrule
A       &Washroom & 23 Aug &\rcp  \\
B       &Office & 27 Aug  &\rcp\\
C      & Rotana Lobby & 27 Aug &\rcp\\
D      & Bell Captain & 4 Sep &\rcp\\
E      &Lift Lobby & 5 Sep &\rcp\\
F      &Lobby Lounge & 2 Sep & \rcp  \\
G     & Kitchen  & 4 Sep &\rcp\\
H     &Lift Lobby& 5 Sep &\rcp\\
I    &Washroom & 7 Sep &\rcp\\
J    &Washroom & 9 Sep &\rcp\\
K     &Entrance Lobby & 10 Sep&\rcp \\
L     &Pre-function &12 Sep &ID not reviewed\\
M    &Ballroom & 13 Sep &ID not reviewed\\
N &BOH &29 Sep &\rcp \\
O &BOH &29 Sep &\rcp \\
P &BOH  &9 Sep &\rcp \\
Q &Q-TEL  &15 Sep &\rcp \\
R &Loading Dock &9 Sep &RCP required\\
S &Book Store &16 Sep & --- \\
\bottomrule
\end{tabular}
\caption{Rotana,  Ground floor  target completion dates}
\end{center}
\end{table}




%% Level 1
\begin{figure*}[htbp]
 \includegraphics[width=\linewidth]{level1}
  \caption{Level 1 sequence of works.}
  \label{fig:level1}
\end{figure*}




%% Merweb Level 1
%
\begin{figure*}[htbp]
 \includegraphics[width=\linewidth]{MWLevel1}
  \caption{Merweb Level 1 sequence of works.}
  \label{fig:level7}
\end{figure*}

%% Merweb Level 2
%
\begin{figure*}[htbp]
 \includegraphics[width=\linewidth]{MWLevel2}
  \caption{Merweb Level 2 sequence of works.}
  \label{fig:MWlevel2}
\end{figure*}

%% Merweb Level 3
%
\begin{figure*}[htbp]
 \includegraphics[width=\linewidth]{MWLevel3}
  \caption{Merweb Level 3 sequence of works.}
  \label{fig:MWlevel3}
\end{figure*}


%% Merweb Level 4
%
\begin{figure*}[htbp]
 \includegraphics[width=\linewidth]{MWLevel4}
  \caption{Merweb Level 4 sequence of works.}
  \label{fig:MWlevel4}
\end{figure*}


%% Merweb Level 5
%
\begin{figure*}[htbp]
 \includegraphics[width=\linewidth]{MWLevel5}
  \caption{Merweb Level 5 sequence of works.}
  \label{fig:MWlevel2}
\end{figure*}


%% Merweb Level 6
%
\begin{figure*}[htbp]
 \includegraphics[width=\linewidth]{MWLevel6}
  \caption{Merweb Level 6 sequence of works.}
  \label{fig:MWlevel6}
\end{figure*}

\chapter{Merweb Podia}

\newthought{Although} we did manage to reach a good level of completion, numerous Engineer's Instructions have affected the works in the Merweb Podium areas, as well as premature demobilization of QMMC, which has affected the completion of Level 6. We suggest a similar approach to that taken for the other two Towers, by targeting the \textit{easier} floors first.

\section{Current Status}

Basements, Ground Floor, Levels 4,5 and 6 are substantially completed and systems are now being commissioned progressively. 
There are remaining works on Levels 1, 2 and 3. We need to
verify all inspections on completed floors and complete snagging. Any remaining minor third works need to be completed; such minor works include installation of exit lights, access control devices and final fix items for Fire Alarm. Final fixes in staircases need to be completed.


\subsection{Basement 3}

The Plantroom areas are covered under the sections of the report for systems. The Car parking area is covered by the section on Car Park Ventilation systems and works have been completed.


\section{Basement 2}

Area needs a plan to be emptied from materials. Shaft works for ducting are incomplete. Car Park Ventilation (Jet fans have been installed, but not 
electrically connected).

\section{Basement 1}

Substantially completed. 

\section{Ground Floor}

Substantially completed.

\section{Level 1}
Inspection related activities and EIs that follow onto Civil Works.

\section{Level 2}
As for Level 1
\section{Level 3}
As for Level 1

\section{Level 4}

This level houses the Calorifier and Domestic Plant rooms. The Plant room works, which are covered under the relevant systems have been completed and inspected. They and ready for commissioning once gas becomes available. Electrical works have been completed on this floor, inspected, pre-commissioned and activated.

\section{Level 5}

This level houses the ETS Plant rooms, they have been fully completed.

\section{Level 6}

These works have been contracted out to QMMC. A number of EIs have also affected this area. QMMC has demobilized from Site and area is being completed by HLS. Area is substantially completed.

\section{Other Areas}

\subsection{Ramps}

Sprinkler and light fittings have been substantially completed.








%

%\chapter{Kahraama Power On}
 \label{ch:kahraama}

Kahraama power on is expected to be in September for Rotana and early October for Shangri-la and Merweb. The work involved is the testing and commissioning of 22 panels and 12  transformers.

We expect small deviations form the above dates due to the Ramadan holidays and the difficulty to arrange inspections by Kahraama during this period.

Power-on was expected earlier but has been affected by:

\begin{enumerate}
\item Completion of cable termination both ends and cable pulling. \sidenote{Affected by commercial issues with MBK}
\item  Installation of tripping batteries\sidenote{These have been purchased and a submission made to HOK to incorporate these permanently into the works}.
\item  Completion of FM 200\sidenote{Commercial issues with Nafco have now been resolved}.
\item  Complete repairs to Schneider Panels\sidenote{Original supplier not responding. Action has been taken to appoint a different Supplier}.
\item  An attempt to reach agreement  reach with Schneider Doha for the testing and commissioning of the MV panels and Transformers failed due to excessive costs. Agrrement with ATS.
\end{enumerate}

Target days are as shown below:

\begin{table}[htbp]
\small
\begin{tabular}{p{3.2cm}p{5.0cm} p{2.0cm} }
\toprule
\multicolumn{3}{c}{{Rotana Substation Milestone Dates}}\\
\midrule
Activity	&Details/Remarks	&Target Date                \\
\midrule
Stage 1           			& Blue Card Inspection	 &			  \\
  ~~~\em{Cabling}		&			             &\done\\
  ~~~\em{Terminations}	&			             &\done\\
  ~~~\em{Panels}		&			             &\done\\
  ~~~\em{Civil works}	&			             &\done\\
  ~~~\em{Transformers}        &Temperature Controllers (repair/replace)  &\done\\	
  ~~~ \em{Earthing}	            &Complete Earthing and earthing pits	&\done\\
  ~~~ \em{Kahraama insp}. & &\done \\
\midrule		

Stage 2                                   & Yellow Card Inspection	&\\
~~~\em{Switchgear}             & Handles to be replaced, minor components.  
                                       Full factory style tests to be carried out, by ATS &13th Aug\\
~~~\em{Cabling}                  &Completed       & \done \\
~~~\em{Tripping batteries}  &Received& 17th Aug \\
~~~\em{ATS}  		&Testing and Commisioning&1-10th Aug\\
~~~\em{HVAC}         		&Ordered & 7th Aug\\
~~~\em{FM-200}      		&on-going& 15th Aug\\
~~~\em{Civil works} 		&on-going&15th Aug\\
~~~\em{Kahraama insp.}	&				&15-20th Aug\\
~~~\em{Paper work} 	& 				&27th Aug\\
\bottomrule
\end{tabular}
\caption{Kahaama power-on milestone dates for Rotana}
\end{table} 
	



\begin{table}[htbp]
\small
\begin{tabular}{p{3.2cm}p{5.0cm} p{2.0cm} }
\toprule
\multicolumn{3}{c}{{Shangri-la Substation Milestone Dates}}\\
\midrule
Activity	&Details/Remarks	&Target Date                \\
\midrule
Stage 1           &Blue Card Inspection	 &			  \\
  ~~~\em{Cabling}		&			&\done\\
  ~~~\em{Terminations}	&			&\done\\
  ~~~\em{Panels}		&			&\done\\
  ~~~\em{Civil works}		&			&\done\\
  ~~~\em{Transformers}            &Temperature Controllers (repair/replace)  &\done\\	
  ~~~ \em{Earthing}	             &Complete Earthing and earthing pits	&\done\\
  ~~~ \em{Kahraama insp}. & &\done \\
\midrule		

Stage 2                        &Yellow Card Inspection	&\\
~~~\em{Switchgear}            &Handles to be replaced, minor components.  
                                       Full factory style tests to be carried out, by Verger. &1-7th Aug\\
~~~\em{Cabling}                 &Completed       & \done \\
~~~\em{Tripping batteries} &Ordered& 7th Aug \\
~~~\em{Schneider}  &Testing and Commisioning&1-10th Aug\\
~~~\em{HVAC}         &Ordered & 7th Aug\\
~~~\em{FM-200}      &on-going& 15th Aug\\
~~~\em{Civil works} &on-going&15th Aug\\
~~~\em{Kahraama insp.} &&15-20th Aug\\
~~~\em{Paper work} & &27th Aug\\
\bottomrule
\end{tabular}
\caption{Kahaama power-on milestone dates for Shangri-la}
\end{table} 

\begin{table}[htbp]
\small
\begin{tabular}{p{3.2cm}p{5.0cm} p{2.0cm} }
\toprule
\multicolumn{3}{c}{{Merweb Substation Milestone Dates}}\\
\midrule
Activity	&Details/Remarks	&Target Date                \\
\midrule
Stage 1           &Blue Card Inspection	 &			  \\
  ~~~\em{Cabling}		&			&\done\\
  ~~~\em{Terminations}	&			&\done\\
  ~~~\em{Panels}		&			&\done\\
  ~~~\em{Civil works}		&			&\done\\
  ~~~\em{Transformers}            &Temperature Controllers (repair/replace)  &\done\\	
  ~~~ \em{Earthing}	             &Complete Earthing and earthing pits	&\done\\
  ~~~ \em{Kahraama insp}. & &\done \\
\midrule		

Stage 2                        &Yellow Card Inspection	&\\
~~~\em{Switchgear}            &Handles to be replaced, minor components.  
                                       Full factory style tests to be carried out, by Verger. &1-7th Aug\\
~~~\em{Cabling}                 &Completed       & \done \\
~~~\em{Tripping batteries} &Ordered& 7th Aug \\
~~~\em{Schneider}  &Testing and Commisioning&1-10th Aug\\
~~~\em{HVAC}         &Ordered & 7th Aug\\
~~~\em{FM-200}      &on-going& 15th Aug\\
~~~\em{Civil works} &on-going&15th Aug\\
~~~\em{Kahraama insp.} &&15-20th Aug\\
~~~\em{Paper work} & &27th Aug\\
\bottomrule
\end{tabular}
\caption{Kahaama power-on milestone dates for Merweb}
\end{table} 


We confirm the following points agreed during the meeting held at the site office on 31 Aug 2010-08-31
 
The following programme for testing and commissioning was agreed:-

1)      Commence 14th Sept 2010 in Rotana Substation
2)      Complete T&C 21st Sept 2010
3)      Commence T&C Shangri La 22nd Sept 2010
4)      Complete T&C Shangri La 29th Sept 2010
5)      Commence T&C Merweb 30th Sept 2010
6)      Complete T&C Shangri La 7th Oct 2010
 
Every endeavour will be made to improve these dates.
 








			
%

%
\chapter{Car Park Ventilation}
\label{carparkventilation}

\newthought{The Car Park ventilation system} has been completed and the Manufacturer has brought specialist technicians from Spain to test the system. However, an issue arose during commissioning and following the Supplier's recommendation an additional repeater panel has to be installed. Currently we are awaiting for confirmation of delivery dates. The system will need to be re-tested. We expect that these works will be completed between 1 Feb - 15 March.

\begin{figure*}
\includegraphics[width=\textwidth]{CPV-002}
%\vskip{2pt}
%\includegraphics[width=\textwidth]{CPV-002}
\end{figure*}

\section{Current Status (Physical Works) }

\begin{table}
\begin{tabular}{l p{1.5cm} p{1.5cm} p{1.5cm} p{1.5cm}}
\toprule
Level    & Mechanical Installation & Electrical Installation
         & CO Monitors &FD Interface\\
\midrule
L5  &\checked &\checked &\\
L4  &\checked &\checked &\\
L3  &\checked &\checked &\\
B2  &\checked &\checked &\\
B3  &\checked &\checked &\\
\bottomrule
\caption{Physical Works Status. All completed with the exception of minor works at Level 5, where existing Site Offices do not permit the installation.}
\end{tabular}
\end{table}

\begin{figure*}
\includegraphics[width=\textwidth]{CPV-001}
\caption{The car parks are currently being cleared from materials, in
readiness for smoke extract acceptance tests. All Jet Fans have been
installed.}
\end{figure*}

\section{Constraints and Issues}

\begin{enumerate}
\item Client's column painting in parkings was not completed. Final CO monitoring devices have been installed and are prone 
to damages during paint works. Client has been informed via letter.
\end{enumerate}

\section{Documentation}

\begin{table}[h]
\begin{tabular}{lll}
\toprule
Item  & Detail & Status\\
\midrule
1     & T\&C Method Statement & Approved B\\
2     & O\&M Manual           & For Approval\\
3     & Inspections Summary   &            \\
      &  \hfill\hfill\hfill\hfill Electrical    &  \\
      &  \hfill\hfill\hfill\hfill Mechanical    &  \\
4     & Commissioning Report  &            \\
\bottomrule
\end{tabular}
\end{table}


\section{Current Status (Commissioning) }
The current status for commissioning is shown in Table . The
table details the major steps to commissioning the system. More
detailed reports are available. Meeting with IES 11 Mar 2013. Commissioning 21 Mar 2013.

\begin{table}
\begin{tabular}{lll}
\toprule
Item & Detail  & T\&C\\
\midrule
1 & Extract Fans  & \checked\\
2 & Jet Fans      & \checked\\
3 & CO system     & \Cross \fire 21 Mar 2013\\
4 & BMS Interface & 10 Mar 2013\\
5 & FA Interface  & 10 Mar 2013\\
6 & Smoke Test    & 10 Mar 2013\\
\bottomrule
\caption{Commissioning Status for Car park ventilation.}
\end{tabular}
\end{table}



\chapter{Staircase Pressurization}
\label{pressurization}


\newthought{The Staircase Pressurization} systems are expected to be completed by the 25 February 2012. Physical works have been completed and commissioning has been completed for Rotana and Shangrila. The Merweb has been delayed due to the installation of sensor wire. 

          

\section{Constraints}


\begin{marginfigure}
\includegraphics[width=\textwidth]{staircase-pressurization}
\caption{Staircase pressurization systems for Rotana and Shangrila have been commissioned, however the inspections have been returned status 'C' due to being 4Pa less than design pressure at bottom of staircase.}
\end{marginfigure}

\begin{marginfigure}
\includegraphics[width=\textwidth]{staircase-pressurization-01}
\caption{Staircase pressurization systems for Rotana and Shangrila have been commissioned, however the inspections have been returned status 'C' due to being 4Pa less than design pressure at bottom of staircase.}
\end{marginfigure}

No further constraints exist in regards to the commissioning of this service, however temporary doors and closures need to be changed to permanent by HLG before the final inspection by Civil Defence. An issue with the Engineer regarding performance tests is expected to be resolved once full integration tests are completed or earlier. (See Section on Life Safety Integration Acceptance Tests).












\newcommand{\keextracton}{25.10.2010}
\newcommand{\ecolon}{25.10.2010}           % ecological units on
\newcommand{\podiumon}{26.10.2010}
\newcommand{\toweron}{15.10.2010}
\newcommand{\kdpoweron}{20.10.2010}

\chapter{Kitchen Extract Ventilation System }
%\begin{marginfigure}
%  \includegraphics[width=\linewidth]{boilers}
 % \caption{Calorifier plant-room in Merweb.}
 % \label{fig:marginfig1}
%\end{marginfigure}

\newthought{The Kitchen Extract System} has been physically completed with the exception of Merweb Tower, where the subcontractor is behind schedule. Commissioning is following installation of kitchen hoods by Askar.

Commissioning is currently carried out by HLS/CML. Final verification will be done by manufacturer. 

\TODO{Ordering of extra set of filters for damaged unit. Identification of areas being served on unit. Method Statement. set commissioning dates.}
\out{Rotana kitchen extract}{25 Oct 2010}
\out{Shangri-la kitchen extract}{25 Oct 2010}
\out{Merweb kitchen extract}{25 Oct 2010}


           \begin{margintable} 
	    \begin{tabular}{lcl}
	      \toprule
	      System    & Key Milestone Date  \\
	      \midrule

                 Rotana Kitchen Extract       &  \ecolon  \\   
                 Shangri-la Kitchen Extract   & \ecolon\\
	     Merweb Kitchen Extract       &  \ecolon  \\
	      
	      \bottomrule
	    \end{tabular}
           \caption{Kitchen Extract Ventilation Key Dates}
           \label{tbl:KEkeydates}
            \end{margintable}
 

\section{Current Status}

The kitchen extract system is made up of a series of black steel ductwork, extract fans and Ecological Units. The latter treat the air to  remove odours and contaminants, so as to enable the discharge of air at lower levels than the roof of the Towers. Most of the units are at Level~6, whereas the main kitchens are at Basement~1. 

The Ecological Units have been installed, but they have not been connected as yet electrically and the ductwork is still pending. Final discharge louvers still need to be detailed. 

\section{Constraints}

Current constraints are human resources for ductwork installation and certain design issues involving co-ordination. Delivery of MCC panels for electrical hook-up is expected by end of August. Delivery of black steel ducting has started, mock-ups approved by the Engineer. This ductwork was originally to be manufactured with flanged joints (as per Phase II). However, the Engineer insisted that this ductwork is now of an \emph{all welded} type construction.  This is expected to slow down construction works. 

None of these constraints are impossible to resolve within a reasonable time. However, unless a full audit of these systems takes place  and we ensure that no redundant re-works are carried out due to co-ordination issues, completion dates can be met. We also expect MEPE to increase the number of personnel working on these systems, once all design, co-ordination and approval issues are resolved. 

\ramadaneffect


\section{Work Organization}
The work is organized (currently) in sections. An overall responsible person from a systems point of view is Ritzie. For the section under subcontracts (Ecological Units) is George Georgiou.


\section{Sequence for Switch-on}

In order to enable progressive commissioning an early switch on of Ecological Units is to be targetted. Key switch-on dates will  follow the  sequence shown in table \ref{tbl:switchon}. This Table also indicates the key personnel involved.

\begin{table*}[htbp] 
\begin{tabular}{llp{2.5cm}}
\toprule
Activity & Milestone Date & Responsible person\\
\midrule
{\em Rotana} &&\\
Electrical power-on  &\kdpoweron &NH\\
ECU on  &\toweron   &RA\\
\midrule
{\em Shangri-la} &&\\
Electrical power-on    &\kdpoweron &NH\\
ECU on   &\toweron  &RA \\
\midrule
{\em Merweb} &&\\
Electrical power-on &\kdpoweron & NH\\
ECU on  &\toweron  &RA\\
\bottomrule
\end{tabular}
\caption{Kitchen ventilation power on}
\label{tbl:switchon}
\end{table*}


\section{Stainless steel works}

Stainless steel ductwork is being installed to serve kitchen dish-washing equipment and laudry exhausts. 
These works have been subcontracted to QMMC. They are currently behind schedule for insulation. \TODO{write contractual letter for delays of stainless, steel works and insulation. Chase Ashraf for an estimate of the works.}









}
%\input{./Sections/smokeexhaust}

\chapter{Drainage System}

\newthought{The Drainage System is already at an advanced stage}. Work that could be completed has been completed. Areas that are currently incomplete are third fix activities that follow finishes.


\newthought{Drainage Plantrooms} such as the Grey Water Treatment Plant have been completed in terms of Mechanical Works and the balance works.

From a systems point of view our aim is to be able to pour water at the highest point
in the Tower and see it flowing into the Municipal Drain or the Grey Water Treatment Plant etc.
as soon as possible. 

The installation can be divided into the following broad sections

\begin{table} 
\begin{tabular}{ll}
	      \toprule
	      Plantroom   &  Milestone Date  \\
	      \midrule
	      Flushing/Grey Water       &    $30^{th}$ August 2010  \\
	      Rainwater System      &    20th Sep 2010  \\
	      Sump Pits    &    30th August 2010  \\
	     Condensate Drains &15th October\\
	     External Works &15th October \\
	     Sanitary Fixtures installation &Depends on HEE \\ 
	      \bottomrule
\end{tabular}
             \caption{Organization of drainage activities}
\end{table}

\section{Issues}

A number of issues are outstanding and need resolution with the Client. These are:

\begin{tabular}{lll}
Item  &Detail       &Remarks  \\
\end{tabular}


\section{Documentation}
The documentation status is shown below. The drawings status
is reported under a different section.

\begin{tabular}{lll}
\toprule
Item &  Detail & Status \\
\midrule
1    & T\&C Method Statement Grey Water & "C"\\
2    & T\&C Method Statement Drainage & "C"\\
2    & O\&M                  & "C" \\
\bottomrule
\end{tabular}

















        






\chapter{Domestic Water System - Cold Side}

\newthought{The Domestic Water System} is physically completed. Commissioning can partially, proceed without the final fix be completed. However, we would prefer to have the installation fully completed. The Kahraama connection has also be completed. Disinfection of the systems. Target dates for setting the systems into operation are shown below:

\begin{center}
           \begin{table} 
	    \begin{tabular}{lcl}
	      \toprule
	      Plantroom   &  Details & Milestone Date  \\
	      \midrule
	      B3 Domestic plantroom      &  $26^{th}$ Sep 2010  \\
	      L4 Domestic plantroom      &  20th Sep 2010  \\
	      Interconnections           &  15th August 2010  \\
	      Final fix  && \\ 
	      \bottomrule
	    \end{tabular}
           \caption{Domestic water sub-systems}
            \end{table}
   \end{center}


\section{Current Status}

All water tanks have been installed at Level 4. These remain to be fully tested and inspected. The physical installation is completed substantially. 

\section{Outstanding Works, impeding commissioning}

\begin{table}[ht]
\begin{tabular}{lll}
1  &BMS Connections & \\
2  &Testing of Water tanks &Constraint is with Metito\\
3  &Other                  &\\
4  &Covers for water tanks &\\
5  &Side Access Doors to water tanks   &\\
\end{tabular}
\caption{Summary of Outstanding Works}
\end{table}

\section{Constraints}
Current constraints revolve around Main Contractor's and fit-out Contractors final fix activities. 

\section{Closure Sequence}

The Closure and Commissioning sequence will follow the overall pattern for the Building. 
\bigskip



%\begin{fullwidth}
%\begin{tikzpicture}[scale=1.15]
%\begin{ganttchart}[vgrid,hgrid,x unit=3.5mm,
%   y unit chart=3.5mm, bar height=0.5, bar label font=\footnotesize,
%   ]{12}
%\gantttitle{Domestic Water 2012}{12} \\
%\gantttitlelist{1,...,12}{1} \\
%\ganttgroup{Closure}{1}{7} \\
%\ganttbar{Water Tanks Testing}{1}{2} \\
%\ganttlinkedbar{Overall pressure test}{3}{7} \ganttnewline
%\ganttmilestone[milestone={fill=purple}]{Milestone}{7} \ganttnewline
%\ganttbar[bar={fill=orange, draw=none}]{Final Task}{8}{12}\\
%\ganttlink{elem2}{elem3}
%\ganttlink{elem3}{elem4}
%\activity{Pump alignment}{5}{1}
%\activity{Circulate to towers}{6}{}
%\activity{Controls}{7}{}
%\activity{HX Commissioning}{7}{1}
%\activity{Disinfection}{8}{1}
%\activity{Flush}{9}{10}
%\activity{Set to operation}{9}{10}
%\end{ganttchart}
%\end{tikzpicture}
%\begin{tikzpicture}[scale=1.15]
%\begin{ganttchart}[vgrid,hgrid,x unit=3.5mm,
%   y unit chart=3.5mm, bar height=0.5, bar label font=\footnotesize,
%   ]{12}
%\gantttitle{Domestic Water 2012}{12} \\
%\gantttitlelist{1,...,12}{1} \\
%\ganttgroup{Closure}{1}{7} \\
%\ganttbar{Water Tanks Testing}{1}{2} \\
%\ganttlinkedbar{Overall pressure test}{3}{7} \ganttnewline
%\ganttmilestone[milestone={fill=purple}]{Milestone}{7} \ganttnewline
%\ganttbar[bar={fill=orange, draw=none}]{Final Task}{8}{12}\\
%\ganttlink{elem2}{elem3}
%\ganttlink{elem3}{elem4}
%\activity{Pump alignment}{5}{1}
%\activity{Circulate to towers}{6}{}
%\activity{Controls}{7}{}
%\activity{HX Commissioning}{7}{1}
%\activity{Disinfection}{8}{1}
%\activity{Flush}{9}{10}
%\activity{Set to operation}{9}{10}
%\end{ganttchart}
%\end{tikzpicture}
%\end{fullwidth}
%
%\begin{fullwidth}
%\begin{figure*}
%\begin{tikzpicture}[scale=1.15]
%\begin{ganttchart}[vgrid,hgrid,x unit=3.5mm,
%   y unit chart=3.5mm, bar height=0.5, bar label font=\footnotesize,
%   ]{12}
%%\begin{ganttitle}
%%   \newcounter{weekcount}
%%   \setcounter{weekcount}{1}
%%   \whiledo{\value{weekcount}<13}
%%  {%
%%   \mon{\theweekcount}%
%%   \numtitle{1}{1}{4}{1}
%%   \stepcounter{weekcount}% 
%%  }      
%%\end{ganttitle}
%% 
%% We define some new colors
%\definecolor{red}{rgb}{1,0,0}
%% activities are plotted here
%\xdef\prop{color=orange}
%
%\ganttbar[bar={fill =green}]{GR-RO-ECU-1 - horizontal}{0}{8}\\
%\ganttbar[bar={fill =DarkGreen}]{GR-RO-ECU-1 - vertical}{0}{8}\\
%
%\ganttbar[bar={fill =green}]{L1-RO-ECU-1 - horizontal}{8}{6}\\
%\ganttbar[bar={fill =purple}]{L1-RO-ECU-1 - vertical}{7}{6}\\

%\ganttbar[color=blue]{L1-RO-ECU-2 - horizontal}{0}{6}
%%% Level 6
%\ganttbar[color=violet]{L6-RO-ECU-1 - horizontal}{5}{4}
%\ganttbar[color=purple]{L6-RO-ECU-1 - vertical}{0}{8}
%
%\ganttbar[color=blue]{L6-RO-ECU-2 (horizontal)}{8}{6}
%\ganttbar[color=violet]{L6-RO-ECU-2 (vertical)}{8}{6}
%
%
%\ganttbar[color=orange]{L6-RO-ECU-4 (horizontal)}{5}{6}
%\ganttbar[color=DarkGreen]{L6-RO-ECU-4 (vertical)}{7}{6}
%
%
%\ganttbar[color=purple]{L7-RO-ECU-1}{4}{3}
%
%\ganttbar[color=brown]{Commissioning}{16}{6}

%\end{ganttchart}
%\end{tikzpicture}
%\colorbox{green}{\textcolor{black}{Team-1}} Currently working in B1 (Andreas)\\[3pt]
%\colorbox{DarkGreen}{\textcolor{white}{Team-2}} Currently working in L6 shafts going down (Fitos)\\[3pt]
%\colorbox{purple}{\textcolor{white}{Team-3}} Currently working in L6 shafts going down (Fitos)\\[3pt]
%\colorbox{blue}{\textcolor{white}{Team-4}} Currently working in B1 to be moved to L1 (Andreas)\\[3pt]
%\colorbox{violet}{\textcolor{white}{Team-5}} New Team HS, (Fitos)\\[3pt]
%\colorbox{orange}{\textcolor{white}{Team-6}} New Team HS, (Andreas)\\[3pt]
%\caption{Rotana installation and commissioning program.}
%\label{plan}
%\end{figure*}
%
%\end{fullwidth}
%
%
%EI-Python systems for all Hotels affects ceilings.
%






















\input{./Sections/domestichot}

\chapter{Fire Fighting}
\index{FireFighting! General}


\newthought{The Fire Protection System} has been substantially
completed with the exception of where ceilings have not been completed. These areas include vestibules in Towers, some lift lobbies in podia and droppers into walk-in cold stores. The
PlantRoom has been energized and tested, as well as the majority of the rest of the network, including Towers. The extend of the system is detailed in \tref{fireitems}.

\begin{figure*}
\includegraphics[width=\linewidth]{./pic/F-001}
\caption{All items for the Fire PlantRooms are ready. The systems have been pre-commissioned and
switched on. Main constraint remaining is the completion of internal plantforms in tanks.}
\end{figure*}
 

\section{Current Detailed Status and Targets}

The current status is shown in Tables \ref{fire1}, \ref{fire2} and \ref{fire3}. We are now in the process of submitting official WIRs for commissioning. Repair works on Rotana Tower have been completed and we currently working on an overtime basis to complete them on Shangrila. On Merweb the work is undertaken by AJE. The sprinkler status is shown in Table. Unfortunately final fix items cannot be completed until all vestibules and other finishing related items are completed.

\begin{table}[htbp]\label{fire1}
\begin{center}
\begin{tabular}{cccccc}
\toprule
\multicolumn{5}{c}{Rotana}\\
\midrule
Level	&Risers    &ZCV	      &FHC  &\WIR & Target\\
\midrule
B3	&\checkmark     &\checkmark	 &\checkmark & &\fire \\
B2	&\checkmark     &\checkmark	 &\checkmark & &\fire \\
B1	&\checkmark     &\checkmark	 &\checkmark & &\fire \\
GL	&\checkmark     &\checkmark	 &\checkmark & &\fire \\
L1	&\checkmark     &\checkmark	 &\checkmark & &\fire \\
L2	&\checkmark     &\checkmark	 &\checkmark & &\fire \\
L3	&\checkmark     &\checkmark	 &\checkmark & &\fire \\
L4	&\checkmark     &\checkmark	 &\checkmark & &\fire \\
L5	&\checkmark     &\checkmark	 &\checkmark & &\fire \\
L6	&\checkmark     &\checkmark	 &\checkmark & &\fire \\
L7	&\checkmark     &\checkmark	 &\checkmark & &\fire \\
\midrule
L8-14 &\checkmark     &\checkmark	 &\checkmark &\ch & \\
L15	&\checkmark     &\checkmark	 &\checkmark  &&\fire \\ 	 
L16-38	&\checkmark     &\checkmark	 &\checkmark & &\fire \\	 	 
L39-47	&\checkmark     &\checkmark	 &\checkmark & &\fire \\	 	 
\bottomrule
\end{tabular}
\caption{Rotana Risers, zone valves and FHC completion status.}
\end{center}
\end{table}

\begin{table}[htbp]\label{fire2}
\begin{center}
\begin{tabular}{cccccc}
\toprule
\multicolumn{5}{c}{Shangri-la}\\
\midrule
Level	&Risers &ZCV	&FHC  &\WIR &Target\\
\midrule
B3	&\checkmark     &\checkmark	 &\checkmark  &&\fire 12 Mar 13\\
B2	&\checkmark     &\checkmark	 &\checkmark  &&\fire 12 Mar 13\\
B1	&\checkmark     &\checkmark	 &\checkmark  &&\fire 12 Mar 13\\
GL	&\checkmark     &\checkmark	 &\checkmark  &&\fire 12 Mar 13\\
L1	&\checkmark     &\checkmark	 &\checkmark  &&\fire 12 Mar 13\\
L2	&\checkmark     &\checkmark	 &\checkmark  &&\fire 12 Mar 13\\
L3	&\checkmark     &\checkmark	 &\checkmark  &&\fire 12 Mar 13\\
L4	&\checkmark     &\checkmark	 &\checkmark  &&\fire 12 Mar 13\\
L5	&\checkmark     &\checkmark	 &\checkmark  &&\fire 12 Mar 13\\
L6	&\checkmark     &\checkmark	 &\checkmark  &&\fire 12 Mar 13\\
L7	&\checkmark     &\checkmark	 &\checkmark  &&\fire 12 Mar 13\\
\midrule
L8-14	&\checkmark     &\checkmark	 &\checkmark  &&9 Mar 13\\
L15	&\checkmark     &\checkmark	 &\checkmark  &&\fire 22 Mar 13\\	 	 
L16-38	&\checkmark     &\checkmark	 &\checkmark  &&\fire 23 Mar 13\\ 	 
L39-L47	&\checkmark     &\checkmark	 &\checkmark  &&\fire 25 Mar 13\\ 	 	 
 \bottomrule
\end{tabular}
\caption{Shangrila Risers, zone valves and FHC targets}
\end{center}
\end{table}

\begin{table}[htbp]\label{fire3}
\begin{center}
\begin{tabular}{cccccc}
\toprule
\multicolumn{5}{c}{Merweb}\\
\midrule
Level	&Risers &ZCV	&FHC  &\WIR&Target\\
\midrule
B3	&\checkmark     &\checkmark	 &\checkmark  &&\fire 12 Mar 13\\
B2	&\checkmark     &\checkmark	 &\checkmark  &&\fire 12 Mar 13\\
B1	&\checkmark     &\checkmark	 &\checkmark  &&\fire 22 Mar 13\\
GL	&\checkmark     &\checkmark	 &\checkmark  &&\fire 26 Mar 13\\
L1	&\checkmark     &\checkmark	 &\checkmark  &&\fire 15 Apr 13\\
L2	&\checkmark     &\checkmark	 &\checkmark  &&\fire 15 Apr 13\\
L3	&\checkmark     &\checkmark	 &\checkmark  &&\fire 31 Mar 13\\
L4	&\checkmark     &\checkmark	 &\checkmark  &&\fire 26 Mar 13\\
L5	&\checkmark     &\checkmark	 &\checkmark  &&\fire 26 Mar 13\\
L6	&\checkmark     &\checkmark	 &\checkmark  &&\fire 26 Mar 13\\
\midrule
L7-14	 &\checkmark     &\checkmark	 &\checkmark  &&\fire 12 Mar 13\\
L-15	     &\checkmark     &\checkmark	 &\checkmark  &&\fire 12 Mar 13\\
L16-38	&\checkmark     &\checkmark	      &\checkmark  &&\fire 13 Mar 13\\
L39-47	&\checkmark     &\checkmark	      &\checkmark  &&\fire 19 Mar 13\\
\bottomrule
\end{tabular}
\caption{Merweb Risers, zone valves and FHC targets}
\end{center}
\end{table}

\section{Helipad Foam Installations}

The helipad foam installations---have been completed by HLS---as Nafco failed to complete. There is still a missing part (hydraulic valve) and the foam still needs to be delivered to Site. Upon completion of the above we estimate that we can commission the system within 2-3 days. The system is best to be commissioned in the presence of the Helipad Supplier.

\hl{Foam still to be delivered despite numerous reminders to Naffco.}



\section{Constraints}

We summarize the current constraints in achieving the target completion date as below:
\medskip

\begin{table}[htbp]
{\RaggedRight\resetinc
\begin{tabular}{lp{3.5cm}ll}
\toprule
Item    & Detail  & Action &Date \\
\midrule
\inc &Completion of Water Tank platforms and water tanks. &HLG & 28 Feb 20013\\
\inc &Completion of vestibule areas by Dragoni. &Dragoni & Unknown \\
\inc &Completion of other areas under Dragoni scope. &Dragoni & Unknown\\
\bottomrule
\end{tabular}}
\caption{Constraints for Fire Protection Systems}
\end{table}

\section{FM-200 Systems}

The FM-200 systems have been tested in their majority and we expect to finalize retesting of some of the rooms, where leakages exceeded limits.
This has been caused by poor construction specifications, as partitioned walls cannot be sealed adequately. We expect remedial works to be completed soon to enable retesting and reinspections.

The Rotana Security Room is expected
to be redesigned to accommodate the new equipment for the CCTV Systems, but as yet we had no feedback from the Engineer. 

\newcounter{firep}
\setcounter{firep}{0}
\def\ddd{\stepcounter{firep}}

\begin{table}[htbp]
\begin{tabular}{llllp{2.8cm}}
\toprule
SN	&Room 	      &Tower/Level	& \WIR &Target\\ 
\midrule
1	&Fire Control Room	  &Rotana GF	  &\ch& \ddd\\
2	&Fire Control Room	  &Shangrila/GF&\ch&  \ddd\\
3	&Fire Control Room	  &Merweb/GF	&\ch& \ddd\\
\midrule
4	&PABX Room	      &Shangrila/B1	&\ch& retest \ddd\\
5	&IT Room	           &Shangrila/B1	  &\ch&retest \ddd\\
6	&PABX Room	      &Rotana/L3	    &\ch&retest \ddd\\
7	&IT Room	           &Rotana/L3	     &\ch&retest \ddd\\
\midrule
8	&UPS Room	           &Rotana/L3	     &\ch  &retest \ddd\\
9	&UPS Room	           &Shangrila/L3	&\ch  &retest \ddd\\
\midrule
10	&Diesel Generator Room	&Rotana/B3	&\ch& \ddd\\
11	&Diesel Generator Room	&Shangrila/B3	 &\ch& \ddd\\
12	&Diesel Generator Room	&Merweb/B3	&\ch& \ddd\\
\midrule
13	&BMS Room	            &Rotana / B1	&\ch& \ddd\\
14	&BMS Room	            &Shangrila/B1	&\ch& \ddd\\
15	&BMS Room	            &Merweb/B1	&\ch& \ddd\\
\midrule
16	&Security Room	&Rotana/GF	&&\Danger Possible changes of layout of room. \\
17	&Security Room	&Shangrila/GF	&&29.03.2013 \\
\midrule
18	&QTEL Room	&Shangrila/GF	&& 31.03.2013\\
19	&QTEL Room	&Rotana/GF	&& 31.03.2013\\
20	&QTEL Room	&Merweb/GF	&& 31.03.2013\\
\bottomrule
\end{tabular}
\caption{FM-200 Systems, integrity testing.}
\label{FM-200}
\end{table}




%\clearpage
%
%\begin{tikzpicture}[scale=1.15]
%\begin{ganttchart}[vgrid,hgrid,x unit=3.5mm,
%   y unit chart=3.5mm, bar height=0.5, bar label font=\footnotesize,
%   ]{24}
%\gantttitle{Fire Fighting, FHC, HD and Sprinkler System \vbox{2012}}{24} \\
%\gantttitlelist{1,...,24}{1} \\
%\activity[purple]{Merweb Tower}{1}{12}
%\activity[purple]{Merweb Podium}{1}{8}
%\activity[violet]{Rotana Tower}{1}{6}
%\activity[violet]{Rotana Podium}{1}{7}
%\activity[violet]{Shangri-la Tower}{1}{6}
%\activity[violet]{Shangri-la Podium}{1}{7}

%\ganttbar{Water Tanks Testing}{1}{2} \\
%\activity[purple]{Overall pressure test}{3}{7} \ganttnewline
%\ganttmilestone[milestone={fill=purple}]{Milestone}{7} \ganttnewline
%\ganttbar[bar={fill=orange, draw=none}]{Final Task}{8}{12}\\
%\ganttlink{elem2}{elem3}
%\ganttlink{elem3}{elem4}
%\activity{Pump alignment}{5}{1}
%\activity{Circulate to towers}{6}{}
%\activity{Controls}{7}{}
%\activity{HX Commissioning}{7}{1}
%\activity{Disinfection}{8}{1}
%\activity{Flush}{9}{10}
%\activity{Set to operation}{9}{10}
%\end{ganttchart}
%\end{tikzpicture}



%

\chapter{UPS Installation}
\label{ups}\index{UPS Installation}

\begin{update}
\centerline{\textbf{26 April 2013}}

All works have been  completed including Testing and Commissioning.\sidenote{Merweb scheduled for Inspection 30 April 2013.}
\end{update}

\newthought{UPS physical installation works} have been completed. However, given the
age of the equipment and the disruption in Merweb with the EI adding a new staircase
finalization of the works and Testing and Commissioning have been delayed.

\begin{figure}
\includegraphics[width=\textwidth]{UPS-001}
\caption{UPS systems have been tested and commissioned and now going through the process of Work Inspections.}
\end{figure}

\section{Current Status}


\begin{table}[h]
\centering
{\RaggedRight

\begin{tabular}{llccp{2cm}}
\toprule
Item &Detail &Installation &T\&C &Milestone\\
\midrule
1    & Rotana         & \checkmark &\checkmark    & \checkmark \\
2    & Shangrila      & \checkmark &\checkmark    & \ch\\
3    & Merweb         & \checkmark &\checkmark    &\ch \\   
\bottomrule
\end{tabular}
\caption{UPS Installation}
}
\vspace{24pt}
\end{table}

The sequence for commissioning is as follows:

\begin{table}[h]

\RaggedRight
\centering
\begin{tabular}{lp{4.5cm}cl}
\toprule
Item &Details &Status &Milestone\\
\midrule
1     &Documentation check for all inspections &\checkmark &\\
2     &FM-200 &\checkmark &\\
3     &Lights &\checkmark &\\
4     &Small power &\checkmark &\\
5     &T\&C Method Statement &\checkmark &\\
6     &O\&M Manual &\checkmark &\\
7     &Batteries to be charged and replaced as necessary &\checkmark &\\
8     &Pre-commissioning of all panels & &\\
9     &Switch-on& &\\
10    & Load-test (hiring of load banks).& &\\
11    & Interfaces tests                     & &Pending generator\\
12    & Automatic operation                 & &\\
\bottomrule
\end{tabular}
\end{table}






















%
\bgroup
\small
\label{chap:listofservices}

\begin{longtable}{llllp{3.9cm}}
%\begin{tabular}{llll}
\toprule
Ref	&Package	&Commiss.	&Perc. & Commissioning Status\\
     &         &Entity    &Compl.\\
\midrule
1.00	&\textbf{Mechanical}	&	&\\
1.01	&Chilled Water Systems	&CML	&25\% & Started. Valves arrived for Merweb risers. Hoping to free all Towers by end of January. \\
1.02	&Ductwork  &CML		&25\%  & Started. 30\% of all AHUs cleared for balancing.\\
1.03  &Kitchen Extract System &CML &25\% &Not started. Black steel ductwork still to complete. Level 8 louvres/plenum need to be completed. Expected start 13th February 2012.\\
1.04	&Car Park Ventilation 	&CML &25\% &Not started. Low priority for physical works. Expect to start commissioning 5th Floor 13th Feb. 2012.\\
1.05	&Gas Fired steam Boilers &Specialist&25\% &not started\\	
1.06	&Condenser Water System 	&CML&25 & Part of chilled water. Valves to be installed.\%     \\
1.07    &Cooling Tower &Specialist&25\% & Final parts for cooling Tower not cleared for delivery by Supplier. \\
\midrule
2.00	&\textbf{Electrical}		&& 70\% complete\\
2.01	&Standby Generators	&Specialist&100\% & Started on flushing for cooling towers. Commercial issues to be sorted out.\\	
2.02	&Medium Voltage System	&HS&25\% & Good progress all over. Panels missing and on order.\\	
2.03	&Low Voltage System		&HS&25\% & Started and commissioning progressively.\\
2.04	&Earthing \& Lightning Protection &HS&100\% &earthing mostly completed. Lightning protection, missing part on Shangrila\\		
2.05	&Lighting \& Emergency Lighting  &HS&100\% & Where possible started. Fit-out needs to be completed in a few areas.\\		
2.06	&Aircraft Warning System &HS&100\% & Ready. Supplier requested to visit and commission. Merweb support to be done.\\		
2.07	&UPS System	&Specialist&100\% & Not started. Incomplete and commercial issues.\\	
2.08 &Power Factor  &Specialist &100\% & Not started. Commercial issue.\\
\midrule
3.00	&Public Health		&&\\
3.01	&Potable Water	&HS&25\% & Not started. Ready for Kahraama possibly this week. Testing of Tanks, meeting with Metito, 16th Jan. \\	
3.02	&Above Ground Drainage &HS&25\% & Started.\\		
3.03	&Below Ground Drainage  &HS&25\% & Completed. Sump pumps commissioned.\\		
3.04	&Water Features	     &Specialist&100\% & not started\\	
3.05	&Gas Supply Installations	&Gasco&100\% & Not started. Can only start after Kitchen equipment. \\	
3.06 &Gray Water Treatment Plant&Metito&100\% & Not started. Will agree dates with Metito.\\
\midrule		
4.00	&Fire Defense &Nafco&100\% & Started. Naffco to supply missing electrical items. Low level zones cleared. Physical works to be completed.\\	
4.01	&FHC, FH and Sprinkler Installations	&Nafco&100\% & Started on Rotana. Nozzles still to be installed.\\	
4.02	&Smoke Exhaust		&HS/CML&100\% & started \\
4.03	&Staircase Pressurization          &HS/CML&100\% & Started in Rotana, measurements are low.\\		
4.04	&Fire Alarm		&Specialist&100\% & Not started.\\
4.05	&FM-200		&Nafco&100\% & Started on new schedule.\\
4.06	&Heliport Foam System	 &Nafco&100\% & Not physically ready\\	
\midrule
5.00	&Specialist Services		&&\\
5.01	&Building Management System	&Shajan/HS&25\% & Started. Shajan under performing.\\	
5.02	&Security Access Control \& CCTV	&Specialist/HS&100\% & Not started control rooms not ready.\\	
5.03	&CO Monitoring		&Specialist&100\% & Not started. Level 5 to be commissioned 13th February.\\
5.04	&Car Calling System		&Specialist&100\% & Unknown\\
5.05	&Lighting Control		&HS/Specilaist&25\% & not started\\
5.06	&Structured Cabling		&Specialist&25\% & Not started. Physical works incomplete.\\
5.07	&PA \& Background Music System &Specialist&25\% & Not started\\		
5.08	&IPTV \& Satellite System	&Specialist&100\% & not started. Order just closed. \\	
5.09	&Room Management System	&Specialist/HS&25\% & Not started. Constrained with Dragoni's works.\\
5.10 &Intercom                 &Specilaist/HS&25\% & Not started\\
5.11 &Audiovisual              &Specialist/HS&25\% & Not in scope\\
5.12 &Telephone                &Specialist/HS&25\% & Not started. Qtel room not ready. HVAC not delivered.\\
5.13 &Wi-fi                    &Specialist/HS&25\% & Not started\\
\midrule
6.00 &Other                    &			&\\
6.01 & BMU                     &Specialist/HEE &25\% & Not started\\
6.02 & Swimming Pool           &Specialist     &25\% & Not started\\
6.03 &Irrigation System        &Specialist     &25\% & Not started\\
\bottomrule
%\end{tabular}			
\end{longtable}
\egroup



%\input{./Sections/lvsystems}
\chapter{Building Management System}
\label{bms}\index{Building Management System}
%\begin{marginfigure}%
%  \includegraphics[width=\linewidth]{staircase-pressurization}
%  \caption{Equipment show evidence of corrosion and neglect. Starting these equipment can present problems. Spares need to be sourced as early as possible.}
%  \label{fig:marginfig1}
%\end{marginfigure}

\begin{update}
\centerline{\textbf{Upadate 13 Mar 2013}}

The BMS System is currently on target. Since the agreement with the Engineer was made to only
inspect 25\% our Engineering Team has managed to speed up the works. The Rotana central
BMS workstation was connected to the network on the 12 Mar 2013 and the system demonstrated
graphically. All UNCs for Rotana will be now activated and further commissioning and inspections will
take place. We expect to increase the speed further once more technicians are on board. It is estimated
that software programming of the graphics and workstations will take 60 days. The Supplier has 
agreed to provide one additional Engineer for this task.
\end{update}

\newthought{Physical installation is substantially completed} and approximately 76\% of the T\&C of the \ac{BMS} devices has been carried out, so far without any major issues. Remaining physical activities revolve around the
completion of the Kitchen equipment subcontractor's works (condenser systems), water features and elevators. A number of \acp{FCU} in Podia need to have control cards installed\sidenote{Already ordered.}. The BMS rooms are now ready and we are hoping to start setting up the central system and activation of software and graphics by end March. A number of activities in Merweb have been delayed due to non-performance by JEM.

\section{Current Status}

The current status of commissioning activities is shown in
Table~\ref{tbl:bmsstatus}. The nature of the works is such
that it is difficult to make prediction as to an exact date
for completion. However, our current plan is to complete all Civil Defence related monitoring points before 30 April 2013 and the balance by 30 May 2013\sidenote{Provided works by \acp{ODC} are completed on time}. In the detailed tables that follow we have not included the status for FCUs, which we will do on our next update. However, detailed lists are available.


\begin{table}[htbp]
\begin{tabular}{llll}
\toprule
Building      &No. Points. &T\&C     &Inspected\\
\midrule
Rotana       & 8619        &7363      & 5609 (85.43\%) \\
Shagrila     & 6250        &3943      & 2065 (63.09\%) \\
Merweb     & 7739        &5408       & 1673 (69.88\%) \\
\midrule
Total          &22608       &16714     &9347  (73.93\%) \\
\bottomrule 
\caption{BMS status based on point count}
\label{tbl:bmsstatus}
\end{tabular}
\end{table}

\section{Current Constraints} 

Current constraints are listed below and can be viewed in more detail
in the status tables.

\begin{enumerate}
\item Gas related points (Gasco not fully ready yet).
\item Askar related points (equipment not on site).
\item Missing materials fcu cards.
\item Boiler commissioning. Can only be done once gas is available on
      site.
\end{enumerate}

\subsection{Rotana}
In Rotana from 8636 points 7567 have been commissioned. From these 5977 have been inspected by the TPI. As agreed with the Engineer the balance of the points will be verified as inspected only internally. We expect this to improve the progress of works, as it frees our Technicians and Engineers from duplicating inspection related works. A detailed breakdown of the works is shown in Table~\ref{RObms}. 
\bigskip



{\small\RaggedRight
\let\ch\checkmark
\def\fire{{\LARGE\color{red}\Fire}}
\def\Danger{{\LARGE\color{red}\danger}}
\long\def\askar{\parbox{4cm}{\Danger \RaggedRight Askar condensers.}}
\captionof{table}{Rotana BMS status}
\begin{longtable}{lp{3cm}p{2.9cm}llllllp{2.8cm}}

\toprule
S/No&DDC Panel&Area &\textcircled{1}&\textcircled{2}
               &\textcircled{3}&\textcircled{4}

               &\WIR & Remarks\\
\midrule
\inc & DDCP-RO-L47-01	&L47DDCP-RO-AHU-1, ELEVATORS &\ch&\ch&\ch&\ch&&\Danger Cables balance by KONE.\\

\inc & DDCP-RO-L46-01	&L46DDCP-RO-AH2, L46DDCP-RO-FAF-1,2,3.CHW Header Pressure.&\ch&\ch&\ch&\ch&\ch&\\
\inc & DDCP-RO-L46-02	&L46DDCP-RO-AH1, L46DDCP-RO-FAF-4,5,6&\ch&\ch&\ch&\ch&\ch&\\
\inc & DDCP-RO-L45-01	&L45DDCP-RO-SEF-1,2,ELEVATORS&\ch&\ch&\ch&\ch&\ch&\\
\inc & DDCP-RO-L44-01	&L44DDCP-RO-AH1,4,L44DDCP-RO-EAF-1,2,3,4.,SMD&\ch&\ch&\ch&\ch&\ch&\\
\inc & DDCP-RO-L40-01	&motorised dampers&\ch&\ch&\ch&\ch&\ch&\fire\\
\inc & DDCP-RO-L37-01	&motorised dampers&\ch&\ch&\ch&\ch&\ch&\fire\\
\inc & DDCP-RO-L34-01	&motorised dampers&\ch&\ch&\ch&\ch&\ch&\fire\\
\inc & DDCP-RO-L31-01	&motorised dampers&\ch&\ch&\ch&\ch&\ch&\fire\\
\inc & DDCP-RO-L28-01	&motorised dampers \& Elevators&\ch&\ch&\ch&\ch&\ch&\fire\\
\inc & DDCP-RO-L25-01	&motorised dampers&\ch&\ch&\ch&\ch&\ch&\fire\\
\inc & DDCP-RO-L22-01	&motorised dampers&\ch&\ch&\ch&\ch&\ch&\fire\\
\inc & DDCP-RO-L19-01	&motorised dampers&\ch&\ch&\ch&\ch&\ch&\fire\\
\inc & DDCP-RO-L16-01	&motorised dampers&\ch&\ch&\ch&\ch&\ch&\fire\\
\inc & DDCP-RO-L12-01	&motorised dampers&\ch&\ch&\ch&\ch&\ch&\fire\\
\inc & DDCP-RO-L09-01	&motorised dampers&\ch&\ch&\ch&\ch&\ch&\fire\\
\midrule
\inc & DDCP-RO-L07-01	&L7DDCP-RO-AH1,ECU-1,EAF-1,2,3,4&\ch&\ch&\ch&\ch&\ch&\\
\midrule

\inc & DDCP-RO-L06-01	&L6DDCP-RO-AH4,L6DDCP-RO-EAF-1,2,3,4,5,6 ECU-1,2,4&\ch&\ch&\ch&\ch&\ch&\fire\\

\inc & DDCP-RO-L06-02	&L6DDCP-RO-AH-1,2,3,SEF-1,2,3&\ch&\ch&\ch&\ch&\ch&\\

\inc & DDCP-RO-L06-03	&SWIMMING POOL PUMPS&\ch&\ch&\ch&\ch&\ch&\\
\inc & DDCP-RO-L06-04	&RAHU&\ch&\ch&\ch&\ch&\ch&\\
\midrule

\inc & DDCP-RO-L05-01	&L5DDCP-RO-AH1,EAF-1,FAF1,JET FANS&\ch&\ch&\ch&\ch&\ch&\fire\\

\inc & DDCP-RO-L05-02	&L5DDCP-RO-AH10,L5DDCP-RO-EAF-2,3,4, L5DDCP-RO-SEF-1, ELEVATORS&\ch&\ch&\ch&\ch&\ch&\Danger \fire Kone\\

\inc & DDCP-RO-L05-03	&L5DDCP-RO-AH-2,3,4,5,6,8&\ch&\ch&\ch&\ch&\ch&\\
\midrule

\inc & DDCP-RO-L05-04	&CHILLED WATER SYSTEM&\ch&\ch&\ch&\ch&\ch&\\

\inc & DDCP-RO-L05-05	&EAF, Jet fan&\ch&\ch&\ch&\ch&\ch&\fire\\

\inc & DDCP-RO-L04-01	&L4DDCP-RO-AH-1,SEF-1,EAF-1&\ch&\ch&\ch&\ch&\ch&\\

\inc & DDCP-RO-L04-02	&L4DDCP-RO-HYDZ,L4DDCP-RO-HYFHZ,L4DDCP-RO-PT&\ch&\ch&\ch&\ch&\ch&\\

\inc & DDCP-RO-L04-03	&HOT WATER SYSTEM&\ch&\ch&\ch&\ch&\ch&\\

\inc & DDCP-RO-L04-04	&L4DDCP-RO-AH-2, L4DDCP-RO-FAF-1, JET FANS, EAF-2, 3&\ch&\ch&\ch&\ch&\ch&\fire\\

\inc & DDCP-RO-L04-05	&cancelled&\ch&\ch&\ch&\ch&\ch&\\

\inc & DDCP-RO-L04-06	&HOA and TRIP for HYD \&HYZ pumps-EI 718&\ch&\ch&\ch&\ch&\ch&\\
\midrule

\inc & DDCP-RO-L03-01	&L3DDCP-RO-FAF-2,EAF-1,UPS,PAC&\ch&\ch&\ch&\ch&\ch&\\
\inc & DDCP-RO-L03-02	&L3DDCP-RO-EAF-2,FAF-1,ECU-1,L3DDCP-RO-JF 1-9&\ch&\ch&\ch&\ch&\ch&\\

\inc & DDCP-RO-L03-03	&COOLING TOWERS&\ch&\ch&\ch&\ch&&\fire\\

\inc & DDCP-RO-L03-04	&EAF and jet fan&\ch&\ch&\ch&\ch&\ch&\fire\\
\midrule

\inc & DDCP-RO-L02-01	&AHU&\ch&\ch&\ch&\ch&\ch&\\

\inc & DDCP-RO-L02-02	&Cold Stores&\ch&\ch&\ch&&&\askar\\

\inc & DDCP-RO-L02-03	&L2DDCP-RO-AH-5&\ch&\ch&\ch&\ch&\ch&\\
\midrule


\inc & DDCP-RO-L01-01	&L1DDCP-RO-AH-5,6,SEF-3&\ch&\ch&\ch&\ch&\ch&\\

\inc & DDCP-RO-L01-02	&L1DDCP-RO-AH3,4,SEF-1,2&\ch&\ch&\ch&\ch&\ch&\\

\inc & DDCP-RO-L01-03	&L1-RP-AH-1,5,ECU-1,2,EAF-3,5,Cold Stores&\ch&\ch&\ch&\ch&&\askar\\

\inc & DDCP-RO-L01-04	&L1DDCP-RO-EAF-1,4,SEF-1&\ch&\ch&\ch&\ch&\ch&\\

\inc & DDCP-RO-L01-05	&GRDDCP-RO-EAF-03,04,Cold Stores&\ch&\ch&\ch&\ch&&\askar\\

\midrule

\inc & DDCP-RO-GR-01	    &GRDDCP-RO-AH-2,4,LPG System&\ch&\ch&\ch&&&\Danger Gasco \\

\inc & DDCP-RO-GR-02	    &GRDDCP-RO-AH-1,EAF-6,Cold Stores&\ch&\ch&\ch&\ch&&\askar\\

\inc & DDCP-RO-GR-03	    &GRDDCP-RO-AH-3,EAF-1,2,6,SEF-1,3.,ELEVATORS&\ch&\ch&\ch&\ch&&\Danger Kone\\

\inc & DDCP-RO-GR-04	    &LT PANELS&\ch&\ch&\ch&\ch&\ch&\\

\midrule

\inc & DDCP-RO-B1-01	    &B1DDCP-RO-AH-9,3,EAF-3,JET FANS&\ch&\ch&\ch&\ch&\ch&\fire\\

\inc & DDCP-RO-B1-02	    &B1DDCP-RO-AH-1,B1DDCP-RO-EAF-1,2,6&\ch&\ch&\ch&\ch&\ch&\\

\inc & DDCP-RO-B1-03	    &B1DDCP-RO-AH-4,5,13&\ch&\ch&\ch&\ch&\ch&\\

\inc & DDCP-RO-B1-04	    &Shifted to B2 and now is B2DDCP-RO-02&\ch&\ch&\ch&\ch&\ch&\\

\inc & DDCP-RO-B1-04	    &Cold Stores+EAF&\ch&\ch&\ch&\ch&&\askar\\

\inc & DDCP-RO-B2-01	    &B2DDCP-RO-AH-5,FAF-2,FAF-3,EAF-1,JET FANS&\ch&\ch&\ch&\ch&\ch&\fire\\

\inc & DDCP-RO-B2-02	    &Fountain pump.WT level monitoring&\ch&\ch&\ch&\ch&&\Danger Water features contractor.\\
\midrule

\inc & DDCP-RO-B3-01	    &Plumbing Plant Room&\ch&\ch&\ch&\ch&\ch&\\

\inc & DDCP-RO-B3-02	    &Boiler Room&\ch&\ch&&&& 10 Apr 2013\\

\inc & DDCP-RO-B3-03	    &District Cooling Room&\ch&\ch&\ch&\ch&\ch&\\

\inc & DDCP-RO-B3-04	    &DG SETS&\ch&\ch&\ch&&& \mbox{\fire 7 Mar 2013}\\
\inc & DDCP-RO-B3-05	    &Gray Water Plant Room&\ch&\ch&\ch&&&15 May 2013\\
\inc & DDCP-RO-B3-06	    &FAF-2,JET FANS&\ch&\ch&\ch&\ch&\ch&\fire\\

\inc & DDCP-RO-B3-07	    &B3DDCP-RO-PSP-01,FAF-1&\ch&\ch&\ch&\ch&\ch&\fire\\

\inc & DDCP-RO-B3-08	    &Secondory CHW pump&\ch&\ch&\ch&\ch&&\\

\inc & DDCP-RO-B3-09	    &Submersible pump&\ch&\ch&\ch&&& 15 May 2013\\
\bottomrule
\end{longtable}
\label{RObms}
}

\newpage

\subsection{Shangrila}

In Shangrila from 6252 points\sidenote{Shangrila has less I/O points that the other Towers because the Tower rooms are being served by the \acp{RCU}.} 4017 have been commissioned. From these 2192
have been inspected by the TPI. As agreed with the Engineer the balance of the points will be verified as inspected only internally. We expect this to improve the progress of works, as it frees our Technicians and Engineers from duplicating inspection related works. A detailed breakdown of the works is shown in Table~\ref{SLbms}. 
\bigskip


{\small\RaggedRight
\let\ch\checkmark
\def\fire{{\LARGE\color{red}\Fire}}
\def\Danger{{\LARGE\color{red}\danger}}
\long\def\askar{\parbox{4cm}{\Danger \RaggedRight Askar condensers.}}
\captionof{table}{Shangrila BMS status}
\begin{longtable}{lp{3cm}p{2.9cm}lllllp{2.8cm}p{2.8cm}}

\toprule
S/No&DDC Panel&Area &\textcircled{1}&\textcircled{2}
               &\textcircled{3}&\textcircled{4}
               &\WIR & Remarks\\
\midrule



\inc & DDCP-SL-L48-01	&Aviation lights, Elevators&\ch&\ch&\ch&\ch&&To be commissioned together with aviation lights.\\

\inc & DDCP-SL-L47-01	&L-47-AHU-1,2 AND L-46-FAF-1,2,3,4,5,6&\ch&\ch&\ch&\ch&&\fire\\

\inc & DDCP-SL-L46-01	&L46-SL-EAF-3,L46-SL-SEF-1,2&\ch&\ch&\ch&\ch&&\fire\\

\inc & DDCP-SL-L45-01	&L45-SL-EAF-1,2,3&\ch&\ch&\ch&\ch&&\fire\\

\inc & DDCP-SL-L45-02	&AHU&\ch&\ch&\ch&\ch&&\fire\\

\inc & DDCP-SL-L44-01	&L44-SL-AH1,4,L44-SL-ECU-1&\ch&\ch&\ch&\ch&&\\
\inc & DDCP-SL-L44-02	&L44-SL-AH2, AH3,L44-SL-EAF-2&\ch&\ch&\ch&\ch&&\\
\midrule

\inc & DDCP-SL-L42-01	&Cold Stores,L42-SL-EAF-1&\ch&\ch&\ch&&&\askar\\
\inc & DDCP-SL-L41-01	&L41-SL-AH-1&\ch&\ch&\ch&\ch&&\\

\inc & DDCP-SL-L40-01	&motorised dampers&\ch&\ch&\ch&\ch&&\fire\\
\inc & DDCP-SL-L37-01	&motorised dampers&\ch&\ch&\ch&\ch&&\fire\\
\inc & DDCP-SL-L34-01	&motorised dampers&\ch&\ch&\ch&\ch&&\fire\\
\inc & DDCP-SL-L31-01	&motorised dampers&\ch&\ch&\ch&\ch&&\fire\\
\inc & DDCP-SL-L28-01	&motorised dampers&\ch&\ch&\ch&\ch&&\fire\\
\inc & DDCP-SL-L25-01	&motorised dampers&\ch&\ch&\ch&\ch&&\fire\\
\inc & DDCP-SL-L22-01	&motorised dampers&\ch&\ch&\ch&\ch&&\fire\\
\inc & DDCP-SL-L19-01	&motorised dampers&\ch&\ch&\ch&\ch&&\fire\\
\inc & DDCP-SL-L16-01	&motorised dampers&\ch&\ch&\ch&\ch&&\fire\\
\inc & DDCP-SL-L12-01	&motorised dampers&\ch&\ch&\ch&\ch&&\fire\\
\inc & DDCP-SL-L09-01	&motorised dampers&\ch&\ch&\ch&\ch&&\fire\\
\midrule

\inc & DDCP-SL-L07-01	&L7-SL-AH1,3,Cold Stores&\ch&\ch&\ch&\ch&&\askar\\
\inc & DDCP-SL-L07-02	&L7-SL-EAF-2,3,4,L7-SL-ECU-1&\ch&\ch&\ch&\ch&&\fire\\
\inc & DDCP-SL-L07-03	&L7-SL-EAF-1,4B,L7-SL-FAF-2&\ch&\ch&\ch&\ch&&\\
\inc & DDCP-SL-L07-04	&Water meter&\ch&\ch&\ch&&&\\
\midrule

\inc & DDCP-SL-L06-01	&L6-SL-AH3,L6-SL-EAF-1,2&\ch&\ch&\ch&\ch&&\\
\inc & DDCP-SL-L06-02	&L6-SL-AH-4&\ch&\ch&\ch&\ch&&\\
\inc & DDCP-SL-L06-03	&Swimming pool pumps&\ch&\ch&\ch&\ch&&\\
\inc & DDCP-SL-L06-04	&Water meter&\ch&\ch&\ch&\ch&&\\
\inc & DDCP-SL-L06-05	&RAHU&\ch&\ch&\ch&\ch&&\\
\midrule

\inc & DDCP-SL-L05-01	&L5-SL-ECU-1,2,3,4,L5-SL-EAF-3,6&\ch&\ch&\ch&\ch&&\fire\\
\inc & DDCP-SL-L05-02	&L5-SL-AH2,6,10,L5-SL-SEF-1,L5-SL-EAF-4,5&\ch&\ch&\ch&\ch&&\\
\inc & DDCP-SL-L05-03	&L5-SL-AH1,3,4,5,7,8,L5-SL-SEF-5&\ch&\ch&\ch&\ch&&\\
\inc & DDCP-SL-L05-04	&Chilled Water System &\ch&\ch&\ch&\ch&&\\
\inc & DDCP-SL-L05-05	&L5-SL-AH9,ELEV&\ch&\ch&\ch&\ch&&\\
\inc & DDCP-SL-L05-06	&ECU,FAF,JET FAN.&\ch&\ch&\ch&\ch&&\fire\\
\midrule


\inc & DDCP-SL-L04-01	&L4-SL-ECU-1,2,L4-SL-AH-4,L4-SL-SEF-2,L4-SL-EAF-3&\ch&\ch&\ch&\ch&&\\
\inc & DDCP-SL-L04-02	&L4-SL-HYDZ,L4-SL-HYFHZ,L4-SL-PT&\ch&\ch&\ch&\ch&&\\
\inc & DDCP-SL-L04-03	&L4-SL-AH1,2,L4-SL-EAF-1,2,L4-SL-SEF-1&\ch&\ch&\ch&\ch&&\\

\inc & DDCP-SL-L04-04	&HOT WATER SYSTEM&\ch&\ch&\ch&\ch&&\\
\inc & DDCP-SL-L04-05	&Water meter&\ch&\ch&\ch&\ch&&\\
\inc & DDCP-SL-L04-06	&EAF,FAF,JET FAN&\ch&\ch&\ch&\ch&&\fire\\
\midrule


\inc & DDCP-SL-L03-01	&L3-SL-ECU-1,2,L3-SL-EAF-1,2,3,4,UPS&\ch&\ch&\ch&\ch&&\fire\\
\inc & DDCP-SL-L03-02	&L3-SL-EAF-5,6,7,L3-SL-JF 1-9&\ch&\ch&\ch&\ch&&\\
\inc & DDCP-SL-L03-03	&Cooling Towers&\ch&\ch&\ch&\ch&&\\
\midrule

\inc & DDCP-SL-L02-01	&L2-SL-AH11,L2-SL-SEF-1,Cold Stores&\ch&\ch&\ch&\ch&&\askar\\
\inc & DDCP-SL-L02-02	&L2-SL-AH1,2,3,4,9&\ch&\ch&\ch&\ch&&\\
\midrule

\inc & DDCP-SL-L01-01	&L1-SL-AH1,2,3,12,FAF-2,SEF-2&\ch&\ch&\ch&\ch&&\\
\inc & DDCP-SL-L01-02	&L1-SL-AH7,10,L1-SL-ECU-01,Cold Stores&\ch&\ch&\ch&\ch&&\fire\\
\inc & DDCP-SL-L01-03	&L1-SL-EAF-1,3,4,L1-SL-SEF-5,Cold Stores&\ch&\ch&\ch&\ch&&\askar\\

\inc & DDCP-SL-L01-04	&GR-SL-EAF-4,5,L1-SL-SEF-3,4&\ch&\ch&\ch&\ch&&\\
\inc & DDCP-SL-L01-05	&LT PANELS&\ch&\ch&\ch&\ch&&\\
\inc & DDCP-SL-L01-06	&Water meter&\ch&\ch&\ch&\ch&&\\
\midrule


\inc & DDCP-SL-GR-01	&GR-SL-AH-7A,7B,LPG SYTEM&\ch&\ch&\ch&\ch&&\\
\inc & DDCP-SL-GR-02	&GR-SL-AH-4,GR-SL-EA-1,GR-SL-SEF-1&\ch&\ch&\ch&\ch&&\\
\inc & DDCP-SL-GR-03	&GR-SL-AH-2,9,EAF-2,6 Cold Stores&\ch&\ch&\ch&\ch&&\\
\inc & DDCP-SL-GR-04	&MV PANELS&\ch&\ch&\ch&\ch&&\\
\midrule


\inc & DDCP-SL-B1-01	&B1-SL-AH-5,B1-SL-EAF-6,7A&\ch&\ch&\ch&\ch&&\\
\inc & DDCP-SL-B1-02	&B1-SL-AH-7,FAF-6&\ch&\ch&\ch&\ch&&\\
\inc & DDCP-SL-B1-03	&B1-SL-AH-1,15,B1-SL-EAF-3,4,5,9&\ch&\ch&\ch&\ch&&\\
\inc & DDCP-SL-B1-04	&B1-SL-AH-3,4,6,10, EAF-7,10&\ch&\ch&\ch&\ch&&\\
\inc & DDCP-SL-B1-05	&Water Fountain Pumps &\ch&\ch&\ch&\ch&& Water features contractor.\\
\inc & DDCP-SL-B1-06	&Cold Stores&\ch&\ch&\ch&\ch&&\askar\\
\midrule

\inc & DDCP-SL-B2-01	&FAF-3&\ch&\ch&\ch&\ch&&\\
\inc & DDCP-SL-B2-02	&EAF&\ch&\ch&\ch&\ch&&\\
\midrule

\inc & DDCP-SL-B3-01	&Plumbing Plant Room&\ch&\ch&\ch&\ch&&\\
\inc & DDCP-SL-B3-02	&Boiler Room&\ch&\ch&\ch&\ch&&\\
\inc & DDCP-SL-B3-03   &District Cooling Plant Room&\ch&\ch&\ch&\ch&&\\
\inc & DDCP-SL-B3-04	&DG SETS&\ch&\ch&\ch&\ch&&\\
\inc & DDCP-SL-B3-06	&B3-SL-PSP-01,JET FANS&\ch&\ch&\ch&\ch&&\\
\inc & DDCP-SL-B3-05	&Grey Water Plant&\ch&\ch&\ch&\ch&&\\


\bottomrule
\end{longtable}
\label{SLbms}
}


\newpage

\subsection{Merweb}

In Merweb from 7833 points 5701 have been commissioned. From these 2613
have been inspected by the TPI. As agreed with the Engineer the balance of the points will be verified as inspected only internally. We expect this to improve the progress of works, as it frees our Technicians and Engineers from duplicating inspection related works. A detailed breakdown of the works is shown in Table~\ref{SLbms}. 
\bigskip

{\small\RaggedRight
\let\ch\checkmark
\def\fire{{\LARGE\color{red}\Fire}}
\def\Danger{{\LARGE\color{red}\danger}}
\long\def\askar{\parbox{4cm}{\Danger \RaggedRight Askar condensers.}}

\captionof{table}{Merweb BMS status}
\begin{longtable}{lp{3cm}p{2.9cm}lllllp{2.8cm}p{2.8cm}}

\toprule
S/No&DDC Panel&Area &\textcircled{1}&\textcircled{2}
               &\textcircled{3}&\textcircled{4}
               &\WIR & Remarks\\
\midrule

\inc	&	DDCP-MW-L46-01	&	L46-MW-AH-1,2, L46-MW-FAF-1,2,3,4,5,6,Elevators	&\ch&\ch&\ch&\ch&&\fire\\

\inc	&	DDCP-MW-L45-01	&	L45-MW-EAF-1,L45-MW-SEF-1,2,3,4,Elevators	&\ch&\ch&\ch&\ch&&\fire\\

\inc	&	DDCP-MW-L44-01	&	L44-MW-EAF-1,2,3,4,5,6,L44-MW-AH1,AH2	&\ch&\ch&\ch&\textcolor{teal}{\ch}&&\fire\\

\inc	&	DDCP-MW-L43-01	&	L43-SL-AH2,AH4, L43-MW-EAF-1	&\ch&\ch&\ch&\ch&&\fire\\

\inc	&	DDCP-MW-L42-01	&	Swimming Pool Systems	&\ch&\ch&\ch&\ch&&\Danger swimming pool contractor\\

\inc	&	DDCP-MW-L40-01	&	motorised dampers	&\ch&\ch&\ch&\ch&&\fire 20 Mar 2013\\

\inc	&	DDCP-MW-L38-01	&	motorised dampers	
&\ch&\ch&\ch&&&\fire 22 Mar 2013\\


\inc	&	DDCP-MW-L35-01	&	motorised dampers	&\ch&\ch&\ch&\ch&&\fire 24 Mar 2013\\


\inc	&	DDCP-MW-L32-01	&	motorised dampers	&\ch&\ch&\ch&\ch&&\fire 26 Mar 2013\\

\inc	&	DDCP-MW-L29-01	&	motorised dampers	&\ch&\ch&\ch&\ch&&\fire 28 Mar 2013\\

\inc	&	DDCP-MW-L26-01	&	motorised dampers	&\ch&\ch&\ch&\ch&&\fire 30 Mar 2013\\

\inc	&	DDCP-MW-L24-01	&	L24-MW-AH1,2,3,SEF-1,2,ECU-1,Cold Stores,EAF-1	&\ch&\ch&\ch&\ch&&\fire\askar 1 Apr 2013\\

\inc	&	DDCP-MW-L23-01	&	motorised dampers	&\ch&\ch&\ch&\ch&&\fire 3 Apr 2013\\

\inc	&	DDCP-MW-L20-01	&	motorised dampers	&\ch&\ch&\ch&\ch&&\fire 4 Apr 2013\\

\inc	&	DDCP-MW-L17-01	&	motorised dampers	&\ch&\ch&\ch&\ch&&\fire 5 Apr 2013\\

\inc	&	DDCP-MW-L14-01	&	motorised dampers	&\ch&\ch&\ch&\ch&&\fire 6 Apr 2013\\
\inc	&	DDCP-MW-L10-01	&	motorised dampers	&\ch&\ch&\ch&\ch&&\fire 7 Apr 2013\\

\inc	&	DDCP-MW-L07-01	&	motorised dampers	&\ch&\ch&\ch&\ch&&\fire 8 Apr 2013\\

\inc	&	DDCP-MW-L05-01	&	L5-MW-AH-1,AH-2	&\ch&\ch&\ch&\ch&&\fire 9 Apr 2013\\

\inc	&	DDCP-MW-L05-02	&	L5-MW-AH5,Chilled Water System	&\ch&\ch&\ch&\ch&&\\
\inc	&	DDCP-MW-L05-03	&	L5-MW-AH6,EAF-1,2,3,4,L5-ECU-1,2	&\ch&\ch&\ch&\ch&&\\
\inc	&	DDCP-MW-L05-04	&	LV PANELS	&\ch&\ch&\ch&\ch&&\\

\inc	&	DDCP-MW-L04-01	&	L4-MW-ECU-1,L4-MW-AH-1	&\ch&\ch&\ch&\ch&&\\

\inc	&	DDCP-MW-L04-02	&	L4-MW-HYDZ,L4-MW-HYFHZ,L4-MW-PT	&\ch&\ch&\ch&\ch&&\\

\inc	&	DDCP-MW-L04-03	&	L4-MW-AH3,4,5,6	&\ch&\ch&\ch&\ch&&\\

\inc	&	DDCP-MW-L04-04	&	HOT WATER SYSTEM	&\ch&\ch&\ch&\ch&&\\

\inc	&	DDCP-MW-L03-01	&	L3-MW-EAF-1,2,3,4,5,6,L3-MW-FAF-1,ELEVATORS	&\ch&\ch&\ch&\ch&&\\

\inc	&	DDCP-MW-L03-02	&	Cooling Tower FOR DG	&\ch&\ch&\ch&\ch&&\\

\inc	&	DDCP-MW-L03-03	&	Swimming Pool Systems	&\ch&\ch&\ch&\ch&&\Danger swimming pool contractor\\

\inc	&	DDCP-MW-L02-01	&	L2-MW-AH-1,2	&\ch&\ch&\ch&\ch&&\\

\inc	&	DDCP-MW-L02-02	&	L2-MW-AH-7,Cold Stores,SEF-1,2,EAF-3	&\ch&\ch&\ch&\ch&&\askar\\

\inc	&	DDCP-MW-L02-03	&	L2-MW-AH-4,5,6,L2-MW-EAF-1,2	&\ch&\ch&\ch&\ch&&\\

\inc	&	DDCP-MW-L01-01	&	L1-MW-AH4A,6,7,EAF2,3,4	&\ch&\ch&\ch&\ch&&\\

\inc	&	DDCP-MW-L01-02	&	L1-MW-AH-1,2,3,Cold Stores, EAF-1	&\ch&\ch&\ch&\ch&&\askar\\

\inc	&	DDCP-MW-L01-03	&	L1-MW-AH5A,L1-MW-SEF-1,2, VAV	&\ch&\ch&\ch&\ch&&\\

\inc	&	DDCP-MW-GR-01	&	GR-MW-AH1,AH2,GR-MW-EAF3,4,SEF-1,LPG SYSTEM	&\ch&\ch&\ch&\ch&&\Danger Gasco\\

\inc	&	DDCP-MW-GR-02	&	LT PANEL,HV PANEL,Cold Stores,EAF-1,2	&\ch&\ch&\ch&\ch&& \askar\\

\inc	&	DDCP-MW-B1-01	&	B1-MW-EAF-1,FAF-1,PAC	&\ch&\ch&\ch&\ch&&\\

\inc	&	DDCP-MW-B1-02	&	B1-MW-AH-4	&\ch&\ch&\ch&\ch&&\\

\inc	&	DDCP-MW-B1-03	&	Water Fountain Pumps
	&\ch&\ch&\ch&\ch&&\Danger water features contractor\\

\inc	&	DDCP-MW-B2-01	&	B2-MW-AH-01A,B2-MW-FAF-01	&\ch&\ch&\ch&\ch&&\fire\\

\inc	&	DDCP-MW-B2-02	&	B2-RO-EAF1, B2-RO-FAF-1, B2-RO-JF-1-16	&\ch&\ch&\ch&\ch&&\fire\\

\inc	&	DDCP-MW-B2-03	&	B2-SL-EAF1, B2-SL-EAF-2, B2-SL-FAF-1, B2-SL-JF-1-16	&\ch&\ch&\ch&\ch&&\fire\\

\inc	&	DDCP-MW-B3-01	&	BOILER ROOM	&\ch&\ch&\ch&\ch&&\\

\inc	&	DDCP-MW-B3-02	&	DIESEL GENERATOR	&\ch&\ch&\ch&\ch&&\fire\\

\inc	&	DDCP-MW-B3-03	&	Plumbing Plant Room	&\ch&\ch&\ch&\ch&&\fire\\

\inc	&	DDCP-MW-B3-04	&	District Cooling Room	&\ch&\ch&\ch&\ch&\ch&\\

\inc	&	DDCP-MW-B3-05	&	Grey Water Plant Room	&\ch&\ch&\ch&\ch&&\\

\inc	&	DDCP-MW-B3-06	&	B3-MW-AH3A,B3-MW-PSP	&\ch&\ch&\ch&\ch&&\\


\bottomrule
\end{longtable}
\label{SLbms}
}








 

%% ADD SIGNAGE

%\chapter{Civil Defence Approval}
\parindent1em
\label{civildefence}
Civil Defence Inspections can only take place when the Building Operations have been completed, when the full Fire Fighting,  Fire Alarm, Staircase Pressurization and Smoke Exhaust systems have been completed tested and commissioned. We expect that this will only be possible at the earliest by end April  2013. However, we will re-evaluate these dates once the works progress a bit further. 


The Qatar Civil Defence Department if approached can unofficially inspect the works earlier in order to identify problem areas earlier. However, at this stage it is bound to catch rather known problems than unknown problems.

The Civil Defence related sections have been discussed under separate headings and will not be repeated here. See \tref{cdd} for status and details.


\begin{table}[h]
\begin{tabular}{lr}
\toprule
\textbf{Description}	 &\textbf{Qty}\\
\midrule 	 
\textbf{Fire Fighting System}	 &\\
Sprinkler Points	    &26,078\\
Fire Hose Cabinets	    &457 \\
Zone Control Valves	    &393 \\
Pressure Reducing Valves	 &288 \\
Breeching Inlets	        &3\\ 
Fire Hydrants	            &13\\ 
OS\&Y Valves	               &86\\ 
\textbf{Alarm Check Valves}	              &18 \\
Check Valves / Non Return Valves &18 \\
Fire Extinguishers Dry Powder	 &691 \\
Fire Extinguishers CO2	           &457 \\
Fire Pumps	                     &12 \\
Jockey Pumps	                     &6 \\
\midrule 	 
\textbf{Fire Alarm System}	 &\\
Smoke \& Heat Detectors	 &7,573 \\
Monitor / Control Modules	       &2,252 \\
Sounders / Speakers / Voice Evacuation	   &2,630 \\
Main Fire Alarm Control Panel	         &9 \\
Conventional Fire Alarm Panel	        &16 \\
Voice Evacuation Panel	                  &9 \\
\midrule 
\textbf{FM200 System}	 &\\
Fire Extinguisher Control Panel	&36 \\
FM200 Cylinder with Agent	            &38 \\
\midrule 	 
\textbf{Foam System}	 &\\
Deluge Valve with complete trimming	 &2 \\
Bladder Tank	                        &2 \\
Foam Chemical	                   &100 Gal\\
\bottomrule
\end{tabular}
\caption{Items comprising the Fire Protection System}
\label{fireitems}
\end{table}

\tref{fireitems} lists all the devices related to Fire Life Safety code requirements and indicates clearly the extend of the challenge to complete the installation and commissioning within the stipulated time. Terminal devices are still to be installed in Merweb and a great number of them are still to be delivered. Of special concern are the large number of devices related to the Fire Alarm system (approximately 13,000) which are prone to errors and false alarms.




\medskip

\captionof{table}{Summary CDD requirements.}
\label{cdd}
{\small\RaggedRight
\begin{longtable}{p{2.3cm}p{4.2cm}p{4.2cm}}

\toprule
Fire Protection    &Firefighting system.  & Substantially ready.\\
    &Fire suppression FM~200 system. & 65\%\\
	&Fire alarm \& voice evacuation system. & ongoing\\
	&Fire alarm control rooms. & 75\%\\
	&Fire Water tanks & \Danger Expected Completion 20 Mar 2013 \\
	&Evacuation Map & HLG Scope\\
	&FA Matrix & Submitted\\
	&Life Safety \& fire protection cause and 
         effect report & \ch\\
	&Function \& Sequence of operation & On-going\\
\midrule

HVAC	&Staircase Pressurization System & Rotana, Shangrila tested, Merweb 15.03.2013. Retesting necessary on other two.\\
	&ECUs, SEF, EAF, AHU, FAF, ECU, SMD, FD etc… &\\
	&Car Park Ventilation & See \S\ref{carparkventilation}\\
	&Function \& sequence of operation &\\
\midrule


Electrical	&Exit \& emergency lights. &\\
	&Power Diesel generator \& ATS. &\\
	&DG Tanks. &\\
	&Access control system.  &\\
	&CPMS. &\\
	&UPS system. & See \refs{ups}\\
	&Lift sequence of operation &\\
\midrule

Documents	&Drawings &\\
	&Method Statements &\ch \\
	&Sequence of operation &\ch \\
	&Products certificates &\ch \\
	&T\&C reports, &\\
	&Undertaking letter to CDD &\\
	&Certificate of HLG \& NAFFCO &\\
        & Certificate Al-Misnad              &\ch 9 Mar 2013\\
	&MEP Function \& sequence of operation manual/Report, Interface \& Integrity reports &ongoing\\
	&Stamped \& approved Shop Drawing &\ch \\
\bottomrule
\end{longtable}
}

\section{Civil Defence Inspection Checklist base on Phase II Inspection}

Comments that need action based on Phase II inspection.



\begin{table}[htbp]
\resetinc
{\RaggedRight\small
\begin{tabular}{p{0.5cm}p{4.5cm}ll}
\toprule
item  & Activity  & Status  & Remarks\\
\midrule
\inc & Fire exit staircase identification, inside stair floor with clear identification marking, "TO BASEMENT NO EXIT", "NO ROOF ACCESS" signages etc. & & 30 Mar 2013\\
\inc & Identification marking for mechanical rooms, fire pump room, electrical room, Fire Lift, Fire Command Center and other important building/services rooms. && 30 Mar 2013\\
\inc & "In Case of Fire Do not Use Lift" in arabic identification marks fro mechanical room, fire pump room, exit staircases, exit access routes, as per NFPA 101            &&30 Mar 2013\\
\inc & Label marking for all access doors for Fire Dampers & &30 Mar 2013\\

\bottomrule

\end{tabular}}
\caption{Emergency Lights, signs and markings. Works under HLG.}
\end{table}








\end{document}
%\input{./Sections/municipalconnections}

%\input{./sections/logistics}
%\appendix


%\input{./Sections/gantcharts}
%\input{./Sections/charts}
%\input{/.Sections/heatgains}

\input{./Sections/mechanicalengineers}
%% Example
%% 
   
   \out{Close Ceilings Rotana Floor 10}{Line 2}
   \out{Rotana,  power-on}{12$^{th}$ Sept 2010}
   \out{Rotana Qatar Cool switch-on}{12th September}


%% we now read the file and dostuff with it
%% 
\chapter{Summary of  Key Dates}
\immediate\closeout\tempfile
\keydate { Rotana kitchen extract}{25 Feb 2013}
\keydate { Shangri-la kitchen extract}{15 Mar 2013}
\keydate { Merweb kitchen extract}{30 Mar 2013}
\keydate { Rotana Tower Completion : }{15 Sep 2010}


\lipsum[1]

\backmatter
%\bibliography{sample-handout}
%\bibliographystyle{plainnat}
\printindex



%% finally we close the file
%%




\end{document}
