\documentclass[11pt,twocolumn]{ltugboat}
\usepackage[utf8]{inputenc}
\usepackage[T1]{fontenc}
\usepackage[latin,english]{babel}
\usepackage{phd}
\usepackage{marginnote}
\def\hlred{{\color{Magenta}}}
\usepackage{doccommands}
\usepackage{morelogos}
\usepackage{hyperref}
\hypersetup{pdftex,
  bookmarks,
  raiselinks,
  pageanchor,
  hyperindex,
  colorlinks,
  allcolors=theblue, 
  %anchorcolor= blue,
  %filecolor=blue,
  urlcolor=theblue,
  linkcolor= theblue,
  pdftitle={My Title},
 }
\title{A Discourse Upon The Origin And The Foundation Of The Inequality Among
Mankind }
\author{By J. J. Rousseau}
%\date{Konigsberg in Prussia, 30th September, 1784.}
\begin{document}

\maketitle

\begin{figure*}
\includegraphics[width=\textwidth]{C:/Users/user/Dropbox/pic/F-001.jpg}
\end{figure*}

\section{INTRODUCTORY NOTE}

Jean Jacques Rousseau was born at Geneva, June~28, 1712, the son of a
watchmaker of French origin. His education was irregular, and though
he tried many professions---including engraving, music, and
teaching---he found it difficult to support himself in any of them. The
discovery of his talent as a writer came with the winning of a prize
offered by the Academy of Dijon for a discourse on the question,
"Whether the progress of the sciences and of letters has tended to
corrupt or to elevate morals." He argued so brilliantly that the
tendency of civilization was degrading that he became at once famous.
The discourse here printed on the causes of inequality among men was
written in a similar competition.

He now concentrated his powers upon literature, producing two novels,
"La Nouvelle Heloise," the forerunner and parent of endless
sentimental and picturesque fictions; and "Emile, ou l'Education," a
work which has had enormous influence on the theory and practise of
pedagogy down to our own time and in which the Savoyard Vicar appears,
who is used as the mouthpiece for Rousseau's own religious ideas. "Le
Contrat Social" (1762) elaborated the doctrine of the discourse on
inequality. Both historically and philosophically it is unsound; but
it was the chief literary source of the enthusiasm for liberty,
fraternity, and equality, which inspired the leaders of the French
Revolution, and its effects passed far beyond France.

His most famous work, the "Confessions," was published after his
death. This book is a mine of information as to his life, but it is
far from trustworthy; and the picture it gives of the author's
personality and conduct, though painted in such a way as to make it
absorbingly interesting, is often unpleasing in the highest degree.
But it is one of the great autobiographies of the world.

During Rousseau's later years he was the victim of the delusion of
persecution; and although he was protected by a succession of good
friends, he came to distrust and quarrel with each in turn. He died at
Ermenonville, near Paris, July 2, 1778, the most widely influential
French writer of his age.

The Savoyard Vicar and his "Profession of Faith" are introduced into
"Emile" not, according to the author, because he wishes to exhibit his
principles as those which should be taught, but to give an example of
the way in which religious matters should be discussed with the young.
Nevertheless, it is universally recognized that these opinions are
Rousseau's own, and represent in short form his characteristic
attitude toward religious belief. The Vicar himself is believed to
combine the traits of two Savoyard priests whom Rousseau knew in his
youth. The more important was the Abbe Gaime, whom he had known at
Turin; the other, the Abbe Gatier, who had taught him at Annecy.




\section*{QUESTION PROPOSED BY THE ACADEMY OF DIJON}

What is the Origin of the Inequality among Mankind; and whether such
Inequality is authorized by the Law of Nature?



\section*{A DISCOURSE UPON THE ORIGIN AND THE FOUNDATION OF THE INEQUALITY AMONG
MANKIND}

'Tis of man I am to speak; and the very question, in answer to which I
am to speak of him, sufficiently informs me that I am going to speak
to men; for to those alone, who are not afraid of honouring truth, it
belongs to propose discussions of this kind. I shall therefore
maintain with confidence the cause of mankind before the sages, who
invite me to stand up in its defence; and I shall think myself happy,
if I can but behave in a manner not unworthy of my subject and of my
judges.

I conceive two species of inequality among men; one which I call
natural, or physical inequality, because it is established by nature,
and consists in the difference of age, health, bodily strength, and
the qualities of the mind, or of the soul; the other which may be
termed moral, or political inequality, because it depends on a kind of
convention, and is established, or at least authorized, by the common
consent of mankind. This species of inequality consists in the
different privileges, which some men enjoy, to the prejudice of
others, such as that of being richer, more honoured, more powerful,
and even that of exacting obedience from them.

It were absurd to ask, what is the cause of natural inequality, seeing
the bare definition of natural inequality answers the question: it
would be more absurd still to enquire, if there might not be some
essential connection between the two species of inequality, as it
would be asking, in other words, if those who command are necessarily
better men than those who obey; and if strength of body or of mind,
wisdom or virtue are always to be found in individuals, in the same
proportion with power, or riches: a question, fit perhaps to be
discussed by slaves in the hearing of their masters, but unbecoming
free and reasonable beings in quest of truth.

What therefore is precisely the subject of this discourse? It is to
point out, in the progress of things, that moment, when, right taking
place of violence, nature became subject to law; to display that chain
of surprising events, in consequence of which the strong submitted to
serve the weak, and the people to purchase imaginary ease, at the
expense of real happiness.

The philosophers, who have examined the foundations of society, have,
every one of them, perceived the necessity of tracing it back to a
state of nature, but not one of them has ever arrived there. Some of
them have not scrupled to attribute to man in that state the ideas of
justice and injustice, without troubling their heads to prove, that he
really must have had such ideas, or even that such ideas were useful
to him: others have spoken of the natural right of every man to keep
what belongs to him, without letting us know what they meant by the
word belong; others, without further ceremony ascribing to the
strongest an authority over the weakest, have immediately struck out
government, without thinking of the time requisite for men to form any
notion of the things signified by the words authority and government.
All of them, in fine, constantly harping on wants, avidity,
oppression, desires and pride, have transferred to the state of nature
ideas picked up in the bosom of society. In speaking of savages they
described citizens. Nay, few of our own writers seem to have so much
as doubted, that a state of nature did once actually exit; though it
plainly appears by Sacred History, that even the first man,
immediately furnished as he was by God himself with both instructions
and precepts, never lived in that state, and that, if we give to the
books of Moses that credit which every Christian philosopher ought to
give to them, we must deny that, even before the deluge, such a state
ever existed among men, unless they fell into it by some extraordinary
event: a paradox very difficult to maintain, and altogether impossible
to prove.

Let us begin therefore, by laying aside facts, for they do not affect
the question. The researches, in which we may engage on this occasion,
are not to be taken for historical truths, but merely as hypothetical
and conditional reasonings, fitter to illustrate the nature of things,
than to show their true origin, like those systems, which our
naturalists daily make of the formation of the world. Religion
commands us to believe, that men, having been drawn by God himself out
of a state of nature, are unequal, because it is his pleasure they
should be so; but religion does not forbid us to draw conjectures
solely from the nature of man, considered in itself, and from that of
the beings which surround him, concerning the fate of mankind, had
they been left to themselves. This is then the question I am to
answer, the question I propose to examine in the present discourse. As
mankind in general have an interest in my subject, I shall endeavour
to use a language suitable to all nations; or rather, forgetting the
circumstances of time and place in order to think of nothing but the
men I speak to, I shall suppose myself in the Lyceum of Athens,
repeating the lessons of my masters before the Platos and the
Xenocrates of that famous seat of philosophy as my judges, and in
presence of the whole human species as my audience.

O man, whatever country you may belong to, whatever your opinions may
be, attend to my words; you shall hear your history such as I think I
have read it, not in books composed by those like you, for they are
liars, but in the book of nature which never lies. All that I shall
repeat after her, must be true, without any intermixture of falsehood,
but where I may happen, without intending it, to introduce my own
conceits. The times I am going to speak of are very remote. How much
you are changed from what you once were! 'Tis in a manner the life of
your species that I am going to write, from the qualities which you
have received, and which your education and your habits could deprave,
but could not destroy. There is, I am sensible, an age at which every
individual of you would choose to stop; and you will look out for the
age at which, had you your wish, your species had stopped. Uneasy at
your present condition for reasons which threaten your unhappy
posterity with still greater uneasiness, you will perhaps wish it were
in your power to go back; and this sentiment ought to be considered,
as the panegyric of your first parents, the condemnation of your
contemporaries, and a source of terror to all those who may have the
misfortune of succeeding you.




\section*{DISCOURSE FIRST PART}

However important it may be, in order to form a proper judgment of the
natural state of man, to consider him from his origin, and to examine
him, as it were, in the first embryo of the species; I shall not
attempt to trace his organization through its successive approaches to
perfection: I shall not stop to examine in the animal system what he
might have been in the beginning, to become at last what he actually
is; I shall not inquire whether, as Aristotle thinks, his neglected
nails were no better at first than crooked talons; whether his whole
body was not, bear-like, thick covered with rough hair; and whether,
walking upon all-fours, his eyes, directed to the earth, and confined
to a horizon of a few paces extent, did not at once point out the
nature and limits of his ideas. I could only form vague, and almost
imaginary, conjectures on this subject. Comparative anatomy has not as
yet been sufficiently improved; neither have the observations of
natural philosophy been sufficiently ascertained, to establish upon
such foundations the basis of a solid system. For this reason, without
having recourse to the supernatural informations with which we have
been favoured on this head, or paying any attention to the changes,
that must have happened in the conformation of the interior and
exterior parts of man's body, in proportion as he applied his members
to new purposes, and took to new aliments, I shall suppose his
conformation to have always been, what we now behold it; that he
always walked on two feet, made the same use of his hands that we do
of ours, extended his looks over the whole face of nature, and
measured with his eyes the vast extent of the heavens.

If I strip this being, thus constituted, of all the supernatural gifts
which he may have received, and of all the artificial faculties, which
we could not have acquired but by slow degrees; if I consider him, in
a word, such as he must have issued from the hands of nature; I see an
animal less strong than some, and less active than others, but, upon
the whole, the most advantageously organized of any; I see him
satisfying the calls of hunger under the first oak, and those of
thirst at the first rivulet; I see him laying himself down to sleep at
the foot of the same tree that afforded him his meal; and behold, this
done, all his wants are completely supplied.

The earth left to its own natural fertility and covered with immense
woods, that no hatchet ever disfigured, offers at every step food and
shelter to every species of animals. Men, dispersed among them,
observe and imitate their industry, and thus rise to the instinct of
beasts; with this advantage, that, whereas every species of beasts is
confined to one peculiar instinct, man, who perhaps has not any that
particularly belongs to him, appropriates to himself those of all
other animals, and lives equally upon most of the different aliments,
which they only divide among themselves; a circumstance which
qualifies him to find his subsistence, with more ease than any of
them.

Men, accustomed from their infancy to the inclemency of the weather,
and to the rigour of the different seasons; inured to fatigue, and
obliged to defend, naked and without arms, their life and their prey
against the other wild inhabitants of the forest, or at least to avoid
their fury by flight, acquire a robust and almost unalterable habit of
body; the children, bringing with them into the world the excellent
constitution of their parents, and strengthening it by the same
exercises that first produced it, attain by this means all the vigour
that the human frame is capable of. Nature treats them exactly in the
same manner that Sparta treated the children of her citizens; those
who come well formed into the world she renders strong and robust, and
destroys all the rest; differing in this respect from our societies,
in which the state, by permitting children to become burdensome to
their parents, murders them all without distinction, even in the wombs
of their mothers.

The body being the only instrument that savage man is acquainted with,
he employs it to different uses, of which ours, for want of practice,
are incapable; and we may thank our industry for the loss of that
strength and agility, which necessity obliges him to acquire. Had he a
hatchet, would his hand so easily snap off from an oak so stout a
branch? Had he a sling, would it dart a stone to so great a distance?
Had he a ladder, would he run so nimbly up a tree? Had he a horse,
would he with such swiftness shoot along the plain? Give civilized man
but time to gather about him all his machines, and no doubt he will be
an overmatch for the savage: but if you have a mind to see a contest
still more unequal, place them naked and unarmed one opposite to the
other; and you will soon discover the advantage there is in
perpetually having all our forces at our disposal, in being constantly
prepared against all events, and in always carrying ourselves, as it
were, whole and entire about us.

Hobbes would have it that man is naturally void of fear, and always
intent upon attacking and fighting. An illustrious philosopher thinks
on the contrary, and Cumberland and Puffendorff likewise affirm it,
that nothing is more fearful than man in a state of nature, that he is
always in a tremble, and ready to fly at the first motion he
perceives, at the first noise that strikes his ears. This, indeed, may
be very true in regard to objects with which he is not acquainted; and
I make no doubt of his being terrified at every new sight that
presents itself, as often as he cannot distinguish the physical good
and evil which he may expect from it, nor compare his forces with the
dangers he has to encounter; circumstances that seldom occur in a
state of nature, where all things proceed in so uniform a manner, and
the face of the earth is not liable to those sudden and continual
changes occasioned in it by the passions and inconstancies of
collected bodies. But savage man living among other animals without
any society or fixed habitation, and finding himself early under a
necessity of measuring his strength with theirs, soon makes a
comparison between both, and finding that he surpasses them more in
address, than they surpass him in strength, he learns not to be any
longer in dread of them. Turn out a bear or a wolf against a sturdy,
active, resolute savage, (and this they all are,) provided with stones
and a good stick; and you will soon find that the danger is at least
equal on both sides, and that after several trials of this kind, wild
beasts, who are not fond of attacking each other, will not be very
fond of attacking man, whom they have found every whit as wild as
themselves. As to animals who have really more strength than man has
address, he is, in regard to them, what other weaker species are, who
find means to subsist notwithstanding; he has even this great
advantage over such weaker species, that being equally fleet with
them, and finding on every tree an almost inviolable asylum, he is
always at liberty to take it or leave it, as he likes best, and of
course to fight or to fly, whichever is most agreeable to him. To this
we may add that no animal naturally makes war upon man, except in the
case of self-defence or extreme hunger; nor ever expresses against him
any of these violent antipathies, which seem to indicate that some
particular species are intended by nature for the food of others.

But there are other more formidable enemies, and against which man is
not provided with the same means of defence; I mean natural
infirmities, infancy, old age, and sickness of every kind, melancholy
proofs of our weakness, whereof the two first are common to all
animals, and the last chiefly attends man living in a state of
society. It is even observable in regard to infancy, that the mother
being able to carry her child about with her, wherever she goes, can
perform the duty of a nurse with a great deal less trouble, than the
females of many other animals, who are obliged to be constantly going
and coming with no small labour and fatigue, one way to look out for
their own subsistence, and another to suckle and feed their young
ones. True it is that, if the woman happens to perish, her child is
exposed to the greatest danger of perishing with her; but this danger
is common to a hundred other species, whose young ones require a great
deal of time to be able to provide for themselves; and if our infancy
is longer than theirs, our life is longer likewise; so that, in this
respect too, all things are in a manner equal; not but that there are
other rules concerning the duration of the first age of life, and the
number of the young of man and other animals, but they do not belong
to my subject. With old men, who stir and perspire but little, the
demand for food diminishes with their abilities to provide it; and as
a savage life would exempt them from the gout and the rheumatism, and
old age is of all ills that which human assistance is least capable of
alleviating, they would at last go off, without its being perceived by
others that they ceased to exist, and almost without perceiving it
themselves.

In regard to sickness, I shall not repeat the vain and false
declamations made use of to discredit medicine by most men, while they
enjoy their health; I shall only ask if there are any solid
observations from which we may conclude that in those countries where
the healing art is most neglected, the mean duration of man's life is
shorter than in those where it is most cultivated? And how is it
possible this should be the case, if we inflict more diseases upon
ourselves than medicine can supply us with remedies! The extreme
inequalities in the manner of living of the several classes of
mankind, the excess of idleness in some, and of labour in others, the
facility of irritating and satisfying our sensuality and our
appetites, the too exquisite and out of the way aliments of the rich,
which fill them with fiery juices, and bring on indigestions, the
unwholesome food of the poor, of which even, bad as it is, they very
often fall short, and the want of which tempts them, every opportunity
that offers, to eat greedily and overload their stomachs; watchings,
excesses of every kind, immoderate transports of all the passions,
fatigues, waste of spirits, in a word, the numberless pains and
anxieties annexed to every condition, and which the mind of man is
constantly a prey to; these are the fatal proofs that most of our ills
are of our own making, and that we might have avoided them all by
adhering to the simple, uniform and solitary way of life prescribed to
us by nature. Allowing that nature intended we should always enjoy
good health, I dare almost affirm that a state of reflection is a
state against nature, and that the man who meditates is a depraved
animal. We need only call to mind the good constitution of savages,
of those at least whom we have not destroyed by our strong liquors; we
need only reflect, that they are strangers to almost every disease,
except those occasioned by wounds and old age, to be in a manner
convinced that the history of human diseases might be easily composed
by pursuing that of civil societies. Such at least was the opinion of
Plato, who concluded from certain remedies made use of or approved by
Podalyrus and Macaon at the Siege of Troy, that several disorders,
which these remedies were found to bring on in his days, were not
known among men at that remote period.

Man therefore, in a state of nature where there are so few sources of
sickness, can have no great occasion for physic, and still less for
physicians; neither is the human species more to be pitied in this
respect, than any other species of animals. Ask those who make hunting
their recreation or business, if in their excursions they meet with
many sick or feeble animals. They meet with many carrying the marks of
considerable wounds, that have been perfectly well healed and closed
up; with many, whose bones formerly broken, and whose limbs almost
torn off, have completely knit and united, without any other surgeon
but time, any other regimen but their usual way of living, and whose
cures were not the less perfect for their not having been tortured
with incisions, poisoned with drugs, or worn out by diet and
abstinence. In a word, however useful medicine well administered may
be to us who live in a state of society, it is still past doubt, that
if, on the one hand, the sick savage, destitute of help, has nothing
to hope from nature, on the other, he has nothing to fear but from his
disease; a circumstance, which oftens renders his situation preferable
to ours.

Let us therefore beware of confounding savage man with the men, whom
we daily see and converse with. Nature behaves towards all animals
left to her care with a predilection, that seems to prove how jealous
she is of that prerogative. The horse, the cat, the bull, nay the ass
itself, have generally a higher stature, and always a more robust
constitution, more vigour, more strength and courage in their forests
than in our houses; they lose half these advantages by becoming
domestic animals; it looks as if all our attention to treat them
kindly, and to feed them well, served only to bastardize them. It is
thus with man himself. In proportion as he becomes sociable and a
slave to others, he becomes weak, fearful, mean-spirited, and his soft
and effeminate way of living at once completes the enervation of his
strength and of his courage. We may add, that there must be still a
wider difference between man and man in a savage and domestic
condition, than between beast and beast; for as men and beasts have
been treated alike by nature, all the conveniences with which men
indulge themselves more than they do the beasts tamed by them, are so
many particular causes which make them degenerate more sensibly.

Nakedness therefore, the want of houses, and of all these
unnecessaries, which we consider as so very necessary, are not such
mighty evils in respect to these primitive men, and much less still
any obstacle to their preservation. Their skins, it is true, are
destitute of hair; but then they have no occasion for any such
covering in warm climates; and in cold climates they soon learn to
apply to that use those of the animals they have conquered; they have
but two feet to run with, but they have two hands to defend themselves
with, and provide for all their wants; it costs them perhaps a great
deal of time and trouble to make their children walk, but the mothers
carry them with ease; an advantage not granted to other species of
animals, with whom the mother, when pursued, is obliged to abandon her
young ones, or regulate her steps by theirs. In short, unless we admit
those singular and fortuitous concurrences of circumstances, which I
shall speak of hereafter, and which, it is very possible, may never
have existed, it is evident, in every state of the question, that the
man, who first made himself clothes and built himself a cabin,
supplied himself with things which he did not much want, since he had
lived without them till then; and why should he not have been able to
support in his riper years, the same kind of life, which he had
supported from his infancy?

Alone, idle, and always surrounded with danger, savage man must be
fond of sleep, and sleep lightly like other animals, who think but
little, and may, in a manner, be said to sleep all the time they do
not think: self-preservation being almost his only concern, he must
exercise those faculties most, which are most serviceable in attacking
and in defending, whether to subdue his prey, or to prevent his
becoming that of other animals: those organs, on the contrary, which
softness and sensuality can alone improve, must remain in a state of
rudeness, utterly incompatible with all manner of delicacy; and as his
senses are divided on this point, his touch and his taste must be
extremely coarse and blunt; his sight, his hearing, and his smelling
equally subtle: such is the animal state in general, and accordingly
if we may believe travellers, it is that of most savage nations. We
must not therefore be surprised, that the Hottentots of the Cape of
Good Hope, distinguish with their naked eyes ships on the ocean, at as
great a distance as the Dutch can discern them with their glasses; nor
that the savages of America should have tracked the Spaniards with
their noses, to as great a degree of exactness, as the best dogs could
have done; nor that all these barbarous nations support nakedness
without pain, use such large quantities of Piemento to give their food
a relish, and drink like water the strongest liquors of Europe.

As yet I have considered man merely in his physical capacity; let us
now endeavour to examine him in a metaphysical and moral light.

I can discover nothing in any mere animal but an ingenious machine, to
which nature has given senses to wind itself up, and guard, to a
certain degree, against everything that might destroy or disorder it.
I perceive the very same things in the human machine, with this
difference, that nature alone operates in all the operations of the
beast, whereas man, as a free agent, has a share in his. One chooses
by instinct; the other by an act of liberty; for which reason the
beast cannot deviate from the rules that have been prescribed to it,
even in cases where such deviation might be useful, and man often
deviates from the rules laid down for him to his prejudice. Thus a
pigeon would starve near a dish of the best flesh-meat, and a cat on a
heap of fruit or corn, though both might very well support life with
the food which they thus disdain, did they but bethink themselves to
make a trial of it: it is in this manner dissolute men run into
excesses, which bring on fevers and death itself; because the mind
depraves the senses, and when nature ceases to speak, the will still
continues to dictate.

All animals must be allowed to have ideas, since all animals have
senses; they even combine their ideas to a certain degree, and, in
this respect, it is only the difference of such degree, that
constitutes the difference between man and beast: some philosophers
have even advanced, that there is a greater difference between some
men and some others, than between some men and some beasts; it is not
therefore so much the understanding that constitutes, among animals,
the specifical distinction of man, as his quality of a free agent.
Nature speaks to all animals, and beasts obey her voice. Man feels the
same impression, but he at the same time perceives that he is free to
resist or to acquiesce; and it is in the consciousness of this
liberty, that the spirituality of his soul chiefly appears: for
natural philosophy explains, in some measure, the mechanism of the
senses and the formation of ideas; but in the power of willing, or
rather of choosing, and in the consciousness of this power, nothing
can be discovered but acts, that are purely spiritual, and cannot be
accounted for by the laws of mechanics.

But though the difficulties, in which all these questions are
involved, should leave some room to dispute on this difference between
man and beast, there is another very specific quality that
distinguishes them, and a quality which will admit of no dispute; this
is the faculty of improvement; a faculty which, as circumstances
offer, successively unfolds all the other faculties, and resides among
us not only in the species, but in the individuals that compose it;
whereas a beast is, at the end of some months, all he ever will be
during the rest of his life; and his species, at the end of a thousand
years, precisely what it was the first year of that long period. Why
is man alone subject to dotage? Is it not, because he thus returns to
his primitive condition? And because, while the beast, which has
acquired nothing and has likewise nothing to lose, continues always in
possession of his instinct, man, losing by old age, or by accident,
all the acquisitions he had made in consequence of his perfectibility,
thus falls back even lower than beasts themselves? It would be a
melancholy necessity for us to be obliged to allow, that this
distinctive and almost unlimited faculty is the source of all man's
misfortunes; that it is this faculty, which, though by slow degrees,
draws them out of their original condition, in which his days would
slide away insensibly in peace and innocence; that it is this faculty,
which, in a succession of ages, produces his discoveries and mistakes,
his virtues and his vices, and, at long run, renders him both his own
and nature's tyrant. It would be shocking to be obliged to commend, as
a beneficent being, whoever he was that first suggested to the
Oronoco Indians the use of those boards which they bind on the
temples of their children, and which secure to them the enjoyment of
some part at least of their natural imbecility and happiness.

Savage man, abandoned by nature to pure instinct, or rather
indemnified for that which has perhaps been denied to him by faculties
capable of immediately supplying the place of it, and of raising him
afterwards a great deal higher, would therefore begin with functions
that were merely animal: to see and to feel would be his first
condition, which he would enjoy in common with other animals. To will
and not to will, to wish and to fear, would be the first, and in a
manner, the only operations of his soul, till new circumstances
occasioned new developments.

Let moralists say what they will, the human understanding is greatly
indebted to the passions, which, on their side, are likewise
universally allowed to be greatly indebted to the human understanding.
It is by the activity of our passions, that our reason improves: we
covet knowledge merely because we covet enjoyment, and it is
impossible to conceive why a man exempt from fears and desires should
take the trouble to reason. The passions, in their turn, owe their
origin to our wants, and their increase to our progress in science;
for we cannot desire or fear anything, but in consequence of the ideas
we have of it, or of the simple impulses of nature; and savage man,
destitute of every species of knowledge, experiences no passions but
those of this last kind; his desires never extend beyond his physical
wants; he knows no goods but food, a female, and rest; he fears no
evil but pain, and hunger; I say pain, and not death; for no animal,
merely as such, will ever know what it is to die, and the knowledge of
death, and of its terrors, is one of the first acquisitions made by
man, in consequence of his deviating from the animal state.

I could easily, were it requisite, cite facts in support of this
opinion, and show, that the progress of the mind has everywhere kept
pace exactly with the wants, to which nature had left the inhabitants
exposed, or to which circumstances had subjected them, and
consequently to the passions, which inclined them to provide for these
wants. I could exhibit in Egypt the arts starting up, and extending
themselves with the inundations of the Nile; I could pursue them in
their progress among the Greeks, where they were seen to bud forth,
grow, and rise to the heavens, in the midst of the sands and rocks of
Attica, without being able to take root on the fertile banks of the
Eurotas; I would observe that, in general, the inhabitants of the
north are more industrious than those of the south, because they can
less do without industry; as if nature thus meant to make all things
equal, by giving to the mind that fertility she has denied to the
soil.

But exclusive of the uncertain testimonies of history, who does not
perceive that everything seems to remove from savage man the
temptation and the means of altering his condition? His imagination
paints nothing to him; his heart asks nothing from him. His moderate
wants are so easily supplied with what he everywhere finds ready to
his hand, and he stands at such a distance from the degree of
knowledge requisite to covet more, that he can neither have foresight
nor curiosity. The spectacle of nature, by growing quite familiar to
him, becomes at last equally indifferent. It is constantly the same
order, constantly the same revolutions; he has not sense enough to
feel surprise at the sight of the greatest wonders; and it is not in
his mind we must look for that philosophy, which man must have to know
how to observe once, what he has every day seen. His soul, which
nothing disturbs, gives itself up entirely to the consciousness of its
actual existence, without any thought of even the nearest futurity;
and his projects, equally confined with his views, scarce extend to
the end of the day. Such is, even at present, the degree of foresight
in the Caribbean: he sells his cotton bed in the morning, and comes in
the evening, with tears in his eyes, to buy it back, not having
foreseen that he should want it again the next night.

The more we meditate on this subject, the wider does the distance
between mere sensation and the most simple knowledge become in our
eyes; and it is impossible to conceive how man, by his own powers
alone, without the assistance of communication, and the spur of
necessity, could have got over so great an interval. How many ages
perhaps revolved, before men beheld any other fire but that of the
heavens? How many different accidents must have concurred to make them
acquainted with the most common uses of this element? How often have
they let it go out, before they knew the art of reproducing it? And
how often perhaps has not every one of these secrets perished with the
discoverer? What shall we say of agriculture, an art which requires so
much labour and foresight; which depends upon other arts; which, it is
very evident, cannot be practised but in a society, if not a formed
one, at least one of some standing, and which does not so much serve
to draw aliments from the earth, for the earth would yield them
without all that trouble, as to oblige her to produce those things,
which we like best, preferably to others? But let us suppose that men
had multiplied to such a degree, that the natural products of the
earth no longer sufficed for their support; a supposition which, by
the bye, would prove that this kind of life would be very advantageous
to the human species; let us suppose that, without forge or anvil, the
instruments of husbandry had dropped from the heavens into the hands
of savages, that these men had got the better of that mortal aversion
they all have for constant labour; that they had learned to foretell
their wants at so great a distance of time; that they had guessed
exactly how they were to break the earth, commit their seed to it, and
plant trees; that they had found out the art of grinding their corn,
and improving by fermentation the juice of their grapes; all
operations which we must allow them to have learned from the gods,
since we cannot conceive how they should make such discoveries of
themselves; after all these fine presents, what man would be mad
enough to cultivate a field, that may be robbed by the first comer,
man or beast, who takes a fancy to the produce of it. And would any
man consent to spend his day in labour and fatigue, when the rewards
of his labour and fatigue became more and more precarious in
proportion to his want of them? In a word, how could this situation
engage men to cultivate the earth, as long as it was not parcelled out
among them, that is, as long as a state of nature subsisted.

Though we should suppose savage man as well versed in the art of
thinking, as philosophers make him; though we were, after them, to
make him a philosopher himself, discovering of himself the sublimest
truths, forming to himself, by the most abstract arguments, maxims of
justice and reason drawn from the love of order in general, or from
the known will of his Creator: in a word, though we were to suppose
his mind as intelligent and enlightened, as it must, and is, in fact,
found to be dull and stupid; what benefit would the species receive
from all these metaphysical discoveries, which could not be
communicated, but must perish with the individual who had made them?
What progress could mankind make in the forests, scattered up and down
among the other animals? And to what degree could men mutually improve
and enlighten each other, when they had no fixed habitation, nor any
need of each other's assistance; when the same persons scarcely met
twice in their whole lives, and on meeting neither spoke to, or so
much as knew each other?

Let us consider how many ideas we owe to the use of speech; how much
grammar exercises, and facilitates the operations of the mind; let us,
besides, reflect on the immense pains and time that the first
invention of languages must have required: Let us add these
reflections to the preceding; and then we may judge how many thousand
ages must have been requisite to develop successively the operations,
which the human mind is capable of producing.

I must now beg leave to stop one moment to consider the perplexities
attending the origin of languages. I might here barely cite or repeat
the researches made, in relation to this question, by the Abbe de
Condillac, which all fully confirm my system, and perhaps even
suggested to me the first idea of it. But, as the manner, in which the
philosopher resolves the difficulties of his own starting, concerning
the origin of arbitrary signs, shows that he supposes, what I doubt,
namely a kind of society already established among the inventors of
languages; I think it my duty, at the same time that I refer to his
reflections, to give my own, in order to expose the same difficulties
in a light suitable to my subject. The first that offers is how
languages could become necessary; for as there was no correspondence
between men, nor the least necessity for any, there is no conceiving
the necessity of this invention, nor the possibility of it, if it was
not indispensable. I might say, with many others, that languages are
the fruit of the domestic intercourse between fathers, mothers, and
children: but this, besides its not answering any difficulties, would
be committing the same fault with those, who reasoning on the state of
nature, transfer to it ideas collected in society, always consider
families as living together under one roof, and their members as
observing among themselves an union, equally intimate and permanent
with that which we see exist in a civil state, where so many common
interests conspire to unite them; whereas in this primitive state, as
there were neither houses nor cabins, nor any kind of property, every
one took up his lodging at random, and seldom continued above one
night in the same place; males and females united without any
premeditated design, as chance, occasion, or desire brought them
together, nor had they any great occasion for language to make known
their thoughts to each other. They parted with the same ease. The
mother suckled her children, when just born, for her own sake; but
afterwards out of love and affection to them, when habit and custom
had made them dear to her; but they no sooner gained strength enough
to run about in quest of food than they separated even from her of
their own accord; and as they scarce had any other method of not
losing each other, than that of remaining constantly in each other's
sight, they soon came to such a pass of forgetfulness, as not even to
know each other, when they happened to meet again. I must further
observe that the child having all his wants to explain, and
consequently more things to say to his mother, than the mother can
have to say to him, it is he that must be at the chief expense of
invention, and the language he makes use of must be in a great measure
his own work; this makes the number of languages equal to that of the
individuals who are to speak them; and this multiplicity of languages
is further increased by their roving and vagabond kind of life, which
allows no idiom time enough to acquire any consistency; for to say
that the mother would have dictated to the child the words he must
employ to ask her this thing and that, may well enough explain in what
manner languages, already formed, are taught, but it does not show us
in what manner they are first formed.

Let us suppose this first difficulty conquered: Let us for a moment
consider ourselves at this side of the immense space, which must have
separated the pure state of nature from that in which languages became
necessary, and let us, after allowing such necessity, examine how
languages could begin to be established. A new difficulty this, still
more stubborn than the preceding; for if men stood in need of speech
to learn to think, they must have stood in still greater need of the
art of thinking to invent that of speaking; and though we could
conceive how the sounds of the voice came to be taken for the
conventional interpreters of our ideas we should not be the nearer
knowing who could have been the interpreters of this convention for
such ideas, as, in consequence of their not having any sensible
objects, could not be made manifest by gesture or voice; so that we
can scarce form any tolerable conjectures concerning the birth of this
art of communicating our thoughts, and establishing a correspondence
between minds: a sublime art which, though so remote from its origin,
philosophers still behold at such a prodigious distance from its
perfection, that I never met with one of them bold enough to affirm it
would ever arrive there, though the revolutions necessarily produced
by time were suspended in its favour; though prejudice could be
banished from, or would be at least content to sit silent in the
presence of our academies, and though these societies should
consecrate themselves, entirely and during whole ages, to the study of
this intricate object.

The first language of man, the most universal and most energetic of
all languages, in short, the only language he had occasion for, before
there was a necessity of persuading assembled multitudes, was the cry
of nature. As this cry was never extorted but by a kind of instinct in
the most urgent cases, to implore assistance in great danger, or
relief in great sufferings, it was of little use in the common
occurrences of life, where more moderate sentiments generally prevail.
When the ideas of men began to extend and multiply, and a closer
communication began to take place among them, they laboured to devise
more numerous signs, and a more extensive language: they multiplied
the inflections of the voice, and added to them gestures, which are,
in their own nature, more expressive, and whose meaning depends less
on any prior determination. They therefore expressed visible and
movable objects by gestures and those which strike the ear, by
imitative sounds: but as gestures scarcely indicate anything except
objects that are actually present or can be easily described, and
visible actions; as they are not of general use, since darkness or the
interposition of an opaque medium renders them useless; and as besides
they require attention rather than excite it: men at length bethought
themselves of substituting for them the articulations of voice, which,
without having the same relation to any determinate object, are, in
quality of instituted signs, fitter to represent all our ideas; a
substitution, which could only have been made by common consent, and
in a manner pretty difficult to practise by men, whose rude organs
were unimproved by exercise; a substitution, which is in itself more
difficult to be conceived, since the motives to this unanimous
agreement must have been somehow or another expressed, and speech
therefore appears to have been exceedingly requisite to establish the
use of speech.

We must allow that the words, first made use of by men, had in their
minds a much more extensive signification, than those employed in
languages of some standing, and that, considering how ignorant they
were of the division of speech into its constituent parts; they at
first gave every word the meaning of an entire proposition. When
afterwards they began to perceive the difference between the subject
and attribute, and between verb and noun, a distinction which required
no mean effort of genius, the substantives for a time were only so
many proper names, the infinitive was the only tense, and as to
adjectives, great difficulties must have attended the development of
the idea that represents them, since every adjective is an abstract
word, and abstraction is an unnatural and very painful operation.

At first they gave every object a peculiar name, without any regard to
its genus or species, things which these first institutors of language
were in no condition to distinguish; and every individual presented
itself solitary to their minds, as it stands in the table of nature.
If they called one oak A, they called another oak B: so that their
dictionary must have been more extensive in proportion as their
knowledge of things was more confined. It could not but be a very
difficult task to get rid of so diffuse and embarrassing a
nomenclature; as in order to marshal the several beings under common
and generic denominations, it was necessary to be first acquainted
with their properties, and their differences; to be stocked with
observations and definitions, that is to say, to understand natural
history and metaphysics, advantages which the men of these times could
not have enjoyed.

Besides, general ideas cannot be conveyed to the mind without the
assistance of words, nor can the understanding seize them without the
assistance of propositions. This is one of the reasons, why mere
animals cannot form such ideas, nor ever acquire the perfectibility
which depends on such an operation. When a monkey leaves without the
least hesitation one nut for another, are we to think he has any
general idea of that kind of fruit, and that he compares these two
individual bodies with his archetype notion of them? No, certainly;
but the sight of one of these nuts calls back to his memory the
sensations which he has received from the other; and his eyes,
modified after some certain manner, give notice to his palate of the
modification it is in its turn going to receive. Every general idea is
purely intellectual; let the imagination tamper ever so little with
it, it immediately becomes a particular idea. Endeavour to represent
to yourself the image of a tree in general, you never will be able to
do it; in spite of all your efforts it will appear big or little, thin
or tufted, of a bright or a deep colour; and were you master to see
nothing in it, but what can be seen in every tree, such a picture
would no longer resemble any tree. Beings perfectly abstract are
perceivable in the same manner, or are only conceivable by the
assistance of speech. The definition of a triangle can alone give you
a just idea of that figure: the moment you form a triangle in your
mind, it is this or that particular triangle and no other, and you
cannot avoid giving breadth to its lines and colour to its area. We
must therefore make use of propositions; we must therefore speak to
have general ideas; for the moment the imagination stops, the mind
must stop too, if not assisted by speech. If therefore the first
inventors could give no names to any ideas but those they had already,
it follows that the first substantives could never have been anything
more than proper names.

But when by means, which I cannot conceive, our new grammarians began
to extend their ideas, and generalize their words, the ignorance of
the inventors must have confined this method to very narrow bounds;
and as they had at first too much multiplied the names of individuals
for want of being acquainted with the distinctions called genus and
species, they afterwards made too few genera and species for want of
having considered beings in all their differences; to push the
divisions far enough, they must have had more knowledge and experience
than we can allow them, and have made more researches and taken more
pains, than we can suppose them willing to submit to. Now if, even at
this present time, we every day discover new species, which had before
escaped all our observations, how many species must have escaped the
notice of men, who judged of things merely from their first
appearances! As to the primitive classes and the most general notions,
it were superfluous to add that these they must have likewise
overlooked: how, for example, could they have thought of or understood
the words, matter, spirit, substance, mode, figure, motion, since even
our philosophers, who for so long a time have been constantly
employing these terms, can themselves scarcely understand them, and
since the ideas annexed to these words being purely metaphysical, no
models of them could be found in nature?

I stop at these first advances, and beseech my judges to suspend their
lecture a little, in order to consider, what a great way language has
still to go, in regard to the invention of physical substantives
alone, (though the easiest part of language to invent,) to be able to
express all the sentiments of man, to assume an invariable form, to
bear being spoken in public and to influence society: I earnestly
entreat them to consider how much time and knowledge must have been
requisite to find out numbers, abstract words, the aorists, and all
the other tenses of verbs, the particles, and syntax, the method of
connecting propositions and arguments, of forming all the logic of
discourse. For my own part, I am so scared at the difficulties that
multiply at every step, and so convinced of the almost demonstrated
impossibility of languages owing their birth and establishment to
means that were merely human, that I must leave to whoever may please
to take it up, the task of discussing this difficult problem. "Which
was the most necessary, society already formed to invent languages, or
languages already invented to form society?"

But be the case of these origins ever so mysterious, we may at least
infer from the little care which nature has taken to bring men
together by mutual wants, and make the use of speech easy to them, how
little she has done towards making them sociable, and how little she
has contributed to anything which they themselves have done to become
so. In fact, it is impossible to conceive, why, in this primitive
state, one man should have more occasion for the assistance of
another, than one monkey, or one wolf for that of another animal of
the same species; or supposing that he had, what motive could induce
another to assist him; or even, in this last case, how he, who wanted
assistance, and he from whom it was wanted, could agree among
themselves upon the conditions. Authors, I know, are continually
telling us, that in this state man would have been a most miserable
creature; and if it is true, as I fancy I have proved it, that he must
have continued many ages without either the desire or the opportunity
of emerging from such a state, this their assertion could only serve
to justify a charge against nature, and not any against the being
which nature had thus constituted; but, if I thoroughly understand
this term miserable, it is a word, that either has no meaning, or
signifies nothing, but a privation attended with pain, and a suffering
state of body or soul; now I would fain know what kind of misery can
be that of a free being, whose heart enjoys perfect peace, and body
perfect health? And which is aptest to become insupportable to those
who enjoy it, a civil or a natural life? In civil life we can scarcely
meet a single person who does not complain of his existence; many even
throw away as much of it as they can, and the united force of divine
and human laws can hardly put bounds to this disorder. Was ever any
free savage known to have been so much as tempted to complain of life,
and lay violent hands on himself? Let us therefore judge with less
pride on which side real misery is to be placed. Nothing, on the
contrary, must have been so unhappy as savage man, dazzled by flashes
of knowledge, racked by passions, and reasoning on a state different
from that in which he saw himself placed. It was in consequence of a
very wise Providence, that the faculties, which he potentially
enjoyed, were not to develop themselves but in proportion as there
offered occasions to exercise them, lest they should be superfluous or
troublesome to him when he did not want them, or tardy and useless
when he did. He had in his instinct alone everything requisite to live
in a state of nature; in his cultivated reason he has barely what is
necessary to live in a state of society.

It appears at first sight that, as there was no kind of moral
relations between men in this state, nor any known duties, they could
not be either good or bad, and had neither vices nor virtues, unless
we take these words in a physical sense, and call vices, in the
individual, the qualities which may prove detrimental to his own
preservation, and virtues those which may contribute to it; in which
case we should be obliged to consider him as most virtuous, who made
least resistance against the simple impulses of nature. But without
deviating from the usual meaning of these terms, it is proper to
suspend the judgment we might form of such a situation, and be upon
our guard against prejudice, till, the balance in hand, we have
examined whether there are more virtues or vices among civilized men;
or whether the improvement of their understanding is sufficient to
compensate the damage which they mutually do to each other, in
proportion as they become better informed of the services which they
ought to do; or whether, upon the whole, they would not be much
happier in a condition, where they had nothing to fear or to hope from
each other, than in that where they had submitted to an universal
subserviency, and have obliged themselves to depend for everything
upon the good will of those, who do not think themselves obliged to
give anything in return.

But above all things let us beware concluding with Hobbes, that man,
as having no idea of goodness, must be naturally bad; that he is
vicious because he does not know what virtue is; that he always
refuses to do any service to those of his own species, because he
believes that none is due to them; that, in virtue of that right which
he justly claims to everything he wants, he foolishly looks upon
himself as proprietor of the whole universe. Hobbes very plainly saw
the flaws in all the modern definitions of natural right: but the
consequences, which he draws from his own definition, show that it is,
in the sense he understands it, equally exceptionable. This author, to
argue from his own principles, should say that the state of nature,
being that where the care of our own preservation interferes least
with the preservation of others, was of course the most favourable to
peace, and most suitable to mankind; whereas he advances the very
reverse in consequence of his having injudiciously admitted, as
objects of that care which savage man should take of his preservation,
the satisfaction of numberless passions which are the work of society,
and have rendered laws necessary. A bad man, says he, is a robust
child. But this is not proving that savage man is a robust child; and
though we were to grant that he was, what could this philosopher infer
from such a concession? That if this man, when robust, depended on
others as much as when feeble, there is no excess that he would not be
guilty of. He would make nothing of striking his mother when she
delayed ever so little to give him the breast; he would claw, and
bite, and strangle without remorse the first of his younger brothers,
that ever so accidentally jostled or otherwise disturbed him. But
these are two contradictory suppositions in the state of nature, to be
robust and dependent. Man is weak when dependent, and his own master
before he grows robust. Hobbes did not consider that the same cause,
which hinders savages from making use of their reason, as our
jurisconsults pretend, hinders them at the same time from making an
ill use of their faculties, as he himself pretends; so that we may say
that savages are not bad, precisely because they don't know what it is
to be good; for it is neither the development of the understanding,
nor the curb of the law, but the calmness of their passions and their
ignorance of vice that hinders them from doing ill: \textit{tantus plus in
illis proficit vitiorum ignorantia, quam in his cognito virtutis}.
There is besides another principle that has escaped Hobbes, and which,
having been given to man to moderate, on certain occasions, the blind
and impetuous sallies of self-love, or the desire of self-preservation
previous to the appearance of that passion, allays the ardour, with
which he naturally pursues his private welfare, by an innate
abhorrence to see beings suffer that resemble him. I shall not surely
be contradicted, in granting to man the only natural virtue, which the
most passionate detractor of human virtues could not deny him, I mean
that of pity, a disposition suitable to creatures weak as we are, and
liable to so many evils; a virtue so much the more universal, and
withal useful to man, as it takes place in him of all manner of
reflection; and so natural, that the beasts themselves sometimes give
evident signs of it. Not to speak of the tenderness of mothers for
their young; and of the dangers they face to screen them from danger;
with what reluctance are horses known to trample upon living bodies;
one animal never passes unmoved by the dead carcass of another animal
of the same species: there are even some who bestow a kind of
sepulture upon their dead fellows; and the mournful lowings of cattle,
on their entering the slaughter-house, publish the impression made
upon them by the horrible spectacle they are there struck with. It is
with pleasure we see the author of the fable of the bees, forced to
acknowledge man a compassionate and sensible being; and lay aside, in
the example he offers to confirm it, his cold and subtle style, to
place before us the pathetic picture of a man, who, with his hands
tied up, is obliged to behold a beast of prey tear a child from the
arms of his mother, and then with his teeth grind the tender limbs,
and with his claws rend the throbbing entrails of the innocent victim.
What horrible emotions must not such a spectator experience at the
sight of an event which does not personally concern him? What anguish
must he not suffer at his not being able to assist the fainting mother
or the expiring infant?

Such is the pure motion of nature, anterior to all manner of
reflection; such is the force of natural pity, which the most
dissolute manners have as yet found it so difficult to extinguish,
since we every day see, in our theatrical representation, those men
sympathize with the unfortunate and weep at their sufferings, who, if
in the tyrant's place, would aggravate the torments of their enemies.
Mandeville was very sensible that men, in spite of all their morality,
would never have been better than monsters, if nature had not given
them pity to assist reason: but he did not perceive that from this
quality alone flow all the social virtues, which he would dispute
mankind the possession of. In fact, what is generosity, what clemency,
what humanity, but pity applied to the weak, to the guilty, or to the
human species in general? Even benevolence and friendship, if we judge
right, will appear the effects of a constant pity, fixed upon a
particular object: for to wish that a person may not suffer, what is
it but to wish that he may be happy? Though it were true that
commiseration is no more than a sentiment, which puts us in the place
of him who suffers, a sentiment obscure but active in the savage,
developed but dormant in civilized man, how could this notion affect
the truth of what I advance, but to make it more evident. In fact,
commiseration must be so much the more energetic, the more intimately
the animal, that beholds any kind of distress, identifies himself with
the animal that labours under it. Now it is evident that this
identification must have been infinitely more perfect in the state of
nature than in the state of reason. It is reason that engenders
self-love, and reflection that strengthens it; it is reason that makes
man shrink into himself; it is reason that makes him keep aloof from
everything that can trouble or afflict him: it is philosophy that
destroys his connections with other men; it is in consequence of her
dictates that he mutters to himself at the sight of another in
distress, You may perish for aught I care, nothing can hurt me.
Nothing less than those evils, which threaten the whole species, can
disturb the calm sleep of the philosopher, and force him from his bed.
One man may with impunity murder another under his windows; he has
nothing to do but clap his hands to his ears, argue a little with
himself to hinder nature, that startles within him, from identifying
him with the unhappy sufferer. Savage man wants this admirable talent;
and for want of wisdom and reason, is always ready foolishly to obey
the first whispers of humanity. In riots and street-brawls the
populace flock together, the prudent man sneaks off. They are the
dregs of the people, the poor basket and barrow-women, that part the
combatants, and hinder gentle folks from cutting one another's
throats.

It is therefore certain that pity is a natural sentiment, which, by
moderating in every individual the activity of self-love, contributes
to the mutual preservation of the whole species. It is this pity
which hurries us without reflection to the assistance of those we see
in distress; it is this pity which, in a state of nature, stands for
laws, for manners, for virtue, with this advantage, that no one is
tempted to disobey her sweet and gentle voice: it is this pity which
will always hinder a robust savage from plundering a feeble child, or
infirm old man, of the subsistence they have acquired with pain and
difficulty, if he has but the least prospect of providing for himself
by any other means: it is this pity which, instead of that sublime
maxim of argumentative justice, Do to others as you would have others
do to you, inspires all men with that other maxim of natural goodness
a great deal less perfect, but perhaps more useful, Consult your own
happiness with as little prejudice as you can to that of others. It is
in a word, in this natural sentiment, rather than in fine-spun
arguments, that we must look for the cause of that reluctance which
every man would experience to do evil, even independently of the
maxims of education. Though it may be the peculiar happiness of
Socrates and other geniuses of his stamp, to reason themselves into
virtue, the human species would long ago have ceased to exist, had it
depended entirely for its preservation on the reasonings of the
individuals that compose it.

With passions so tame, and so salutary a curb, men, rather wild than
wicked, and more attentive to guard against mischief than to do any to
other animals, were not exposed to any dangerous dissensions: As they
kept up no manner of correspondence with each other, and were of
course strangers to vanity, to respect, to esteem, to contempt; as
they had no notion of what we call Meum and Tuum, nor any true idea of
justice; as they considered any violence they were liable to, as an
evil that could be easily repaired, and not as an injury that deserved
punishment; and as they never so much as dreamed of revenge, unless
perhaps mechanically and unpremeditatedly, as a dog who bites the
stone that has been thrown at him; their disputes could seldom be
attended with bloodshed, were they never occasioned by a more
considerable stake than that of subsistence: but there is a more
dangerous subject of contention, which I must not leave unnoticed.

Among the passions which ruffle the heart of man, there is one of a
hot and impetuous nature, which renders the sexes necessary to each
other; a terrible passion which despises all dangers, bears down all
obstacles, and to which in its transports it seems proper to destroy
the human species which it is destined to preserve. What must become
of men abandoned to this lawless and brutal rage, without modesty,
without shame, and every day disputing the objects of their passion at
the expense of their blood?

We must in the first place allow that the more violent the passions,
the more necessary are laws to restrain them: but besides that the
disorders and the crimes, to which these passions daily give rise
among us, sufficiently grove the insufficiency of laws for that
purpose, we would do well to look back a little further and examine,
if these evils did not spring up with the laws themselves; for at this
rate, though the laws were capable of repressing these evils, it is
the least that might be expected from them, seeing it is no more than
stopping the progress of a mischief which they themselves have
produced.

Let us begin by distinguishing between what is moral and what is
physical in the passion called love. The physical part of it is that
general desire which prompts the sexes to unite with each other; the
moral part is that which determines that desire, and fixes it upon a
particular object to the exclusion of all others, or at least gives it
a greater degree of energy for this preferred object. Now it is easy
to perceive that the moral part of love is a factitious sentiment,
engendered by society, and cried up by the women with great care and
address in order to establish their empire, and secure command to that
sex which ought to obey. This sentiment, being founded on certain
notions of beauty and merit which a savage is not capable of having,
and upon comparisons which he is not capable of making, can scarcely
exist in him: for as his mind was never in a condition to form
abstract ideas of regularity and proportion, neither is his heart
susceptible of sentiments of admiration and love, which, even without
our perceiving it, are produced by our application of these ideas; he
listens solely to the dispositions implanted in him by nature, and not
to taste which he never was in a way of acquiring; and every woman
answers his purpose.

Confined entirely to what is physical in love, and happy enough not to
know these preferences which sharpen the appetite for it, at the same
time that they increase the difficulty of satisfying such appetite,
men, in a state of nature, must be subject to fewer and less violent
fits of that passion, and of course there must be fewer and less
violent disputes among them in consequence of it. The imagination
which causes so many ravages among us, never speaks to the heart of
savages, who peaceably wait for the impulses of nature, yield to these
impulses without choice and with more pleasure than fury; and whose
desires never outlive their necessity for the thing desired.

Nothing therefore can be more evident, than that it is society alone,
which has added even to love itself as well as to all the other
passions, that impetuous ardour, which so often renders it fatal to
mankind; and it is so much the more ridiculous to represent savages
constantly murdering each other to glut their brutality, as this
opinion is diametrically opposite to experience, and the Caribbeans,
the people in the world who have as yet deviated least from the state
of nature, are to all intents and purposes the most peaceable in their
amours, and the least subject to jealousy, though they live in a
burning climate which seems always to add considerably to the activity
of these passions.

As to the inductions which may be drawn, in respect to several species
of animals, from the battles of the males, who in all seasons cover
our poultry yards with blood, and in spring particularly cause our
forests to ring again with the noise they make in disputing their
females, we must begin by excluding all those species, where nature
has evidently established, in the relative power of the sexes,
relations different from those which exist among us: thus from the
battle of cocks we can form no induction that will affect the human
species. In the species, where the proportion is better observed,
these battles must be owing entirely to the fewness of the females
compared with the males, or, which is all one, to the exclusive
intervals, during which the females constantly refuse the addresses of
the males; for if the female admits the male but two months in the
year, it is all the same as if the number of females were five-sixths
less than what it is: now neither of these cases is applicable to the
human species, where the number of females generally surpasses that of
males, and where it has never been observed that, even among savages,
the females had, like those of other animals, stated times of passion
and indifference, Besides, among several of these animals the whole
species takes fire all at once, and for some days nothing is, to be
seen among them but confusion, tumult, disorder and bloodshed; a state
unknown to the human species where love is never periodical. We can
not therefore conclude from the battles of certain animals for the
possession of their females, that the same would be the case of man in
a state of nature; and though we might, as these contests do not
destroy the other species, there is at least equal room to think they
would not be fatal to ours; nay it is very probable that they would
cause fewer ravages than they do in society, especially in those
countries where, morality being as yet held in some esteem, the
jealousy of lovers, and the vengeance of husbands every day produce
duels, murders and even worse crimes; where the duty of an eternal
fidelity serves only to propagate adultery; and the very laws of
continence and honour necessarily contribute to increase
dissoluteness, and multiply abortions.

Let us conclude that savage man, wandering about in the forests,
without industry, without speech, without any fixed residence, an
equal stranger to war and every social connection, without standing in
any shape in need of his fellows, as well as without any desire of
hurting them, and perhaps even without ever distinguishing them
individually one from the other, subject to few passions, and finding
in himself all he wants, let us, I say, conclude that savage man thus
circumstanced had no knowledge or sentiment but such as are proper to
that condition, that he was alone sensible of his real necessities,
took notice of nothing but what it was his interest to see, and that
his understanding made as little progress as his vanity. If he
happened to make any discovery, he could the less communicate it as he
did not even know his children. The art perished with the inventor;
there was neither education nor improvement; generations succeeded
generations to no purpose; and as all constantly set out from the same
point, whole centuries rolled on in the rudeness and barbarity of the
first age; the species was grown old, while the individual still
remained in a state of childhood.

If I have enlarged so much upon the supposition of this primitive
condition, it is because I thought it my duty, considering what
ancient errors and inveterate prejudices I have to extirpate, to dig
to the very roots, and show in a true picture of the state of nature,
how much even natural inequality falls short in this state of that
reality and influence which our writers ascribe to it.

In fact, we may easily perceive that among the differences, which
distinguish men, several pass for natural, which are merely the work
of habit and the different kinds of life adopted by men living in a
social way. Thus a robust or delicate constitution, and the strength
and weakness which depend on it, are oftener produced by the hardy or
effeminate manner in which a man has been brought up, than by the
primitive constitution of his body. It is the same thus in regard to
the forces of the mind; and education not only produces a difference
between those minds which are cultivated and those which are not, but
even increases that which is found among the first in proportion to
their culture; for let a giant and a dwarf set out in the same path,
the giant at every step will acquire a new advantage over the dwarf.
Now, if we compare the prodigious variety in the education and manner
of living of the different orders of men in a civil state, with the
simplicity and uniformity that prevails in the animal and savage life,
where all the individuals make use of the same aliments, live in the
same manner, and do exactly the same things, we shall easily conceive
how much the difference between man and man in the state of nature
must be less than in the state of society, and how much every
inequality of institution must increase the natural inequalities of
the human species.

But though nature in the distribution of her gifts should really
affect all the preferences that are ascribed to her, what advantage
could the most favoured derive from her partiality, to the prejudice
of others, in a state of things, which scarce admitted any kind of
relation between her pupils? Of what service can beauty be, where
there is no love? What will wit avail people who don't speak, or craft
those who have no affairs to transact? Authors are constantly crying
out, that the strongest would oppress the weakest; but let them
explain what they mean by the word oppression. One man will rule with
violence, another will groan under a constant subjection to all his
caprices: this is indeed precisely what I observe among us, but I
don't see how it can be said of savage men, into whose heads it would
be a harder matter to drive even the meaning of the words domination
and servitude. One man might, indeed, seize on the fruits which
another had gathered, on the game which another had killed, on the
cavern which another had occupied for shelter; but how is it possible
he should ever exact obedience from him, and what chains of dependence
can there be among men who possess nothing? If I am driven from one
tree, I have nothing to do but look out for another; if one place is
made uneasy to me, what can hinder me from taking up my quarters
elsewhere? But suppose I should meet a man so much superior to me in
strength, and withal so wicked, so lazy and so barbarous as to oblige
me to provide for his subsistence while he remains idle; he must
resolve not to take his eyes from me a single moment, to bind me fast
before he can take the least nap, lest I should kill him or give him
the slip during his sleep: that is to say, he must expose himself
voluntarily to much greater troubles than what he seeks to avoid, than
any he gives me. And after all, let him abate ever so little of his
vigilance; let him at some sudden noise but turn his head another way;
I am already buried in the forest, my fetters are broke, and he never
sees me again.

But without insisting any longer upon these details, every one must
see that, as the bonds of servitude are formed merely by the mutual
dependence of men one upon another and the reciprocal necessities
which unite them, it is impossible for one man to enslave another,
without having first reduced him to a condition in which he can not
live without the enslaver's assistance; a condition which, as it does
not exist in a state of nature, must leave every man his own master,
and render the law of the strongest altogether vain and useless.

Having proved that the inequality, which may subsist between man and
man in a state of nature, is almost imperceivable, and that it has
very little influence, I must now proceed to show its origin, and
trace its progress, in the successive developments of the human mind.
After having showed, that perfectibility, the social virtues, and the
other faculties, which natural man had received \textit{in potentia}, could
never be developed of themselves, that for that purpose there was a
necessity for the fortuitous concurrence of several foreign causes,
which might never happen, and without which he must have eternally
remained in his primitive condition; I must proceed to consider and
bring together the different accidents which may have perfected the
human understanding by debasing the species, render a being wicked by
rendering him sociable, and from so remote a term bring man at last
and the world to the point in which we now see them.

I must own that, as the events I am about to describe might have
happened many different ways, my choice of these I shall assign can be
grounded on nothing but mere conjecture; but besides these conjectures
becoming reasons, when they are not only the most probable that can be
drawn from the nature of things, but the only means we can have of
discovering truth, the consequences I mean to deduce from mine will
not be merely conjectural, since, on the principles I have just
established, it is impossible to form any other system, that would not
supply me with the same results, and from which I might not draw the
same conclusions.

This will authorize me to be the more concise in my reflections on the
manner, in which the lapse of time makes amends for the little
verisimilitude of events; on the surprising power of very trivial
causes, when they act without intermission; on the impossibility there
is on the one hand of destroying certain Hypotheses, if on the other
we can not give them the degree of certainty which facts must be
allowed to possess; on its being the business of history, when two
facts are proposed, as real, to be connected by a chain of
intermediate facts which are either unknown or considered as such, to
furnish such facts as may actually connect them; and the business of
philosophy, when history is silent, to point out similar facts which
may answer the same purpose; in fine on the privilege of similitude,
in regard to events, to reduce facts to a much smaller number of
different classes than is generally imagined. It suffices me to offer
these objects to the consideration of my judges; it suffices me to
have conducted my inquiry in such a manner as to save common readers
the trouble of considering them.




SECOND PART

The first man, who, after enclosing a piece of ground, took it into
his head to say, "This is mine," and found people simple enough to
believe him, was the true founder of civil society. How many crimes,
how many wars, how many murders, how many misfortunes and horrors,
would that man have saved the human species, who pulling up the stakes
or filling up the ditches should have cried to his fellows: Be sure
not to listen to this imposter; you are lost, if you forget that the
fruits of the earth belong equally to us all, and the earth itself to
nobody! But it is highly probable that things were now come to such a
pass, that they could not continue much longer in the same way; for as
this idea of property depends on several prior ideas which could only
spring up gradually one after another, it was not formed all at once
in the human mind: men must have made great progress; they must have
acquired a great stock of industry and knowledge, and transmitted and
increased it from age to age before they could arrive at this last
term of the state of nature. Let us therefore take up things a little
higher, and collect into one point of view, and in their most natural
order, this slow succession of events and mental improvements.

The first sentiment of man was that of his existence, his first care
that of preserving it. The productions of the earth yielded him all
the assistance he required; instinct prompted him to make use of them.
Among the various appetites, which made him at different times
experience different modes of existence, there was one that excited
him to perpetuate his species; and this blind propensity, quite void
of anything like pure love or affection, produced nothing but an act
that was merely animal. The present heat once allayed, the sexes took
no further notice of each other, and even the child ceased to have any
tie in his mother, the moment he ceased to want her assistance.

Such was the condition of infant man; such was the life of an animal
confined at first to pure sensations, and so far from harbouring any
thought of forcing her gifts from nature, that he scarcely availed
himself of those which she offered to him of her own accord. But
difficulties soon arose, and there was a necessity for learning how to
surmount them: the height of some trees, which prevented his reaching
their fruits; the competition of other animals equally fond of the
same fruits; the fierceness of many that even aimed at his life; these
were so many circumstances, which obliged him to apply to bodily
exercise. There was a necessity for becoming active, swift-footed, and
sturdy in battle. The natural arms, which are stones and the branches
of trees, soon offered themselves to his assistance. He learned to
surmount the obstacles of nature, to contend in case of necessity with
other animals, to dispute his subsistence even with other men, or
indemnify himself for the loss of whatever he found himself obliged to
part with to the strongest.

In proportion as the human species grew more numerous, and extended
itself, its pains likewise multiplied and increased. The difference
of soils, climates and seasons, might have forced men to observe some
difference in their way of living. Bad harvests, long and severe
winters, and scorching summers which parched up all the fruits of the
earth, required extraordinary exertions of industry. On the sea shore,
and the banks of rivers, they invented the line and the hook, and
became fishermen and ichthyophagous. In the forests they made
themselves bows and arrows, and became huntsmen and warriors. In the
cold countries they covered themselves with the skins of the beasts
they had killed; thunder, a volcano, or some happy accident made them
acquainted with fire, a new resource against the rigours of winter:
they discovered the method of preserving this element, then that of
reproducing it, and lastly the way of preparing with it the flesh of
animals, which heretofore they devoured raw from the carcass.

This reiterated application of various beings to himself, and to one
another, must have naturally engendered in the mind of man the idea of
certain relations. These relations, which we express by the words,
great, little, strong, weak, swift, slow, fearful, bold, and the like,
compared occasionally, and almost without thinking of it, produced in
him some kind of reflection, or rather a mechanical prudence, which
pointed out to him the precautions most essential to his preservation
and safety.

The new lights resulting from this development increased his
superiority over other animals, by making him sensible of it. He laid
himself out to ensnare them; he played them a thousand tricks; and
though several surpassed him in strength or in swiftness, he in time
became the master of those that could be of any service to him, and a
sore enemy to those that could do him any mischief. 'Tis thus, that
the first look he gave into himself produced the first emotion of
pride in him; 'tis thus that, at a time he scarce knew how to
distinguish between the different ranks of existence, by attributing
to his species the first rank among animals in general, he prepared
himself at a distance to pretend to it as an individual among those of
his own species in particular.

Though other men were not to him what they are to us, and he had
scarce more intercourse with them than with other animals, they were
not overlooked in his observations. The conformities, which in time
he might discover between them, and between himself and his female,
made him judge of those he did not perceive; and seeing that they all
behaved as himself would have done in similar circumstances, he
concluded that their manner of thinking and willing was quite
conformable to his own; and this important truth, when once engraved
deeply on his mind, made him follow, by a presentiment as sure as any
logic, and withal much quicker, the best rules of conduct, which for
the sake of his own safety and advantage it was proper he should
observe towards them.

Instructed by experience that the love of happiness is the sole
principle of all human actions, he found himself in a condition to
distinguish the few cases, in which common interest might authorize
him to build upon the assistance of his fellows, and those still
fewer, in which a competition of interests might justly render it
suspected. In the first case he united with them in the same flock, or
at most by some kind of free association which obliged none of its
members, and lasted no longer than the transitory necessity that had
given birth to it. In the second case every one aimed at his own
private advantage, either by open force if he found himself strong
enough, or by cunning and address if he thought himself too weak to
use violence.

Such was the manner in which men might have insensibly acquired some
gross idea of their mutual engagements and the advantage of fulfilling
them, but this only as far as their present and sensible interest
required; for as to foresight they were utter strangers to it, and far
from troubling their heads about a distant futurity, they scarce
thought of the day following. Was a deer to be taken? Every one saw
that to succeed he must faithfully stand to his post; but suppose a
hare to have slipped by within reach of any one of them, it is not to
be doubted but he pursued it without scruple, and when he had seized
his prey never reproached himself with having made his companions miss
theirs.

We may easily conceive that such an intercourse scarce required a more
refined language than that of crows and monkeys, which flock together
almost in the same manner. Inarticulate exclamations, a great many
gestures, and some imitative sounds, must have been for a long time
the universal language of mankind, and by joining to these in every
country some articulate and conventional sounds, of which, as I have
already hinted, it is not very easy to explain the institution, there
arose particular languages, but rude, imperfect, and such nearly as
are to be found at this day among several savage nations. My pen
straightened by the rapidity of time, the abundance of things I have
to say, and the almost insensible progress of the first improvements,
flies like an arrow over numberless ages, for the slower the
succession of events, the quicker I may allow myself to be in relating
them.

At length, these first improvements enabled man to improve at a
greater rate. Industry grew perfect in proportion as the mind became
more enlightened. Men soon ceasing to fall asleep under the first
tree, or take shelter in the first cavern, lit upon some hard and
sharp kinds of stone resembling spades or hatchets, and employed them
to dig the ground, cut down trees, and with the branches build huts,
which they afterwards bethought themselves of plastering over with
clay or dirt. This was the epoch of a first revolution, which produced
the establishment and distinction of families, and which introduced a
species of property, and along with it perhaps a thousand quarrels and
battles. As the strongest however were probably the first to make
themselves cabins, which they knew they were able to defend, we may
conclude that the weak found it much shorter and safer to imitate than
to attempt to dislodge them: and as to those, who were already
provided with cabins, no one could have any great temptation to seize
upon that of his neighbour, not so much because it did not belong to
him, as because it could be of no service to him; and as besides to
make himself master of it, he must expose himself to a very sharp
conflict with the present occupiers.

The first developments of the heart were the effects of a new
situation, which united husbands and wives, parents and children,
under one roof; the habit of living together gave birth to the
sweetest sentiments the human species is acquainted with, conjugal and
paternal love. Every family became a little society, so much the more
firmly united, as a mutual attachment and liberty were the only bonds
of it; and it was now that the sexes, whose way of life had been
hitherto the same, began to adopt different manners and customs. The
women became more sedentary, and accustomed themselves to stay at home
and look after the children, while the men rambled abroad in quest of
subsistence for the whole family. The two sexes likewise by living a
little more at their ease began to lose somewhat of their usual
ferocity and sturdiness; but if on the one hand individuals became
less able to engage separately with wild beasts, they on the other
were more easily got together to make a common resistance against
them.

In this new state of things, the simplicity and solitariness of man's
life, the limitedness of his wants, and the instruments which he had
invented to satisfy them, leaving him a great deal of leisure, he
employed it to supply himself with several conveniences unknown to his
ancestors; and this was the first yoke he inadvertently imposed upon
himself, and the first source of mischief which he prepared for his
children; for besides continuing in this manner to soften both body
and mind, these conveniences having through use lost almost all their
aptness to please, and even degenerated into real wants, the privation
of them became far more intolerable than the possession of them had
been agreeable; to lose them was a misfortune, to possess them no
happiness.

Here we may a little better discover how the use of speech insensibly
commences or improves in the bosom of every family, and may likewise
from conjectures concerning the manner in which divers particular
causes might have propagated language, and accelerated its progress by
rendering it every day more and more necessary. Great inundations or
earthquakes surrounded inhabited districts with water or precipices,
portions of the continent were by revolutions of the globe torn off
and split into islands. It is obvious that among men thus collected,
and forced to live together, a common idiom must have started up much
sooner, than among those who freely wandered through the forests of
the main land. Thus it is very possible that the inhabitants of the
islands formed in this manner, after their first essays in navigation,
brought among us the use of speech; and it is very probable at least
that society and languages commenced in islands and even acquired
perfection there, before the inhabitants of the continent knew
anything of either.

Everything now begins to wear a new aspect. Those who heretofore
wandered through the woods, by taking to a more settled way of life,
gradually flock together, coalesce into several separate bodies, and
at length form in every country distinct nations, united in character
and manners, not by any laws or regulations, but by an uniform manner
of life, a sameness of provisions, and the common influence of the
climate. A permanent neighborhood must at last infallibly create some
connection between different families. The transitory commerce
required by nature soon produced, among the youth of both sexes living
in contiguous cabins, another kind of commerce, which besides being
equally agreeable is rendered more durable by mutual intercourse. Men
begin to consider different objects, and to make comparisons; they
insensibly acquire ideas of merit and beauty, and these soon produce
sentiments of preference. By seeing each other often they contract a
habit, which makes it painful not to see each other always. Tender and
agreeable sentiments steal into the soul, and are by the smallest
opposition wound up into the most impetuous fury: Jealousy kindles
with love; discord triumphs; and the gentlest of passions requires
sacrifices of human blood to appease it.

In proportion as ideas and sentiments succeed each other, and the head
and the heart exercise themselves, men continue to shake off their
original wildness, and their connections become more intimate and
extensive. They now begin to assemble round a great tree: singing and
dancing, the genuine offspring of love and leisure, become the
amusement or rather the occupation of the men and women, free from
care, thus gathered together. Every one begins to survey the rest, and
wishes to be surveyed himself; and public esteem acquires a value. He
who sings or dances best; the handsomest, the strongest, the most
dexterous, the most eloquent, comes to be the most respected: this was
the first step towards inequality, and at the same time towards vice.
From these first preferences there proceeded on one side vanity and
contempt, on the other envy and shame; and the fermentation raised by
these new leavens at length produced combinations fatal to happiness
and innocence.

Men no sooner began to set a value upon each other, and know what
esteem was, than each laid claim to it, and it was no longer safe for
any man to refuse it to another. Hence the first duties of civility
and politeness, even among savages; and hence every voluntary injury
became an affront, as besides the mischief, which resulted from it as
an injury, the party offended was sure to find in it a contempt for
his person more intolerable than the mischief itself. It was thus that
every man, punishing the contempt expressed for him by others in
proportion to the value he set upon himself, the effects of revenge
became terrible, and men learned to be sanguinary and cruel. Such
precisely was the degree attained by most of the savage nations with
whom we are acquainted. And it is for want of sufficiently
distinguishing ideas, and observing at how great a distance these
people were from the first state of nature, that so many authors have
hastily concluded that man is naturally cruel, and requires a regular
system of police to be reclaimed; whereas nothing can be more gentle
than he in his primitive state, when placed by nature at an equal
distance from the stupidity of brutes, and the pernicious good sense
of civilized man; and equally confined by instinct and reason to the
care of providing against the mischief which threatens him, he is
withheld by natural compassion from doing any injury to others, so far
from being ever so little prone even to return that which he has
received. For according to the axiom of the wise Locke, Where there is
no property, there can be no injury.

But we must take notice, that the society now formed and the relations
now established among men required in them qualities different from
those, which they derived from their primitive constitution; that as a
sense of morality began to insinuate itself into human actions, and
every man, before the enacting of laws, was the only judge and avenger
of the injuries he had received, that goodness of heart suitable to
the pure state of nature by no means suited infant society; that it
was necessary punishments should become severer in the same proportion
that the opportunities of offending became more frequent, and the
dread of vengeance add strength to the too weak curb of the law. Thus,
though men were become less patient, and natural compassion had
already suffered some alteration, this period of the development of
the human faculties, holding a just mean between the indolence of the
primitive state, and the petulant activity of self-love, must have
been the happiest and most durable epoch. The more we reflect on this
state, the more convinced we shall be, that it was the least subject
of any to revolutions, the best for man, and that nothing could have
drawn him out of it but some fatal accident, which, for the public
good, should never have happened. The example of the savages, most of
whom have been found in this condition, seems to confirm that mankind
was formed ever to remain in it, that this condition is the real youth
of the world, and that all ulterior improvements have been so many
steps, in appearance towards the perfection of individuals, but in
fact towards the decrepitness of the species.

As long as men remained satisfied with their rustic cabins; as long as
they confined themselves to the use of clothes made of the skins of
other animals, and the use of thorns and fish-bones, in putting these
skins together; as long as they continued to consider feathers and
shells as sufficient ornaments, and to paint their bodies of different
colours, to improve or ornament their bows and arrows, to form and
scoop out with sharp-edged stones some little fishing boats, or clumsy
instruments of music; in a word, as long as they undertook such works
only as a single person could finish, and stuck to such arts as did
not require the joint endeavours of several hands, they lived free,
healthy, honest and happy, as much as their nature would admit, and
continued to enjoy with each other all the pleasures of an independent
intercourse; but from the moment one man began to stand in need of
another's assistance; from the moment it appeared an advantage for one
man to possess the quantity of provisions requisite for two, all
equality vanished; property started up; labour became necessary; and
boundless forests became smiling fields, which it was found necessary
to water with human sweat, and in which slavery and misery were soon
seen to sprout out and grow with the fruits of the earth.

Metallurgy and agriculture were the two arts whose invention produced
this great revolution. With the poet, it is gold and silver, but with
the philosopher it is iron and corn, which have civilized men, and
ruined mankind. Accordingly both one and the other were unknown to the
savages of America, who for that very reason have always continued
savages; nay other nations seem to have continued in a state of
barbarism, as long as they continued to exercise one only of these
arts without the other; and perhaps one of the best reasons that can
be assigned, why Europe has been, if not earlier, at least more
constantly and better civilized than the other quarters of the world,
is that she both abounds most in iron and is best qualified to
produce corn.

It is a very difficult matter to tell how men came to know anything of
iron, and the art of employing it: for we are not to suppose that they
should of themselves think of digging it out of the mines, and
preparing it for fusion, before they knew what could be the result of
such a process. On the other hand, there is the less reason to
attribute this discovery to any accidental fire, as mines are formed
nowhere but in dry and barren places, and such as are bare of trees
and plants, so that it looks as if nature had taken pains to keep from
us so mischievous a secret. Nothing therefore remains but the
extraordinary circumstance of some volcano, which, belching forth
metallic substances ready fused, might have given the spectators a
notion of imitating that operation of nature; and after all we must
suppose them endued with an extraordinary stock of courage and
foresight to undertake so painful a work, and have, at so great a
distance, an eye to the advantages they might derive from it;
qualities scarcely suitable but to heads more exercised, than those of
such discoverers can be supposed to have been.

As to agriculture, the principles of it were known a long time before
the practice of it took place, and it is hardly possible that men,
constantly employed in drawing their subsistence from trees and
plants, should not have early hit on the means employed by nature for
the generation of vegetables; but in all probability it was very late
before their industry took a turn that way, either because trees,
which with their land and water game supplied them with sufficient
food, did not require their attention; or because they did not know
the use of corn; or because they had no instruments to cultivate it;
or because they were destitute of foresight in regard to future
necessities; or in fine, because they wanted means to hinder others
from running away with the fruit of their labours. We may believe that
on their becoming more industrious they began their agriculture by
cultivating with sharp stones and pointed sticks a few pulse or roots
about their cabins; and that it was a long time before they knew the
method of preparing corn, and were provided with instruments necessary
to raise it in large quantities; not to mention the necessity there
is, in order to follow this occupation and sow lands, to consent to
lose something at present to gain a great deal hereafter; a precaution
very foreign to the turn of man's mind in a savage state, in which, as
I have already taken notice, he can hardly foresee his wants from
morning to night.

For this reason the invention of other arts must have been necessary
to oblige mankind to apply to that of agriculture. As soon as men
were wanted to fuse and forge iron, others were wanted to maintain
them. The more hands were employed in manufactures, the fewer hands
were left to provide subsistence for all, though the number of mouths
to be supplied with food continued the same; and as some required
commodities in exchange for their iron, the rest at last found out the
method of making iron subservient to the multiplication of
commodities. Hence on the one hand husbandry and agriculture, and on
the other the art of working metals and of multiplying the uses of
them.

To the tilling of the earth the distribution of it necessarily
succeeded, and to property once acknowledged, the first rules of
justice: for to secure every man his own, every man must have
something. Moreover, as men began to extend their views to futurity,
and all found themselves in possession of more or less goods capable
of being lost, every one in particular had reason to fear, lest
reprisals should be made on him for any injury he might do to others.
This origin is so much the more natural, as it is impossible to
conceive how property can flow from any other source but industry; for
what can a man add but his labour to things which he has not made, in
order to acquire a property in them? 'Tis the labour of the hands
alone, which giving the husbandman a title to the produce of the land
he has tilled gives him a title to the land itself, at least till he
has gathered in the fruits of it, and so on from year to year; and
this enjoyment forming a continued possession is easily transformed
into a property. The ancients, says Grotius, by giving to Ceres the
epithet of Legislatrix, and to a festival celebrated in her honour the
name of Thesmorphoria, insinuated that the distribution of lands
produced a new kind of right; that is, the right of property different
from that which results from the law of nature.

Things thus circumstanced might have remained equal, if men's talents
had been equal, and if, for instance, the use of iron, and the
consumption of commodities had always held an exact proportion to each
other; but as this proportion had no support, it was soon broken. The
man that had most strength performed most labour; the most dexterous
turned his labour to best account; the most ingenious found out
methods of lessening his labour; the husbandman required more iron, or
the smith more corn, and while both worked equally, one earned a great
deal by his labour, while the other could scarce live by his. It is
thus that natural inequality insensibly unfolds itself with that
arising from a variety of combinations, and that the difference among
men, developed by the difference of their circumstances, becomes more
sensible, more permanent in its effects, and begins to influence in
the same proportion the condition of private persons.

Things once arrived at this period, it is an easy matter to imagine
the rest. I shall not stop to describe the successive inventions of
other arts, the progress of language, the trial and employments of
talents, the inequality of fortunes, the use or abuse of riches, nor
all the details which follow these, and which every one may easily
supply. I shall just give a glance at mankind placed in this new order
of things.

Behold then all our faculties developed; our memory and imagination at
work, self-love interested; reason rendered active; and the mind
almost arrived at the utmost bounds of that perfection it is capable
of. Behold all our natural qualities put in motion; the rank and
condition of every man established, not only as to the quantum of
property and the power of serving or hurting others, but likewise as
to genius, beauty, strength or address, merit or talents; and as these
were the only qualities which could command respect, it was found
necessary to have or at least to affect them. It was requisite for men
to be thought what they really were not. To be and to appear became
two very different things, and from this distinction sprang pomp and
knavery, and all the vices which form their train. On the other hand,
man, heretofore free and independent, was now in consequence of a
multitude of new wants brought under subjection, as it were, to all
nature, and especially to his fellows, whose slave in some sense he
became even by becoming their master; if rich, he stood in need of
their services, if poor, of their assistance; even mediocrity itself
could not enable him to do without them. He must therefore have been
continually at work to interest them in his happiness, and make them,
if not really, at least apparently find their advantage in labouring
for his: this rendered him sly and artful in his dealings with some,
imperious and cruel in his dealings with others, and laid him under
the necessity of using ill all those whom he stood in need of, as
often as he could not awe them into a compliance with his will, and
did not find it his interest to purchase it at the expense of real
services. In fine, an insatiable ambition, the rage of raising their
relative fortunes, not so much through real necessity, as to over-top
others, inspire all men with a wicked inclination to injure each
other, and with a secret jealousy so much the more dangerous, as to
carry its point with the greater security, it often puts on the face
of benevolence. In a word, sometimes nothing was to be seen but a
contention of endeavours on the one hand, and an opposition of
interests on the other, while a secret desire of thriving at the
expense of others constantly prevailed. Such were the first effects of
property, and the inseparable attendants of infant inequality.

Riches, before the invention of signs to represent them, could scarce
consist in anything but lands and cattle, the only real goods which
men can possess. But when estates increased so much in number and in
extent as to take in whole countries and touch each other, it became
impossible for one man to aggrandise himself but at the expense of
some other; and the supernumerary inhabitants, who were too weak or
too indolent to make such acquisitions in their turn, impoverished
without losing anything, because while everything about them changed
they alone remained the same, were obliged to receive or force their
subsistence from the hands of the rich. And hence began to flow,
according to the different characters of each, domination and slavery,
or violence and rapine. The rich on their side scarce began to taste
the pleasure of commanding, when they preferred it to every other; and
making use of their old slaves to acquire new ones, they no longer
thought of anything but subduing and enslaving their neighbours; like
those ravenous wolves, who having once tasted human flesh, despise
every other food, and devour nothing but men for the future.

It is thus that the most powerful or the most wretched, respectively
considering their power and wretchedness as a kind of title to the
substance of others, even equivalent to that of property, the equality
once broken was followed by the most shocking disorders. It is thus
that the usurpations of the rich, the pillagings of the poor, and the
unbridled passions of all, by stifling the cries of natural
compassion, and the as yet feeble voice of justice, rendered man
avaricious, wicked and ambitious. There arose between the title of the
strongest, and that of the first occupier a perpetual conflict, which
always ended in battery and bloodshed. Infant society became a scene
of the most horrible warfare: Mankind thus debased and harassed, and
no longer able to retreat, or renounce the unhappy acquisitions it had
made; labouring, in short merely to its confusion by the abuse of
those faculties, which in themselves do it so much honour, brought
itself to the very brink of ruin and destruction.

\begin{quotation}
\textlatin{%
  Attonitus novitate mali, divesque miserque,
  Effugere optat opes; et quoe modo voverat, odit.
}
\end{quotation}

But it is impossible that men should not sooner or later have made
reflections on so wretched a situation, and upon the calamities with
which they were overwhelmed. The rich in particular must have soon
perceived how much they suffered by a perpetual war, of which they
alone supported all the expense, and in which, though all risked life,
they alone risked any substance. Besides, whatever colour they might
pretend to give their usurpations, they sufficiently saw that these
usurpations were in the main founded upon false and precarious titles,
and that what they had acquired by mere force, others could again by
mere force wrest out of their hands, without leaving them the least
room to complain of such a proceeding. Even those, who owed all their
riches to their own industry, could scarce ground their acquisitions
upon a better title. It availed them nothing to say, 'Twas I built
this wall; I acquired this spot by my labour. Who traced it out for
you, another might object, and what right have you to expect payment
at our expense for doing that we did not oblige you to do? Don't you
know that numbers of your brethren perish, or suffer grievously for
want of what you possess more than suffices nature, and that you
should have had the express and unanimous consent of mankind to
appropriate to yourself of their common, more than was requisite for
your private subsistence? Destitute of solid reasons to justify, and
sufficient force to defend himself; crushing individuals with ease,
but with equal ease crushed by numbers; one against all, and unable,
on account of mutual jealousies, to unite with his equals against
banditti united by the common hopes of pillage; the rich man, thus
pressed by necessity, at last conceived the deepest project that ever
entered the human mind: this was to employ in his favour the very
forces that attacked him, to make allies of his enemies, to inspire
them with other maxims, and make them adopt other institutions as
favourable to his pretensions, as the law of nature was unfavourable
to them.

With this view, after laying before his neighbours all the horrors of
a situation, which armed them all one against another, which rendered
their possessions as burdensome as their wants were intolerable, and
in which no one could expect any safety either in poverty or riches,
he easily invented specious arguments to bring them over to his
purpose. "Let us unite," said he, "to secure the weak from
oppression, restrain the ambitious, and secure to every man the
possession of what belongs to him: Let us form rules of justice and
peace, to which all may be obliged to conform, which shall not except
persons, but may in some sort make amends for the caprice of fortune,
by submitting alike the powerful and the weak to the observance of
mutual duties. In a word, instead of turning our forces against
ourselves, let us collect them into a sovereign power, which may
govern us by wise laws, may protect and defend all the members of the
association, repel common enemies, and maintain a perpetual concord
and harmony among us."

Much fewer words of this kind were sufficient to draw in a parcel of
rustics, whom it was an easy matter to impose upon, who had besides
too many quarrels among themselves to live without arbiters, and too
much avarice and ambition to live long without masters. All offered
their necks to the yoke in hopes of securing their liberty; for though
they had sense enough to perceive the advantages of a political
constitution, they had not experience enough to see beforehand the
dangers of it; those among them, who were best qualified to foresee
abuses, were precisely those who expected to benefit by them; even the
soberest judged it requisite to sacrifice one part of their liberty to
ensure the other, as a man, dangerously wounded in any of his limbs,
readily parts with it to save the rest of his body.

Such was, or must have been, had man been left to himself, the origin
of society and of the laws, which increased the fetters of the weak,
and the strength of the rich; irretrievably destroyed natural liberty,
fixed for ever the laws of property and inequality; changed an artful
usurpation into an irrevocable title; and for the benefit of a few
ambitious individuals subjected the rest of mankind to perpetual
labour, servitude, and misery. We may easily conceive how the
establishment of a single society rendered that of all the rest
absolutely necessary, and how, to make head against united forces, it
became necessary for the rest of mankind to unite in their turn.
Societies once formed in this manner, soon multiplied or spread to
such a degree, as to cover the face of the earth; and not to leave a
corner in the whole universe, where a man could throw off the yoke,
and withdraw his head from under the often ill-conducted sword which
he saw perpetually hanging over it. The civil law being thus become
the common rule of citizens, the law of nature no longer obtained but
among the different societies, in which, under the name of the law of
nations, it was qualified by some tacit conventions to render commerce
possible, and supply the place of natural compassion, which, losing by
degrees all that influence over societies which it originally had over
individuals, no longer exists but in some great souls, who consider
themselves as citizens of the world, and forcing the imaginary
barriers that separate people from people, after the example of the
Sovereign Being from whom we all derive our existence, make the whole
human race the object of their benevolence.

Political bodies, thus remaining in a state of nature among
themselves, soon experienced the inconveniences which had obliged
individuals to quit it; and this state became much more fatal to these
great bodies, than it had been before to the individuals which now
composed them. Hence those national wars, those battles, those
murders, those reprisals, which make nature shudder and shock reason;
hence all those horrible prejudices, which make it a virtue and an
honour to shed human blood. The worthiest men learned to consider the
cutting the throats of their fellows as a duty; at length men began to
butcher each other by thousands without knowing for what; and more
murders were committed in a single action, and more horrible disorders
at the taking of a single town, than had been committed in the state
of nature during ages together upon the whole face of the earth. Such
are the first effects we may conceive to have arisen from the division
of mankind into different societies. Let us return to their
institution.

I know that several writers have assigned other origins of political
society; as for instance, the conquests of the powerful, or the union
of the weak; and it is no matter which of these causes we adopt in
regard to what I am going to establish; that, however, which I have
just laid down, seems to me the most natural, for the following
reasons: First, because, in the first case, the right of conquest
being in fact no right at all, it could not serve as a foundation for
any other right, the conqueror and the conquered ever remaining with
respect to each other in a state of war, unless the conquered,
restored to the full possession of their liberty, should freely choose
their conqueror for their chief. Till then, whatever capitulations
might have been made between them, as these capitulations were founded
upon violence, and of course \textit{de facto} null and void, there could not
have existed in this hypothesis either a true society, or a political
body, or any other law but that of the strongest. Second, because
these words strong and weak, are ambiguous in the second case; for
during the interval between the establishment of the right of property
or prior occupation and that of political government, the meaning of
these terms is better expressed by the words poor and rich, as before
the establishment of laws men in reality had no other means of
reducing their equals, but by invading the property of these equals,
or by parting with some of their own property to them. Third, because
the poor having nothing but their liberty to lose, it would have been
the height of madness in them to give up willingly the only blessing
they had left without obtaining some consideration for it: whereas the
rich being sensible, if I may say so, in every part of their
possessions, it was much easier to do them mischief, and therefore
more incumbent upon them to guard against it; and because, in fine, it
is but reasonable to suppose, that a thing has been invented by him to
whom it could be of service rather than by him to whom it must prove
detrimental.

Government in its infancy had no regular and permanent form. For want
of a sufficient fund of philosophy and experience, men could see no
further than the present inconveniences, and never thought of
providing remedies for future ones, but in proportion as they arose.
In spite of all the labours of the wisest legislators, the political
state still continued imperfect, because it was in a manner the work
of chance; and, as the foundations of it were ill laid, time, though
sufficient to discover its defects and suggest the remedies for them,
could never mend its original vices. Men were continually repairing;
whereas, to erect a good edifice, they should have begun as Lycurgus
did at Sparta, by clearing the area, and removing the old materials.
Society at first consisted merely of some general conventions which
all the members bound themselves to observe, and for the performance
of which the whole body became security to every individual.
Experience was necessary to show the great weakness of such a
constitution, and how easy it was for those, who infringed it, to
escape the conviction or chastisement of faults, of which the public
alone was to be both the witness and the judge; the laws could not
fail of being eluded a thousand ways; inconveniences and disorders
could not but multiply continually, till it was at last found
necessary to think of committing to private persons the dangerous
trust of public authority, and to magistrates the care of enforcing
obedience to the people: for to say that chiefs were elected before
confederacies were formed, and that the ministers of the laws existed
before the laws themselves, is a supposition too ridiculous to deserve
I should seriously refute it.

It would be equally unreasonable to imagine that men at first threw
themselves into the arms of an absolute master, without any conditions
or consideration on his side; and that the first means contrived by
jealous and unconquered men for their common safety was to run hand
over head into slavery. In fact, why did they give themselves
superiors, if it was not to be defended by them against oppression,
and protected in their lives, liberties, and properties, which are in
a manner the constitutional elements of their being? Now in the
relations between man and man, the worst that can happen to one man
being to see himself at the discretion of another, would it not have
been contrary to the dictates of good sense to begin by making over to
a chief the only things for the preservation of which they stood in
need of his assistance? What equivalent could he have offered them
for so fine a privilege? And had he presumed to exact it on pretense
of defending them, would he not have immediately received the answer
in the apologue? What worse treatment can we expect from an enemy? It
is therefore past dispute, and indeed a fundamental maxim of political
law, that people gave themselves chiefs to defend their liberty and
not be enslaved by them. If we have a prince, said Pliny to Trajan, it
is in order that he may keep us from having a master.

Political writers argue in regard to the love of liberty with the same
philosophy that philosophers do in regard to the state of nature; by
the things they see they judge of things very different which they
have never seen, and they attribute to men a natural inclination to
slavery, on account of the patience with which the slaves within their
notice carry the yoke; not reflecting that it is with liberty as with
innocence and virtue, the value of which is not known but by those who
possess them, though the relish for them is lost with the things
themselves. I know the charms of your country, said Brasidas to a
satrap who was comparing the life of the Spartans with that of the
Persepolites; but you can not know the pleasures of mine.

As an unbroken courser erects his mane, paws the ground, and rages at
the bare sight of the bit, while a trained horse patiently suffers
both whip and spur, just so the barbarian will never reach his neck to
the yoke which civilized man carries without murmuring but prefers the
most stormy liberty to a calm subjection. It is not therefore by the
servile disposition of enslaved nations that we must judge of the
natural dispositions of man for or against slavery, but by the
prodigies done by every free people to secure themselves from
oppression. I know that the first are constantly crying up that peace
and tranquillity they enjoy in their irons, and that \textit{miserrimam
servitutem pacem appellant}: but when I see the others sacrifice
pleasures, peace, riches, power, and even life itself to the
preservation of that single jewel so much slighted by those who have
lost it; when I see free-born animals through a natural abhorrence of
captivity dash their brains out against the bars of their prison; when
I see multitudes of naked savages despise European pleasures, and
brave hunger, fire and sword, and death itself to preserve their
independency; I feel that it belongs not to slaves to argue concerning
liberty.

As to paternal authority, from which several have derived absolute
government and every other mode of society, it is sufficient, without
having recourse to Locke and Sidney, to observe that nothing in the
world differs more from the cruel spirit of despotism that the
gentleness of that authority, which looks more to the advantage of him
who obeys than to the utility of him who commands; that by the law of
nature the father continues master of his child no longer than the
child stands in need of his assistance; that after that term they
become equal, and that then the son, entirely independent of the
father, owes him no obedience, but only respect. Gratitude is indeed
a duty which we are bound to pay, but which benefactors can not exact.
Instead of saying that civil society is derived from paternal
authority, we should rather say that it is to the former that the
latter owes its principal force: No one individual was acknowledged as
the father of several other individuals, till they settled about him.
The father's goods, which he can indeed dispose of as he pleases, are
the ties which hold his children to their dependence upon him, and he
may divide his substance among them in proportion as they shall have
deserved his attention by a continual deference to his commands. Now
the subjects of a despotic chief, far from having any such favour to
expect from him, as both themselves and all they have are his
property, or at least are considered by him as such, are obliged to
receive as a favour what he relinquishes to them of their own
property. He does them justice when he strips them; he treats them
with mercy when he suffers them to live. By continuing in this manner
to compare facts with right, we should discover as little solidity as
truth in the voluntary establishment of tyranny; and it would be a
hard matter to prove the validity of a contract which was binding only
on one side, in which one of the parties should stake everything and
the other nothing, and which could turn out to the prejudice of him
alone who had bound himself.

This odious system is even, at this day, far from being that of wise
and good monarchs, and especially of the kings of France, as may be
seen by divers passages in their edicts, and particularly by that of a
celebrated piece published in 1667 in the name and by the orders of
Louis XIV. "Let it therefore not be said that the sovereign is not
subject to the laws of his realm, since, that he is, is a maxim of the
law of nations which flattery has sometimes attacked, but which good
princes have always defended as the tutelary divinity of their realms.
How much more reasonable is it to say with the sage Plato, that the
perfect happiness of a state consists in the subjects obeying their
prince, the prince obeying the laws, and the laws being equitable and
always directed to the good of the public?" I shall not stop to
consider, if, liberty being the most noble faculty of man, it is not
degrading one's nature, reducing one's self to the level of brutes,
who are the slaves of instinct, and even offending the author of one's
being, to renounce without reserve the most precious of his gifts, and
submit to the commission of all the crimes he has forbid us, merely to
gratify a mad or a cruel master; and if this sublime artist ought to
be more irritated at seeing his work destroyed than at seeing it
dishonoured. I shall only ask what right those, who were not afraid
thus to degrade themselves, could have to subject their dependants to
the same ignominy, and renounce, in the name of their posterity,
blessings for which it is not indebted to their liberality, and
without which life itself must appear a burthen to all those who are
worthy to live.

Puffendorf says that, as we can transfer our property from one to
another by contracts and conventions, we may likewise divest ourselves
of our liberty in favour of other men. This, in my opinion, is a very
poor way of arguing; for, in the first place, the property I cede to
another becomes by such cession a thing quite foreign to me, and the
abuse of which can no way affect me; but it concerns me greatly that
my liberty is not abused, and I can not, without incurring the guilt
of the crimes I may be forced to commit, expose myself to become the
instrument of any. Besides, the right of property being of mere human
convention and institution, every man may dispose as he pleases of
what he possesses: But the case is otherwise with regard to the
essential gifts of nature, such as life and liberty, which every man
is permitted to enjoy, and of which it is doubtful at least whether
any man has a right to divest himself: By giving up the one, we
degrade our being; by giving up the other we annihilate it as much as
it is our power to do so; and as no temporal enjoyments can indemnify
us for the loss of either, it would be at once offending both nature
and reason to renounce them for any consideration. But though we could
transfer our liberty as we do our substance, the difference would be
very great with regard to our children, who enjoy our substance but by
a cession of our right; whereas liberty being a blessing, which as men
they hold from nature, their parents have no right to strip them of
it; so that as to establish slavery it was necessary to do violence to
nature, so it was necessary to alter nature to perpetuate such a
right; and the jurisconsults, who have gravely pronounced that the
child of a slave comes a slave into the world, have in other words
decided, that a man does not come a man into the world.

It therefore appears to me incontestably true, that not only
governments did not begin by arbitrary power, which is but the
corruption and extreme term of government, and at length brings it
back to the law of the strongest, against which governments were at
first the remedy, but even that, allowing they had commenced in this
manner, such power being illegal in itself could never have served as
a foundation to the rights of society, nor of course to the inequality
of institution.

I shall not now enter upon the inquiries which still remain to be made
into the nature of the fundamental pacts of every kind of government,
but, following the common opinion, confine myself in this place to the
establishment of the political body as a real contract between the
multitude and the chiefs elected by it. A contract by which both
parties oblige themselves to the observance of the laws that are
therein stipulated, and form the bands of their union. The multitude
having, on occasion of the social relations between them, concentered
all their wills in one person, all the articles, in regard to which
this will explains itself, become so many fundamental laws, which
oblige without exception all the members of the state, and one of
which laws regulates the choice and the power of the magistrates
appointed to look to the execution of the rest. This power extends to
everything that can maintain the constitution, but extends to nothing
that can alter it. To this power are added honours, that may render
the laws and the ministers of them respectable; and the persons of the
ministers are distinguished by certain prerogatives, which may make
them amends for the great fatigues inseparable from a good
administration. The magistrate, on his side, obliges himself not to
use the power with which he is intrusted but conformably to the
intention of his constituents, to maintain every one of them in the
peaceable possession of his property, and upon all occasions prefer
the good of the public to his own private interest.

Before experience had demonstrated, or a thorough knowledge of the
human heart had pointed out, the abuses inseparable from such a
constitution, it must have appeared so much the more perfect, as those
appointed to look to its preservation were themselves most concerned
therein; for magistracy and its rights being built solely on the
fundamental laws, as soon as these ceased to exist, the magistrates
would cease to be lawful, the people would no longer be bound to obey
them, and, as the essence of the state did not consist in the
magistrates but in the laws, the members of it would immediately
become entitled to their primitive and natural liberty.

A little reflection would afford us new arguments in confirmation of
this truth, and the nature of the contract might alone convince us
that it can not be irrevocable: for if there was no superior power
capable of guaranteeing the fidelity of the contracting parties and of
obliging them to fulfil their mutual engagements, they would remain
sole judges in their own cause, and each of them would always have a
right to renounce the contract, as soon as he discovered that the
other had broke the conditions of it, or that these conditions ceased
to suit his private convenience. Upon this principle, the right of
abdication may probably be founded. Now, to consider as we do nothing
but what is human in this institution, if the magistrate, who has all
the power in his own hands, and who appropriates to himself all the
advantages of the contract, has notwithstanding a right to divest
himself of his authority; how much a better right must the people, who
pay for all the faults of its chief, have to renounce their dependence
upon him. But the shocking dissensions and disorders without number,
which would be the necessary consequence of so dangerous a privilege,
show more than anything else how much human governments stood in need
of a more solid basis than that of mere reason, and how necessary it
was for the public tranquillity, that the will of the Almighty should
interpose to give to sovereign authority, a sacred and inviolable
character, which should deprive subjects of the mischievous right to
dispose of it to whom they pleased. If mankind had received no other
advantages from religion, this alone would be sufficient to make them
adopt and cherish it, since it is the means of saving more blood than
fanaticism has been the cause of spilling. But to resume the thread of
our hypothesis.

The various forms of government owe their origin to the various
degrees of inequality between the members, at the time they first
coalesced into a political body. Where a man happened to be eminent
for power, for virtue, for riches, or for credit, he became sole
magistrate, and the state assumed a monarchical form; if many of
pretty equal eminence out-topped all the rest, they were jointly
elected, and this election produced an aristocracy; those, between
whose fortune or talents there happened to be no such disproportion,
and who had deviated less from the state of nature, retained in common
the supreme administration, and formed a democracy. Time demonstrated
which of these forms suited mankind best. Some remained altogether
subject to the laws; others soon bowed their necks to masters. The
former laboured to preserve their liberty; the latter thought of
nothing but invading that of their neighbours, jealous at seeing
others enjoy a blessing which themselves had lost. In a word, riches
and conquest fell to the share of the one, and virtue and happiness to
that of the other.

In these various modes of government the offices at first were all
elective; and when riches did not preponderate, the preference was
given to merit, which gives a natural ascendant, and to age, which is
the parent of deliberateness in council, and experience in execution.
The ancients among the Hebrews, the Geronts of Sparta, the Senate of
Rome, nay, the very etymology of our word seigneur, show how much gray
hairs were formerly respected. The oftener the choice fell upon old
men, the oftener it became necessary to repeat it, and the more the
trouble of such repetitions became sensible; electioneering took
place; factions arose; the parties contracted ill blood; civil wars
blazed forth; the lives of the citizens were sacrificed to the
pretended happiness of the state; and things at last came to such a
pass, as to be ready to relapse into their primitive confusion. The
ambition of the principal men induced them to take advantage of these
circumstances to perpetuate the hitherto temporary charges in their
families; the people already inured to dependence, accustomed to ease
and the conveniences of life, and too much enervated to break their
fetters, consented to the increase of their slavery for the sake of
securing their tranquillity; and it is thus that chiefs, become
hereditary, contracted the habit of considering magistracies as a
family estate, and themselves as proprietors of those communities, of
which at first they were but mere officers; to call their
fellow-citizens their slaves; to look upon them, like so many cows or
sheep, as a part of their substance; and to style themselves the peers
of Gods, and Kings of Kings.

By pursuing the progress of inequality in these different revolutions,
we shall discover that the establishment of laws and of the right of
property was the first term of it; the institution of magistrates the
second; and the third and last the changing of legal into arbitrary
power; so that the different states of rich and poor were authorized
by the first epoch; those of powerful and weak by the second; and by
the third those of master and slave, which formed the last degree of
inequality, and the term in which all the rest at last end, till new
revolutions entirely dissolve the government, or bring it back nearer
to its legal constitution.

To conceive the necessity of this progress, we are not so much to
consider the motives for the establishment of political bodies, as the
forms these bodies assume in their administration; and the
inconveniences with which they are essentially attended; for those
vices, which render social institutions necessary, are the same which
render the abuse of such institutions unavoidable; and as (Sparta
alone excepted, whose laws chiefly regarded the education of children,
and where Lycurgus established such manners and customs, as in a great
measure made laws needless,) the laws, in general less strong than the
passions, restrain men without changing them; it would be no hard
matter to prove that every government, which carefully guarding
against all alteration and corruption should scrupulously comply with
the ends of its institution, was unnecessarily instituted; and that a
country, where no one either eluded the laws, or made an ill use of
magistracy, required neither laws nor magistrates.

Political distinctions are necessarily attended with civil
distinctions. The inequality between the people and the chiefs
increase so fast as to be soon felt by the private members, and
appears among them in a thousand shapes according to their passions,
their talents, and the circumstances of affairs. The magistrate can
not usurp any illegal power without making himself creatures, with
whom he must divide it. Besides, the citizens of a free state suffer
themselves to be oppressed merely in proportion as, hurried on by a
blind ambition, and looking rather below than above them, they come to
love authority more than independence. When they submit to fetters,
'tis only to be the better able to fetter others in their turn. It is
no easy matter to make him obey, who does not wish to command; and the
most refined policy would find it impossible to subdue those men, who
only desire to be independent; but inequality easily gains ground
among base and ambitious souls, ever ready to run the risks of
fortune, and almost indifferent whether they command or obey, as she
proves either favourable or adverse to them. Thus then there must have
been a time, when the eyes of the people were bewitched to such a
degree, that their rulers needed only to have said to the most pitiful
wretch, "Be great you and all your posterity," to make him immediately
appear great in the eyes of every one as well as in his own; and his
descendants took still more upon them, in proportion to their removes
from him: the more distant and uncertain the cause, the greater the
effect; the longer line of drones a family produced, the more
illustrious it was reckoned.

Were this a proper place to enter into details, I could easily explain
in what manner inequalities in point of credit and authority become
unavoidable among private persons the moment that, united into one
body, they are obliged to compare themselves one with another, and to
note the differences which they find in the continual use every man
must make of his neighbour. These differences are of several kinds;
but riches, nobility or rank, power and personal merit, being in
general the principal distinctions, by which men in society measure
each other, I could prove that the harmony or conflict between these
different forces is the surest indication of the good or bad original
constitution of any state: I could make it appear that, as among these
four kinds of inequality, personal qualities are the source of all the
rest, riches is that in which they ultimately terminate, because,
being the most immediately useful to the prosperity of individuals,
and the most easy to communicate, they are made use of to purchase
every other distinction. By this observation we are enabled to judge
with tolerable exactness, how much any people has deviated from its
primitive institution, and what steps it has still to make to the
extreme term of corruption. I could show how much this universal
desire of reputation, of honours, of preference, with which we are all
devoured, exercises and compares our talents and our forces: how much
it excites and multiplies our passions; and, by creating an universal
competition, rivalship, or rather enmity among men, how many
disappointments, successes, and catastrophes of every kind it daily
causes among the innumerable pretenders whom it engages in the same
career. I could show that it is to this itch of being spoken of, to
this fury of distinguishing ourselves which seldom or never gives us a
moment's respite, that we owe both the best and the worst things among
us, our virtues and our vices, our sciences and our errors, our
conquerors and our philosophers; that is to say, a great many bad
things to a very few good ones. I could prove, in short, that if we
behold a handful of rich and powerful men seated on the pinnacle of
fortune and greatness, while the crowd grovel in obscurity and want,
it is merely because the first prize what they enjoy but in the same
degree that others want it, and that, without changing their
condition, they would cease to be happy the minute the people ceased
to be miserable.

But these details would alone furnish sufficient matter for a more
considerable work, in which might be weighed the advantages and
disadvantages of every species of government, relatively to the rights
of man in a state of nature, and might likewise be unveiled all the
different faces under which inequality has appeared to this day, and
may hereafter appear to the end of time, according to the nature of
these several governments, and the revolutions time must unavoidably
occasion in them. We should then see the multitude oppressed by
domestic tyrants in consequence of those very precautions taken by
them to guard against foreign masters. We should see oppression
increase continually without its being ever possible for the oppressed
to know where it would stop, nor what lawful means they had left to
check its progress. We should see the rights of citizens, and the
liberties of nations extinguished by slow degrees, and the groans, and
protestations and appeals of the weak treated as seditious murmurings.
We should see policy confine to a mercenary portion of the people the
honour of defending the common cause. We should see imposts made
necessary by such measures, the disheartened husbandman desert his
field even in time of peace, and quit the plough to take up the sword.
We should see fatal and whimsical rules laid down concerning the point
of honour. We should see the champions of their country sooner or
later become her enemies, and perpetually holding their poniards to
the breasts of their fellow citizens. Nay, the time would come when
they might be heard to say to the oppressor of their country:

  Pectore si fratris gladium juguloque parentis
  Condere me jubeas, gravidoeque in viscera partu
  Conjugis, in vita peragam tamen omnia dextra.

From the vast inequality of conditions and fortunes, from the great
variety of passions and of talents, of useless arts, of pernicious
arts, of frivolous sciences, would issue clouds of prejudices equally
contrary to reason, to happiness, to virtue. We should see the chiefs
foment everything that tends to weaken men formed into societies by
dividing them; everything that, while it gives society an air of
apparent harmony, sows in it the seeds of real division; everything
that can inspire the different orders with mutual distrust and hatred
by an opposition of their rights and interest, and of course
strengthen that power which contains them all.

'Tis from the bosom of this disorder and these revolutions, that
despotism gradually rearing up her hideous crest, and devouring in
every part of the state all that still remained sound and untainted,
would at last issue to trample upon the laws and the people, and
establish herself upon the ruins of the republic. The times
immediately preceding this last alteration would be times of calamity
and trouble: but at last everything would be swallowed up by the
monster; and the people would no longer have chiefs or laws, but only
tyrants. At this fatal period all regard to virtue and manners would
likewise disappear; for despotism, \textit{cui ex honesto nulla est spes},
tolerates no other master, wherever it reigns; the moment it speaks,
probity and duty lose all their influence, and the blindest obedience
is the only virtue the miserable slaves have left them to practise.

This is the last term of inequality, the extreme point which closes
the circle and meets that from which we set out. 'Tis here that all
private men return to their primitive equality, because they are no
longer of any account; and that, the subjects having no longer any law
but that of their master, nor the master any other law but his
passions, all notions of good and principles of justice again
disappear. 'Tis here that everything returns to the sole law of the
strongest, and of course to a new state of nature different from that
with which we began, in as much as the first was the state of nature
in its purity, and the last the consequence of excessive corruption.
There is, in other respects, so little difference between these two
states, and the contract of government is so much dissolved by
despotism, that the despot is no longer master than he continues the
strongest, and that, as soon as his slaves can expel him, they may do
it without his having the least right to complain of their using him
ill. The insurrection, which ends in the death or despotism of a
sultan, is as juridical an act as any by which the day before he
disposed of the lives and fortunes of his subjects. Force alone upheld
him, force alone overturns him. Thus all things take place and succeed
in their natural order; and whatever may be the upshot of these hasty
and frequent revolutions, no one man has reason to complain of
another's injustice, but only of his own indiscretion or bad fortune.

By thus discovering and following the lost and forgotten tracks, by
which man from the natural must have arrived at the civil state; by
restoring, with the intermediate positions which I have been just
indicating, those which want of leisure obliges me to suppress, or
which my imagination has not suggested, every attentive reader must
unavoidably be struck at the immense space which separates these two
states. 'Tis in this slow succession of things he may meet with the
solution of an infinite number of problems in morality and politics,
which philosophers are puzzled to solve. He will perceive that, the
mankind of one age not being the mankind of another, the reason why
Diogenes could not find a man was, that he sought among his
cotemporaries the man of an earlier period: Cato, he will then see,
fell with Rome and with liberty, because he did not suit the age in
which he lived; and the greatest of men served only to astonish that
world, which would have cheerfully obeyed him, had he come into it
five hundred years earlier. In a word, he will find himself in a
condition to understand how the soul and the passions of men by
insensible alterations change as it were their nature; how it comes to
pass, that at the long run our wants and our pleasures change objects;
that, original man vanishing by degrees, society no longer offers to
our inspection but an assemblage of artificial men and factitious
passions, which are the work of all these new relations, and have no
foundation in nature. Reflection teaches us nothing on that head, but
what experience perfectly confirms. Savage man and civilised man
differ so much at bottom in point of inclinations and passions, that
what constitutes the supreme happiness of the one would reduce the
other to despair. The first sighs for nothing but repose and liberty;
he desires only to live, and to be exempt from labour; nay, the
ataraxy of the most confirmed Stoic falls short of his consummate
indifference for every other object. On the contrary, the citizen
always in motion, is perpetually sweating and toiling, and racking his
brains to find out occupations still more laborious: He continues a
drudge to his last minute; nay, he courts death to be able to live, or
renounces life to acquire immortality. He cringes to men in power whom
he hates, and to rich men whom he despises; he sticks at nothing to
have the honour of serving them; he is not ashamed to value himself on
his own weakness and the protection they afford him; and proud of his
chains, he speaks with disdain of those who have not the honour of
being the partner of his bondage. What a spectacle must the painful
and envied labours of an European minister of state form in the eyes
of a Caribbean! How many cruel deaths would not this indolent savage
prefer to such a horrid life, which very often is not even sweetened
by the pleasure of doing good? But to see the drift of so many cares,
his mind should first have affixed some meaning to these words power
and reputation; he should be apprised that there are men who consider
as something the looks of the rest of mankind, who know how to be
happy and satisfied with themselves on the testimony of others sooner
than upon their own. In fact, the real source of all those
differences, is that the savage lives within himself, whereas the
citizen, constantly beside himself, knows only how to live in the
opinion of others; insomuch that it is, if I may say so, merely from
their judgment that he derives the consciousness of his own existence.
It is foreign to my subject to show how this disposition engenders so
much indifference for good and evil, notwithstanding so many and such
fine discourses of morality; how everything, being reduced to
appearances, becomes mere art and mummery; honour, friendship, virtue,
and often vice itself, which we at last learn the secret to boast of;
how, in short, ever inquiring of others what we are, and never daring
to question ourselves on so delicate a point, in the midst of so much
philosophy, humanity, and politeness, and so many sublime maxims, we
have nothing to show for ourselves but a deceitful and frivolous
exterior, honour without virtue, reason without wisdom, and pleasure
without happiness. It is sufficient that I have proved that this is
not the original condition of man, and that it is merely the spirit of
society, and the inequality which society engenders, that thus change
and transform all our natural inclinations.

I have endeavoured to exhibit the origin and progress of inequality,
the institution and abuse of political societies, as far as these
things are capable of being deduced from the nature of man by the mere
light of reason, and independently of those sacred maxims which give
to the sovereign authority the sanction of divine right. It follows
from this picture, that as there is scarce any inequality among men in
a state of nature, all that which we now behold owes its force and its
growth to the development of our faculties and the improvement of our
understanding, and at last becomes permanent and lawful by the
establishment of property and of laws. It likewise follows that moral
inequality, authorised by any right that is merely positive, clashes
with natural right, as often as it does not combine in the same
proportion with physical inequality: a distinction which sufficiently
determines, what we are able to think in that respect of that kind of
inequality which obtains in all civilised nations, since it is
evidently against the law of nature that infancy should command old
age, folly conduct wisdom, and a handful of men should be ready to
choke with superfluities, while the famished multitude want the
commonest necessaries of life.


\end{document}

[Transcriber's Note: Some words which appear to be potential typos are
printed as such in the original book: These possible words include
cotemporaries and oftens. The paragraph starting with the words "This
odius system is even" contains unmatched quotes, which have been
reproduced as they appeared in the orginal. This work was transcribed
from a anthology (Harvard Classics Volume 34) published in 1910. The
editor of the entire series was Charles W. Eliot. The name of the
translator was not given, nor was the name of the author of the
introduction. Indented lines indicate embedded verse that should not
be re-wrapped.]





End of the Project Gutenberg EBook of A Discourse Upon The Origin And The
Foundation Of The Inequality Among Mankind, by Jean Jacques Rousseau

*** END OF THIS PROJECT GUTENBERG EBOOK INEQUALITY AMONG MANKIND ***

***** This file should be named 11136.txt or 11136.zip *****
This and all associated files of various formats will be found in:
        http://www.gutenberg.net/1/1/1/3/11136/


Updated editions will replace the previous one--the old editions
will be renamed.

Creating the works from public domain print editions means that no
one owns a United States copyright in these works, so the Foundation
(and you!) can copy and distribute it in the United States without
permission and without paying copyright royalties.  Special rules,
set forth in the General Terms of Use part of this license, apply to
copying and distributing Project Gutenberg-tm electronic works to
protect the PROJECT GUTENBERG-tm concept and trademark.  Project
Gutenberg is a registered trademark, and may not be used if you
charge for the eBooks, unless you receive specific permission.  If you
do not charge anything for copies of this eBook, complying with the
rules is very easy.  You may use this eBook for nearly any purpose
such as creation of derivative works, reports, performances and
research.  They may be modified and printed and given away--you may do
practically ANYTHING with public domain eBooks.  Redistribution is
subject to the trademark license, especially commercial
redistribution.



*** START: FULL LICENSE ***

THE FULL PROJECT GUTENBERG LICENSE
PLEASE READ THIS BEFORE YOU DISTRIBUTE OR USE THIS WORK

To protect the Project Gutenberg-tm mission of promoting the free
distribution of electronic works, by using or distributing this work
(or any other work associated in any way with the phrase "Project
Gutenberg"), you agree to comply with all the terms of the Full Project
Gutenberg-tm License (available with this file or online at
http://gutenberg.net/license).


Section 1.  General Terms of Use and Redistributing Project Gutenberg-tm
electronic works

1.A.  By reading or using any part of this Project Gutenberg-tm
electronic work, you indicate that you have read, understand, agree to
and accept all the terms of this license and intellectual property
(trademark/copyright) agreement.  If you do not agree to abide by all
the terms of this agreement, you must cease using and return or destroy
all copies of Project Gutenberg-tm electronic works in your possession.
If you paid a fee for obtaining a copy of or access to a Project
Gutenberg-tm electronic work and you do not agree to be bound by the
terms of this agreement, you may obtain a refund from the person or
entity to whom you paid the fee as set forth in paragraph 1.E.8.

1.B.  "Project Gutenberg" is a registered trademark.  It may only be
used on or associated in any way with an electronic work by people who
agree to be bound by the terms of this agreement.  There are a few
things that you can do with most Project Gutenberg-tm electronic works
even without complying with the full terms of this agreement.  See
paragraph 1.C below.  There are a lot of things you can do with Project
Gutenberg-tm electronic works if you follow the terms of this agreement
and help preserve free future access to Project Gutenberg-tm electronic
works.  See paragraph 1.E below.

1.C.  The Project Gutenberg Literary Archive Foundation ("the Foundation"
or PGLAF), owns a compilation copyright in the collection of Project
Gutenberg-tm electronic works.  Nearly all the individual works in the
collection are in the public domain in the United States.  If an
individual work is in the public domain in the United States and you are
located in the United States, we do not claim a right to prevent you from
copying, distributing, performing, displaying or creating derivative
works based on the work as long as all references to Project Gutenberg
are removed.  Of course, we hope that you will support the Project
Gutenberg-tm mission of promoting free access to electronic works by
freely sharing Project Gutenberg-tm works in compliance with the terms of
this agreement for keeping the Project Gutenberg-tm name associated with
the work.  You can easily comply with the terms of this agreement by
keeping this work in the same format with its attached full Project
Gutenberg-tm License when you share it without charge with others.

1.D.  The copyright laws of the place where you are located also govern
what you can do with this work.  Copyright laws in most countries are in
a constant state of change.  If you are outside the United States, check
the laws of your country in addition to the terms of this agreement
before downloading, copying, displaying, performing, distributing or
creating derivative works based on this work or any other Project
Gutenberg-tm work.  The Foundation makes no representations concerning
the copyright status of any work in any country outside the United
States.

1.E.  Unless you have removed all references to Project Gutenberg:

1.E.1.  The following sentence, with active links to, or other immediate
access to, the full Project Gutenberg-tm License must appear prominently
whenever any copy of a Project Gutenberg-tm work (any work on which the
phrase "Project Gutenberg" appears, or with which the phrase "Project
Gutenberg" is associated) is accessed, displayed, performed, viewed,
copied or distributed:

This eBook is for the use of anyone anywhere at no cost and with
almost no restrictions whatsoever.  You may copy it, give it away or
re-use it under the terms of the Project Gutenberg License included
with this eBook or online at www.gutenberg.net

1.E.2.  If an individual Project Gutenberg-tm electronic work is derived
from the public domain (does not contain a notice indicating that it is
posted with permission of the copyright holder), the work can be copied
and distributed to anyone in the United States without paying any fees
or charges.  If you are redistributing or providing access to a work
with the phrase "Project Gutenberg" associated with or appearing on the
work, you must comply either with the requirements of paragraphs 1.E.1
through 1.E.7 or obtain permission for the use of the work and the
Project Gutenberg-tm trademark as set forth in paragraphs 1.E.8 or
1.E.9.

1.E.3.  If an individual Project Gutenberg-tm electronic work is posted
with the permission of the copyright holder, your use and distribution
must comply with both paragraphs 1.E.1 through 1.E.7 and any additional
terms imposed by the copyright holder.  Additional terms will be linked
to the Project Gutenberg-tm License for all works posted with the
permission of the copyright holder found at the beginning of this work.

1.E.4.  Do not unlink or detach or remove the full Project Gutenberg-tm
License terms from this work, or any files containing a part of this
work or any other work associated with Project Gutenberg-tm.

1.E.5.  Do not copy, display, perform, distribute or redistribute this
electronic work, or any part of this electronic work, without
prominently displaying the sentence set forth in paragraph 1.E.1 with
active links or immediate access to the full terms of the Project
Gutenberg-tm License.

1.E.6.  You may convert to and distribute this work in any binary,
compressed, marked up, nonproprietary or proprietary form, including any
word processing or hypertext form.  However, if you provide access to or
distribute copies of a Project Gutenberg-tm work in a format other than
"Plain Vanilla ASCII" or other format used in the official version
posted on the official Project Gutenberg-tm web site (www.gutenberg.net),
you must, at no additional cost, fee or expense to the user, provide a
copy, a means of exporting a copy, or a means of obtaining a copy upon
request, of the work in its original "Plain Vanilla ASCII" or other
form.  Any alternate format must include the full Project Gutenberg-tm
License as specified in paragraph 1.E.1.

1.E.7.  Do not charge a fee for access to, viewing, displaying,
performing, copying or distributing any Project Gutenberg-tm works
unless you comply with paragraph 1.E.8 or 1.E.9.

1.E.8.  You may charge a reasonable fee for copies of or providing
access to or distributing Project Gutenberg-tm electronic works provided
that

- You pay a royalty fee of 20% of the gross profits you derive from
     the use of Project Gutenberg-tm works calculated using the method
     you already use to calculate your applicable taxes.  The fee is
     owed to the owner of the Project Gutenberg-tm trademark, but he
     has agreed to donate royalties under this paragraph to the
     Project Gutenberg Literary Archive Foundation.  Royalty payments
     must be paid within 60 days following each date on which you
     prepare (or are legally required to prepare) your periodic tax
     returns.  Royalty payments should be clearly marked as such and
     sent to the Project Gutenberg Literary Archive Foundation at the
     address specified in Section 4, "Information about donations to
     the Project Gutenberg Literary Archive Foundation."

- You provide a full refund of any money paid by a user who notifies
     you in writing (or by e-mail) within 30 days of receipt that s/he
     does not agree to the terms of the full Project Gutenberg-tm
     License.  You must require such a user to return or
     destroy all copies of the works possessed in a physical medium
     and discontinue all use of and all access to other copies of
     Project Gutenberg-tm works.

- You provide, in accordance with paragraph 1.F.3, a full refund of any
     money paid for a work or a replacement copy, if a defect in the
     electronic work is discovered and reported to you within 90 days
     of receipt of the work.

- You comply with all other terms of this agreement for free
     distribution of Project Gutenberg-tm works.

1.E.9.  If you wish to charge a fee or distribute a Project Gutenberg-tm
electronic work or group of works on different terms than are set
forth in this agreement, you must obtain permission in writing from
both the Project Gutenberg Literary Archive Foundation and Michael
Hart, the owner of the Project Gutenberg-tm trademark.  Contact the
Foundation as set forth in Section 3 below.

1.F.

1.F.1.  Project Gutenberg volunteers and employees expend considerable
effort to identify, do copyright research on, transcribe and proofread
public domain works in creating the Project Gutenberg-tm
collection.  Despite these efforts, Project Gutenberg-tm electronic
works, and the medium on which they may be stored, may contain
"Defects," such as, but not limited to, incomplete, inaccurate or
corrupt data, transcription errors, a copyright or other intellectual
property infringement, a defective or damaged disk or other medium, a
computer virus, or computer codes that damage or cannot be read by
your equipment.

1.F.2.  LIMITED WARRANTY, DISCLAIMER OF DAMAGES - Except for the "Right
of Replacement or Refund" described in paragraph 1.F.3, the Project
Gutenberg Literary Archive Foundation, the owner of the Project
Gutenberg-tm trademark, and any other party distributing a Project
Gutenberg-tm electronic work under this agreement, disclaim all
liability to you for damages, costs and expenses, including legal
fees.  YOU AGREE THAT YOU HAVE NO REMEDIES FOR NEGLIGENCE, STRICT
LIABILITY, BREACH OF WARRANTY OR BREACH OF CONTRACT EXCEPT THOSE
PROVIDED IN PARAGRAPH F3.  YOU AGREE THAT THE FOUNDATION, THE
TRADEMARK OWNER, AND ANY DISTRIBUTOR UNDER THIS AGREEMENT WILL NOT BE
LIABLE TO YOU FOR ACTUAL, DIRECT, INDIRECT, CONSEQUENTIAL, PUNITIVE OR
INCIDENTAL DAMAGES EVEN IF YOU GIVE NOTICE OF THE POSSIBILITY OF SUCH
DAMAGE.

1.F.3.  LIMITED RIGHT OF REPLACEMENT OR REFUND - If you discover a
defect in this electronic work within 90 days of receiving it, you can
receive a refund of the money (if any) you paid for it by sending a
written explanation to the person you received the work from.  If you
received the work on a physical medium, you must return the medium with
your written explanation.  The person or entity that provided you with
the defective work may elect to provide a replacement copy in lieu of a
refund.  If you received the work electronically, the person or entity
providing it to you may choose to give you a second opportunity to
receive the work electronically in lieu of a refund.  If the second copy
is also defective, you may demand a refund in writing without further
opportunities to fix the problem.

1.F.4.  Except for the limited right of replacement or refund set forth
in paragraph 1.F.3, this work is provided to you 'AS-IS' WITH NO OTHER
WARRANTIES OF ANY KIND, EXPRESS OR IMPLIED, INCLUDING BUT NOT LIMITED TO
WARRANTIES OF MERCHANTIBILITY OR FITNESS FOR ANY PURPOSE.

1.F.5.  Some states do not allow disclaimers of certain implied
warranties or the exclusion or limitation of certain types of damages.
If any disclaimer or limitation set forth in this agreement violates the
law of the state applicable to this agreement, the agreement shall be
interpreted to make the maximum disclaimer or limitation permitted by
the applicable state law.  The invalidity or unenforceability of any
provision of this agreement shall not void the remaining provisions.

1.F.6.  INDEMNITY - You agree to indemnify and hold the Foundation, the
trademark owner, any agent or employee of the Foundation, anyone
providing copies of Project Gutenberg-tm electronic works in accordance
with this agreement, and any volunteers associated with the production,
promotion and distribution of Project Gutenberg-tm electronic works,
harmless from all liability, costs and expenses, including legal fees,
that arise directly or indirectly from any of the following which you do
or cause to occur: (a) distribution of this or any Project Gutenberg-tm
work, (b) alteration, modification, or additions or deletions to any
Project Gutenberg-tm work, and (c) any Defect you cause.


Section  2.  Information about the Mission of Project Gutenberg-tm

Project Gutenberg-tm is synonymous with the free distribution of
electronic works in formats readable by the widest variety of computers
including obsolete, old, middle-aged and new computers.  It exists
because of the efforts of hundreds of volunteers and donations from
people in all walks of life.

Volunteers and financial support to provide volunteers with the
assistance they need, is critical to reaching Project Gutenberg-tm's
goals and ensuring that the Project Gutenberg-tm collection will
remain freely available for generations to come.  In 2001, the Project
Gutenberg Literary Archive Foundation was created to provide a secure
and permanent future for Project Gutenberg-tm and future generations.
To learn more about the Project Gutenberg Literary Archive Foundation
and how your efforts and donations can help, see Sections 3 and 4
and the Foundation web page at http://www.pglaf.org.


Section 3.  Information about the Project Gutenberg Literary Archive
Foundation

The Project Gutenberg Literary Archive Foundation is a non profit
501(c)(3) educational corporation organized under the laws of the
state of Mississippi and granted tax exempt status by the Internal
Revenue Service.  The Foundation's EIN or federal tax identification
number is 64-6221541.  Its 501(c)(3) letter is posted at
http://pglaf.org/fundraising.  Contributions to the Project Gutenberg
Literary Archive Foundation are tax deductible to the full extent
permitted by U.S. federal laws and your state's laws.

The Foundation's principal office is located at 4557 Melan Dr. S.
Fairbanks, AK, 99712., but its volunteers and employees are scattered
throughout numerous locations.  Its business office is located at
809 North 1500 West, Salt Lake City, UT 84116, (801) 596-1887, email
business@pglaf.org.  Email contact links and up to date contact
information can be found at the Foundation's web site and official
page at http://pglaf.org

For additional contact information:
     Dr. Gregory B. Newby
     Chief Executive and Director
     gbnewby@pglaf.org

Section 4.  Information about Donations to the Project Gutenberg
Literary Archive Foundation

Project Gutenberg-tm depends upon and cannot survive without wide
spread public support and donations to carry out its mission of
increasing the number of public domain and licensed works that can be
freely distributed in machine readable form accessible by the widest
array of equipment including outdated equipment.  Many small donations
($1 to $5,000) are particularly important to maintaining tax exempt
status with the IRS.

The Foundation is committed to complying with the laws regulating
charities and charitable donations in all 50 states of the United
States.  Compliance requirements are not uniform and it takes a
considerable effort, much paperwork and many fees to meet and keep up
with these requirements.  We do not solicit donations in locations
where we have not received written confirmation of compliance.  To
SEND DONATIONS or determine the status of compliance for any
particular state visit http://pglaf.org

While we cannot and do not solicit contributions from states where we
have not met the solicitation requirements, we know of no prohibition
against accepting unsolicited donations from donors in such states who
approach us with offers to donate.

International donations are gratefully accepted, but we cannot make
any statements concerning tax treatment of donations received from
outside the United States.  U.S. laws alone swamp our small staff.

Please check the Project Gutenberg Web pages for current donation
methods and addresses.  Donations are accepted in a number of other
ways including including checks, online payments and credit card
donations.  To donate, please visit: http://pglaf.org/donate


Section 5.  General Information About Project Gutenberg-tm electronic
works.

Professor Michael S. Hart is the originator of the Project Gutenberg-tm
concept of a library of electronic works that could be freely shared
with anyone.  For thirty years, he produced and distributed Project
Gutenberg-tm eBooks with only a loose network of volunteer support.

Project Gutenberg-tm eBooks are often created from several printed
editions, all of which are confirmed as Public Domain in the U.S.
unless a copyright notice is included.  Thus, we do not necessarily
keep eBooks in compliance with any particular paper edition.

Each eBook is in a subdirectory of the same number as the eBook's
eBook number, often in several formats including plain vanilla ASCII,
compressed (zipped), HTML and others.

Corrected EDITIONS of our eBooks replace the old file and take over
the old filename and etext number.  The replaced older file is renamed.
VERSIONS based on separate sources are treated as new eBooks receiving
new filenames and etext numbers.

Most people start at our Web site which has the main PG search facility:

     http://www.gutenberg.net

This Web site includes information about Project Gutenberg-tm,
including how to make donations to the Project Gutenberg Literary
Archive Foundation, how to help produce our new eBooks, and how to
subscribe to our email newsletter to hear about new eBooks.

EBooks posted prior to November 2003, with eBook numbers BELOW #10000,
are filed in directories based on their release date.  If you want to
download any of these eBooks directly, rather than using the regular
search system you may utilize the following addresses and just
download by the etext year.

     http://www.gutenberg.net/etext06

    (Or /etext 05, 04, 03, 02, 01, 00, 99,
     98, 97, 96, 95, 94, 93, 92, 92, 91 or 90)

EBooks posted since November 2003, with etext numbers OVER #10000, are
filed in a different way.  The year of a release date is no longer part
of the directory path.  The path is based on the etext number (which is
identical to the filename).  The path to the file is made up of single
digits corresponding to all but the last digit in the filename.  For
example an eBook of filename 10234 would be found at:

     http://www.gutenberg.net/1/0/2/3/10234

or filename 24689 would be found at:
     http://www.gutenberg.net/2/4/6/8/24689

An alternative method of locating eBooks:
     http://www.gutenberg.net/GUTINDEX.ALL


The Project Gutenberg EBook of A Discourse Upon The Origin And The
Foundation Of The Inequality Among Mankind, by Jean Jacques Rousseau

This eBook is for the use of anyone anywhere at no cost and with
almost no restrictions whatsoever.  You may copy it, give it away or
re-use it under the terms of the Project Gutenberg License included
with this eBook or online at www.gutenberg.net


Title: A Discourse Upon The Origin And The Foundation Of
         The Inequality Among Mankind

Author: Jean Jacques Rousseau

Release Date: February 17, 2004 [EBook #11136]

Language: English

Character set encoding: ASCII

*** START OF THIS PROJECT GUTENBERG EBOOK INEQUALITY AMONG MANKIND ***


