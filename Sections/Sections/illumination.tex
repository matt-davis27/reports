\section{SAFETY RULES DURING TESTING AND OPERATION OF EQUIPMENTt}
\begin{enumerate}
\item	In order to ensure safe conditions during testing the following rules have to be followed:
\item	Persons engaged with the commissioning of equipment must have  and have received proper and adequate training related to procedures and test equipment utilized for the particular testing activity.  
\item	Safety procedures for energization of MCC’s for Tag On \& Lock Out Procedures .
\item	Test instruments and test leads should be maintained in a safe condition and the test methods not to create danger either for the operator or for other persons.
\item	Where tests are required to circuits which are live for amp readings then the tightness of the cable terminations will be checked without power prior to live amp readings.
        \item     Ensure that warning labels are installed on MCC’s and plant.
	\item     Ensure that all plant drives have been pre-commissioned by the BMS contractor 
 and the drive supplier prior to commissioning.
\item	When readings are being taken for amperage and running speeds a support engineer    will be in attendance for safety reasons.
\end{enumerate}


\section{LIGHTING}
\subsection{Area Lighting}

The goal of area lighting should be to produce an overall level of illumination sufficient for personnel and equipment to move around and complete work activities safely.
\begin{enumerate}
\item {The types of work activities to be completed will assist in determining the lighting requirements.}
\item {The area to be illuminated for the movement of material by hand is much different than the movement of material by crane.}
\item {A night shift maintenance crew of 5-10 people will require lighting of a specific area vs. a production crew 50-100 will require a larger area to be illuminated.}
\item{When setting up lighting, consideration will be given to the layout to ensure that all areas receive light from at least two directions to prevent shadows.}
\item{Lighting should be mounted on poles or towers and spaced for optimal light distribution.}
\item{Bulbs shall be protected from breakage as designed by the manufacturer or as required by regulating agencies.}
\item{Lighting fixtures shall be positioned to prevent employees from coming into contact with the fixtures during work operations.}
\item{Lighting supply cords and cables shall be installed and maintained in a manner consistent with good electrical wiring practices.}
\end{enumerate}

The following table indicates the minimum levels of area illumination required for the type of work indicated:

\begin{table}
\begin{center}
\begin{tabular}{|l| c|c|}
\noalign{}\hline
\bf {Work Activity} & \bf{LUX Measurement} & \bf{Foot-candle Measurement} \\
\noalign{}\hline
Interior movement only &10 &1.0\\
\noalign{}\hline
Handling material &30 & 3.0\\
\noalign{}\hline
General rough work &30 &3.0\\
\noalign{}\hline
Mechanical work &50 &5.0\\
\noalign{}\hline
Electrical work &50 &5.0\\
\noalign{}\hline
\end{tabular}
\caption{Illumination Levels}
\end{center}
\end{table}


\subsection{General Requirements}

The following general aspects of area lighting should be followed:
\begin{enumerate}
\item	All points of exit, pathways, and muster points shall be clearly illuminated and marked. 
\item	Ladder access and egress shall be clearly illuminated.
\item Illumination shall be measured by a light meter, calibrated in lux, during the light setup and periodically thereafter.
\item	Tower lighting shall be located in a manner that illuminates all work areas.
\item	Lighting should be provided in confined spaces, and back-up lighting shall be provided in case of power failure.
\item Lighting poles and other metal poles shall be earthed (grounded) and the circuit fitted with residual current devices.
\item Cables supporting temporary lighting shall be routed to ensure they do not present a hazard.
\item All lighting fixtures shall be installed in a secure manner to prevent accidental movement or falling.
\item Broken or defective bulbs shall be promptly replaced.  All lights used for illumination shall be protected from accidental contact or breakage.  Metal-case sockets shall be grounded.
\end{enumerate}


\subsection{Emergency Arrangements}
\begin{enumerate}
\item	General: Site general safety procedures will be followed.  In case of emergency the Safety Officer or the most senior employee present will stop the work, and make necessary emergency arrangements and report to the assembly points as described during Site Induction Course.
	
\item    Fire Precautions: No significant fire risks associated with this activity but general 
fire precaution measures will be taken as per Main Contractor instructions to MEP.

\item	Communication from site areas to Site Office via mobile phones issued to all engineers.  In emergency situations the users of hand-held 2-way radios to be notified, if deemed that this can speed up notification of relevant personnel and/or the Site Clinic.

\end{enumerate}









