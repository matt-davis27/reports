%\newcommand{\genon}{25.10.2010}
%\newcommand{\flueon}{25.10.2010}           % ecological units on
%\renewcommand{\toweron}{26.10.2010}



\chapter{Standby Generators }
\begin{marginfigure}
  \includegraphics[width=\linewidth]{463}
  \caption{Cooling tower water pipes in Basement 2.}
  \label{fig:generators}
\end{marginfigure}

\newthought{The Standby Generators} physical works within the Generator Plantrooms have been completed with the exception of identification, snags and some ancillary services. The cooling tower cooling systems serving the Generators have also been completed. There are still Design issues to be clarified regarding the power feed to the pumps in these plan rooms. Cladding works remain for chilled water piping passing through these Plant rooms \TODO{Chase QMMC weekly on the issue.}

           \begin{margintable} 
	    \begin{tabular}{lcl}
	      \toprule
	      System    & Key Milestone Date  \\
	      \midrule
                  Rotana Kitchen Extract       &  \ecolon  \\   
                 Shangri-la Kitchen Extract   & \ecolon\\
	     Merweb Kitchen Extract       &  \ecolon  \\
	      
	      \bottomrule
	    \end{tabular}
           \caption{Kitchen Extract Ventilation Key Dates}
           \label{tbl:KEkeydates}
            \end{margintable}
 

\section{Current Status}

The generators are in place, as well as most of the electrical support installations, such as cable trays and cable ladders. Mechanical pipping works for the Cooling Tower circuits have started. Major work left to carry out includes all Panel installations, completing the Cooling Tower circuits\sidenote{\hl{Considerable progress has been achieved}.}, the generator exhaust piping and the fuel oil system. \sidenote{The fuel oil system has still not be designed, although an EI has been outstanding for more than a year. This is expected to be a three month cycle to approve, procure and install. \hl{Drawings have now been completed and issued to HOK}}

\section{Constraints}

Current constraints are human resources for ductwork installation and certain design issues involving co-ordination. Delivery of MCC panels for electrical hook-up is expected by end of August. Delivery of black steel ducting has started, mock-ups approved by the Engineer. This ductwork is flanged\sidenote{Specification calls for an all-welded construction. HOK has accepted a flanged construction, subject to ductwork pressure testing} and gaskets are still to be approved and procured. \sidenote{These have to be high temperature gaskets}. The ductwork is to be pressure tested. This is expected to slow down construction works. \sidenote{Duct pressure testing equipment are still to be ordered}

None of these constraints are impossible to resolve within a reasonable time. However, unless a full audit of these systems takes place  and we ensure that no redundant re-works are carried out due to co-ordination issues, competion dates can be met. We also expect MEPE to increase the number of personnel working on these systems, once all design, co-ordination and approval issues are resolved at the beginning of August. 

We have not factored the Ramadan slow-down in the key-dates, having assumed that any effect will be compensated by additional overtime in the months of Septemeber and October. 


           \begin{table}[htbp] 
	    \begin{tabular}{l p{4cm}l}
	      \toprule
	     Item & Cut-off date  \\
	      \midrule
                  1 &Electrical design audit & 5 Aug\\
                  2 &Mechanical design audit &5Aug \\
	      3  &Design Completed and Approved Shop drawings to be issued             &  $8^{th}$ Aug 2010  \\
                  4  &Procurement gaskets & 5th August\\
                  5  &BMS Schematics & 5 August \\
                  6  &DDC Panels delivered & 20th Sept \\
                  7  &MCC Panels delivered & 1st Sept\\
	      \bottomrule
	    \end{tabular}
           \caption{Stanby Generators:  major constraints}
            \end{table}



\section{Work Organization}
Due to the importance of these works and the fact that the installation is lagging in progress behind other works a PM\sidenote{This will be requested from Specon UAE} needs to be appointed to be fully responsible for this section of the works.

The work has been subdivided for convenience as follows:

\begin{table}[htbp] 
	    \begin{tabular}{l p{4cm}l}
	      \toprule
	     Item & Cut-off date  \\
	      \midrule
                  1 &Flushing cooling towers & 5 Aug\\
                  2 &Mechanical works (Cooling Tower) &5Aug \\
                  3 &Mechanical works (Fuel system)     &       \\
                  4 &Mechanical works                &       \\
                  5 & Fire protection                              &        \\
                  6 & Cable pulling                                 &       \\
	       \bottomrule
	    \end{tabular}
           \caption{Kitchen Ventilation:  major constraints}
            \end{table}


\section{Sequence for Switch-on}
\normalsize
In order to enable progressive commissioning early switch-on will be targetted. Although it may not be possible to achieve overall switch-on within the time frame envsisaged, sub-systems can be tested earlier on for example, the Cooling Tower circuit, the Chemical Treatment for same and the Fuel system. On the Electrical side, it will only be possible to carry full tests, once a substantial electrical load can be ready for switch-on. However, we expect all cable testing and inspections to be completed by the 30th October.

\section{Constraints}
\subsection{Fuel system}
\begin{enumerate}
\item Woquod tanks are not in place.\TODO{Sripathy action list}
\item Day tanks have not been delivered.
\end{enumerate}
\subsection{Commercial Issues}

The appointment of a subcontractor for commissioning has been issued to HLG on the . We have not had a response as yet. This is expected to delay commissioning with a knock-on effect on the Civil Defence Inspections.
\index{Civil Defence Inspection! Constraints! Generator subcontractor.}



















