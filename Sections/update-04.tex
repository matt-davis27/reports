
\cleardoublepage

\chapter*{Update}

Given the current constraints of the Project, we are satisfied that the general direction of the Project is doing well. Physical Installations in Towers and Podia have been completed in their majority and third fix activities are following the pace of both the Main Contractor as well as Fit-out activities. 

EIs, RFIs and other constraints have slowed down the works and precluded a reasonable implementation of our Closure and Commissioning Plan.

\section*{Objectives}

The aim of this report is to summarize the status of the works
and propose a reasonable plan for Closure of the Works and the implementation of a Commissioning Plan.

I general we are proposing a focus where areas can be fully completed, as well as the enabling of commissioning of all systems to take place. We suggest to start focusing on the
Shangri-la Hotel first, as this presents more difficulties.

\begin{longtable}{lll}
\toprule
Item  &Area   & Milestone \\
\midrule
1     & Podium Shangrila & 30 Jan 2013\\
2     & Tower Shangrila  & 15 Feb 2013\\
3     & Podium Rotana    & 16 Feb 2013\\
4     & Tower Rotana     & 16 Mar 2013\\
5     & Merweb Podium    & 17 Mar 2013\\
6     & Tower Merweb     & 30 Apr 2013\\
\bottomrule
\end{longtable}

We suggest a 4-7 week cycle for each Hotel at which point we expect that all physical works will close (including comments, FORs and NCRs) and the enabling of all systems to be commissioned. A small Team can remain to continue closing any issues that arise out of these activities and to follow-up on activities delayed by the Fit-out subcontractor.

Within Podia the following sequence is preferred, whereas in the Towers a top to bottom sequence will be followed.

\begin{table}[htbp]
\centering
\begin{tabular}{ccc}
\toprule
Item  &Area   & Milestone \\
\midrule
1     & B1 & 20 Jan\\
2     & B3 & 20 Jan \\
3     & B2 & 20 Jan \\
4     & L6 & 20 Jan\\
5     & L4 & 27 Jan\\
6     & L3 & 27 Jan\\
7     & L7 &  5 Feb\\
8     & L2 & 10 Feb\\
9     & L1 & 10 Feb\\
10    & L5 & 10 Feb\\
11    & GF & 20 Feb\\
\bottomrule
\end{tabular}
\caption{Floor sequence of works.}
\end{table}

Sequence as shown is not that important and the logic behind it, is to enable other trades, such as Dragoni to complete the
works before concentrating in the area. Also we expect the CCTV
changes to affect these areas and we prefer to work in them
once all the other areas are completed.

More details can be found in the relevant area and system sections of the report.

\begin{comment}

\section{Personnel}


 Table~\ref{tbl:manpower} summarizes the manpower currently on the Project and Figure~\ref{fig:manpower} shows the expected manpower histogram, required to complete the Project on time. Figure \ref{fig:manpower2} shows the estimated deployment of personnel to be involved with commissioning and closure issues, such as snagging, minor repairs, maintenance and operation of the various installations. \hl{Manpower has now re-mobilized, except Crompton who are still under resourced. We understand this issue has now been resolved and will increase their manpower.}

\def\Z{\phantom{Z}}
\begin{table}[htbp]
\begin{center}
\begin{tabular}{l r r}
\toprule
Company       &Tradesmen & Tradesmen\\
~                 & 30 January & 15 February\\
\midrule
Specon          &503           & 681\\
Al-Jaber         &181           &182\\
Crompton       &   0           &59\\
QMMC            &112          &113\\
Nafco             &5              &16\\
MEP               &111          &113\\
 & \underline{\phantom{1075}}&\underline{\phantom{1075}}\\
Subtotal         &912           &1201\\

Hired manpower      &               &\\
\Z ERE                     &53           &51\\
\Z Exblowra             &34            &39\\
\Z Deotech               &24           &24\\
\Z GVG                    &9             &10\\
\Z Thermo               &7             &13\\
                          & \underline{\phantom{1075}}& \underline{\phantom{1075}}\\
\Z Total hire manpower &127     &137\\
~&&\\
Total                      & \underline{\underline{1039}}       &\underline{\underline{1338}}\\
                             & &\\
\bottomrule
\end{tabular}
\caption{Average Manpower as of January 2013}
\label{tbl:manpower}
\end{center}
\end{table}






\begin{figure}[htbp]
\vspace{1.2cm}
\includegraphics[width=16.8cm]{./graphics/manpowercommissioning}
\label{fig:manpower2}
\caption{Manpower histogram, showing the required manpower for commissioning and snagging.}
\end{figure}
\end{fullwidth}
 Table~\ref{tbl:manpower} summarizes the manpower currently on the Project and Figure~\ref{fig:manpower} shows the expected manpower histogram, required to complete the Project on time. Figure \ref{fig:manpower2} shows the estimated deployment of personnel to be involved with commissioning and closure issues, such as snagging, minor repairs, maintenance and operation of the various installations. \hl{Manpower has now re-mobilized, except Crompton who are still under resourced. We understand this issue has now been resolved and will increase their manpower.}

\def\Z{\phantom{Z}}
\begin{table}[htbp]
\begin{center}
\begin{tabular}{l r r}
\toprule
Company       &Tradesmen & Tradesmen\\
~                 & 30 January & 15 February\\
\midrule
Specon          &503           & 681\\
Al-Jaber         &181           &182\\
Crompton       &   0           &59\\
QMMC            &112          &113\\
Nafco             &5              &16\\
MEP               &111          &113\\
Nhidi              & 37           &37\\
 & \underline{\phantom{1075}}&\underline{\phantom{1075}}\\
Subtotal         &912           &1201\\

Hired manpower      &               &\\
\Z ERE                     &53           &51\\
\Z Exblowra             &34            &39\\
\Z Deotech               &24           &24\\
\Z GVG                    &9             &10\\
\Z Thermo               &7             &13\\
                          & \underline{\phantom{1075}}& \underline{\phantom{1075}}\\
\Z Total hire manpower &127     &137\\
~&&\\
Total                      & \underline{\underline{1039}}       &\underline{\underline{1338}}\\
                             & &\\
\bottomrule
\end{tabular}
\caption{Average Manpower as of January 2011}
\label{tbl:manpower}
\end{center}
\end{table}




\section*{Current Constraints affecting completion}

\begin{enumerate}
\item Disruption due to Commercial issues.
\item Continued disruption with EIs and other changes.
\item Premature demobiliziation.
\item Demobilization of Crompton due to Commercial issues.
\item Slow down by all sub-contractors due to commercial issues.
\item BMS Supplier
\item Fire Protection
\end{enumerate}

\section*{Demobilization}

Since October the staff and technicians has been reduced substantially. 

\section*{Recommendations}
Our recommendation is to focus our attention and resources as per the original Plan, Tower by Tower.

Although we are trying our best to continue the works despite the disruptions experienced with the above issues. Decisive steps need to be taken to remedy the situation.

\begin{enumerate}
\item All orders ready to be delivered and pending due to commercial constraints need to be cleared. It will be easier to negotiate terms with Creditors, rather than keep the orders back.

\item Subcontractor payments and VO approvals need to be processed.


\item A decision to stop all works on EIs received after August needs to be taken.

\item Last minute additional resources will be necessary.

\item A larger cash account must be allocated to the Project to eneable procurement of urgent materials that have either been missed or are subject to commercial issues.

\end{enumerate}
\end{comment}
