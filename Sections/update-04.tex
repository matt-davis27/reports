
%\cleardoublepage

%% add the heading to the Table of Contents
%\addcontentsline{toc}{chapter}{Executive Summary}
\makeatletter
\newenvironment{update}{\quotation}{\endquotation}
\makeatother
\chapter{Executive Summary}

\begin{update}
\noindent\centerline{\textbf{Update 13 Mar 2013}}

Currently most activities are on target with the exception of the generator commissioning, which we postponed for later in the month in order to procure new batteries, which the Supplier advised. The
CO Monitoring system has been scheduled for the 23 Apr 2013. It is expected that it can be
commissioned within two to three days. We have released orders for additional manpower, but
the manpower is only expected to arrive after the 17 Mar 2013. This is delaying closure works.

Still of concern is underperformance for fans and AHUs. We had a meeting with IES who are confident
that most of the issues can be resolved via changes to impeller blades. Some of the CML measurements
were suspect and we requested both CML as well as the Supplier to re-check them.

\subsection*{Record of Delays}
RCU unit commissioning has progressed with all software downloaded to the devices (a time consuming process). Further commissioning cannot take place until Dragoni hands over the apartments and guest rooms back. Once the RCU units are commissioned no unauthorized personnel will be allowed to enter the rooms. Dragoni still has impeded works for final washand basin connections and bath speaker installations.


\end{update}

\newthought{Given the current constraints} of the Project, we are satisfied that the general direction of the Project is doing well and we will be in a position to offer the works for Civil Defence Inspection at end of April and shortly thereafter to  handover the buildings to the Operators. 

Physical installations in Towers and Podia have been completed in their majority and final fix activities are following the pace of the Fit-out activities carried out by the Owner's Direct Contractors, such as Dragoni, Askar and others.  A number of EIs remain to be executed and these are
expected to be completed after the Civil Defence Inspection.

EIs, RFIs and other constraints have slowed down the works and obstructed a reasonable implementation of our original Closure and Commissioning Plan. Commissioning is carried out on works completed

The aim of this report is to summarize the status of the works
and propose a reasonable plan for Closure of the Works and the implementation of a Commissioning Plan.

Below we list milestone dates for substantial completion. Substantial defined as completion of all physical works, excluding certain EIs and works affected by delays due to Owner's Direct Contractors. As for commissioning, our definition of substantial is to complete all activities relating to Civil Defence Inspection. The dates are tight, given the extend of the works, but we are actively working in eliminating risks and constraints.

\begin{longtable}{llll}
\toprule
Item  &Area   & Milestone \\
\midrule
\inc     & Podium Rotana      & 28 Mar 2013\\
\inc     & Tower Rotana       & 28 Mar 2013\\
\inc     & Podium Shangrila  & 28 Mar 2013\\
\inc     & Tower Shangrila    & 28 Mar 2013\\
\inc     & Merweb Podium    & 30 Apr 2013\\
\inc     & Tower Merweb      & 30 Apr 2013\\
\bottomrule
\end{longtable}

\section*{Current Constraints affecting completion}

\begin{enumerate}
\item Continued disruption with EIs and other changes. These EIs will be scheduled for completion after the Civil Defence Inspection (see Tables~\ref{EIsphase3a}\&\ref{EIsphase3b}.
\item Fire Protection (issues with Naffco). Currently we are meeting Naffco regularly and their new management is genuinely trying to solve the issues, however issues remain.
\item Zeroone resources for data and telephone services (only partly Civil Defence related).
\item Delivery of materials for Fire Alarm. \sidenote{All materials have been ordered and detailed lists are available and being tracked regularly.}
\item Imbalances in resources between the Main Contractor, MEP Subcontractors and Finishing Contractor. This needs to be addressed in order to achieve completion targets.
\end{enumerate}

\section*{Current Action}

The following steps are being implemented to eliminate the risks of not achieving the target dates.

\begin{enumerate}
\item  Instituted over-time works for all tradesmen.
\item  Deployment of Electrical Commissioning Manager from HLS UAE to Site. (Already mobilized to Site).
\item  Financial incentive to all subcontractor's to allow for submission of claims every two weeks.
\item  Hiring of additional 20 electricians from ERE.
\item  Deployment of further 50 technicians to be hired from manpower supply companies.
\item  Deployment of a Mechanical Engineer with commissioning skills.
(Already mobilized to Site).
\item  Deployment of one Mechanical Engineer for carrying out smoke tests.\sidenote{This is a time consuming task that needs to be repeated for all the building floors.}
\item HLG to urgently obtain commitments from the other parties, such as Dragoni and other Owner's Direct Subcontractors, that they will provide access and complete their work according to our programme to enable Civil Defence approval by end of April. The requirements for access are as detailed in numerous of our letters and in Table \ref{masterplan} in this report.
\end{enumerate}


\section*{Personnel}


 Table~\ref{personnel} summarizes the manpower currently on the Project. This is inadequate at the moment to complete the works as planned. We require at least another 30 electricians, 25 duct erectors and 15 plumbers.\sidenote{Electricians are needed for final fix activities, especially for ELV services, duct erectors is mostly supplementary labour for commissioning works on Towers and Podia (duct leakage). Plumbers are needed for Tower rectification works and final fixes.}

\def\Z{\phantom{Z}}
\begin{table}[htbp]
\begin{center}
\begin{tabular}{l r r}
\toprule
Company           &Tradesmen  \\
~                 &25 Feb \\
\midrule
HLS                &219         \\
Al-Jaber          &34         \\
Crompton        &20           \\
QMMC            &16        \\
Nafco             &9           \\
ERE (subcontract)               &10          \\
                 
                  &\underline{\phantom{1075}}\\
Subtotal          &\textbf{287}        \\

Hired manpower    &         \\
\Z ERE            &21        \\
          
                 
&&\\
Total                      &\textbf{308}       \\
                           
\bottomrule
\end{tabular}
\caption{Average Manpower as of January 2013}
\label{personnel}
\end{center}
\end{table}



\section*{EI Management}

Unfortunately the Engineer continues to issue amendments and additions to the works. In Table~\ref{EIsphase3a} and Table~\ref{EIsphase3b}  we list
the outstanding EIs and the expected completion date. The completion date for Engineer's Instructions is tentative as
is based on other trades completing prerequisite works on time.   
\bigskip


\resetinc
\begin{tabular}{lllp{4.5cm}}
\toprule
Item  &Description &No  &Remarks\\
\midrule
\inc &Completed EIs  &69 &\\
\inc &EIS completed 85\%-95\%  &32  &Final fix, subject mostly to ODC\\
\inc &EIS completed 50\%-85\%  &1  & ODC /Civil\\
\midrule
\inc &Outstanding EIS                          &1 &EI-232, B3 water for laundry boiler (on hold)\\
\bottomrule
\end{tabular}

\section*{Detailed Action Plan}

In the sections that follow we describe the detailed methodology and status of works for each individual service. \tref{masterplan} shows the current status of Testing \& Commissioning.

\section*{List of Services}
\label{masterplan}

\captionof{table}{List of Services and approximate status of T\&C.}
\begin{longtable}{lllp{4.3cm}@{}}
%\begin{tabular}{llll}
\toprule
Ref	&Package	&T\&C	&Remarks\\
\midrule
1.00	&\textbf{Mechanical}	&	&\\
1.01	&Chilled Water Systems	&97\%	&Remaining, works closing of paper work or Askar related.\\
1.02	&HVAC AHU  & 80\%		&  Remaining works, where final fixes are not completed. However, about 30\% of the units will have to have pulleys and or motors changed.\\
1.03    &Kitchen Extract System &75\%& Following installation of ASKAR hoods\\
1.03	&Car Park Ventilation 	&85\%&Smoke Tests remaining and some fans need remeasurements.\\
1.04	&Gas Fired steam Boilers &50\%&Not possible to commission until we have gas on site. Electrical systems tested and commissioned.		\\
1.05	&Condenser Water System 	&0\%&ASKAR have still not installed any of the related equipment.     \\
1.06    &Cooling Tower &80\%&Chemical Treatment still to be commissioned. Live test still to be carried out, in tandem with generator sets. \\
\midrule
2.00	&\textbf{Electrical}		&&\\
2.01	&Standby Generators	&30\%& Final live tests still remaining. \\	
2.02	&Medium Voltage System	&100\%&\\	
2.03	&Low Voltage System		&90\%&WIRs remaining.\\
2.04	&Earthing \& Lightning Protection &80\%&Lightning protection remaining. Currently manufacturer is not available.\\		
2.05	&Lighting \& Emergency Lighting  &70\%& Exit lights, progressively carried out. Emergency lighting to be commissioned after Generator live test.\\		
2.06	&Aircraft Warning System &0\%& Supplier issue.\\		
2.07	&UPS System	&70\%& Constraints (civil works due to EI) at Merweb.\\	
\midrule
3.00	&\textbf{Public Health}		&&\\
3.01	&Potable Water	&&\\	
3.02	&Above Ground Drainage &80\%&20\% to be carried out once all sanitary fittings are completed.\\		
3.03	&Below Ground Drainage  &100\%&\\		
3.04	&Water Features	     &0\%& HLG scope\\	
3.05	&Gas Supply Installations	&0\%& Cannot get gas until Civil Defence Inspection.\\	
3.06  & Gray Water Treatment &0\% &    \\
\midrule		
4.00	&\textbf{Fire Defense} &&\\	
4.01	&FHC, FH and Sprinkler Installations.	&80\%&Closing paper work and work progresses with Final fixes.\\	
4.02	&Smoke Exhaust		&50\%&Fans underperforming, leakage on ducts. \\
4.03	&Staircase Pressurization          &85\%&All completed but retesting is necessary.\\		
4.04	&Fire Alarm		&50\%& Works are progressively carried out (see \S\ref{sec:firealarm}).\\
4.05	&FM-200		&40\%&Integrity tests being carried out.\\
4.06	&Heliport Foam System	 &0\%&Issue with missing materilas (Naffco).\\	
\midrule
5.00	&\textbf{Specialist Services}		&&\\
5.01	&Building Management System	&75\%& Going well.\\	
5.02	&Security Access Control \& CCTV	&0\%&Client scope, now.\\	
5.03	&CO Monitoring		& 0\%&Manufacturer \\
5.04	&Car Calling System		&0\%&manufacturer related.\\
5.05	&Lighting Control		&0\%&Subject to finishes\\
5.06	&Structured Cabling		&40\%\%&Progressively tested.\\
5.07	&PA \& Background Music System &0\%&Part with FA will be ready by deadline\\		
5.08	&IPTV \& Satellite System	&0\%&Physical works dirsupted.\\	
5.09	&Room Management System	&0\%&Dragoni related.\\

\bottomrule
%\end{tabular}			
\end{longtable}



\section*{Conclusions}

We have outlined a plan that we are confident we can meet from our end. We are hoping that all parties involved can continue their co-operation with HLG \& HLS to remove existing constraints and or new issues that may arise during finalization to bring the Project to a successful conclusion. We also hope that the Direct Owner's Contractors can increase their pace of works to enable us to complete works without any further delays.



