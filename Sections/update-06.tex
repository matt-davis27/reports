
%\cleardoublepage

%% add the heading to the Table of Contents
%\addcontentsline{toc}{chapter}{Executive Summary}
\makeatletter
\newenvironment{update}{\quotation}{\endquotation}
\makeatother
\chapter*{Executive Summary}




The Civil Defence Inspection took place on the 22  May 2013 and the MEP issues that arose out of the inspection (except those related to design) were of a minor nature. These issues we have now mostly closed and will certainly have them completed within the month.

Given the current action of the Owner, the Site has taken steps to reduce costs, by stopping all hired manpower labour and the redeployment of 25 of our own staff. In addition many of the subcontractors have reduced both staff and labour (Towers) due to commercial issues. Outstanding orders for materials to cover EIs, maintenance and issues that arose during commissioning are on hold.

Given the constraints we are experiencing the current outlook for the completion of the works is not very good. The Owner's subcontractors---experiencing their own commercial issues---are not able to close open works at a rate than one can see completion of the works before the end of the year. The ex-HOK personnel that the Client now employed as "Engineer", continue to be hostile towards the Contractor with very little co-operation and impeding inspections and approvals by means of minor comments. They are also delaying return of submitted as-built and other drawings (currently about 1200 drawings are with the Engineer for approval, some of them in excess of five months).

The works are estimated to be 98.5\% completed with the balance 1.5\% representing commissioning issues, snagging and closure of areas where we are constrained by Owner's Direct Contractors. Some works, such as the Access Control Systems and Telephony we are delayed by our own subcontractors who are not performing in a satisfactory manner.

Currently we employ 200 technicians  out of which about 30 are employed as Operators in different shifts and approximately another 15 carry out maintenance works. With the balance 150 technicians we are completing balance commissioning activities, rectification of snags and final fix activities, as the Owner's Contractors complete their areas.

Given the green light to re-institute overtime and removing all constraints for handover we see a duration of 3 months to completion and another six months for clearing paper work with the Engineer and possibly new requirements by the Operators.

We do not see the buildings open to the public for at least another year, as the CCTV installations (now subcontracted to Johnson's Control by the Owner) have not as yet started. These works will necessitate the opening of ceilings (including decorative ceilings). 

Should the situation remain as is, our suggestion is to demobilize all but 50 technicians, 1 Superintentent and 2 Engineers by around the 25th of August. By then we project that any activities remaining can easily be completed by a much smaller Team during the duration that Johnson's Control complete their works. 

\end{document}

\section*{Personnel}


 Table~\ref{personnel} summarizes the manpower currently on the Project. This is inadequate at the moment to complete the works as planned. We require at least another 30 electricians, 25 duct erectors and 15 plumbers.\sidenote{Electricians are needed for final fix activities, especially for ELV services, duct erectors is mostly supplementary labour for commissioning works on Towers and Podia (duct leakage). Plumbers are needed for Tower rectification works and final fixes.}

\def\Z{\phantom{Z}}
\begin{table}[htbp]
\begin{center}
\begin{tabular}{l r r}
\toprule
Company           &Tradesmen  \\
~                 &25 Feb \\
\midrule
HLS                &219         \\
Al-Jaber          &34         \\
Crompton        &20           \\
QMMC            &16        \\
Nafco             &9           \\
ERE (subcontract)               &10          \\
                 
                  &\underline{\phantom{1075}}\\
Subtotal          &\textbf{287}        \\

Hired manpower    &         \\
\Z ERE            &21        \\
          
                 
&&\\
Total                      &\textbf{308}       \\
                           
\bottomrule
\end{tabular}
\caption{Average Manpower as of January 2013}
\label{personnel}
\end{center}
\end{table}



\section*{EI Management}

Unfortunately the Engineer continues to issue amendments and additions to the works. In Table~\ref{EIsphase3a} and Table~\ref{EIsphase3b}  we list
the outstanding EIs and the expected completion date. The completion date for Engineer's Instructions is tentative as
is based on other trades completing prerequisite works on time.   
\bigskip


\captionof{table}{Phase 3a, outstanding EIs}
\label{EIsphase3a}
\def\hcolor{\arrayrulecolor{LightGray}\midrule}
\resetinc
{


\begin{longtable}{@{}lllp{5.5cm}l}
\toprule
Item  &Description &No  &Remarks & Targt date\\
\midrule
\inc &Completed (100\%) &174 &\\
%\inc &EIS completed 85\%-95\%  & 174 &\\
\inc &Completed 50\%-85\%  & 51 & Final fix. We expect ODCs to complete by 10 Apr 2013 in order for us to complete our works by the deadline. & 31 May 2013\\
\midrule
\inc &Outstanding EIS                          &22 &EI-822, TSL/TRO dishwasher extract system for Argentinean and Teatro Restaurant. (rec Nov 2012) & 15.06.13\\
\hcolor
      &                                                 &     &\EI{EI-819}, TSL/TRO - CCTV System as per new MOI regulations.& Owner\\
\hcolor
      &                                                 &     &\EI{EI-814}, TSL/TRO Power provision for external lighting. (rec Nov 2012)& 20 Apr 2013\\
\hcolor
      &                                                 &     &EI-745, TSL/TRO B1 \&L5, soft water boiler (on hold).&\\
\hcolor
      &                                                 &     &EI 750, TSL/TRO Tag details& on-going\\
\hcolor
       &                                                &     &EI-HLG-001 TSL Level 3 Revised office Layout (rec 2 Jan 2013)& 30 May 2013\\
\hcolor
      &                                                 &     &EI-HLG-002  TSL GR external signage power provisions (rec 2 Jan 2013)& 30 May 2013\\ 
\hcolor
      &      
    &    &EI-HLG-003 F\&B Ansul Piranha Suppression and Fire Alarm (rec 5 Jan 2013)&\fire  30 Apr 2013\\
\hcolor
      &                                                 &     &EI-HLG-004 PABX Equipment dedicated clean earth. (rec 30 Dec 2013) & 15 Mar 2013\\
      &       
                                            &     &EI-HLG-005 Ansul Piranha Suppression \& Fire Alarm. (rec 5 Jan 2013)& \fire 30 Apr 2013\\
\hcolor
      &                                                 &     &EI DIWAN-002 Rotana Sewage tank. (rec 2 Feb 2013) & 30 May 2013\\\hcolor
      &                                                 &     &EI DIWAN-003 Shangri-la All-Day Dining lights (rec 3 Feb 2013).& 30 May 2013\\\hcolor
      &                                                 &     &EI-DIWAN-006 SL GF Additional flag poles. (rec 7 Feb 2013)& 30 May 2013\\\hcolor
      &                                                 &    &EI-DIWAN-008a Rotana LPG changes (rec 3.03.2013)  &\fire 30 Apr 2013  \\\hcolor
     &                                                 &    &EI-DIWAN-010 Rotana Provision of master switch (rec 14 Feb 2013) & 30 May 2013  \\\hcolor
     &                                                 &    &EI-DIWAN-011 Rotana \& Shangrila Dishwasher Exhaust fans. (rec 16 Feb 2013)& 30 May 2013\\\hcolor
 &                                                 &    &EI-DIWAN-011a Rotana \& Shangrila Dishwasher Exhaust fans. (rec 26 Feb 2013)& 30 May 2013\\\hcolor
 &                                                 &    &EI-DIWAN-012 Shangrila Corridor Lighting Lvl 8-17 (rec 11 Feb 13) & 30 May 2013\\\hcolor

&                                                 &    &EI-DIWAN-013 Lvl 42 Horizon Lounge (rec 13 Feb 2013) & 30 May 2013\\\hcolor
&                                                 &    &EI-DIWAN-014 Ballroom Prefunction and bridal room (21 Feb 2013)  & 30 May 2013\\\hcolor
&                                                 &    &EI-DIWAN-017 Staff Dining (rec 26 Feb 2013)  & 30 May 2013\\\hcolor

 &                                                     &    &EI-DIWAN-019 TSL Artwork Corridor (rec 26 Feb 2013)  &\Danger HLG inst. Not to Proceed  \\

\arrayrulecolor{black}
\midrule
\inc & Total                                         &247 &&\\
\bottomrule
\end{longtable}
\vspace*{1.5cm}
}


\pagebreak
\caption{Phase 3b, outstanding EIs}
\label{EIsphase3b}
\resetinc
\begin{longtable}{r lll p{4.5cm} l}
\toprule
S.no. &Description &No  &EI    &Remarks    & Target Date\\
\midrule
\setcounter{step}{68}%
\inc &Completed  (100\%) &69& & \\
\addtocounter{step}{32}%
\inc &Completed 85\%-95\%  &33  &
      &Final fix. We expect ODCs to complete by 10 Apr 2013 in order
        for us to complete our works by 31 May 2013. 
      & 31 May 2013 \\
\midrule
\inc &                       & &EI-232& B3 water for laundry boiler (on hold)&\\\hcolor
\inc &                                                 &  &HLG-001& L24 Fire Alarm (9 Jan 2013) &\fire 30 Apr 2013  \\\hcolor
\inc &                                                 &  &HLG-002& Merweb Sewage Tank (26 Jan 2013)& 30 May 2013    \\ \hcolor
\inc &                                                 &  &HLG-003& Lvl 24 Electrical Enclosure (5 Feb 2013)& 30 May 2013\\                                                 \hcolor
\inc &                                                 &  &HLG-005& Lvl 24 Services coordination (17 Feb 2013)&\fire 30 Mar 2013\\
\arrayrulecolor{black}\midrule
\inc & Total                                         & &\the\value{step}&\\
\bottomrule
\end{longtable}

\




\section*{Detailed Action Plan}

In the sections that follow we describe the detailed methodology and status of works for each individual service. \tref{masterplan} shows the current status of Testing \& Commissioning.

\section*{List of Services}


\captionof{table}{List of Services and approximate status of T\&C.}
\label{masterplan}
{\RaggedRight\small
\begin{longtable}{lllp{4.3cm}@{}}
%\begin{tabular}{llll}
\toprule
Ref	&Package	&T\&C	&Remarks\\
\midrule
1.00	&\textbf{Mechanical}	&	&\\
1.01	&Chilled Water Systems	&97\%	&Remaining, works closing of paper work or Askar related. See \refs{chilledwater}\\

1.02	&HVAC AHU  & 80\%		&  Remaining works, where final fixes are not completed. However, about 30\% of the units will have to have pulleys and or motors changed. See \refs{HVAC}.\\

1.03    &Kitchen Extract System &75\%& Following installation of ASKAR hoods. See \refs{kitchenextract}\\

1.03	&Car Park Ventilation 	&80\%&Smoke Tests remaining and some fans need remeasurements. See \refs{carparkventilation}\\

1.04	&Gas Fired Steam Boilers &50\%&Not possible to commission until we have gas on site. Gas will only be made available after Civil Defence inspection.	\\

1.05	&Condenser Water System 	&0\%&ASKAR have still not installed any of the related equipment. See \refs{chilledwaterconstraints}.    \\

1.06    &Cooling Tower &80\%&Chemical Treatment still to be commissioned. Live test still to be carried out, in tandem with generator sets. \ref{generators}.  \\
\midrule

2.00	&\textbf{Electrical}		&&\\
2.01	&Standby Generators	&30\%& Final live tests still remaining. See \sref{generators}. \\
	
2.02	&Medium Voltage System	&100\%&\\	

2.03	&Low Voltage System		&90\%&WIRs remaining.\\

2.04	&Earthing \& Lightning Protection &80\%&Lightning protection remaining. Expecetd to be commissioned by 30 Mar 2013.\\		

2.05	&Lighting \& Emergency Lighting  &70\%& Exit lights, progressively carried out. Emergency lighting to be commissioned after Generator live test.\\		

2.06	&Aircraft Warning System &0\%& Expected to be commissioned by 20 Apr 2013.\\	
	
2.07	&UPS System	&70\%&  Merweb remaining due to EI, expected to be commissioned by 30 Mar 2013.\\	
\midrule
3.00	&\textbf{Public Health}		&&\\
3.01	&Potable Water	&65\%& Delays due to Final fixture installation. Expected to start commissioning late April.\\	

3.02	&Above Ground Drainage &80\%&20\% to be carried out once all sanitary fittings are completed.\\		
3.03	&Below Ground Drainage  &100\%&\\		
3.04	&Water Features	     &0\%& HLG scope\\	
3.05	&Gas Supply Installations	&0\%& Cannot get gas until Civil Defence Inspection. Earliest start date 10 May 2013.\\	
3.06  & Gray Water Treatment &0\% &EI works for changing sectional tanks to concrete. Earliest start 20 April 2013.  \\
\midrule		
4.00	&\textbf{Fire Defense} &&\\	
4.01	&FHC, FH and Sprinkler Installations.	&80\%&Closing paper work and work progresses with Final fixes.\\	
4.02	&Smoke Exhaust		&70\%&Fans underperforming, leakage on ducts. \\
4.03	&Staircase Pressurization          &85\%&All completed but retesting is necessary.\\		
4.04	&Fire Alarm		&50\%& Works are progressively carried out (see \S\ref{sec:firealarm}). Also awaiting for materials.\\

4.05	&FM-200		&70\%&Integrity tests being carried out. Going well. \\

4.06	&Heliport Foam System	 &0\%&Issue with missing materials (Naffco).\\	
\midrule
5.00	&\textbf{Specialist Services}		&&\\
5.01	&Building Management System	&75\%& Going well.\\	
5.02	&Security Access Control \& CCTV	&0\%&Client scope, now.\\	
5.03	&CO Monitoring		& 30\%&Manufacturer to start commissioning 13 Mar 2013 \\
5.04	&Car Calling System		&0\%&Manufacturer ready to commission.\\

5.05	&Lighting Control		&0\%&Subject to finishes\\

5.06	&Structured Cabling		&40\%&Progressively tested. Affected by integration EI. Expected completion 20-30 Apr 2013\\

5.07	&PA \& Background Music System &0\%& Not in HLS scope. Interfacing with  FA ongoing.\\		
5.08	&IPTV \& Satellite System	&0\%&Physical works dirsupted by EI as for 5.06.\\	
5.09	&Room Management System	&30\%&Delayed due to Dragoni works.\\

\bottomrule
%\end{tabular}			
\end{longtable}
}



\section*{Conclusions}

We have outlined a plan that we are confident we can meet from our end. We are hoping that all parties involved can continue their co-operation with HLG \& HLS to remove existing constraints and or new issues that may arise during finalization to bring the Project to a successful conclusion. We also hope that the Direct Owner's Contractors can increase their pace of works to enable us to complete works without any further delays.



