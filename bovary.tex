\documentclass{tufte-book}
\usepackage[utf8]{inputenc}
\usepackage[T1]{fontenc}
\usepackage[latin,english]{babel}
\usepackage{phd}
\def\hlred{{\color{Magenta}}}

\usepackage{hyperref}
\hypersetup{pdftex,
  bookmarks,
  raiselinks,
  pageanchor,
  hyperindex,
  colorlinks,
  allcolors=theblue, 
  %anchorcolor= blue,
  %filecolor=blue,
  urlcolor=theblue,
  linkcolor= theblue,
  pdftitle={My Title},
 }
\title{Madame Bovary }
\author{Gustave Flaubert}
%\date{Konigsberg in Prussia, 30th September, 1784.}

\newcommand\hemdash{\hbox{---}\kern-.5em---}  
\makeatletter
\def\@fnsymbol#1{%
   \ifcase#1\or \TextOrMath\textasteriskcentered *\or
   \TextOrMath \textdagger \dagger\or
   \TextOrMath \textdaggerdbl \ddagger \or
   \TextOrMath \textsection  \mathsection\or
   \TextOrMath \textparagraph \mathparagraph\or
   \TextOrMath \textbardbl \|\or
   \TextOrMath {\textasteriskcentered\textasteriskcentered}{**}\or
   \TextOrMath {\textdagger\textdagger}{\dagger\dagger}\or
   \TextOrMath {\textdaggerdbl\textdaggerdbl}{\ddagger\ddagger}\else
   \@ctrerr \fi
}
\renewcommand\thefootnote{\@fnsymbol\c@footnote}%
    \def\@makefnmark{\rlap{\@textsuperscript{\normalfont\@thefnmark}}}%
    \long\def\@makefntext#1{\parindent 1em\noindent
            \hb@xt@1.8em{%
                \hss\@textsuperscript{\normalfont\@thefnmark}}#1}
\begin{document}

\maketitle


MADAME BOVARY

By Gustave Flaubert

Translated from the French by Eleanor Marx-Aveling


To Marie-Antoine-Jules Senard Member of the Paris Bar, Ex-President
of the National Assembly, and Former Minister of the Interior Dear and
Illustrious Friend, Permit me to inscribe your name at the head of this
book, and above its dedication; for it is to you, before all, that I
owe its publication. Reading over your magnificent defence, my work has
acquired for myself, as it were, an unexpected authority.

Accept, then, here, the homage of my gratitude, which, how great soever
it is, will never attain the height of your eloquence and your devotion.

Gustave Flaubert Paris, 12 April 1857




MADAME BOVARY




\part*{Part I}



\chapter*{Chapter One}

We were in class when the head-master came in, followed by a "new
fellow," not wearing the school uniform, and a school servant carrying a
large desk. Those who had been asleep woke up, and every one rose as if
just surprised at his work.

The head-master made a sign to us to sit down. Then, turning to the
class-master, he said to him in a low voice\hemdash

"Monsieur Roger, here is a pupil whom I recommend to your care; he'll be
in the second. If his work and conduct are satisfactory, he will go into
one of the upper classes, as becomes his age."

The "new fellow," standing in the corner behind the door so that he
could hardly be seen, was a country lad of about fifteen, and taller
than any of us. His hair was cut square on his forehead like a village
chorister's; he looked reliable, but very ill at ease. Although he was
not broad-shouldered, his short school jacket of green cloth with black
buttons must have been tight about the arm-holes, and showed at the
opening of the cuffs red wrists accustomed to being bare. His legs, in
blue stockings, looked out from beneath yellow trousers, drawn tight by
braces, He wore stout, ill-cleaned, hob-nailed boots.

We began repeating the lesson. He listened with all his ears, as
attentive as if at a sermon, not daring even to cross his legs or lean
on his elbow; and when at two o'clock the bell rang, the master was
obliged to tell him to fall into line with the rest of us.

When we came back to work, we were in the habit of throwing our caps on
the ground so as to have our hands more free; we used from the door to
toss them under the form, so that they hit against the wall and made a
lot of dust: it was "the thing."

But, whether he had not noticed the trick, or did not dare to attempt
it, the "new fellow," was still holding his cap on his knees even after
prayers were over. It was one of those head-gears of composite order, in
which we can find traces of the bearskin, shako, billycock hat, sealskin
cap, and cotton night-cap; one of those poor things, in fine, whose
dumb ugliness has depths of expression, like an imbecile's face. Oval,
stiffened with whalebone, it began with three round knobs; then came in
succession lozenges of velvet and rabbit-skin separated by a red band;
after that a sort of bag that ended in a cardboard polygon covered with
complicated braiding, from which hung, at the end of a long thin cord,
small twisted gold threads in the manner of a tassel. The cap was new;
its peak shone.

"Rise," said the master.

He stood up; his cap fell. The whole class began to laugh. He stooped to
pick it up. A neighbor knocked it down again with his elbow; he picked
it up once more.

"Get rid of your helmet," said the master, who was a bit of a wag.

There was a burst of laughter from the boys, which so thoroughly put the
poor lad out of countenance that he did not know whether to keep his cap
in his hand, leave it on the ground, or put it on his head. He sat down
again and placed it on his knee.

"Rise," repeated the master, "and tell me your name."

The new boy articulated in a stammering voice an unintelligible name.

"Again!"

The same sputtering of syllables was heard, drowned by the tittering of
the class.

"Louder!" cried the master; "louder!"

The "new fellow" then took a supreme resolution, opened an inordinately
large mouth, and shouted at the top of his voice as if calling someone
in the word "Charbovari."

A hubbub broke out, rose in crescendo with bursts of shrill voices (they
yelled, barked, stamped, repeated "Charbovari! Charbovari"), then died
away into single notes, growing quieter only with great difficulty, and
now and again suddenly recommencing along the line of a form whence rose
here and there, like a damp cracker going off, a stifled laugh.

However, amid a rain of impositions, order was gradually re-established
in the class; and the master having succeeded in catching the name of
"Charles Bovary," having had it dictated to him, spelt out, and re-read,
at once ordered the poor devil to go and sit down on the punishment form
at the foot of the master's desk. He got up, but before going hesitated.

"What are you looking for?" asked the master.

"My c-a-p," timidly said the "new fellow," casting troubled looks round
him.

"Five hundred lines for all the class!" shouted in a furious voice
stopped, like the Quos ego\footnote{A quotation from the Aeneid signifying a threat.}, a fresh outburst. "Silence!" continued the
master indignantly, wiping his brow with his handkerchief, which he
had just taken from his cap. "As to you, 'new boy,' you will conjugate
'ridiculus sum'\footnote{I am ridiculous.} twenty times."

Then, in a gentler tone, "Come, you'll find your cap again; it hasn't
been stolen."

     

Quiet was restored. Heads bent over desks, and the "new fellow" remained
for two hours in an exemplary attitude, although from time to time some
paper pellet flipped from the tip of a pen came bang in his face. But he
wiped his face with one hand and continued motionless, his eyes lowered.

In the evening, at preparation, he pulled out his pens from his desk,
arranged his small belongings, and carefully ruled his paper. We saw him
working conscientiously, looking up every word in the dictionary, and
taking the greatest pains. Thanks, no doubt, to the willingness he
showed, he had not to go down to the class below. But though he knew his
rules passably, he had little finish in composition. It was the cure
of his village who had taught him his first Latin; his parents, from
motives of economy, having sent him to school as late as possible.

His father, Monsieur Charles Denis Bartolome Bovary, retired
assistant-surgeon-major, compromised about 1812 in certain conscription
scandals, and forced at this time to leave the service, had taken
advantage of his fine figure to get hold of a dowry of sixty thousand
francs that offered in the person of a hosier's daughter who had fallen
in love with his good looks. A fine man, a great talker, making his
spurs ring as he walked, wearing whiskers that ran into his moustache,
his fingers always garnished with rings and dressed in loud colours,
he had the dash of a military man with the easy go of a commercial
traveller.

Once married, he lived for three or four years on his wife's fortune,
dining well, rising late, smoking long porcelain pipes, not coming in
at night till after the theatre, and haunting cafes. The father-in-law
died, leaving little; he was indignant at this, "went in for the
business," lost some money in it, then retired to the country, where he
thought he would make money.

But, as he knew no more about farming than calico, as he rode his horses
instead of sending them to plough, drank his cider in bottle instead of
selling it in cask, ate the finest poultry in his farmyard, and greased
his hunting-boots with the fat of his pigs, he was not long in finding
out that he would do better to give up all speculation.

For two hundred francs a year he managed to live on the border of
the provinces of Caux and Picardy, in a kind of place half farm, half
private house; and here, soured, eaten up with regrets, cursing his
luck, jealous of everyone, he shut himself up at the age of forty-five,
sick of men, he said, and determined to live at peace.

His wife had adored him once on a time; she had bored him with a
thousand servilities that had only estranged him the more. Lively once,
expansive and affectionate, in growing older she had become (after the
fashion of wine that, exposed to air, turns to vinegar) ill-tempered,
grumbling, irritable. She had suffered so much without complaint at
first, until she had seem him going after all the village drabs, and
until a score of bad houses sent him back to her at night, weary,
stinking drunk. Then her pride revolted. After that she was silent,
burying her anger in a dumb stoicism that she maintained till her death.
She was constantly going about looking after business matters. She
called on the lawyers, the president, remembered when bills fell due,
got them renewed, and at home ironed, sewed, washed, looked after the
workmen, paid the accounts, while he, troubling himself about nothing,
eternally besotted in sleepy sulkiness, whence he only roused himself
to say disagreeable things to her, sat smoking by the fire and spitting
into the cinders.

When she had a child, it had to be sent out to nurse. When he came home,
the lad was spoilt as if he were a prince. His mother stuffed him
with jam; his father let him run about barefoot, and, playing the
philosopher, even said he might as well go about quite naked like the
young of animals. As opposed to the maternal ideas, he had a certain
virile idea of childhood on which he sought to mould his son, wishing
him to be brought up hardily, like a Spartan, to give him a strong
constitution. He sent him to bed without any fire, taught him to drink
off large draughts of rum and to jeer at religious processions. But,
peaceable by nature, the lad answered only poorly to his notions. His
mother always kept him near her; she cut out cardboard for him, told him
tales, entertained him with endless monologues full of melancholy gaiety
and charming nonsense. In her life's isolation she centered on the
child's head all her shattered, broken little vanities. She dreamed of
high station; she already saw him, tall, handsome, clever, settled as
an engineer or in the law. She taught him to read, and even, on an old
piano, she had taught him two or three little songs. But to all this
Monsieur Bovary, caring little for letters, said, "It was not worth
while. Would they ever have the means to send him to a public school, to
buy him a practice, or start him in business? Besides, with cheek a man
always gets on in the world." Madame Bovary bit her lips, and the child
knocked about the village.

He went after the labourers, drove away with clods of earth the ravens
that were flying about. He ate blackberries along the hedges, minded the
geese with a long switch, went haymaking during harvest, ran about in
the woods, played hop-scotch under the church porch on rainy days, and
at great fetes begged the beadle to let him toll the bells, that he
might hang all his weight on the long rope and feel himself borne upward
by it in its swing. Meanwhile he grew like an oak; he was strong on
hand, fresh of colour.

When he was twelve years old his mother had her own way; he began
lessons. The cure took him in hand; but the lessons were so short and
irregular that they could not be of much use. They were given at spare
moments in the sacristy, standing up, hurriedly, between a baptism and
a burial; or else the cure, if he had not to go out, sent for his pupil
after the Angelus*. They went up to his room and settled down; the
flies and moths fluttered round the candle. It was close, the child
fell asleep, and the good man, beginning to doze with his hands on his
stomach, was soon snoring with his mouth wide open. On other occasions,
when Monsieur le Cure, on his way back after administering the viaticum
to some sick person in the neighbourhood, caught sight of Charles
playing about the fields, he called him, lectured him for a quarter of
an hour and took advantage of the occasion to make him conjugate his
verb at the foot of a tree. The rain interrupted them or an acquaintance
passed. All the same he was always pleased with him, and even said the
"young man" had a very good memory.

     *A devotion said at morning, noon, and evening, at the sound
     of a bell. Here, the evening prayer.

Charles could not go on like this. Madame Bovary took strong steps.
Ashamed, or rather tired out, Monsieur Bovary gave in without a
struggle, and they waited one year longer, so that the lad should take
his first communion.

Six months more passed, and the year after Charles was finally sent to
school at Rouen, where his father took him towards the end of October,
at the time of the St. Romain fair.

It would now be impossible for any of us to remember anything about him.
He was a youth of even temperament, who played in playtime, worked in
school-hours, was attentive in class, slept well in the dormitory,
and ate well in the refectory. He had in loco parentis* a wholesale
ironmonger in the Rue Ganterie, who took him out once a month on Sundays
after his shop was shut, sent him for a walk on the quay to look at
the boats, and then brought him back to college at seven o'clock before
supper. Every Thursday evening he wrote a long letter to his mother with
red ink and three wafers; then he went over his history note-books, or
read an old volume of "Anarchasis" that was knocking about the study.
When he went for walks he talked to the servant, who, like himself, came
from the country.

     *In place of a parent.

By dint of hard work he kept always about the middle of the class; once
even he got a certificate in natural history. But at the end of his
third year his parents withdrew him from the school to make him study
medicine, convinced that he could even take his degree by himself.

His mother chose a room for him on the fourth floor of a dyer's she
knew, overlooking the Eau-de-Robec. She made arrangements for his
board, got him furniture, table and two chairs, sent home for an old
cherry-tree bedstead, and bought besides a small cast-iron stove with
the supply of wood that was to warm the poor child.

Then at the end of a week she departed, after a thousand injunctions to
be good now that he was going to be left to himself.

The syllabus that he read on the notice-board stunned him; lectures
on anatomy, lectures on pathology, lectures on physiology, lectures on
pharmacy, lectures on botany and clinical medicine, and therapeutics,
without counting hygiene and materia medica--all names of whose
etymologies he was ignorant, and that were to him as so many doors to
sanctuaries filled with magnificent darkness.

He understood nothing of it all; it was all very well to listen--he did
not follow. Still he worked; he had bound note-books, he attended all
the courses, never missed a single lecture. He did his little daily task
like a mill-horse, who goes round and round with his eyes bandaged, not
knowing what work he is doing.

To spare him expense his mother sent him every week by the carrier a
piece of veal baked in the oven, with which he lunched when he came back
from the hospital, while he sat kicking his feet against the wall.
After this he had to run off to lectures, to the operation-room, to the
hospital, and return to his home at the other end of the town. In the
evening, after the poor dinner of his landlord, he went back to his
room and set to work again in his wet clothes, which smoked as he sat in
front of the hot stove.

On the fine summer evenings, at the time when the close streets are
empty, when the servants are playing shuttle-cock at the doors, he
opened his window and leaned out. The river, that makes of this quarter
of Rouen a wretched little Venice, flowed beneath him, between the
bridges and the railings, yellow, violet, or blue. Working men, kneeling
on the banks, washed their bare arms in the water. On poles projecting
from the attics, skeins of cotton were drying in the air. Opposite,
beyond the roots spread the pure heaven with the red sun setting. How
pleasant it must be at home! How fresh under the beech-tree! And he
expanded his nostrils to breathe in the sweet odours of the country
which did not reach him.

He grew thin, his figure became taller, his face took a saddened look
that made it nearly interesting. Naturally, through indifference, he
abandoned all the resolutions he had made. Once he missed a lecture; the
next day all the lectures; and, enjoying his idleness, little by little,
he gave up work altogether. He got into the habit of going to the
public-house, and had a passion for dominoes. To shut himself up every
evening in the dirty public room, to push about on marble tables the
small sheep bones with black dots, seemed to him a fine proof of his
freedom, which raised him in his own esteem. It was beginning to see
life, the sweetness of stolen pleasures; and when he entered, he put
his hand on the door-handle with a joy almost sensual. Then many things
hidden within him came out; he learnt couplets by heart and sang them to
his boon companions, became enthusiastic about Beranger, learnt how to
make punch, and, finally, how to make love.

Thanks to these preparatory labours, he failed completely in his
examination for an ordinary degree. He was expected home the same night
to celebrate his success. He started on foot, stopped at the beginning
of the village, sent for his mother, and told her all. She excused
him, threw the blame of his failure on the injustice of the examiners,
encouraged him a little, and took upon herself to set matters straight.
It was only five years later that Monsieur Bovary knew the truth; it was
old then, and he accepted it. Moreover, he could not believe that a man
born of him could be a fool.

So Charles set to work again and crammed for his examination,
ceaselessly learning all the old questions by heart. He passed pretty
well. What a happy day for his mother! They gave a grand dinner.

Where should he go to practice? To Tostes, where there was only one old
doctor. For a long time Madame Bovary had been on the look-out for his
death, and the old fellow had barely been packed off when Charles was
installed, opposite his place, as his successor.

But it was not everything to have brought up a son, to have had him
taught medicine, and discovered Tostes, where he could practice it;
he must have a wife. She found him one--the widow of a bailiff at
Dieppe--who was forty-five and had an income of twelve hundred francs.
Though she was ugly, as dry as a bone, her face with as many pimples as
the spring has buds, Madame Dubuc had no lack of suitors. To attain her
ends Madame Bovary had to oust them all, and she even succeeded in
very cleverly baffling the intrigues of a port-butcher backed up by the
priests.

Charles had seen in marriage the advent of an easier life, thinking he
would be more free to do as he liked with himself and his money. But his
wife was master; he had to say this and not say that in company, to fast
every Friday, dress as she liked, harass at her bidding those patients
who did not pay. She opened his letter, watched his comings and goings,
and listened at the partition-wall when women came to consult him in his
surgery.

She must have her chocolate every morning, attentions without end. She
constantly complained of her nerves, her chest, her liver. The noise of
footsteps made her ill; when people left her, solitude became odious to
her; if they came back, it was doubtless to see her die. When Charles
returned in the evening, she stretched forth two long thin arms from
beneath the sheets, put them round his neck, and having made him sit
down on the edge of the bed, began to talk to him of her troubles: he
was neglecting her, he loved another. She had been warned she would be
unhappy; and she ended by asking him for a dose of medicine and a little
more love.



\chapter*{Chapter Two}

One night towards eleven o'clock they were awakened by the noise of
a horse pulling up outside their door. The servant opened the
garret-window and parleyed for some time with a man in the street below.
He came for the doctor, had a letter for him. Natasie came downstairs
shivering and undid the bars and bolts one after the other. The man left
his horse, and, following the servant, suddenly came in behind her. He
pulled out from his wool cap with grey top-knots a letter wrapped up in
a rag and presented it gingerly to Charles, who rested on his elbow on
the pillow to read it. Natasie, standing near the bed, held the light.
Madame in modesty had turned to the wall and showed only her back.

This letter, sealed with a small seal in blue wax, begged Monsieur
Bovary to come immediately to the farm of the Bertaux to set a broken
leg. Now from Tostes to the Bertaux was a good eighteen miles across
country by way of Longueville and Saint-Victor. It was a dark night;
Madame Bovary junior was afraid of accidents for her husband. So it was
decided the stable-boy should go on first; Charles would start three
hours later when the moon rose. A boy was to be sent to meet him, and
show him the way to the farm, and open the gates for him.

Towards four o'clock in the morning, Charles, well wrapped up in his
cloak, set out for the Bertaux. Still sleepy from the warmth of his bed,
he let himself be lulled by the quiet trot of his horse. When it stopped
of its own accord in front of those holes surrounded with thorns that
are dug on the margin of furrows, Charles awoke with a start, suddenly
remembered the broken leg, and tried to call to mind all the fractures
he knew. The rain had stopped, day was breaking, and on the branches
of the leafless trees birds roosted motionless, their little feathers
bristling in the cold morning wind. The flat country stretched as far as
eye could see, and the tufts of trees round the farms at long intervals
seemed like dark violet stains on the cast grey surface, that on the
horizon faded into the gloom of the sky.

Charles from time to time opened his eyes, his mind grew weary, and,
sleep coming upon him, he soon fell into a doze wherein, his recent
sensations blending with memories, he became conscious of a double
self, at once student and married man, lying in his bed as but now, and
crossing the operation theatre as of old. The warm smell of poultices
mingled in his brain with the fresh odour of dew; he heard the iron
rings rattling along the curtain-rods of the bed and saw his wife
sleeping. As he passed Vassonville he came upon a boy sitting on the
grass at the edge of a ditch.

"Are you the doctor?" asked the child.

And on Charles's answer he took his wooden shoes in his hands and ran on
in front of him.

The general practitioner, riding along, gathered from his guide's talk
that Monsieur Rouault must be one of the well-to-do farmers.

He had broken his leg the evening before on his way home from a
Twelfth-night feast at a neighbour's. His wife had been dead for two
years. There was with him only his daughter, who helped him to keep
house.

The ruts were becoming deeper; they were approaching the Bertaux.

The little lad, slipping through a hole in the hedge, disappeared;
then he came back to the end of a courtyard to open the gate. The
horse slipped on the wet grass; Charles had to stoop to pass under
the branches. The watchdogs in their kennels barked, dragging at their
chains. As he entered the Bertaux, the horse took fright and stumbled.

It was a substantial-looking farm. In the stables, over the top of the
open doors, one could see great cart-horses quietly feeding from new
racks. Right along the outbuildings extended a large dunghill, from
which manure liquid oozed, while amidst fowls and turkeys, five or six
peacocks, a luxury in Chauchois farmyards, were foraging on the top of
it. The sheepfold was long, the barn high, with walls smooth as your
hand. Under the cart-shed were two large carts and four ploughs, with
their whips, shafts and harnesses complete, whose fleeces of blue wool
were getting soiled by the fine dust that fell from the granaries. The
courtyard sloped upwards, planted with trees set out symmetrically, and
the chattering noise of a flock of geese was heard near the pond.

A young woman in a blue merino dress with three flounces came to the
threshold of the door to receive Monsieur Bovary, whom she led to the
kitchen, where a large fire was blazing. The servant's breakfast was
boiling beside it in small pots of all sizes. Some damp clothes were
drying inside the chimney-corner. The shovel, tongs, and the nozzle
of the bellows, all of colossal size, shone like polished steel, while
along the walls hung many pots and pans in which the clear flame of the
hearth, mingling with the first rays of the sun coming in through the
window, was mirrored fitfully.

Charles went up the first floor to see the patient. He found him in his
bed, sweating under his bed-clothes, having thrown his cotton nightcap
right away from him. He was a fat little man of fifty, with white skin
and blue eyes, the forepart of his head bald, and he wore earrings. By
his side on a chair stood a large decanter of brandy, whence he poured
himself a little from time to time to keep up his spirits; but as soon
as he caught sight of the doctor his elation subsided, and instead of
swearing, as he had been doing for the last twelve hours, began to groan
freely.

The fracture was a simple one, without any kind of complication.

Charles could not have hoped for an easier case. Then calling to mind
the devices of his masters at the bedsides of patients, he comforted the
sufferer with all sorts of kindly remarks, those Caresses of the surgeon
that are like the oil they put on bistouries. In order to make some
splints a bundle of laths was brought up from the cart-house. Charles
selected one, cut it into two pieces and planed it with a fragment
of windowpane, while the servant tore up sheets to make bandages, and
Mademoiselle Emma tried to sew some pads. As she was a long time before
she found her work-case, her father grew impatient; she did not answer,
but as she sewed she pricked her fingers, which she then put to her
mouth to suck them. Charles was surprised at the whiteness of her nails.
They were shiny, delicate at the tips, more polished than the ivory of
Dieppe, and almond-shaped. Yet her hand was not beautiful, perhaps not
white enough, and a little hard at the knuckles; besides, it was too
long, with no soft inflections in the outlines. Her real beauty was in
her eyes. Although brown, they seemed black because of the lashes, and
her look came at you frankly, with a candid boldness.

The bandaging over, the doctor was invited by Monsieur Rouault himself
to "pick a bit" before he left.

Charles went down into the room on the ground floor. Knives and forks
and silver goblets were laid for two on a little table at the foot of a
huge bed that had a canopy of printed cotton with figures representing
Turks. There was an odour of iris-root and damp sheets that escaped
from a large oak chest opposite the window. On the floor in corners were
sacks of flour stuck upright in rows. These were the overflow from
the neighbouring granary, to which three stone steps led. By way of
decoration for the apartment, hanging to a nail in the middle of the
wall, whose green paint scaled off from the effects of the saltpetre,
was a crayon head of Minerva in gold frame, underneath which was written
in Gothic letters "To dear Papa."

First they spoke of the patient, then of the weather, of the great cold,
of the wolves that infested the fields at night.

Mademoiselle Rouault did not at all like the country, especially now
that she had to look after the farm almost alone. As the room was
chilly, she shivered as she ate. This showed something of her full lips,
that she had a habit of biting when silent.

Her neck stood out from a white turned-down collar. Her hair, whose
two black folds seemed each of a single piece, so smooth were they, was
parted in the middle by a delicate line that curved slightly with the
curve of the head; and, just showing the tip of the ear, it was joined
behind in a thick chignon, with a wavy movement at the temples that the
country doctor saw now for the first time in his life. The upper part of
her cheek was rose-coloured. She had, like a man, thrust in between two
buttons of her bodice a tortoise-shell eyeglass.

When Charles, after bidding farewell to old Rouault, returned to the
room before leaving, he found her standing, her forehead against the
window, looking into the garden, where the bean props had been knocked
down by the wind. She turned round. "Are you looking for anything?" she
asked.

"My whip, if you please," he answered.

He began rummaging on the bed, behind the doors, under the chairs. It
had fallen to the floor, between the sacks and the wall. Mademoiselle
Emma saw it, and bent over the flour sacks.

Charles out of politeness made a dash also, and as he stretched out his
arm, at the same moment felt his breast brush against the back of the
young girl bending beneath him. She drew herself up, scarlet, and looked
at him over her shoulder as she handed him his whip.

Instead of returning to the Bertaux in three days as he had promised,
he went back the very next day, then regularly twice a week, without
counting the visits he paid now and then as if by accident.

Everything, moreover, went well; the patient progressed favourably; and
when, at the end of forty-six days, old Rouault was seen trying to walk
alone in his "den," Monsieur Bovary began to be looked upon as a man
of great capacity. Old Rouault said that he could not have been cured
better by the first doctor of Yvetot, or even of Rouen.

As to Charles, he did not stop to ask himself why it was a pleasure
to him to go to the Bertaux. Had he done so, he would, no doubt, have
attributed his zeal to the importance of the case, or perhaps to the
money he hoped to make by it. Was it for this, however, that his visits
to the farm formed a delightful exception to the meagre occupations of
his life? On these days he rose early, set off at a gallop, urging on
his horse, then got down to wipe his boots in the grass and put on black
gloves before entering. He liked going into the courtyard, and noticing
the gate turn against his shoulder, the cock crow on the wall, the lads
run to meet him. He liked the granary and the stables; he liked old
Rouault, who pressed his hand and called him his saviour; he like the
small wooden shoes of Mademoiselle Emma on the scoured flags of the
kitchen--her high heels made her a little taller; and when she walked in
front of him, the wooden soles springing up quickly struck with a sharp
sound against the leather of her boots.

She always accompanied him to the first step of the stairs. When his
horse had not yet been brought round she stayed there. They had said
"Good-bye"; there was no more talking. The open air wrapped her round,
playing with the soft down on the back of her neck, or blew to and fro
on her hips the apron-strings, that fluttered like streamers. Once,
during a thaw the bark of the trees in the yard was oozing, the snow on
the roofs of the outbuildings was melting; she stood on the threshold,
and went to fetch her sunshade and opened it. The sunshade of silk of
the colour of pigeons' breasts, through which the sun shone, lighted
up with shifting hues the white skin of her face. She smiled under the
tender warmth, and drops of water could be heard falling one by one on
the stretched silk.

During the first period of Charles's visits to the Bertaux, Madame
Bovary junior never failed to inquire after the invalid, and she had
even chosen in the book that she kept on a system of double entry a
clean blank page for Monsieur Rouault. But when she heard he had a
daughter, she began to make inquiries, and she learnt the Mademoiselle
Rouault, brought up at the Ursuline Convent, had received what is called
"a good education"; and so knew dancing, geography, drawing, how to
embroider and play the piano. That was the last straw.

"So it is for this," she said to herself, "that his face beams when he
goes to see her, and that he puts on his new waistcoat at the risk of
spoiling it with the rain. Ah! that woman! That woman!"

And she detested her instinctively. At first she solaced herself by
allusions that Charles did not understand, then by casual observations
that he let pass for fear of a storm, finally by open apostrophes to
which he knew not what to answer. "Why did he go back to the Bertaux now
that Monsieur Rouault was cured and that these folks hadn't paid yet?
Ah! it was because a young lady was there, some one who know how to
talk, to embroider, to be witty. That was what he cared about; he wanted
town misses." And she went on--

"The daughter of old Rouault a town miss! Get out! Their grandfather was
a shepherd, and they have a cousin who was almost had up at the assizes
for a nasty blow in a quarrel. It is not worth while making such a fuss,
or showing herself at church on Sundays in a silk gown like a countess.
Besides, the poor old chap, if it hadn't been for the colza last year,
would have had much ado to pay up his arrears."

For very weariness Charles left off going to the Bertaux. Heloise made
him swear, his hand on the prayer-book, that he would go there no more
after much sobbing and many kisses, in a great outburst of love. He
obeyed then, but the strength of his desire protested against the
servility of his conduct; and he thought, with a kind of naive
hypocrisy, that his interdict to see her gave him a sort of right to
love her. And then the widow was thin; she had long teeth; wore in all
weathers a little black shawl, the edge of which hung down between her
shoulder-blades; her bony figure was sheathed in her clothes as if they
were a scabbard; they were too short, and displayed her ankles with the
laces of her large boots crossed over grey stockings.

Charles's mother came to see them from time to time, but after a few
days the daughter-in-law seemed to put her own edge on her, and
then, like two knives, they scarified him with their reflections and
observations. It was wrong of him to eat so much.

Why did he always offer a glass of something to everyone who came?
What obstinacy not to wear flannels! In the spring it came about that a
notary at Ingouville, the holder of the widow Dubuc's property, one fine
day went off, taking with him all the money in his office. Heloise,
it is true, still possessed, besides a share in a boat valued at six
thousand francs, her house in the Rue St. Francois; and yet, with all
this fortune that had been so trumpeted abroad, nothing, excepting
perhaps a little furniture and a few clothes, had appeared in the
household. The matter had to be gone into. The house at Dieppe was found
to be eaten up with mortgages to its foundations; what she had placed
with the notary God only knew, and her share in the boat did not exceed
one thousand crowns. She had lied, the good lady! In his exasperation,
Monsieur Bovary the elder, smashing a chair on the flags, accused his
wife of having caused misfortune to the son by harnessing him to such
a harridan, whose harness wasn't worth her hide. They came to Tostes.
Explanations followed. There were scenes. Heloise in tears, throwing her
arms about her husband, implored him to defend her from his parents.

Charles tried to speak up for her. They grew angry and left the house.

But "the blow had struck home." A week after, as she was hanging up some
washing in her yard, she was seized with a spitting of blood, and
the next day, while Charles had his back turned to her drawing the
window-curtain, she said, "O God!" gave a sigh and fainted. She was
dead! What a surprise! When all was over at the cemetery Charles went
home. He found no one downstairs; he went up to the first floor to
their room; saw her dress still hanging at the foot of the alcove; then,
leaning against the writing-table, he stayed until the evening, buried
in a sorrowful reverie. She had loved him after all!



\chapter*{Chapter Three}

One morning old Rouault brought Charles the money for setting his
leg--seventy-five francs in forty-sou pieces, and a turkey. He had heard
of his loss, and consoled him as well as he could.

"I know what it is," said he, clapping him on the shoulder; "I've been
through it. When I lost my dear departed, I went into the fields to be
quite alone. I fell at the foot of a tree; I cried; I called on God; I
talked nonsense to Him. I wanted to be like the moles that I saw on the
branches, their insides swarming with worms, dead, and an end of it.
And when I thought that there were others at that very moment with their
nice little wives holding them in their embrace, I struck great blows on
the earth with my stick. I was pretty well mad with not eating; the very
idea of going to a cafe disgusted me--you wouldn't believe it. Well,
quite softly, one day following another, a spring on a winter, and an
autumn after a summer, this wore away, piece by piece, crumb by crumb;
it passed away, it is gone, I should say it has sunk; for something
always remains at the bottom as one would say--a weight here, at one's
heart. But since it is the lot of all of us, one must not give way
altogether, and, because others have died, want to die too. You must
pull yourself together, Monsieur Bovary. It will pass away. Come to see
us; my daughter thinks of you now and again, d'ye know, and she says
you are forgetting her. Spring will soon be here. We'll have some
rabbit-shooting in the warrens to amuse you a bit."

Charles followed his advice. He went back to the Bertaux. He found all
as he had left it, that is to say, as it was five months ago. The pear
trees were already in blossom, and Farmer Rouault, on his legs again,
came and went, making the farm more full of life.

Thinking it his duty to heap the greatest attention upon the doctor
because of his sad position, he begged him not to take his hat off,
spoke to him in an undertone as if he had been ill, and even pretended
to be angry because nothing rather lighter had been prepared for him
than for the others, such as a little clotted cream or stewed pears. He
told stories. Charles found himself laughing, but the remembrance of his
wife suddenly coming back to him depressed him. Coffee was brought in;
he thought no more about her.

He thought less of her as he grew accustomed to living alone. The new
delight of independence soon made his loneliness bearable. He could now
change his meal-times, go in or out without explanation, and when he was
very tired stretch himself at full length on his bed. So he nursed and
coddled himself and accepted the consolations that were offered him.
On the other hand, the death of his wife had not served him ill in his
business, since for a month people had been saying, "The poor young
man! what a loss!" His name had been talked about, his practice had
increased; and moreover, he could go to the Bertaux just as he liked.
He had an aimless hope, and was vaguely happy; he thought himself better
looking as he brushed his whiskers before the looking-glass.

One day he got there about three o'clock. Everybody was in the fields.
He went into the kitchen, but did not at once catch sight of Emma; the
outside shutters were closed. Through the chinks of the wood the sun
sent across the flooring long fine rays that were broken at the corners
of the furniture and trembled along the ceiling. Some flies on the table
were crawling up the glasses that had been used, and buzzing as they
drowned themselves in the dregs of the cider. The daylight that came in
by the chimney made velvet of the soot at the back of the fireplace, and
touched with blue the cold cinders. Between the window and the hearth
Emma was sewing; she wore no fichu; he could see small drops of
perspiration on her bare shoulders.

After the fashion of country folks she asked him to have something to
drink. He said no; she insisted, and at last laughingly offered to have
a glass of liqueur with him. So she went to fetch a bottle of curacao
from the cupboard, reached down two small glasses, filled one to the
brim, poured scarcely anything into the other, and, after having clinked
glasses, carried hers to her mouth. As it was almost empty she bent
back to drink, her head thrown back, her lips pouting, her neck on the
strain. She laughed at getting none of it, while with the tip of her
tongue passing between her small teeth she licked drop by drop the
bottom of her glass.

She sat down again and took up her work, a white cotton stocking she was
darning. She worked with her head bent down; she did not speak, nor did
Charles. The air coming in under the door blew a little dust over the
flags; he watched it drift along, and heard nothing but the throbbing
in his head and the faint clucking of a hen that had laid an egg in the
yard. Emma from time to time cooled her cheeks with the palms of her
hands, and cooled these again on the knobs of the huge fire-dogs.

She complained of suffering since the beginning of the season from
giddiness; she asked if sea-baths would do her any good; she began
talking of her convent, Charles of his school; words came to them. They
went up into her bedroom. She showed him her old music-books, the little
prizes she had won, and the oak-leaf crowns, left at the bottom of a
cupboard. She spoke to him, too, of her mother, of the country, and even
showed him the bed in the garden where, on the first Friday of every
month, she gathered flowers to put on her mother's tomb. But the
gardener they had never knew anything about it; servants are so stupid!
She would have dearly liked, if only for the winter, to live in town,
although the length of the fine days made the country perhaps even more
wearisome in the summer. And, according to what she was saying, her
voice was clear, sharp, or, on a sudden all languor, drawn out in
modulations that ended almost in murmurs as she spoke to herself, now
joyous, opening big naive eyes, then with her eyelids half closed, her
look full of boredom, her thoughts wandering.

Going home at night, Charles went over her words one by one, trying to
recall them, to fill out their sense, that he might piece out the life
she had lived before he knew her. But he never saw her in his thoughts
other than he had seen her the first time, or as he had just left her.
Then he asked himself what would become of her--if she would be married,
and to whom! Alas! Old Rouault was rich, and she!--so beautiful! But
Emma's face always rose before his eyes, and a monotone, like the
humming of a top, sounded in his ears, "If you should marry after
all! If you should marry!" At night he could not sleep; his throat was
parched; he was athirst. He got up to drink from the water-bottle and
opened the window. The night was covered with stars, a warm wind blowing
in the distance; the dogs were barking. He turned his head towards the
Bertaux.

Thinking that, after all, he should lose nothing, Charles promised
himself to ask her in marriage as soon as occasion offered, but each
time such occasion did offer the fear of not finding the right words
sealed his lips.

Old Rouault would not have been sorry to be rid of his daughter, who was
of no use to him in the house. In his heart he excused her, thinking
her too clever for farming, a calling under the ban of Heaven, since one
never saw a millionaire in it. Far from having made a fortune by it,
the good man was losing every year; for if he was good in bargaining, in
which he enjoyed the dodges of the trade, on the other hand, agriculture
properly so called, and the internal management of the farm, suited him
less than most people. He did not willingly take his hands out of his
pockets, and did not spare expense in all that concerned himself, liking
to eat well, to have good fires, and to sleep well. He liked old cider,
underdone legs of mutton, glorias* well beaten up. He took his meals in
the kitchen alone, opposite the fire, on a little table brought to him
all ready laid as on the stage.

     *A mixture of coffee and spirits.

When, therefore, he perceived that Charles's cheeks grew red if near his
daughter, which meant that he would propose for her one of these days,
he chewed the cud of the matter beforehand. He certainly thought him a
little meagre, and not quite the son-in-law he would have liked, but he
was said to be well brought-up, economical, very learned, and no doubt
would not make too many difficulties about the dowry. Now, as old
Rouault would soon be forced to sell twenty-two acres of "his property,"
as he owed a good deal to the mason, to the harness-maker, and as the
shaft of the cider-press wanted renewing, "If he asks for her," he said
to himself, "I'll give her to him."

At Michaelmas Charles went to spend three days at the Bertaux.

The last had passed like the others in procrastinating from hour to
hour. Old Rouault was seeing him off; they were walking along the road
full of ruts; they were about to part. This was the time. Charles gave
himself as far as to the corner of the hedge, and at last, when past
it--

"Monsieur Rouault," he murmured, "I should like to say something to
you."

They stopped. Charles was silent.

"Well, tell me your story. Don't I know all about it?" said old Rouault,
laughing softly.

"Monsieur Rouault--Monsieur Rouault," stammered Charles.

"I ask nothing better", the farmer went on. "Although, no doubt, the
little one is of my mind, still we must ask her opinion. So you get
off--I'll go back home. If it is 'yes', you needn't return because of
all the people about, and besides it would upset her too much. But so
that you mayn't be eating your heart, I'll open wide the outer shutter
of the window against the wall; you can see it from the back by leaning
over the hedge."

And he went off.

Charles fastened his horse to a tree; he ran into the road and waited.
Half an hour passed, then he counted nineteen minutes by his watch.
Suddenly a noise was heard against the wall; the shutter had been thrown
back; the hook was still swinging.

The next day by nine o'clock he was at the farm. Emma blushed as
he entered, and she gave a little forced laugh to keep herself in
countenance. Old Rouault embraced his future son-in-law. The discussion
of money matters was put off; moreover, there was plenty of time before
them, as the marriage could not decently take place till Charles was out
of mourning, that is to say, about the spring of the next year.

The winter passed waiting for this. Mademoiselle Rouault was busy with
her trousseau. Part of it was ordered at Rouen, and she made herself
chemises and nightcaps after fashion-plates that she borrowed. When
Charles visited the farmer, the preparations for the wedding were talked
over; they wondered in what room they should have dinner; they dreamed
of the number of dishes that would be wanted, and what should be
entrees.

Emma would, on the contrary, have preferred to have a midnight wedding
with torches, but old Rouault could not understand such an idea. So
there was a wedding at which forty-three persons were present, at which
they remained sixteen hours at table, began again the next day, and to
some extent on the days following.



\chapter{Chapter Four}

The guests arrived early in carriages, in one-horse chaises, two-wheeled
cars, old open gigs, waggonettes with leather hoods, and the young
people from the nearer villages in carts, in which they stood up in
rows, holding on to the sides so as not to fall, going at a trot
and well shaken up. Some came from a distance of thirty miles, from
Goderville, from Normanville, and from Cany.

All the relatives of both families had been invited, quarrels between
friends arranged, acquaintances long since lost sight of written to.

From time to time one heard the crack of a whip behind the hedge; then
the gates opened, a chaise entered. Galloping up to the foot of the
steps, it stopped short and emptied its load. They got down from all
sides, rubbing knees and stretching arms. The ladies, wearing bonnets,
had on dresses in the town fashion, gold watch chains, pelerines with
the ends tucked into belts, or little coloured fichus fastened down
behind with a pin, and that left the back of the neck bare. The lads,
dressed like their papas, seemed uncomfortable in their new clothes
(many that day hand-sewed their first pair of boots), and by their
sides, speaking never a work, wearing the white dress of their first
communion lengthened for the occasion were some big girls of fourteen or
sixteen, cousins or elder sisters no doubt, rubicund, bewildered, their
hair greasy with rose pomade, and very much afraid of dirtying their
gloves. As there were not enough stable-boys to unharness all the
carriages, the gentlemen turned up their sleeves and set about it
themselves. According to their different social positions they wore
tail-coats, overcoats, shooting jackets, cutaway-coats; fine tail-coats,
redolent of family respectability, that only came out of the wardrobe
on state occasions; overcoats with long tails flapping in the wind and
round capes and pockets like sacks; shooting jackets of coarse
cloth, generally worn with a cap with a brass-bound peak; very short
cutaway-coats with two small buttons in the back, close together like
a pair of eyes, and the tails of which seemed cut out of one piece by a
carpenter's hatchet. Some, too (but these, you may be sure, would sit at
the bottom of the table), wore their best blouses--that is to say,
with collars turned down to the shoulders, the back gathered into small
plaits and the waist fastened very low down with a worked belt.

And the shirts stood out from the chests like cuirasses! Everyone had
just had his hair cut; ears stood out from the heads; they had been
close-shaved; a few, even, who had had to get up before daybreak, and
not been able to see to shave, had diagonal gashes under their noses or
cuts the size of a three-franc piece along the jaws, which the fresh
air en route had enflamed, so that the great white beaming faces were
mottled here and there with red dabs.

The mairie was a mile and a half from the farm, and they went thither
on foot, returning in the same way after the ceremony in the church.
The procession, first united like one long coloured scarf that undulated
across the fields, along the narrow path winding amid the green corn,
soon lengthened out, and broke up into different groups that loitered to
talk. The fiddler walked in front with his violin, gay with ribbons at
its pegs. Then came the married pair, the relations, the friends, all
following pell-mell; the children stayed behind amusing themselves
plucking the bell-flowers from oat-ears, or playing amongst themselves
unseen. Emma's dress, too long, trailed a little on the ground; from
time to time she stopped to pull it up, and then delicately, with her
gloved hands, she picked off the coarse grass and the thistledowns,
while Charles, empty handed, waited till she had finished. Old Rouault,
with a new silk hat and the cuffs of his black coat covering his hands
up to the nails, gave his arm to Madame Bovary senior. As to Monsieur
Bovary senior, who, heartily despising all these folk, had come simply
in a frock-coat of military cut with one row of buttons--he was passing
compliments of the bar to a fair young peasant. She bowed, blushed,
and did not know what to say. The other wedding guests talked of their
business or played tricks behind each other's backs, egging one another
on in advance to be jolly. Those who listened could always catch the
squeaking of the fiddler, who went on playing across the fields. When
he saw that the rest were far behind he stopped to take breath, slowly
rosined his bow, so that the strings should sound more shrilly, then set
off again, by turns lowering and raising his neck, the better to mark
time for himself. The noise of the instrument drove away the little
birds from afar.

The table was laid under the cart-shed. On it were four sirloins, six
chicken fricassees, stewed veal, three legs of mutton, and in the middle
a fine roast suckling pig, flanked by four chitterlings with sorrel. At
the corners were decanters of brandy. Sweet bottled-cider frothed round
the corks, and all the glasses had been filled to the brim with wine
beforehand. Large dishes of yellow cream, that trembled with the least
shake of the table, had designed on their smooth surface the initials of
the newly wedded pair in nonpareil arabesques. A confectioner of Yvetot
had been intrusted with the tarts and sweets. As he had only just set up
on the place, he had taken a lot of trouble, and at dessert he himself
brought in a set dish that evoked loud cries of wonderment. To begin
with, at its base there was a square of blue cardboard, representing a
temple with porticoes, colonnades, and stucco statuettes all round, and
in the niches constellations of gilt paper stars; then on the second
stage was a dungeon of Savoy cake, surrounded by many fortifications
in candied angelica, almonds, raisins, and quarters of oranges; and
finally, on the upper platform a green field with rocks set in lakes of
jam, nutshell boats, and a small Cupid balancing himself in a chocolate
swing whose two uprights ended in real roses for balls at the top.

Until night they ate. When any of them were too tired of sitting, they
went out for a stroll in the yard, or for a game with corks in the
granary, and then returned to table. Some towards the finish went to
sleep and snored. But with the coffee everyone woke up. Then they began
songs, showed off tricks, raised heavy weights, performed feats with
their fingers, then tried lifting carts on their shoulders, made broad
jokes, kissed the women. At night when they left, the horses, stuffed
up to the nostrils with oats, could hardly be got into the shafts; they
kicked, reared, the harness broke, their masters laughed or swore;
and all night in the light of the moon along country roads there were
runaway carts at full gallop plunging into the ditches, jumping over
yard after yard of stones, clambering up the hills, with women leaning
out from the tilt to catch hold of the reins.

Those who stayed at the Bertaux spent the night drinking in the kitchen.
The children had fallen asleep under the seats.

The bride had begged her father to be spared the usual marriage
pleasantries. However, a fishmonger, one of their cousins (who had even
brought a pair of soles for his wedding present), began to squirt water
from his mouth through the keyhole, when old Rouault came up just in
time to stop him, and explain to him that the distinguished position
of his son-in-law would not allow of such liberties. The cousin all the
same did not give in to these reasons readily. In his heart he accused
old Rouault of being proud, and he joined four or five other guests in
a corner, who having, through mere chance, been several times running
served with the worst helps of meat, also were of opinion they had been
badly used, and were whispering about their host, and with covered hints
hoping he would ruin himself.

Madame Bovary, senior, had not opened her mouth all day. She had been
consulted neither as to the dress of her daughter-in-law nor as to the
arrangement of the feast; she went to bed early. Her husband, instead
of following her, sent to Saint-Victor for some cigars, and smoked till
daybreak, drinking kirsch-punch, a mixture unknown to the company. This
added greatly to the consideration in which he was held.

Charles, who was not of a facetious turn, did not shine at the wedding.
He answered feebly to the puns, doubles entendres*, compliments, and
chaff that it was felt a duty to let off at him as soon as the soup
appeared.

     *Double meanings.

The next day, on the other hand, he seemed another man. It was he who
might rather have been taken for the virgin of the evening before,
whilst the bride gave no sign that revealed anything. The shrewdest did
not know what to make of it, and they looked at her when she passed
near them with an unbounded concentration of mind. But Charles concealed
nothing. He called her "my wife", tutoyed* her, asked for her of
everyone, looked for her everywhere, and often he dragged her into the
yards, where he could be seen from far between the trees, putting his
arm around her waist, and walking half-bending over her, ruffling the
chemisette of her bodice with his head.

     *Used the familiar form of address.

Two days after the wedding the married pair left. Charles, on account of
his patients, could not be away longer. Old Rouault had them driven back
in his cart, and himself accompanied them as far as Vassonville. Here
he embraced his daughter for the last time, got down, and went his way.
When he had gone about a hundred paces he stopped, and as he saw the
cart disappearing, its wheels turning in the dust, he gave a deep sigh.
Then he remembered his wedding, the old times, the first pregnancy of
his wife; he, too, had been very happy the day when he had taken her
from her father to his home, and had carried her off on a pillion,
trotting through the snow, for it was near Christmas-time, and the
country was all white. She held him by one arm, her basket hanging from
the other; the wind blew the long lace of her Cauchois headdress so that
it sometimes flapped across his mouth, and when he turned his head he
saw near him, on his shoulder, her little rosy face, smiling silently
under the gold bands of her cap. To warm her hands she put them from
time to time in his breast. How long ago it all was! Their son would
have been thirty by now. Then he looked back and saw nothing on the
road. He felt dreary as an empty house; and tender memories mingling
with the sad thoughts in his brain, addled by the fumes of the feast, he
felt inclined for a moment to take a turn towards the church. As he was
afraid, however, that this sight would make him yet more sad, he went
right away home.

Monsieur and Madame Charles arrived at Tostes about six o'clock.

The neighbors came to the windows to see their doctor's new wife.

The old servant presented herself, curtsied to her, apologised for not
having dinner ready, and suggested that madame, in the meantime, should
look over her house.



\chapter{Chapter Five}

The brick front was just in a line with the street, or rather the road.
Behind the door hung a cloak with a small collar, a bridle, and a black
leather cap, and on the floor, in a corner, were a pair of leggings,
still covered with dry mud. On the right was the one apartment, that was
both dining and sitting room. A canary yellow paper, relieved at the
top by a garland of pale flowers, was puckered everywhere over the badly
stretched canvas; white calico curtains with a red border hung crossways
at the length of the window; and on the narrow mantelpiece a clock with
a head of Hippocrates shone resplendent between two plate candlesticks
under oval shades. On the other side of the passage was Charles's
consulting room, a little room about six paces wide, with a table,
three chairs, and an office chair. Volumes of the "Dictionary of Medical
Science," uncut, but the binding rather the worse for the successive
sales through which they had gone, occupied almost along the six shelves
of a deal bookcase.

The smell of melted butter penetrated through the walls when he saw
patients, just as in the kitchen one could hear the people coughing in
the consulting room and recounting their histories.

Then, opening on the yard, where the stable was, came a large
dilapidated room with a stove, now used as a wood-house, cellar, and
pantry, full of old rubbish, of empty casks, agricultural implements
past service, and a mass of dusty things whose use it was impossible to
guess.

The garden, longer than wide, ran between two mud walls with espaliered
apricots, to a hawthorn hedge that separated it from the field. In the
middle was a slate sundial on a brick pedestal; four flower beds with
eglantines surrounded symmetrically the more useful kitchen garden bed.
Right at the bottom, under the spruce bushes, was a cure in plaster
reading his breviary.

Emma went upstairs. The first room was not furnished, but in the second,
which was their bedroom, was a mahogany bedstead in an alcove with red
drapery. A shell box adorned the chest of drawers, and on the secretary
near the window a bouquet of orange blossoms tied with white satin
ribbons stood in a bottle. It was a bride's bouquet; it was the other
one's. She looked at it. Charles noticed it; he took it and carried it
up to the attic, while Emma seated in an arm-chair (they were putting
her things down around her) thought of her bridal flowers packed up in
a bandbox, and wondered, dreaming, what would be done with them if she
were to die.

During the first days she occupied herself in thinking about changes in
the house. She took the shades off the candlesticks, had new wallpaper
put up, the staircase repainted, and seats made in the garden round the
sundial; she even inquired how she could get a basin with a jet fountain
and fishes. Finally her husband, knowing that she liked to drive out,
picked up a second-hand dogcart, which, with new lamps and splashboard
in striped leather, looked almost like a tilbury.

He was happy then, and without a care in the world. A meal together,
a walk in the evening on the highroad, a gesture of her hands over her
hair, the sight of her straw hat hanging from the window-fastener, and
many another thing in which Charles had never dreamed of pleasure, now
made up the endless round of his happiness. In bed, in the morning, by
her side, on the pillow, he watched the sunlight sinking into the down
on her fair cheek, half hidden by the lappets of her night-cap. Seen
thus closely, her eyes looked to him enlarged, especially when, on
waking up, she opened and shut them rapidly many times. Black in the
shade, dark blue in broad daylight, they had, as it were, depths of
different colours, that, darker in the centre, grew paler towards the
surface of the eye. His own eyes lost themselves in these depths; he saw
himself in miniature down to the shoulders, with his handkerchief round
his head and the top of his shirt open. He rose. She came to the window
to see him off, and stayed leaning on the sill between two pots of
geranium, clad in her dressing gown hanging loosely about her. Charles,
in the street buckled his spurs, his foot on the mounting stone, while
she talked to him from above, picking with her mouth some scrap of
flower or leaf that she blew out at him. Then this, eddying, floating,
described semicircles in the air like a bird, and was caught before
it reached the ground in the ill-groomed mane of the old white mare
standing motionless at the door. Charles from horseback threw her a
kiss; she answered with a nod; she shut the window, and he set off. And
then along the highroad, spreading out its long ribbon of dust, along
the deep lanes that the trees bent over as in arbours, along paths where
the corn reached to the knees, with the sun on his back and the morning
air in his nostrils, his heart full of the joys of the past night, his
mind at rest, his flesh at ease, he went on, re-chewing his happiness,
like those who after dinner taste again the truffles which they are
digesting.

Until now what good had he had of his life? His time at school, when
he remained shut up within the high walls, alone, in the midst of
companions richer than he or cleverer at their work, who laughed at his
accent, who jeered at his clothes, and whose mothers came to the school
with cakes in their muffs? Later on, when he studied medicine, and never
had his purse full enough to treat some little work-girl who would have
become his mistress? Afterwards, he had lived fourteen months with the
widow, whose feet in bed were cold as icicles. But now he had for life
this beautiful woman whom he adored. For him the universe did not extend
beyond the circumference of her petticoat, and he reproached himself
with not loving her. He wanted to see her again; he turned back quickly,
ran up the stairs with a beating heart. Emma, in her room, was dressing;
he came up on tiptoe, kissed her back; she gave a cry.

He could not keep from constantly touching her comb, her ring, her
fichu; sometimes he gave her great sounding kisses with all his mouth on
her cheeks, or else little kisses in a row all along her bare arm
from the tip of her fingers up to her shoulder, and she put him away
half-smiling, half-vexed, as you do a child who hangs about you.

Before marriage she thought herself in love; but the happiness that
should have followed this love not having come, she must, she thought,
have been mistaken. And Emma tried to find out what one meant exactly in
life by the words felicity, passion, rapture, that had seemed to her so
beautiful in books.



\chapter{Chapter Six}

She had read "Paul and Virginia," and she had dreamed of the little
bamboo-house, the nigger Domingo, the dog Fidele, but above all of the
sweet friendship of some dear little brother, who seeks red fruit for
you on trees taller than steeples, or who runs barefoot over the sand,
bringing you a bird's nest.

When she was thirteen, her father himself took her to town to place
her in the convent. They stopped at an inn in the St. Gervais quarter,
where, at their supper, they used painted plates that set forth the
story of Mademoiselle de la Valliere. The explanatory legends, chipped
here and there by the scratching of knives, all glorified religion, the
tendernesses of the heart, and the pomps of court.

Far from being bored at first at the convent, she took pleasure in the
society of the good sisters, who, to amuse her, took her to the chapel,
which one entered from the refectory by a long corridor. She played very
little during recreation hours, knew her catechism well, and it was she
who always answered Monsieur le Vicaire's difficult questions. Living
thus, without every leaving the warm atmosphere of the classrooms, and
amid these pale-faced women wearing rosaries with brass crosses, she
was softly lulled by the mystic languor exhaled in the perfumes of the
altar, the freshness of the holy water, and the lights of the tapers.
Instead of attending to mass, she looked at the pious vignettes with
their azure borders in her book, and she loved the sick lamb, the sacred
heart pierced with sharp arrows, or the poor Jesus sinking beneath the
cross he carries. She tried, by way of mortification, to eat nothing a
whole day. She puzzled her head to find some vow to fulfil.

When she went to confession, she invented little sins in order that she
might stay there longer, kneeling in the shadow, her hands joined,
her face against the grating beneath the whispering of the priest.
The comparisons of betrothed, husband, celestial lover, and eternal
marriage, that recur in sermons, stirred within her soul depths of
unexpected sweetness.

In the evening, before prayers, there was some religious reading in
the study. On week-nights it was some abstract of sacred history or
the Lectures of the Abbe Frayssinous, and on Sundays passages from the
"Genie du Christianisme," as a recreation. How she listened at first to
the sonorous lamentations of its romantic melancholies reechoing
through the world and eternity! If her childhood had been spent in the
shop-parlour of some business quarter, she might perhaps have opened
her heart to those lyrical invasions of Nature, which usually come to
us only through translation in books. But she knew the country too well;
she knew the lowing of cattle, the milking, the ploughs.

Accustomed to calm aspects of life, she turned, on the contrary, to
those of excitement. She loved the sea only for the sake of its storms,
and the green fields only when broken up by ruins.

She wanted to get some personal profit out of things, and she rejected
as useless all that did not contribute to the immediate desires of her
heart, being of a temperament more sentimental than artistic, looking
for emotions, not landscapes.

At the convent there was an old maid who came for a week each month to
mend the linen. Patronized by the clergy, because she belonged to an
ancient family of noblemen ruined by the Revolution, she dined in the
refectory at the table of the good sisters, and after the meal had a bit
of chat with them before going back to her work. The girls often slipped
out from the study to go and see her. She knew by heart the love songs
of the last century, and sang them in a low voice as she stitched away.

She told stories, gave them news, went errands in the town, and on
the sly lent the big girls some novel, that she always carried in the
pockets of her apron, and of which the good lady herself swallowed
long chapters in the intervals of her work. They were all love, lovers,
sweethearts, persecuted ladies fainting in lonely pavilions, postilions
killed at every stage, horses ridden to death on every page, sombre
forests, heartaches, vows, sobs, tears and kisses, little skiffs by
moonlight, nightingales in shady groves, "gentlemen" brave as lions,
gentle as lambs, virtuous as no one ever was, always well dressed, and
weeping like fountains. For six months, then, Emma, at fifteen years of
age, made her hands dirty with books from old lending libraries.

Through Walter Scott, later on, she fell in love with historical events,
dreamed of old chests, guard-rooms and minstrels. She would have liked
to live in some old manor-house, like those long-waisted chatelaines
who, in the shade of pointed arches, spent their days leaning on the
stone, chin in hand, watching a cavalier with white plume galloping on
his black horse from the distant fields. At this time she had a cult
for Mary Stuart and enthusiastic veneration for illustrious or unhappy
women. Joan of Arc, Heloise, Agnes Sorel, the beautiful Ferroniere, and
Clemence Isaure stood out to her like comets in the dark immensity of
heaven, where also were seen, lost in shadow, and all unconnected, St.
Louis with his oak, the dying Bayard, some cruelties of Louis XI, a
little of St. Bartholomew's Day, the plume of the Bearnais, and always
the remembrance of the plates painted in honour of Louis XIV.

In the music class, in the ballads she sang, there was nothing but
little angels with golden wings, madonnas, lagunes, gondoliers;-mild
compositions that allowed her to catch a glimpse athwart the obscurity
of style and the weakness of the music of the attractive phantasmagoria
of sentimental realities. Some of her companions brought "keepsakes"
given them as new year's gifts to the convent. These had to be hidden;
it was quite an undertaking; they were read in the dormitory. Delicately
handling the beautiful satin bindings, Emma looked with dazzled eyes at
the names of the unknown authors, who had signed their verses for the
most part as counts or viscounts.

She trembled as she blew back the tissue paper over the engraving and
saw it folded in two and fall gently against the page. Here behind the
balustrade of a balcony was a young man in a short cloak, holding in his
arms a young girl in a white dress wearing an alms-bag at her belt; or
there were nameless portraits of English ladies with fair curls, who
looked at you from under their round straw hats with their large clear
eyes. Some there were lounging in their carriages, gliding through
parks, a greyhound bounding along in front of the equipage driven at
a trot by two midget postilions in white breeches. Others, dreaming on
sofas with an open letter, gazed at the moon through a slightly open
window half draped by a black curtain. The naive ones, a tear on their
cheeks, were kissing doves through the bars of a Gothic cage, or,
smiling, their heads on one side, were plucking the leaves of a
marguerite with their taper fingers, that curved at the tips like peaked
shoes. And you, too, were there, Sultans with long pipes reclining
beneath arbours in the arms of Bayaderes; Djiaours, Turkish sabres,
Greek caps; and you especially, pale landscapes of dithyrambic lands,
that often show us at once palm trees and firs, tigers on the right, a
lion to the left, Tartar minarets on the horizon; the whole framed by
a very neat virgin forest, and with a great perpendicular sunbeam
trembling in the water, where, standing out in relief like white
excoriations on a steel-grey ground, swans are swimming about.

And the shade of the argand lamp fastened to the wall above Emma's head
lighted up all these pictures of the world, that passed before her one
by one in the silence of the dormitory, and to the distant noise of some
belated carriage rolling over the Boulevards.

When her mother died she cried much the first few days. She had a
funeral picture made with the hair of the deceased, and, in a letter
sent to the Bertaux full of sad reflections on life, she asked to be
buried later on in the same grave. The goodman thought she must be ill,
and came to see her. Emma was secretly pleased that she had reached at
a first attempt the rare ideal of pale lives, never attained by mediocre
hearts. She let herself glide along with Lamartine meanderings, listened
to harps on lakes, to all the songs of dying swans, to the falling of
the leaves, the pure virgins ascending to heaven, and the voice of
the Eternal discoursing down the valleys. She wearied of it, would not
confess it, continued from habit, and at last was surprised to feel
herself soothed, and with no more sadness at heart than wrinkles on her
brow.

The good nuns, who had been so sure of her vocation, perceived with
great astonishment that Mademoiselle Rouault seemed to be slipping
from them. They had indeed been so lavish to her of prayers, retreats,
novenas, and sermons, they had so often preached the respect due to
saints and martyrs, and given so much good advice as to the modesty of
the body and the salvation of her soul, that she did as tightly reined
horses; she pulled up short and the bit slipped from her teeth. This
nature, positive in the midst of its enthusiasms, that had loved the
church for the sake of the flowers, and music for the words of the
songs, and literature for its passional stimulus, rebelled against
the mysteries of faith as it grew irritated by discipline, a thing
antipathetic to her constitution. When her father took her from school,
no one was sorry to see her go. The Lady Superior even thought that she
had latterly been somewhat irreverent to the community.

Emma, at home once more, first took pleasure in looking after the
servants, then grew disgusted with the country and missed her convent.
When Charles came to the Bertaux for the first time, she thought herself
quite disillusioned, with nothing more to learn, and nothing more to
feel.

But the uneasiness of her new position, or perhaps the disturbance
caused by the presence of this man, had sufficed to make her believe
that she at last felt that wondrous passion which, till then, like a
great bird with rose-coloured wings, hung in the splendour of the skies
of poesy; and now she could not think that the calm in which she lived
was the happiness she had dreamed.



Chapter Seven

She thought, sometimes, that, after all, this was the happiest time
of her life--the honeymoon, as people called it. To taste the full
sweetness of it, it would have been necessary doubtless to fly to those
lands with sonorous names where the days after marriage are full of
laziness most suave. In post chaises behind blue silken curtains to ride
slowly up steep road, listening to the song of the postilion re-echoed
by the mountains, along with the bells of goats and the muffled sound of
a waterfall; at sunset on the shores of gulfs to breathe in the perfume
of lemon trees; then in the evening on the villa-terraces above, hand in
hand to look at the stars, making plans for the future. It seemed to her
that certain places on earth must bring happiness, as a plant peculiar
to the soil, and that cannot thrive elsewhere. Why could not she lean
over balconies in Swiss chalets, or enshrine her melancholy in a Scotch
cottage, with a husband dressed in a black velvet coat with long tails,
and thin shoes, a pointed hat and frills? Perhaps she would have liked
to confide all these things to someone. But how tell an undefinable
uneasiness, variable as the clouds, unstable as the winds? Words failed
her--the opportunity, the courage.

If Charles had but wished it, if he had guessed it, if his look had but
once met her thought, it seemed to her that a sudden plenty would have
gone out from her heart, as the fruit falls from a tree when shaken by
a hand. But as the intimacy of their life became deeper, the greater
became the gulf that separated her from him.

Charles's conversation was commonplace as a street pavement, and
everyone's ideas trooped through it in their everyday garb, without
exciting emotion, laughter, or thought. He had never had the curiosity,
he said, while he lived at Rouen, to go to the theatre to see the actors
from Paris. He could neither swim, nor fence, nor shoot, and one day
he could not explain some term of horsemanship to her that she had come
across in a novel.

A man, on the contrary, should he not know everything, excel in manifold
activities, initiate you into the energies of passion, the refinements
of life, all mysteries? But this one taught nothing, knew nothing,
wished nothing. He thought her happy; and she resented this easy calm,
this serene heaviness, the very happiness she gave him.

Sometimes she would draw; and it was great amusement to Charles to stand
there bolt upright and watch her bend over her cardboard, with eyes
half-closed the better to see her work, or rolling, between her fingers,
little bread-pellets. As to the piano, the more quickly her fingers
glided over it the more he wondered. She struck the notes with aplomb,
and ran from top to bottom of the keyboard without a break. Thus shaken
up, the old instrument, whose strings buzzed, could be heard at the
other end of the village when the window was open, and often the
bailiff's clerk, passing along the highroad bare-headed and in list
slippers, stopped to listen, his sheet of paper in his hand.

Emma, on the other hand, knew how to look after her house. She sent the
patients' accounts in well-phrased letters that had no suggestion of
a bill. When they had a neighbour to dinner on Sundays, she managed to
have some tasty dish--piled up pyramids of greengages on vine leaves,
served up preserves turned out into plates--and even spoke of buying
finger-glasses for dessert. From all this much consideration was
extended to Bovary.

Charles finished by rising in his own esteem for possessing such a wife.
He showed with pride in the sitting room two small pencil sketches by
her that he had had framed in very large frames, and hung up against the
wallpaper by long green cords. People returning from mass saw him at his
door in his wool-work slippers.

He came home late--at ten o'clock, at midnight sometimes. Then he asked
for something to eat, and as the servant had gone to bed, Emma waited
on him. He took off his coat to dine more at his ease. He told her, one
after the other, the people he had met, the villages where he had been,
the prescriptions he had written, and, well pleased with himself, he
finished the remainder of the boiled beef and onions, picked pieces off
the cheese, munched an apple, emptied his water-bottle, and then went to
bed, and lay on his back and snored.

As he had been for a time accustomed to wear nightcaps, his handkerchief
would not keep down over his ears, so that his hair in the morning was
all tumbled pell-mell about his face and whitened with the feathers of
the pillow, whose strings came untied during the night. He always wore
thick boots that had two long creases over the instep running obliquely
towards the ankle, while the rest of the upper continued in a straight
line as if stretched on a wooden foot. He said that "was quite good
enough for the country."

His mother approved of his economy, for she came to see him as formerly
when there had been some violent row at her place; and yet Madame Bovary
senior seemed prejudiced against her daughter-in-law. She thought "her
ways too fine for their position"; the wood, the sugar, and the candles
disappeared as "at a grand establishment," and the amount of firing in
the kitchen would have been enough for twenty-five courses. She put her
linen in order for her in the presses, and taught her to keep an eye on
the butcher when he brought the meat. Emma put up with these lessons.
Madame Bovary was lavish of them; and the words "daughter" and "mother"
were exchanged all day long, accompanied by little quiverings of the
lips, each one uttering gentle words in a voice trembling with anger.

In Madame Dubuc's time the old woman felt that she was still the
favorite; but now the love of Charles for Emma seemed to her a desertion
from her tenderness, an encroachment upon what was hers, and she watched
her son's happiness in sad silence, as a ruined man looks through
the windows at people dining in his old house. She recalled to him as
remembrances her troubles and her sacrifices, and, comparing these with
Emma's negligence, came to the conclusion that it was not reasonable to
adore her so exclusively.

Charles knew not what to answer: he respected his mother, and he loved
his wife infinitely; he considered the judgment of the one infallible,
and yet he thought the conduct of the other irreproachable. When Madam
Bovary had gone, he tried timidly and in the same terms to hazard one or
two of the more anodyne observations he had heard from his mamma. Emma
proved to him with a word that he was mistaken, and sent him off to his
patients.

And yet, in accord with theories she believed right, she wanted to make
herself in love with him. By moonlight in the garden she recited all
the passionate rhymes she knew by heart, and, sighing, sang to him many
melancholy adagios; but she found herself as calm after as before, and
Charles seemed no more amorous and no more moved.

When she had thus for a while struck the flint on her heart without
getting a spark, incapable, moreover, of understanding what she did
not experience as of believing anything that did not present itself
in conventional forms, she persuaded herself without difficulty that
Charles's passion was nothing very exorbitant. His outbursts became
regular; he embraced her at certain fixed times. It was one habit among
other habits, and, like a dessert, looked forward to after the monotony
of dinner.

A gamekeeper, cured by the doctor of inflammation of the lungs, had
given madame a little Italian greyhound; she took her out walking, for
she went out sometimes in order to be alone for a moment, and not to see
before her eyes the eternal garden and the dusty road. She went as far
as the beeches of Banneville, near the deserted pavilion which forms an
angle of the wall on the side of the country. Amidst the vegetation of
the ditch there are long reeds with leaves that cut you.

She began by looking round her to see if nothing had changed since last
she had been there. She found again in the same places the foxgloves and
wallflowers, the beds of nettles growing round the big stones, and
the patches of lichen along the three windows, whose shutters, always
closed, were rotting away on their rusty iron bars. Her thoughts,
aimless at first, wandered at random, like her greyhound, who ran round
and round in the fields, yelping after the yellow butterflies, chasing
the shrew-mice, or nibbling the poppies on the edge of a cornfield.

Then gradually her ideas took definite shape, and, sitting on the grass
that she dug up with little prods of her sunshade, Emma repeated to
herself, "Good heavens! Why did I marry?"

She asked herself if by some other chance combination it would have not
been possible to meet another man; and she tried to imagine what would
have been these unrealised events, this different life, this unknown
husband. All, surely, could not be like this one. He might have been
handsome, witty, distinguished, attractive, such as, no doubt, her old
companions of the convent had married. What were they doing now? In
town, with the noise of the streets, the buzz of the theatres and the
lights of the ballroom, they were living lives where the heart expands,
the senses bourgeon out. But she--her life was cold as a garret whose
dormer window looks on the north, and ennui, the silent spider, was
weaving its web in the darkness in every corner of her heart.

She recalled the prize days, when she mounted the platform to receive
her little crowns, with her hair in long plaits. In her white frock and
open prunella shoes she had a pretty way, and when she went back to her
seat, the gentlemen bent over her to congratulate her; the courtyard was
full of carriages; farewells were called to her through their windows;
the music master with his violin case bowed in passing by. How far all
of this! How far away! She called Djali, took her between her knees, and
smoothed the long delicate head, saying, "Come, kiss mistress; you have
no troubles."

Then noting the melancholy face of the graceful animal, who yawned
slowly, she softened, and comparing her to herself, spoke to her aloud
as to somebody in trouble whom one is consoling.

Occasionally there came gusts of winds, breezes from the sea rolling in
one sweep over the whole plateau of the Caux country, which brought
even to these fields a salt freshness. The rushes, close to the ground,
whistled; the branches trembled in a swift rustling, while their
summits, ceaselessly swaying, kept up a deep murmur. Emma drew her shawl
round her shoulders and rose.

In the avenue a green light dimmed by the leaves lit up the short moss
that crackled softly beneath her feet. The sun was setting; the sky
showed red between the branches, and the trunks of the trees, uniform,
and planted in a straight line, seemed a brown colonnade standing out
against a background of gold. A fear took hold of her; she called Djali,
and hurriedly returned to Tostes by the high road, threw herself into an
armchair, and for the rest of the evening did not speak.

But towards the end of September something extraordinary fell upon her
life; she was invited by the Marquis d'Andervilliers to Vaubyessard.

Secretary of State under the Restoration, the Marquis, anxious to
re-enter political life, set about preparing for his candidature to
the Chamber of Deputies long beforehand. In the winter he distributed a
great deal of wood, and in the Conseil General always enthusiastically
demanded new roads for his arrondissement. During the dog-days he had
suffered from an abscess, which Charles had cured as if by miracle by
giving a timely little touch with the lancet. The steward sent to Tostes
to pay for the operation reported in the evening that he had seen some
superb cherries in the doctor's little garden. Now cherry trees did not
thrive at Vaubyessard; the Marquis asked Bovary for some slips; made it
his business to thank his personally; saw Emma; thought she had a pretty
figure, and that she did not bow like a peasant; so that he did not
think he was going beyond the bounds of condescension, nor, on the other
hand, making a mistake, in inviting the young couple.

On Wednesday at three o'clock, Monsieur and Madame Bovary, seated in
their dog-cart, set out for Vaubyessard, with a great trunk strapped
on behind and a bonnet-box in front of the apron. Besides these Charles
held a bandbox between his knees.

They arrived at nightfall, just as the lamps in the park were being lit
to show the way for the carriages.



Chapter Eight

The chateau, a modern building in Italian style, with two projecting
wings and three flights of steps, lay at the foot of an immense
green-sward, on which some cows were grazing among groups of large trees
set out at regular intervals, while large beds of arbutus, rhododendron,
syringas, and guelder roses bulged out their irregular clusters of
green along the curve of the gravel path. A river flowed under a bridge;
through the mist one could distinguish buildings with thatched roofs
scattered over the field bordered by two gently sloping, well timbered
hillocks, and in the background amid the trees rose in two parallel
lines the coach houses and stables, all that was left of the ruined old
chateau.

Charles's dog-cart pulled up before the middle flight of steps; servants
appeared; the Marquis came forward, and, offering his arm to the
doctor's wife, conducted her to the vestibule.

It was paved with marble slabs, was very lofty, and the sound of
footsteps and that of voices re-echoed through it as in a church.

Opposite rose a straight staircase, and on the left a gallery
overlooking the garden led to the billiard room, through whose door one
could hear the click of the ivory balls. As she crossed it to go to the
drawing room, Emma saw standing round the table men with grave faces,
their chins resting on high cravats. They all wore orders, and smiled
silently as they made their strokes.

On the dark wainscoting of the walls large gold frames bore at
the bottom names written in black letters. She read: "Jean-Antoine
d'Andervilliers d'Yvervonbille, Count de la Vaubyessard and Baron de la
Fresnay, killed at the battle of Coutras on the 20th of October,
1587." And on another: "Jean-Antoine-Henry-Guy d'Andervilliers de
la Vaubyessard, Admiral of France and Chevalier of the Order of St.
Michael, wounded at the battle of the Hougue-Saint-Vaast on the 29th of
May, 1692; died at Vaubyessard on the 23rd of January 1693." One could
hardly make out those that followed, for the light of the lamps lowered
over the green cloth threw a dim shadow round the room. Burnishing the
horizontal pictures, it broke up against these in delicate lines where
there were cracks in the varnish, and from all these great black squares
framed in with gold stood out here and there some lighter portion of the
painting--a pale brow, two eyes that looked at you, perukes flowing over
and powdering red-coated shoulders, or the buckle of a garter above a
well-rounded calf.

The Marquis opened the drawing room door; one of the ladies (the
Marchioness herself) came to meet Emma. She made her sit down by her on
an ottoman, and began talking to her as amicably as if she had known her
a long time. She was a woman of about forty, with fine shoulders, a hook
nose, a drawling voice, and on this evening she wore over her brown hair
a simple guipure fichu that fell in a point at the back. A fair young
woman sat in a high-backed chair in a corner; and gentlemen with flowers
in their buttonholes were talking to ladies round the fire.

At seven dinner was served. The men, who were in the majority, sat down
at the first table in the vestibule; the ladies at the second in the
dining room with the Marquis and Marchioness.

Emma, on entering, felt herself wrapped round by the warm air, a
blending of the perfume of flowers and of the fine linen, of the fumes
of the viands, and the odour of the truffles. The silver dish covers
reflected the lighted wax candles in the candelabra, the cut crystal
covered with light steam reflected from one to the other pale rays;
bouquets were placed in a row the whole length of the table; and in
the large-bordered plates each napkin, arranged after the fashion of a
bishop's mitre, held between its two gaping folds a small oval shaped
roll. The red claws of lobsters hung over the dishes; rich fruit in open
baskets was piled up on moss; there were quails in their plumage; smoke
was rising; and in silk stockings, knee-breeches, white cravat, and
frilled shirt, the steward, grave as a judge, offering ready carved
dishes between the shoulders of the guests, with a touch of the spoon
gave you the piece chosen. On the large stove of porcelain inlaid
with copper baguettes the statue of a woman, draped to the chin, gazed
motionless on the room full of life.

Madame Bovary noticed that many ladies had not put their gloves in their
glasses.

But at the upper end of the table, alone amongst all these women, bent
over his full plate, and his napkin tied round his neck like a child, an
old man sat eating, letting drops of gravy drip from his mouth. His eyes
were bloodshot, and he wore a little queue tied with black ribbon. He
was the Marquis's father-in-law, the old Duke de Laverdiere, once on
a time favourite of the Count d'Artois, in the days of the Vaudreuil
hunting-parties at the Marquis de Conflans', and had been, it was said,
the lover of Queen Marie Antoinette, between Monsieur de Coigny and
Monsieur de Lauzun. He had lived a life of noisy debauch, full of duels,
bets, elopements; he had squandered his fortune and frightened all his
family. A servant behind his chair named aloud to him in his ear the
dishes that he pointed to stammering, and constantly Emma's eyes
turned involuntarily to this old man with hanging lips, as to something
extraordinary. He had lived at court and slept in the bed of queens!
Iced champagne was poured out. Emma shivered all over as she felt
it cold in her mouth. She had never seen pomegranates nor tasted
pineapples. The powdered sugar even seemed to her whiter and finer than
elsewhere.

The ladies afterwards went to their rooms to prepare for the ball.

Emma made her toilet with the fastidious care of an actress on her
debut. She did her hair according to the directions of the hairdresser,
and put on the barege dress spread out upon the bed.

Charles's trousers were tight across the belly.

"My trouser-straps will be rather awkward for dancing," he said.

"Dancing?" repeated Emma.

"Yes!"

"Why, you must be mad! They would make fun of you; keep your place.
Besides, it is more becoming for a doctor," she added.

Charles was silent. He walked up and down waiting for Emma to finish
dressing.

He saw her from behind in the glass between two lights. Her black eyes
seemed blacker than ever. Her hair, undulating towards the ears, shone
with a blue lustre; a rose in her chignon trembled on its mobile stalk,
with artificial dewdrops on the tip of the leaves. She wore a gown of
pale saffron trimmed with three bouquets of pompon roses mixed with
green.

Charles came and kissed her on her shoulder.

"Let me alone!" she said; "you are tumbling me."

One could hear the flourish of the violin and the notes of a horn. She
went downstairs restraining herself from running.

Dancing had begun. Guests were arriving. There was some crushing.

She sat down on a form near the door.

The quadrille over, the floor was occupied by groups of men standing up
and talking and servants in livery bearing large trays. Along the line
of seated women painted fans were fluttering, bouquets half hid smiling
faces, and gold stoppered scent-bottles were turned in partly-closed
hands, whose white gloves outlined the nails and tightened on the flesh
at the wrists. Lace trimmings, diamond brooches, medallion bracelets
trembled on bodices, gleamed on breasts, clinked on bare arms.

The hair, well-smoothed over the temples and knotted at the nape,
bore crowns, or bunches, or sprays of mytosotis, jasmine, pomegranate
blossoms, ears of corn, and corn-flowers. Calmly seated in their places,
mothers with forbidding countenances were wearing red turbans.

Emma's heart beat rather faster when, her partner holding her by the
tips of the fingers, she took her place in a line with the dancers, and
waited for the first note to start. But her emotion soon vanished, and,
swaying to the rhythm of the orchestra, she glided forward with slight
movements of the neck. A smile rose to her lips at certain delicate
phrases of the violin, that sometimes played alone while the other
instruments were silent; one could hear the clear clink of the louis
d'or that were being thrown down upon the card tables in the next room;
then all struck again, the cornet-a-piston uttered its sonorous note,
feet marked time, skirts swelled and rustled, hands touched and parted;
the same eyes falling before you met yours again.

A few men (some fifteen or so), of twenty-five to forty, scattered here
and there among the dancers or talking at the doorways, distinguished
themselves from the crowd by a certain air of breeding, whatever their
differences in age, dress, or face.

Their clothes, better made, seemed of finer cloth, and their hair,
brought forward in curls towards the temples, glossy with more delicate
pomades. They had the complexion of wealth--that clear complexion that
is heightened by the pallor of porcelain, the shimmer of satin, the
veneer of old furniture, and that an ordered regimen of exquisite
nurture maintains at its best. Their necks moved easily in their low
cravats, their long whiskers fell over their turned-down collars, they
wiped their lips upon handkerchiefs with embroidered initials that gave
forth a subtle perfume. Those who were beginning to grow old had an air
of youth, while there was something mature in the faces of the young.
In their unconcerned looks was the calm of passions daily satiated, and
through all their gentleness of manner pierced that peculiar brutality,
the result of a command of half-easy things, in which force is exercised
and vanity amused--the management of thoroughbred horses and the society
of loose women.

A few steps from Emma a gentleman in a blue coat was talking of Italy
with a pale young woman wearing a parure of pearls.

They were praising the breadth of the columns of St. Peter's, Tivoly,
Vesuvius, Castellamare, and Cassines, the roses of Genoa, the Coliseum
by moonlight. With her other ear Emma was listening to a conversation
full of words she did not understand. A circle gathered round a very
young man who the week before had beaten "Miss Arabella" and "Romolus,"
and won two thousand louis jumping a ditch in England. One complained
that his racehorses were growing fat; another of the printers' errors
that had disfigured the name of his horse.

The atmosphere of the ball was heavy; the lamps were growing dim.

Guests were flocking to the billiard room. A servant got upon a chair
and broke the window-panes. At the crash of the glass Madame Bovary
turned her head and saw in the garden the faces of peasants pressed
against the window looking in at them. Then the memory of the Bertaux
came back to her. She saw the farm again, the muddy pond, her father in
a blouse under the apple trees, and she saw herself again as formerly,
skimming with her finger the cream off the milk-pans in the dairy. But
in the refulgence of the present hour her past life, so distinct until
then, faded away completely, and she almost doubted having lived it. She
was there; beyond the ball was only shadow overspreading all the rest.
She was just eating a maraschino ice that she held with her left hand
in a silver-gilt cup, her eyes half-closed, and the spoon between her
teeth.

A lady near her dropped her fan. A gentlemen was passing.

"Would you be so good," said the lady, "as to pick up my fan that has
fallen behind the sofa?"

The gentleman bowed, and as he moved to stretch out his arm, Emma saw
the hand of a young woman throw something white, folded in a triangle,
into his hat. The gentleman, picking up the fan, offered it to the lady
respectfully; she thanked him with an inclination of the head, and began
smelling her bouquet.

After supper, where were plenty of Spanish and Rhine wines, soups a la
bisque and au lait d'amandes*, puddings a la Trafalgar, and all sorts of
cold meats with jellies that trembled in the dishes, the carriages one
after the other began to drive off. Raising the corners of the muslin
curtain, one could see the light of their lanterns glimmering through
the darkness. The seats began to empty, some card-players were still
left; the musicians were cooling the tips of their fingers on their
tongues. Charles was half asleep, his back propped against a door.

     *With almond milk

At three o'clock the cotillion began. Emma did not know how to waltz.
Everyone was waltzing, Mademoiselle d'Andervilliers herself and the
Marquis; only the guests staying at the castle were still there, about a
dozen persons.

One of the waltzers, however, who was familiarly called Viscount, and
whose low cut waistcoat seemed moulded to his chest, came a second time
to ask Madame Bovary to dance, assuring her that he would guide her, and
that she would get through it very well.

They began slowly, then went more rapidly. They turned; all around them
was turning--the lamps, the furniture, the wainscoting, the floor, like
a disc on a pivot. On passing near the doors the bottom of Emma's dress
caught against his trousers.

Their legs commingled; he looked down at her; she raised her eyes to
his. A torpor seized her; she stopped. They started again, and with a
more rapid movement; the Viscount, dragging her along disappeared with
her to the end of the gallery, where panting, she almost fell, and for
a moment rested her head upon his breast. And then, still turning, but
more slowly, he guided her back to her seat. She leaned back against the
wall and covered her eyes with her hands.

When she opened them again, in the middle of the drawing room three
waltzers were kneeling before a lady sitting on a stool.

She chose the Viscount, and the violin struck up once more.

Everyone looked at them. They passed and re-passed, she with rigid body,
her chin bent down, and he always in the same pose, his figure curved,
his elbow rounded, his chin thrown forward. That woman knew how to
waltz! They kept up a long time, and tired out all the others.

Then they talked a few moments longer, and after the goodnights, or
rather good mornings, the guests of the chateau retired to bed.

Charles dragged himself up by the balusters. His "knees were going
up into his body." He had spent five consecutive hours standing
bolt upright at the card tables, watching them play whist, without
understanding anything about it, and it was with a deep sigh of relief
that he pulled off his boots.

Emma threw a shawl over her shoulders, opened the window, and leant out.

The night was dark; some drops of rain were falling. She breathed in the
damp wind that refreshed her eyelids. The music of the ball was still
murmuring in her ears. And she tried to keep herself awake in order to
prolong the illusion of this luxurious life that she would soon have to
give up.

Day began to break. She looked long at the windows of the chateau,
trying to guess which were the rooms of all those she had noticed the
evening before. She would fain have known their lives, have penetrated,
blended with them. But she was shivering with cold. She undressed, and
cowered down between the sheets against Charles, who was asleep.

There were a great many people to luncheon. The repast lasted ten
minutes; no liqueurs were served, which astonished the doctor.

Next, Mademoiselle d'Andervilliers collected some pieces of roll in a
small basket to take them to the swans on the ornamental waters, and
they went to walk in the hot-houses, where strange plants, bristling
with hairs, rose in pyramids under hanging vases, whence, as from
over-filled nests of serpents, fell long green cords interlacing.
The orangery, which was at the other end, led by a covered way to the
outhouses of the chateau. The Marquis, to amuse the young woman, took
her to see the stables.

Above the basket-shaped racks porcelain slabs bore the names of the
horses in black letters. Each animal in its stall whisked its tail when
anyone went near and said "Tchk! tchk!" The boards of the harness room
shone like the flooring of a drawing room. The carriage harness was
piled up in the middle against two twisted columns, and the bits, the
whips, the spurs, the curbs, were ranged in a line all along the wall.

Charles, meanwhile, went to ask a groom to put his horse to. The
dog-cart was brought to the foot of the steps, and, all the parcels
being crammed in, the Bovarys paid their respects to the Marquis and
Marchioness and set out again for Tostes.

Emma watched the turning wheels in silence. Charles, on the extreme edge
of the seat, held the reins with his two arms wide apart, and the little
horse ambled along in the shafts that were too big for him. The loose
reins hanging over his crupper were wet with foam, and the box fastened
on behind the chaise gave great regular bumps against it.

They were on the heights of Thibourville when suddenly some horsemen
with cigars between their lips passed laughing. Emma thought she
recognized the Viscount, turned back, and caught on the horizon only the
movement of the heads rising or falling with the unequal cadence of the
trot or gallop.

A mile farther on they had to stop to mend with some string the traces
that had broken.

But Charles, giving a last look to the harness, saw something on the
ground between his horse's legs, and he picked up a cigar-case with
a green silk border and beblazoned in the centre like the door of a
carriage.

"There are even two cigars in it," said he; "they'll do for this evening
after dinner."

"Why, do you smoke?" she asked.

"Sometimes, when I get a chance."

He put his find in his pocket and whipped up the nag.

When they reached home the dinner was not ready. Madame lost her temper.
Nastasie answered rudely.

"Leave the room!" said Emma. "You are forgetting yourself. I give you
warning."

For dinner there was onion soup and a piece of veal with sorrel.

Charles, seated opposite Emma, rubbed his hands gleefully.

"How good it is to be at home again!"

Nastasie could be heard crying. He was rather fond of the poor girl.
She had formerly, during the wearisome time of his widowhood, kept him
company many an evening. She had been his first patient, his oldest
acquaintance in the place.

"Have you given her warning for good?" he asked at last.

"Yes. Who is to prevent me?" she replied.

Then they warmed themselves in the kitchen while their room was being
made ready. Charles began to smoke. He smoked with lips protruding,
spitting every moment, recoiling at every puff.

"You'll make yourself ill," she said scornfully.

He put down his cigar and ran to swallow a glass of cold water at the
pump. Emma seizing hold of the cigar case threw it quickly to the back
of the cupboard.

The next day was a long one. She walked about her little garden, up
and down the same walks, stopping before the beds, before the espalier,
before the plaster curate, looking with amazement at all these things
of once-on-a-time that she knew so well. How far off the ball seemed
already! What was it that thus set so far asunder the morning of the day
before yesterday and the evening of to-day? Her journey to Vaubyessard
had made a hole in her life, like one of those great crevices that
a storm will sometimes make in one night in mountains. Still she was
resigned. She devoutly put away in her drawers her beautiful dress, down
to the satin shoes whose soles were yellowed with the slippery wax of
the dancing floor. Her heart was like these. In its friction against
wealth something had come over it that could not be effaced.

The memory of this ball, then, became an occupation for Emma.

Whenever the Wednesday came round she said to herself as she awoke, "Ah!
I was there a week--a fortnight--three weeks ago."

And little by little the faces grew confused in her remembrance.

She forgot the tune of the quadrilles; she no longer saw the liveries
and appointments so distinctly; some details escaped her, but the regret
remained with her.



Chapter Nine

Often when Charles was out she took from the cupboard, between the
folds of the linen where she had left it, the green silk cigar case.
She looked at it, opened it, and even smelt the odour of the lining--a
mixture of verbena and tobacco. Whose was it? The Viscount's? Perhaps
it was a present from his mistress. It had been embroidered on some
rosewood frame, a pretty little thing, hidden from all eyes, that had
occupied many hours, and over which had fallen the soft curls of the
pensive worker. A breath of love had passed over the stitches on the
canvas; each prick of the needle had fixed there a hope or a memory, and
all those interwoven threads of silk were but the continuity of the same
silent passion. And then one morning the Viscount had taken it away
with him. Of what had they spoken when it lay upon the wide-mantelled
chimneys between flower-vases and Pompadour clocks? She was at Tostes;
he was at Paris now, far away! What was this Paris like? What a vague
name! She repeated it in a low voice, for the mere pleasure of it; it
rang in her ears like a great cathedral bell; it shone before her eyes,
even on the labels of her pomade-pots.

At night, when the carriers passed under her windows in their carts
singing the "Marjolaine," she awoke, and listened to the noise of the
iron-bound wheels, which, as they gained the country road, was soon
deadened by the soil. "They will be there to-morrow!" she said to
herself.

And she followed them in thought up and down the hills, traversing
villages, gliding along the highroads by the light of the stars. At the
end of some indefinite distance there was always a confused spot, into
which her dream died.

She bought a plan of Paris, and with the tip of her finger on the map
she walked about the capital. She went up the boulevards, stopping at
every turning, between the lines of the streets, in front of the white
squares that represented the houses. At last she would close the lids of
her weary eyes, and see in the darkness the gas jets flaring in the wind
and the steps of carriages lowered with much noise before the peristyles
of theatres.

She took in "La Corbeille," a lady's journal, and the "Sylphe des
Salons." She devoured, without skipping a word, all the accounts of
first nights, races, and soirees, took interest in the debut of a
singer, in the opening of a new shop. She knew the latest fashions, the
addresses of the best tailors, the days of the Bois and the Opera. In
Eugene Sue she studied descriptions of furniture; she read Balzac and
George Sand, seeking in them imaginary satisfaction for her own desires.
Even at table she had her book by her, and turned over the pages
while Charles ate and talked to her. The memory of the Viscount always
returned as she read. Between him and the imaginary personages she made
comparisons. But the circle of which he was the centre gradually widened
round him, and the aureole that he bore, fading from his form, broadened
out beyond, lighting up her other dreams.

Paris, more vague than the ocean, glimmered before Emma's eyes in an
atmosphere of vermilion. The many lives that stirred amid this tumult
were, however, divided into parts, classed as distinct pictures. Emma
perceived only two or three that hid from her all the rest, and in
themselves represented all humanity. The world of ambassadors moved over
polished floors in drawing rooms lined with mirrors, round oval tables
covered with velvet and gold-fringed cloths. There were dresses with
trains, deep mysteries, anguish hidden beneath smiles. Then came the
society of the duchesses; all were pale; all got up at four o'clock; the
women, poor angels, wore English point on their petticoats; and the men,
unappreciated geniuses under a frivolous outward seeming, rode horses to
death at pleasure parties, spent the summer season at Baden, and towards
the forties married heiresses. In the private rooms of restaurants,
where one sups after midnight by the light of wax candles, laughed the
motley crowd of men of letters and actresses. They were prodigal as
kings, full of ideal, ambitious, fantastic frenzy. This was an existence
outside that of all others, between heaven and earth, in the midst of
storms, having something of the sublime. For the rest of the world it
was lost, with no particular place and as if non-existent. The nearer
things were, moreover, the more her thoughts turned away from them.
All her immediate surroundings, the wearisome country, the middle-class
imbeciles, the mediocrity of existence, seemed to her exceptional, a
peculiar chance that had caught hold of her, while beyond stretched, as
far as eye could see, an immense land of joys and passions. She confused
in her desire the sensualities of luxury with the delights of the heart,
elegance of manners with delicacy of sentiment. Did not love, like
Indian plants, need a special soil, a particular temperature? Signs
by moonlight, long embraces, tears flowing over yielded hands, all
the fevers of the flesh and the languors of tenderness could not be
separated from the balconies of great castles full of indolence,
from boudoirs with silken curtains and thick carpets, well-filled
flower-stands, a bed on a raised dias, nor from the flashing of precious
stones and the shoulder-knots of liveries.

The lad from the posting house who came to groom the mare every morning
passed through the passage with his heavy wooden shoes; there were holes
in his blouse; his feet were bare in list slippers. And this was the
groom in knee-britches with whom she had to be content! His work done,
he did not come back again all day, for Charles on his return put up
his horse himself, unsaddled him and put on the halter, while the
servant-girl brought a bundle of straw and threw it as best she could
into the manger.

To replace Nastasie (who left Tostes shedding torrents of tears) Emma
took into her service a young girl of fourteen, an orphan with a sweet
face. She forbade her wearing cotton caps, taught her to address her in
the third person, to bring a glass of water on a plate, to knock before
coming into a room, to iron, starch, and to dress her--wanted to make a
lady's-maid of her. The new servant obeyed without a murmur, so as not
to be sent away; and as madame usually left the key in the sideboard,
Felicite every evening took a small supply of sugar that she ate alone
in her bed after she had said her prayers.

Sometimes in the afternoon she went to chat with the postilions.

Madame was in her room upstairs. She wore an open dressing gown that
showed between the shawl facings of her bodice a pleated chamisette with
three gold buttons. Her belt was a corded girdle with great tassels, and
her small garnet coloured slippers had a large knot of ribbon that fell
over her instep. She had bought herself a blotting book, writing case,
pen-holder, and envelopes, although she had no one to write to; she
dusted her what-not, looked at herself in the glass, picked up a book,
and then, dreaming between the lines, let it drop on her knees. She
longed to travel or to go back to her convent. She wished at the same
time to die and to live in Paris.

Charles in snow and rain trotted across country. He ate omelettes on
farmhouse tables, poked his arm into damp beds, received the tepid
spurt of blood-lettings in his face, listened to death-rattles, examined
basins, turned over a good deal of dirty linen; but every evening he
found a blazing fire, his dinner ready, easy-chairs, and a well-dressed
woman, charming with an odour of freshness, though no one could say
whence the perfume came, or if it were not her skin that made odorous
her chemise.

She charmed him by numerous attentions; now it was some new way of
arranging paper sconces for the candles, a flounce that she altered on
her gown, or an extraordinary name for some very simple dish that the
servant had spoilt, but that Charles swallowed with pleasure to the last
mouthful. At Rouen she saw some ladies who wore a bunch of charms on the
watch-chains; she bought some charms. She wanted for her mantelpiece two
large blue glass vases, and some time after an ivory necessaire with a
silver-gilt thimble. The less Charles understood these refinements
the more they seduced him. They added something to the pleasure of the
senses and to the comfort of his fireside. It was like a golden dust
sanding all along the narrow path of his life.

He was well, looked well; his reputation was firmly established.

The country-folk loved him because he was not proud. He petted the
children, never went to the public house, and, moreover, his morals
inspired confidence. He was specially successful with catarrhs and chest
complaints. Being much afraid of killing his patients, Charles, in fact
only prescribed sedatives, from time to time and emetic, a footbath,
or leeches. It was not that he was afraid of surgery; he bled people
copiously like horses, and for the taking out of teeth he had the
"devil's own wrist."

Finally, to keep up with the times, he took in "La Ruche Medicale,"
a new journal whose prospectus had been sent him. He read it a little
after dinner, but in about five minutes the warmth of the room added to
the effect of his dinner sent him to sleep; and he sat there, his chin
on his two hands and his hair spreading like a mane to the foot of the
lamp. Emma looked at him and shrugged her shoulders. Why, at least, was
not her husband one of those men of taciturn passions who work at their
books all night, and at last, when about sixty, the age of rheumatism
sets in, wear a string of orders on their ill-fitting black coat?
She could have wished this name of Bovary, which was hers, had been
illustrious, to see it displayed at the booksellers', repeated in the
newspapers, known to all France. But Charles had no ambition.

An Yvetot doctor whom he had lately met in consultation had somewhat
humiliated him at the very bedside of the patient, before the assembled
relatives. When, in the evening, Charles told her this anecdote, Emma
inveighed loudly against his colleague. Charles was much touched. He
kissed her forehead with a tear in his eyes. But she was angered with
shame; she felt a wild desire to strike him; she went to open the window
in the passage and breathed in the fresh air to calm herself.

"What a man! What a man!" she said in a low voice, biting her lips.

Besides, she was becoming more irritated with him. As he grew older his
manner grew heavier; at dessert he cut the corks of the empty bottles;
after eating he cleaned his teeth with his tongue; in taking soup
he made a gurgling noise with every spoonful; and, as he was getting
fatter, the puffed-out cheeks seemed to push the eyes, always small, up
to the temples.

Sometimes Emma tucked the red borders of his under-vest unto his
waistcoat, rearranged his cravat, and threw away the dirty gloves he was
going to put on; and this was not, as he fancied, for himself; it
was for herself, by a diffusion of egotism, of nervous irritation.
Sometimes, too, she told him of what she had read, such as a passage in
a novel, of a new play, or an anecdote of the "upper ten" that she
had seen in a feuilleton; for, after all, Charles was something, an
ever-open ear, and ever-ready approbation. She confided many a thing to
her greyhound. She would have done so to the logs in the fireplace or to
the pendulum of the clock.

At the bottom of her heart, however, she was waiting for something to
happen. Like shipwrecked sailors, she turned despairing eyes upon the
solitude of her life, seeking afar off some white sail in the mists of
the horizon. She did not know what this chance would be, what wind would
bring it her, towards what shore it would drive her, if it would be a
shallop or a three-decker, laden with anguish or full of bliss to the
portholes. But each morning, as she awoke, she hoped it would come that
day; she listened to every sound, sprang up with a start, wondered that
it did not come; then at sunset, always more saddened, she longed for
the morrow.

Spring came round. With the first warm weather, when the pear trees
began to blossom, she suffered from dyspnoea.

From the beginning of July she counted how many weeks there were to
October, thinking that perhaps the Marquis d'Andervilliers would give
another ball at Vaubyessard. But all September passed without letters or
visits.

After the ennui of this disappointment her heart once more remained
empty, and then the same series of days recommenced. So now they would
thus follow one another, always the same, immovable, and bringing
nothing. Other lives, however flat, had at least the chance of some
event. One adventure sometimes brought with it infinite consequences and
the scene changed. But nothing happened to her; God had willed it so!
The future was a dark corridor, with its door at the end shut fast.

She gave up music. What was the good of playing? Who would hear her?
Since she could never, in a velvet gown with short sleeves, striking
with her light fingers the ivory keys of an Erard at a concert, feel
the murmur of ecstasy envelop her like a breeze, it was not worth while
boring herself with practicing. Her drawing cardboard and her embroidery
she left in the cupboard. What was the good? What was the good? Sewing
irritated her. "I have read everything," she said to herself. And she
sat there making the tongs red-hot, or looked at the rain falling.

How sad she was on Sundays when vespers sounded! She listened with dull
attention to each stroke of the cracked bell. A cat slowly walking over
some roof put up his back in the pale rays of the sun. The wind on the
highroad blew up clouds of dust. Afar off a dog sometimes howled; and
the bell, keeping time, continued its monotonous ringing that died away
over the fields.

But the people came out from church. The women in waxed clogs, the
peasants in new blouses, the little bare-headed children skipping along
in front of them, all were going home. And till nightfall, five or six
men, always the same, stayed playing at corks in front of the large door
of the inn.

The winter was severe. The windows every morning were covered with
rime, and the light shining through them, dim as through ground-glass,
sometimes did not change the whole day long. At four o'clock the lamp
had to be lighted.

On fine days she went down into the garden. The dew had left on the
cabbages a silver lace with long transparent threads spreading from one
to the other. No birds were to be heard; everything seemed asleep, the
espalier covered with straw, and the vine, like a great sick serpent
under the coping of the wall, along which, on drawing hear, one saw the
many-footed woodlice crawling. Under the spruce by the hedgerow, the
curie in the three-cornered hat reading his breviary had lost his right
foot, and the very plaster, scaling off with the frost, had left white
scabs on his face.

Then she went up again, shut her door, put on coals, and fainting with
the heat of the hearth, felt her boredom weigh more heavily than ever.
She would have liked to go down and talk to the servant, but a sense of
shame restrained her.

Every day at the same time the schoolmaster in a black skullcap opened
the shutters of his house, and the rural policeman, wearing his sabre
over his blouse, passed by. Night and morning the post-horses, three by
three, crossed the street to water at the pond. From time to time the
bell of a public house door rang, and when it was windy one could hear
the little brass basins that served as signs for the hairdresser's shop
creaking on their two rods. This shop had as decoration an old engraving
of a fashion-plate stuck against a windowpane and the wax bust of a
woman with yellow hair. He, too, the hairdresser, lamented his wasted
calling, his hopeless future, and dreaming of some shop in a big
town--at Rouen, for example, overlooking the harbour, near the
theatre--he walked up and down all day from the mairie to the church,
sombre and waiting for customers. When Madame Bovary looked up, she
always saw him there, like a sentinel on duty, with his skullcap over
his ears and his vest of lasting.

Sometimes in the afternoon outside the window of her room, the head of a
man appeared, a swarthy head with black whiskers, smiling slowly, with
a broad, gentle smile that showed his white teeth. A waltz immediately
began and on the organ, in a little drawing room, dancers the size of
a finger, women in pink turbans, Tyrolians in jackets, monkeys in frock
coats, gentlemen in knee-breeches, turned and turned between the sofas,
the consoles, multiplied in the bits of looking glass held together
at their corners by a piece of gold paper. The man turned his handle,
looking to the right and left, and up at the windows. Now and again,
while he shot out a long squirt of brown saliva against the milestone,
with his knee raised his instrument, whose hard straps tired his
shoulder; and now, doleful and drawling, or gay and hurried, the music
escaped from the box, droning through a curtain of pink taffeta under
a brass claw in arabesque. They were airs played in other places at
the theatres, sung in drawing rooms, danced to at night under lighted
lustres, echoes of the world that reached even to Emma. Endless
sarabands ran through her head, and, like an Indian dancing girl on the
flowers of a carpet, her thoughts leapt with the notes, swung from dream
to dream, from sadness to sadness. When the man had caught some coppers
in his cap, he drew down an old cover of blue cloth, hitched his organ
on to his back, and went off with a heavy tread. She watched him going.

But it was above all the meal-times that were unbearable to her, in this
small room on the ground floor, with its smoking stove, its creaking
door, the walls that sweated, the damp flags; all the bitterness in life
seemed served up on her plate, and with smoke of the boiled beef there
rose from her secret soul whiffs of sickliness. Charles was a slow
eater; she played with a few nuts, or, leaning on her elbow, amused
herself with drawing lines along the oilcloth table cover with the point
of her knife.

She now let everything in her household take care of itself, and Madame
Bovary senior, when she came to spend part of Lent at Tostes, was much
surprised at the change. She who was formerly so careful, so dainty,
now passed whole days without dressing, wore grey cotton stockings, and
burnt tallow candles. She kept saying they must be economical since
they were not rich, adding that she was very contented, very happy, that
Tostes pleased her very much, with other speeches that closed the mouth
of her mother-in-law. Besides, Emma no longer seemed inclined to follow
her advice; once even, Madame Bovary having thought fit to maintain that
mistresses ought to keep an eye on the religion of their servants, she
had answered with so angry a look and so cold a smile that the good
woman did not interfere again.

Emma was growing difficult, capricious. She ordered dishes for herself,
then she did not touch them; one day drank only pure milk, the next
cups of tea by the dozen. Often she persisted in not going out, then,
stifling, threw open the windows and put on light dresses. After she had
well scolded her servant she gave her presents or sent her out to see
neighbours, just as she sometimes threw beggars all the silver in her
purse, although she was by no means tender-hearted or easily accessible
to the feelings of others, like most country-bred people, who always
retain in their souls something of the horny hardness of the paternal
hands.

Towards the end of February old Rouault, in memory of his cure, himself
brought his son-in-law a superb turkey, and stayed three days at Tostes.
Charles being with his patients, Emma kept him company. He smoked in the
room, spat on the firedogs, talked farming, calves, cows, poultry, and
municipal council, so that when he left she closed the door on him with
a feeling of satisfaction that surprised even herself. Moreover she no
longer concealed her contempt for anything or anybody, and at times she
set herself to express singular opinions, finding fault with that which
others approved, and approving things perverse and immoral, all of which
made her husband open his eyes widely.

Would this misery last for ever? Would she never issue from it? Yet
she was as good as all the women who were living happily. She had seen
duchesses at Vaubyessard with clumsier waists and commoner ways, and she
execrated the injustice of God. She leant her head against the walls
to weep; she envied lives of stir; longed for masked balls, for violent
pleasures, with all the wildness that she did not know, but that these
must surely yield.

She grew pale and suffered from palpitations of the heart.

Charles prescribed valerian and camphor baths. Everything that was tried
only seemed to irritate her the more.

On certain days she chatted with feverish rapidity, and this
over-excitement was suddenly followed by a state of torpor, in which
she remained without speaking, without moving. What then revived her was
pouring a bottle of eau-de-cologne over her arms.

As she was constantly complaining about Tostes, Charles fancied that her
illness was no doubt due to some local cause, and fixing on this idea,
began to think seriously of setting up elsewhere.

From that moment she drank vinegar, contracted a sharp little cough, and
completely lost her appetite.

It cost Charles much to give up Tostes after living there four years and
"when he was beginning to get on there." Yet if it must be! He took her
to Rouen to see his old master. It was a nervous complaint: change of
air was needed.

After looking about him on this side and on that, Charles learnt that
in the Neufchatel arrondissement there was a considerable market town
called Yonville-l'Abbaye, whose doctor, a Polish refugee, had decamped a
week before. Then he wrote to the chemist of the place to ask the
number of the population, the distance from the nearest doctor, what
his predecessor had made a year, and so forth; and the answer being
satisfactory, he made up his mind to move towards the spring, if Emma's
health did not improve.

One day when, in view of her departure, she was tidying a drawer,
something pricked her finger. It was a wire of her wedding bouquet.
The orange blossoms were yellow with dust and the silver bordered satin
ribbons frayed at the edges. She threw it into the fire. It flared
up more quickly than dry straw. Then it was, like a red bush in the
cinders, slowly devoured. She watched it burn.

The little pasteboard berries burst, the wire twisted, the gold
lace melted; and the shriveled paper corollas, fluttering like black
butterflies at the back of the stove, at least flew up the chimney.

When they left Tostes at the month of March, Madame Bovary was pregnant.





Part II



Chapter One

Yonville-l'Abbaye (so called from an old Capuchin abbey of which not
even the ruins remain) is a market-town twenty-four miles from Rouen,
between the Abbeville and Beauvais roads, at the foot of a valley
watered by the Rieule, a little river that runs into the Andelle after
turning three water-mills near its mouth, where there are a few trout
that the lads amuse themselves by fishing for on Sundays.

We leave the highroad at La Boissiere and keep straight on to the top of
the Leux hill, whence the valley is seen. The river that runs through it
makes of it, as it were, two regions with distinct physiognomies--all on
the left is pasture land, all of the right arable. The meadow stretches
under a bulge of low hills to join at the back with the pasture land of
the Bray country, while on the eastern side, the plain, gently rising,
broadens out, showing as far as eye can follow its blond cornfields. The
water, flowing by the grass, divides with a white line the colour of the
roads and of the plains, and the country is like a great unfolded mantle
with a green velvet cape bordered with a fringe of silver.

Before us, on the verge of the horizon, lie the oaks of the forest of
Argueil, with the steeps of the Saint-Jean hills scarred from top
to bottom with red irregular lines; they are rain tracks, and these
brick-tones standing out in narrow streaks against the grey colour of
the mountain are due to the quantity of iron springs that flow beyond in
the neighboring country.

Here we are on the confines of Normandy, Picardy, and the Ile-de-France,
a bastard land whose language is without accent and its landscape is
without character. It is there that they make the worst Neufchatel
cheeses of all the arrondissement; and, on the other hand, farming is
costly because so much manure is needed to enrich this friable soil full
of sand and flints.

Up to 1835 there was no practicable road for getting to Yonville, but
about this time a cross-road was made which joins that of Abbeville to
that of Amiens, and is occasionally used by the Rouen wagoners on their
way to Flanders. Yonville-l'Abbaye has remained stationary in spite of
its "new outlet." Instead of improving the soil, they persist in keeping
up the pasture lands, however depreciated they may be in value, and
the lazy borough, growing away from the plain, has naturally spread
riverwards. It is seem from afar sprawling along the banks like a
cowherd taking a siesta by the water-side.

At the foot of the hill beyond the bridge begins a roadway, planted with
young aspens, that leads in a straight line to the first houses in the
place. These, fenced in by hedges, are in the middle of courtyards
full of straggling buildings, wine-presses, cart-sheds and distilleries
scattered under thick trees, with ladders, poles, or scythes hung on to
the branches. The thatched roofs, like fur caps drawn over eyes, reach
down over about a third of the low windows, whose coarse convex glasses
have knots in the middle like the bottoms of bottles. Against the
plaster wall diagonally crossed by black joists, a meagre pear-tree
sometimes leans and the ground-floors have at their door a small
swing-gate to keep out the chicks that come pilfering crumbs of bread
steeped in cider on the threshold. But the courtyards grow narrower,
the houses closer together, and the fences disappear; a bundle of
ferns swings under a window from the end of a broomstick; there is a
blacksmith's forge and then a wheelwright's, with two or three new carts
outside that partly block the way. Then across an open space appears a
white house beyond a grass mound ornamented by a Cupid, his finger
on his lips; two brass vases are at each end of a flight of steps;
scutcheons* blaze upon the door. It is the notary's house, and the
finest in the place.

     *The panonceaux that have to be hung over the doors of
     notaries.

The Church is on the other side of the street, twenty paces farther
down, at the entrance of the square. The little cemetery that surrounds
it, closed in by a wall breast high, is so full of graves that the old
stones, level with the ground, form a continuous pavement, on which the
grass of itself has marked out regular green squares. The church was
rebuilt during the last years of the reign of Charles X. The wooden roof
is beginning to rot from the top, and here and there has black hollows
in its blue colour. Over the door, where the organ should be, is a
loft for the men, with a spiral staircase that reverberates under their
wooden shoes.

The daylight coming through the plain glass windows falls obliquely upon
the pews ranged along the walls, which are adorned here and there with
a straw mat bearing beneath it the words in large letters, "Mr.
So-and-so's pew." Farther on, at a spot where the building narrows, the
confessional forms a pendant to a statuette of the Virgin, clothed in
a satin robe, coifed with a tulle veil sprinkled with silver stars, and
with red cheeks, like an idol of the Sandwich Islands; and, finally, a
copy of the "Holy Family, presented by the Minister of the Interior,"
overlooking the high altar, between four candlesticks, closes in the
perspective. The choir stalls, of deal wood, have been left unpainted.

The market, that is to say, a tiled roof supported by some twenty posts,
occupies of itself about half the public square of Yonville. The town
hall, constructed "from the designs of a Paris architect," is a sort of
Greek temple that forms the corner next to the chemist's shop. On
the ground-floor are three Ionic columns and on the first floor a
semicircular gallery, while the dome that crowns it is occupied by a
Gallic cock, resting one foot upon the "Charte" and holding in the other
the scales of Justice.

But that which most attracts the eye is opposite the Lion d'Or inn, the
chemist's shop of Monsieur Homais. In the evening especially its argand
lamp is lit up and the red and green jars that embellish his shop-front
throw far across the street their two streams of colour; then across
them as if in Bengal lights is seen the shadow of the chemist
leaning over his desk. His house from top to bottom is placarded with
inscriptions written in large hand, round hand, printed hand: "Vichy,
Seltzer, Barege waters, blood purifiers, Raspail patent medicine,
Arabian racahout, Darcet lozenges, Regnault paste, trusses, baths,
hygienic chocolate," etc. And the signboard, which takes up all the
breadth of the shop, bears in gold letters, "Homais, Chemist." Then at
the back of the shop, behind the great scales fixed to the counter, the
word "Laboratory" appears on a scroll above a glass door, which about
half-way up once more repeats "Homais" in gold letters on a black
ground.

Beyond this there is nothing to see at Yonville. The street (the only
one) a gunshot in length and flanked by a few shops on either side stops
short at the turn of the highroad. If it is left on the right hand and
the foot of the Saint-Jean hills followed the cemetery is soon reached.

At the time of the cholera, in order to enlarge this, a piece of wall
was pulled down, and three acres of land by its side purchased; but all
the new portion is almost tenantless; the tombs, as heretofore,
continue to crowd together towards the gate. The keeper, who is at once
gravedigger and church beadle (thus making a double profit out of the
parish corpses), has taken advantage of the unused plot of ground to
plant potatoes there. From year to year, however, his small field grows
smaller, and when there is an epidemic, he does not know whether to
rejoice at the deaths or regret the burials.

"You live on the dead, Lestiboudois!" the curie at last said to him one
day. This grim remark made him reflect; it checked him for some time;
but to this day he carries on the cultivation of his little tubers, and
even maintains stoutly that they grow naturally.

Since the events about to be narrated, nothing in fact has changed
at Yonville. The tin tricolour flag still swings at the top of the
church-steeple; the two chintz streamers still flutter in the wind from
the linen-draper's; the chemist's fetuses, like lumps of white amadou,
rot more and more in their turbid alcohol, and above the big door of
the inn the old golden lion, faded by rain, still shows passers-by its
poodle mane.

On the evening when the Bovarys were to arrive at Yonville, Widow
Lefrancois, the landlady of this inn, was so very busy that she sweated
great drops as she moved her saucepans. To-morrow was market-day. The
meat had to be cut beforehand, the fowls drawn, the soup and coffee
made. Moreover, she had the boarders' meal to see to, and that of the
doctor, his wife, and their servant; the billiard-room was echoing with
bursts of laughter; three millers in a small parlour were calling for
brandy; the wood was blazing, the brazen pan was hissing, and on the
long kitchen table, amid the quarters of raw mutton, rose piles of
plates that rattled with the shaking of the block on which spinach was
being chopped.

From the poultry-yard was heard the screaming of the fowls whom the
servant was chasing in order to wring their necks.

A man slightly marked with small-pox, in green leather slippers, and
wearing a velvet cap with a gold tassel, was warming his back at the
chimney. His face expressed nothing but self-satisfaction, and he
appeared to take life as calmly as the goldfinch suspended over his head
in its wicker cage: this was the chemist.

"Artemise!" shouted the landlady, "chop some wood, fill the water
bottles, bring some brandy, look sharp! If only I knew what dessert to
offer the guests you are expecting! Good heavens! Those furniture-movers
are beginning their racket in the billiard-room again; and their van has
been left before the front door! The 'Hirondelle' might run into it when
it draws up. Call Polyte and tell him to put it up. Only think, Monsieur
Homais, that since morning they have had about fifteen games, and drunk
eight jars of cider! Why, they'll tear my cloth for me," she went on,
looking at them from a distance, her strainer in her hand.

"That wouldn't be much of a loss," replied Monsieur Homais. "You would
buy another."

"Another billiard-table!" exclaimed the widow.

"Since that one is coming to pieces, Madame Lefrancois. I tell you again
you are doing yourself harm, much harm! And besides, players now want
narrow pockets and heavy cues. Hazards aren't played now; everything is
changed! One must keep pace with the times! Just look at Tellier!"

The hostess reddened with vexation. The chemist went on--

"You may say what you like; his table is better than yours; and if one
were to think, for example, of getting up a patriotic pool for Poland or
the sufferers from the Lyons floods--"

"It isn't beggars like him that'll frighten us," interrupted the
landlady, shrugging her fat shoulders. "Come, come, Monsieur Homais; as
long as the 'Lion d'Or' exists people will come to it. We've feathered
our nest; while one of these days you'll find the 'Cafe Francais' closed
with a big placard on the shutters. Change my billiard-table!" she went
on, speaking to herself, "the table that comes in so handy for folding
the washing, and on which, in the hunting season, I have slept six
visitors! But that dawdler, Hivert, doesn't come!"

"Are you waiting for him for your gentlemen's dinner?"

"Wait for him! And what about Monsieur Binet? As the clock strikes
six you'll see him come in, for he hasn't his equal under the sun for
punctuality. He must always have his seat in the small parlour. He'd
rather die than dine anywhere else. And so squeamish as he is, and so
particular about the cider! Not like Monsieur Leon; he sometimes comes
at seven, or even half-past, and he doesn't so much as look at what he
eats. Such a nice young man! Never speaks a rough word!"

"Well, you see, there's a great difference between an educated man and
an old carabineer who is now a tax-collector."

Six o'clock struck. Binet came in.

He wore a blue frock-coat falling in a straight line round his thin
body, and his leather cap, with its lappets knotted over the top of
his head with string, showed under the turned-up peak a bald forehead,
flattened by the constant wearing of a helmet. He wore a black cloth
waistcoat, a hair collar, grey trousers, and, all the year round,
well-blacked boots, that had two parallel swellings due to the sticking
out of his big-toes. Not a hair stood out from the regular line of fair
whiskers, which, encircling his jaws, framed, after the fashion of a
garden border, his long, wan face, whose eyes were small and the nose
hooked. Clever at all games of cards, a good hunter, and writing a
fine hand, he had at home a lathe, and amused himself by turning napkin
rings, with which he filled up his house, with the jealousy of an artist
and the egotism of a bourgeois.

He went to the small parlour, but the three millers had to be got out
first, and during the whole time necessary for laying the cloth, Binet
remained silent in his place near the stove. Then he shut the door and
took off his cap in his usual way.

"It isn't with saying civil things that he'll wear out his tongue," said
the chemist, as soon as he was along with the landlady.

"He never talks more," she replied. "Last week two travelers in the
cloth line were here--such clever chaps who told such jokes in the
evening, that I fairly cried with laughing; and he stood there like a
dab fish and never said a word."

"Yes," observed the chemist; "no imagination, no sallies, nothing that
makes the society-man."

"Yet they say he has parts," objected the landlady.

"Parts!" replied Monsieur Homais; "he, parts! In his own line it is
possible," he added in a calmer tone. And he went on--

"Ah! That a merchant, who has large connections, a jurisconsult, a
doctor, a chemist, should be thus absent-minded, that they should become
whimsical or even peevish, I can understand; such cases are cited in
history. But at least it is because they are thinking of something.
Myself, for example, how often has it happened to me to look on the
bureau for my pen to write a label, and to find, after all, that I had
put it behind my ear!"

Madame Lefrancois just then went to the door to see if the "Hirondelle"
were not coming. She started. A man dressed in black suddenly came into
the kitchen. By the last gleam of the twilight one could see that his
face was rubicund and his form athletic.

"What can I do for you, Monsieur le Curie?" asked the landlady, as she
reached down from the chimney one of the copper candlesticks placed
with their candles in a row. "Will you take something? A thimbleful of
Cassis*? A glass of wine?"

     *Black currant liqueur.

The priest declined very politely. He had come for his umbrella, that
he had forgotten the other day at the Ernemont convent, and after
asking Madame Lefrancois to have it sent to him at the presbytery in the
evening, he left for the church, from which the Angelus was ringing.

When the chemist no longer heard the noise of his boots along the
square, he thought the priest's behaviour just now very unbecoming. This
refusal to take any refreshment seemed to him the most odious hypocrisy;
all priests tippled on the sly, and were trying to bring back the days
of the tithe.

The landlady took up the defence of her curie.

"Besides, he could double up four men like you over his knee. Last year
he helped our people to bring in the straw; he carried as many as six
trusses at once, he is so strong."

"Bravo!" said the chemist. "Now just send your daughters to confess to
fellows which such a temperament! I, if I were the Government, I'd have
the priests bled once a month. Yes, Madame Lefrancois, every month--a
good phlebotomy, in the interests of the police and morals."

"Be quiet, Monsieur Homais. You are an infidel; you've no religion."

The chemist answered: "I have a religion, my religion, and I even have
more than all these others with their mummeries and their juggling.
I adore God, on the contrary. I believe in the Supreme Being, in a
Creator, whatever he may be. I care little who has placed us here below
to fulfil our duties as citizens and fathers of families; but I don't
need to go to church to kiss silver plates, and fatten, out of my
pocket, a lot of good-for-nothings who live better than we do. For one
can know Him as well in a wood, in a field, or even contemplating the
eternal vault like the ancients. My God! Mine is the God of Socrates, of
Franklin, of Voltaire, and of Beranger! I am for the profession of faith
of the 'Savoyard Vicar,' and the immortal principles of '89! And I can't
admit of an old boy of a God who takes walks in his garden with a
cane in his hand, who lodges his friends in the belly of whales, dies
uttering a cry, and rises again at the end of three days; things absurd
in themselves, and completely opposed, moreover, to all physical laws,
which prove to us, by the way, that priests have always wallowed in
turpid ignorance, in which they would fain engulf the people with them."

He ceased, looking round for an audience, for in his bubbling over
the chemist had for a moment fancied himself in the midst of the town
council. But the landlady no longer heeded him; she was listening to a
distant rolling. One could distinguish the noise of a carriage mingled
with the clattering of loose horseshoes that beat against the ground,
and at last the "Hirondelle" stopped at the door.

It was a yellow box on two large wheels, that, reaching to the tilt,
prevented travelers from seeing the road and dirtied their shoulders.
The small panes of the narrow windows rattled in their sashes when the
coach was closed, and retained here and there patches of mud amid the
old layers of dust, that not even storms of rain had altogether washed
away. It was drawn by three horses, the first a leader, and when it came
down-hill its bottom jolted against the ground.

Some of the inhabitants of Yonville came out into the square; they all
spoke at once, asking for news, for explanations, for hampers. Hivert
did not know whom to answer. It was he who did the errands of the place
in town. He went to the shops and brought back rolls of leather for
the shoemaker, old iron for the farrier, a barrel of herrings for his
mistress, caps from the milliner's, locks from the hair-dresser's and
all along the road on his return journey he distributed his parcels,
which he threw, standing upright on his seat and shouting at the top of
his voice, over the enclosures of the yards.

An accident had delayed him. Madame Bovary's greyhound had run across
the field. They had whistled for him a quarter of an hour; Hivert had
even gone back a mile and a half expecting every moment to catch sight
of her; but it had been necessary to go on.

Emma had wept, grown angry; she had accused Charles of this misfortune.
Monsieur Lheureux, a draper, who happened to be in the coach with
her, had tried to console her by a number of examples of lost dogs
recognizing their masters at the end of long years. One, he said had
been told of, who had come back to Paris from Constantinople. Another
had gone one hundred and fifty miles in a straight line, and swum four
rivers; and his own father had possessed a poodle, which, after twelve
years of absence, had all of a sudden jumped on his back in the street
as he was going to dine in town.



Chapter Two

Emma got out first, then Felicite, Monsieur Lheureux, and a nurse, and
they had to wake up Charles in his corner, where he had slept soundly
since night set in.

Homais introduced himself; he offered his homages to madame and his
respects to monsieur; said he was charmed to have been able to render
them some slight service, and added with a cordial air that he had
ventured to invite himself, his wife being away.

When Madame Bovary was in the kitchen she went up to the chimney.

With the tips of her fingers she caught her dress at the knee, and
having thus pulled it up to her ankle, held out her foot in its black
boot to the fire above the revolving leg of mutton. The flame lit up the
whole of her, penetrating with a crude light the woof of her gowns, the
fine pores of her fair skin, and even her eyelids, which she blinked now
and again. A great red glow passed over her with the blowing of the wind
through the half-open door.

On the other side of the chimney a young man with fair hair watched her
silently.

As he was a good deal bored at Yonville, where he was a clerk at the
notary's, Monsieur Guillaumin, Monsieur Leon Dupuis (it was he who
was the second habitue of the "Lion d'Or") frequently put back his
dinner-hour in hope that some traveler might come to the inn, with whom
he could chat in the evening. On the days when his work was done early,
he had, for want of something else to do, to come punctually, and endure
from soup to cheese a tete-a-tete with Binet. It was therefore with
delight that he accepted the landlady's suggestion that he should dine
in company with the newcomers, and they passed into the large parlour
where Madame Lefrancois, for the purpose of showing off, had had the
table laid for four.

Homais asked to be allowed to keep on his skull-cap, for fear of coryza;
then, turning to his neighbour--

"Madame is no doubt a little fatigued; one gets jolted so abominably in
our 'Hirondelle.'"

"That is true," replied Emma; "but moving about always amuses me. I like
change of place."

"It is so tedious," sighed the clerk, "to be always riveted to the same
places."

"If you were like me," said Charles, "constantly obliged to be in the
saddle"--

"But," Leon went on, addressing himself to Madame Bovary, "nothing, it
seems to me, is more pleasant--when one can," he added.

"Moreover," said the druggist, "the practice of medicine is not very
hard work in our part of the world, for the state of our roads allows us
the use of gigs, and generally, as the farmers are prosperous, they pay
pretty well. We have, medically speaking, besides the ordinary cases
of enteritis, bronchitis, bilious affections, etc., now and then a
few intermittent fevers at harvest-time; but on the whole, little of a
serious nature, nothing special to note, unless it be a great deal of
scrofula, due, no doubt, to the deplorable hygienic conditions of our
peasant dwellings. Ah! you will find many prejudices to combat, Monsieur
Bovary, much obstinacy of routine, with which all the efforts of your
science will daily come into collision; for people still have recourse
to novenas, to relics, to the priest, rather than come straight to the
doctor or the chemist. The climate, however, is not, truth to tell, bad,
and we even have a few nonagenarians in our parish. The thermometer (I
have made some observations) falls in winter to 4 degrees Centigrade
at the outside, which gives us 24 degrees Reaumur as the maximum, or
otherwise 54 degrees Fahrenheit (English scale), not more. And, as a
matter of fact, we are sheltered from the north winds by the forest of
Argueil on the one side, from the west winds by the St. Jean range on
the other; and this heat, moreover, which, on account of the aqueous
vapours given off by the river and the considerable number of cattle
in the fields, which, as you know, exhale much ammonia, that is to say,
nitrogen, hydrogen and oxygen (no, nitrogen and hydrogen alone), and
which sucking up into itself the humus from the ground, mixing together
all those different emanations, unites them into a stack, so to say,
and combining with the electricity diffused through the atmosphere, when
there is any, might in the long run, as in tropical countries, engender
insalubrious miasmata--this heat, I say, finds itself perfectly tempered
on the side whence it comes, or rather whence it should come--that is to
say, the southern side--by the south-eastern winds, which, having cooled
themselves passing over the Seine, reach us sometimes all at once like
breezes from Russia."

"At any rate, you have some walks in the neighbourhood?" continued
Madame Bovary, speaking to the young man.

"Oh, very few," he answered. "There is a place they call La Pature, on
the top of the hill, on the edge of the forest. Sometimes, on Sundays, I
go and stay there with a book, watching the sunset."

"I think there is nothing so admirable as sunsets," she resumed; "but
especially by the side of the sea."

"Oh, I adore the sea!" said Monsieur Leon.

"And then, does it not seem to you," continued Madame Bovary, "that the
mind travels more freely on this limitless expanse, the contemplation of
which elevates the soul, gives ideas of the infinite, the ideal?"

"It is the same with mountainous landscapes," continued Leon. "A cousin
of mine who travelled in Switzerland last year told me that one could
not picture to oneself the poetry of the lakes, the charm of the
waterfalls, the gigantic effect of the glaciers. One sees pines of
incredible size across torrents, cottages suspended over precipices,
and, a thousand feet below one, whole valleys when the clouds open. Such
spectacles must stir to enthusiasm, incline to prayer, to ecstasy; and I
no longer marvel at that celebrated musician who, the better to inspire
his imagination, was in the habit of playing the piano before some
imposing site."

"You play?" she asked.

"No, but I am very fond of music," he replied.

"Ah! don't you listen to him, Madame Bovary," interrupted Homais,
bending over his plate. "That's sheer modesty. Why, my dear fellow, the
other day in your room you were singing 'L'Ange Gardien' ravishingly. I
heard you from the laboratory. You gave it like an actor."

Leon, in fact, lodged at the chemist's where he had a small room on the
second floor, overlooking the Place. He blushed at the compliment of his
landlord, who had already turned to the doctor, and was enumerating to
him, one after the other, all the principal inhabitants of Yonville. He
was telling anecdotes, giving information; the fortune of the notary
was not known exactly, and "there was the Tuvache household," who made a
good deal of show.

Emma continued, "And what music do you prefer?"

"Oh, German music; that which makes you dream."

"Have you been to the opera?"

"Not yet; but I shall go next year, when I am living at Paris to finish
reading for the bar."

"As I had the honour of putting it to your husband," said the chemist,
"with regard to this poor Yanoda who has run away, you will find
yourself, thanks to his extravagance, in the possession of one of the
most comfortable houses of Yonville. Its greatest convenience for a
doctor is a door giving on the Walk, where one can go in and out unseen.
Moreover, it contains everything that is agreeable in a household--a
laundry, kitchen with offices, sitting-room, fruit-room, and so on. He
was a gay dog, who didn't care what he spent. At the end of the garden,
by the side of the water, he had an arbour built just for the purpose of
drinking beer in summer; and if madame is fond of gardening she will be
able--"

"My wife doesn't care about it," said Charles; "although she has
been advised to take exercise, she prefers always sitting in her room
reading."

"Like me," replied Leon. "And indeed, what is better than to sit by
one's fireside in the evening with a book, while the wind beats against
the window and the lamp is burning?"

"What, indeed?" she said, fixing her large black eyes wide open upon
him.

"One thinks of nothing," he continued; "the hours slip by. Motionless we
traverse countries we fancy we see, and your thought, blending with
the fiction, playing with the details, follows the outline of the
adventures. It mingles with the characters, and it seems as if it were
yourself palpitating beneath their costumes."

"That is true! That is true?" she said.

"Has it ever happened to you," Leon went on, "to come across some vague
idea of one's own in a book, some dim image that comes back to you from
afar, and as the completest expression of your own slightest sentiment?"

"I have experienced it," she replied.

"That is why," he said, "I especially love the poets. I think verse more
tender than prose, and that it moves far more easily to tears."

"Still in the long run it is tiring," continued Emma. "Now I, on the
contrary, adore stories that rush breathlessly along, that frighten one.
I detest commonplace heroes and moderate sentiments, such as there are
in nature."

"In fact," observed the clerk, "these works, not touching the heart,
miss, it seems to me, the true end of art. It is so sweet, amid all
the disenchantments of life, to be able to dwell in thought upon noble
characters, pure affections, and pictures of happiness. For myself,
living here far from the world, this is my one distraction; but Yonville
affords so few resources."

"Like Tostes, no doubt," replied Emma; "and so I always subscribed to a
lending library."

"If madame will do me the honour of making use of it", said the chemist,
who had just caught the last words, "I have at her disposal a library
composed of the best authors, Voltaire, Rousseau, Delille, Walter
Scott, the 'Echo des Feuilletons'; and in addition I receive various
periodicals, among them the 'Fanal de Rouen' daily, having the advantage
to be its correspondent for the districts of Buchy, Forges, Neufchatel,
Yonville, and vicinity."

For two hours and a half they had been at table; for the servant
Artemis, carelessly dragging her old list slippers over the flags,
brought one plate after the other, forgot everything, and constantly
left the door of the billiard-room half open, so that it beat against
the wall with its hooks.

Unconsciously, Leon, while talking, had placed his foot on one of the
bars of the chair on which Madame Bovary was sitting. She wore a small
blue silk necktie, that kept up like a ruff a gauffered cambric collar,
and with the movements of her head the lower part of her face gently
sunk into the linen or came out from it. Thus side by side, while
Charles and the chemist chatted, they entered into one of those vague
conversations where the hazard of all that is said brings you back to
the fixed centre of a common sympathy. The Paris theatres, titles of
novels, new quadrilles, and the world they did not know; Tostes, where
she had lived, and Yonville, where they were; they examined all, talked
of everything till to the end of dinner.

When coffee was served Felicite went away to get ready the room in the
new house, and the guests soon raised the siege. Madame Lefrancois was
asleep near the cinders, while the stable-boy, lantern in hand, was
waiting to show Monsieur and Madame Bovary the way home. Bits of straw
stuck in his red hair, and he limped with his left leg. When he had
taken in his other hand the cure's umbrella, they started.

The town was asleep; the pillars of the market threw great shadows; the
earth was all grey as on a summer's night. But as the doctor's house was
only some fifty paces from the inn, they had to say good-night almost
immediately, and the company dispersed.

As soon as she entered the passage, Emma felt the cold of the plaster
fall about her shoulders like damp linen. The walls were new and the
wooden stairs creaked. In their bedroom, on the first floor, a whitish
light passed through the curtainless windows.

She could catch glimpses of tree tops, and beyond, the fields,
half-drowned in the fog that lay reeking in the moonlight along
the course of the river. In the middle of the room, pell-mell, were
scattered drawers, bottles, curtain-rods, gilt poles, with mattresses
on the chairs and basins on the ground--the two men who had brought the
furniture had left everything about carelessly.

This was the fourth time that she had slept in a strange place.

The first was the day of her going to the convent; the second, of her
arrival at Tostes; the third, at Vaubyessard; and this was the fourth.
And each one had marked, as it were, the inauguration of a new phase in
her life. She did not believe that things could present themselves in
the same way in different places, and since the portion of her life
lived had been bad, no doubt that which remained to be lived would be
better.



Chapter Three

The next day, as she was getting up, she saw the clerk on the Place. She
had on a dressing-gown. He looked up and bowed. She nodded quickly and
reclosed the window.

Leon waited all day for six o'clock in the evening to come, but on going
to the inn, he found no one but Monsieur Binet, already at table. The
dinner of the evening before had been a considerable event for him; he
had never till then talked for two hours consecutively to a "lady." How
then had he been able to explain, and in such language, the number of
things that he could not have said so well before? He was usually
shy, and maintained that reserve which partakes at once of modesty and
dissimulation.

At Yonville he was considered "well-bred." He listened to the arguments
of the older people, and did not seem hot about politics--a remarkable
thing for a young man. Then he had some accomplishments; he painted in
water-colours, could read the key of G, and readily talked literature
after dinner when he did not play cards. Monsieur Homais respected him
for his education; Madame Homais liked him for his good-nature, for
he often took the little Homais into the garden--little brats who were
always dirty, very much spoilt, and somewhat lymphatic, like their
mother. Besides the servant to look after them, they had Justin, the
chemist's apprentice, a second cousin of Monsieur Homais, who had been
taken into the house from charity, and who was useful at the same time
as a servant.

The druggist proved the best of neighbours. He gave Madame Bovary
information as to the trades-people, sent expressly for his own cider
merchant, tasted the drink himself, and saw that the casks were properly
placed in the cellar; he explained how to set about getting in a
supply of butter cheap, and made an arrangement with Lestiboudois, the
sacristan, who, besides his sacerdotal and funeral functions, looked
after the principal gardens at Yonville by the hour or the year,
according to the taste of the customers.

The need of looking after others was not the only thing that urged the
chemist to such obsequious cordiality; there was a plan underneath it
all.

He had infringed the law of the 19th Ventose, year xi., article I, which
forbade all persons not having a diploma to practise medicine; so that,
after certain anonymous denunciations, Homais had been summoned to Rouen
to see the procurer of the king in his own private room; the magistrate
receiving him standing up, ermine on shoulder and cap on head. It was
in the morning, before the court opened. In the corridors one heard
the heavy boots of the gendarmes walking past, and like a far-off noise
great locks that were shut. The druggist's ears tingled as if he were
about to have an apoplectic stroke; he saw the depths of dungeons,
his family in tears, his shop sold, all the jars dispersed; and he was
obliged to enter a cafe and take a glass of rum and seltzer to recover
his spirits.

Little by little the memory of this reprimand grew fainter, and
he continued, as heretofore, to give anodyne consultations in his
back-parlour. But the mayor resented it, his colleagues were jealous,
everything was to be feared; gaining over Monsieur Bovary by his
attentions was to earn his gratitude, and prevent his speaking out later
on, should he notice anything. So every morning Homais brought him "the
paper," and often in the afternoon left his shop for a few moments to
have a chat with the Doctor.

Charles was dull: patients did not come. He remained seated for hours
without speaking, went into his consulting room to sleep, or watched
his wife sewing. Then for diversion he employed himself at home as a
workman; he even tried to do up the attic with some paint which had been
left behind by the painters. But money matters worried him. He had
spent so much for repairs at Tostes, for madame's toilette, and for the
moving, that the whole dowry, over three thousand crowns, had slipped
away in two years.

Then how many things had been spoilt or lost during their carriage from
Tostes to Yonville, without counting the plaster cure, who falling out
of the coach at an over-severe jolt, had been dashed into a thousand
fragments on the pavements of Quincampoix! A pleasanter trouble came
to distract him, namely, the pregnancy of his wife. As the time of her
confinement approached he cherished her the more. It was another bond of
the flesh establishing itself, and, as it were, a continued sentiment
of a more complex union. When from afar he saw her languid walk, and
her figure without stays turning softly on her hips; when opposite one
another he looked at her at his ease, while she took tired poses in her
armchair, then his happiness knew no bounds; he got up, embraced her,
passed his hands over her face, called her little mamma, wanted to
make her dance, and half-laughing, half-crying, uttered all kinds of
caressing pleasantries that came into his head. The idea of having
begotten a child delighted him. Now he wanted nothing. He knew human
life from end to end, and he sat down to it with serenity.

Emma at first felt a great astonishment; then was anxious to be
delivered that she might know what it was to be a mother. But not
being able to spend as much as she would have liked, to have a
swing-bassinette with rose silk curtains, and embroidered caps, in a fit
of bitterness she gave up looking after the trousseau, and ordered the
whole of it from a village needlewoman, without choosing or discussing
anything. Thus she did not amuse herself with those preparations that
stimulate the tenderness of mothers, and so her affection was from the
very outset, perhaps, to some extent attenuated.

As Charles, however, spoke of the boy at every meal, she soon began to
think of him more consecutively.

She hoped for a son; he would be strong and dark; she would call him
George; and this idea of having a male child was like an expected
revenge for all her impotence in the past. A man, at least, is free; he
may travel over passions and over countries, overcome obstacles, taste
of the most far-away pleasures. But a woman is always hampered. At once
inert and flexible, she has against her the weakness of the flesh and
legal dependence. Her will, like the veil of her bonnet, held by a
string, flutters in every wind; there is always some desire that draws
her, some conventionality that restrains.

She was confined on a Sunday at about six o'clock, as the sun was
rising.

"It is a girl!" said Charles.

She turned her head away and fainted.

Madame Homais, as well as Madame Lefrancois of the Lion d'Or, almost
immediately came running in to embrace her. The chemist, as man of
discretion, only offered a few provincial felicitations through the
half-opened door. He wished to see the child and thought it well made.

Whilst she was getting well she occupied herself much in seeking a
name for her daughter. First she went over all those that have Italian
endings, such as Clara, Louisa, Amanda, Atala; she liked Galsuinde
pretty well, and Yseult or Leocadie still better.

Charles wanted the child to be called after her mother; Emma opposed
this. They ran over the calendar from end to end, and then consulted
outsiders.

"Monsieur Leon," said the chemist, "with whom I was talking about it
the other day, wonders you do not chose Madeleine. It is very much in
fashion just now."

But Madame Bovary, senior, cried out loudly against this name of a
sinner. As to Monsieur Homais, he had a preference for all those that
recalled some great man, an illustrious fact, or a generous idea, and it
was on this system that he had baptized his four children. Thus Napoleon
represented glory and Franklin liberty; Irma was perhaps a concession to
romanticism, but Athalie was a homage to the greatest masterpiece of the
French stage. For his philosophical convictions did not interfere
with his artistic tastes; in him the thinker did not stifle the man of
sentiment; he could make distinctions, make allowances for imagination
and fanaticism. In this tragedy, for example, he found fault with the
ideas, but admired the style; he detested the conception, but applauded
all the details, and loathed the characters while he grew enthusiastic
over their dialogue. When he read the fine passages he was transported,
but when he thought that mummers would get something out of them for
their show, he was disconsolate; and in this confusion of sentiments in
which he was involved he would have liked at once to crown Racine with
both his hands and discuss with him for a good quarter of an hour.

At last Emma remembered that at the chateau of Vaubyessard she had heard
the Marchioness call a young lady Berthe; from that moment this name was
chosen; and as old Rouault could not come, Monsieur Homais was requested
to stand godfather. His gifts were all products from his establishment,
to wit: six boxes of jujubes, a whole jar of racahout, three cakes of
marshmallow paste, and six sticks of sugar-candy into the bargain that
he had come across in a cupboard. On the evening of the ceremony there
was a grand dinner; the cure was present; there was much excitement.
Monsieur Homais towards liqueur-time began singing "Le Dieu des bonnes
gens." Monsieur Leon sang a barcarolle, and Madame Bovary, senior, who
was godmother, a romance of the time of the Empire; finally, M. Bovary,
senior, insisted on having the child brought down, and began baptizing
it with a glass of champagne that he poured over its head. This mockery
of the first of the sacraments made the Abbe Bournisien angry; old
Bovary replied by a quotation from "La Guerre des Dieux"; the cure
wanted to leave; the ladies implored, Homais interfered; and they
succeeded in making the priest sit down again, and he quietly went on
with the half-finished coffee in his saucer.

Monsieur Bovary, senior, stayed at Yonville a month, dazzling the
natives by a superb policeman's cap with silver tassels that he wore
in the morning when he smoked his pipe in the square. Being also in the
habit of drinking a good deal of brandy, he often sent the servant
to the Lion d'Or to buy him a bottle, which was put down to his
son's account, and to perfume his handkerchiefs he used up his
daughter-in-law's whole supply of eau-de-cologne.

The latter did not at all dislike his company. He had knocked about the
world, he talked about Berlin, Vienna, and Strasbourg, of his soldier
times, of the mistresses he had had, the grand luncheons of which he had
partaken; then he was amiable, and sometimes even, either on the stairs,
or in the garden, would seize hold of her waist, crying, "Charles, look
out for yourself."

Then Madame Bovary, senior, became alarmed for her son's happiness, and
fearing that her husband might in the long-run have an immoral influence
upon the ideas of the young woman, took care to hurry their departure.
Perhaps she had more serious reasons for uneasiness. Monsieur Bovary was
not the man to respect anything.

One day Emma was suddenly seized with the desire to see her little
girl, who had been put to nurse with the carpenter's wife, and, without
looking at the calendar to see whether the six weeks of the Virgin were
yet passed, she set out for the Rollets' house, situated at the extreme
end of the village, between the highroad and the fields.

It was mid-day, the shutters of the houses were closed and the slate
roofs that glittered beneath the fierce light of the blue sky seemed to
strike sparks from the crest of the gables. A heavy wind was blowing;
Emma felt weak as she walked; the stones of the pavement hurt her; she
was doubtful whether she would not go home again, or go in somewhere to
rest.

At this moment Monsieur Leon came out from a neighbouring door with a
bundle of papers under his arm. He came to greet her, and stood in the
shade in front of the Lheureux's shop under the projecting grey awning.

Madame Bovary said she was going to see her baby, but that she was
beginning to grow tired.

"If--" said Leon, not daring to go on.

"Have you any business to attend to?" she asked.

And on the clerk's answer, she begged him to accompany her. That same
evening this was known in Yonville, and Madame Tuvache, the mayor's
wife, declared in the presence of her servant that "Madame Bovary was
compromising herself."

To get to the nurse's it was necessary to turn to the left on leaving
the street, as if making for the cemetery, and to follow between little
houses and yards a small path bordered with privet hedges. They were
in bloom, and so were the speedwells, eglantines, thistles, and the
sweetbriar that sprang up from the thickets. Through openings in
the hedges one could see into the huts, some pigs on a dung-heap, or
tethered cows rubbing their horns against the trunk of trees. The two,
side by side walked slowly, she leaning upon him, and he restraining
his pace, which he regulated by hers; in front of them a swarm of midges
fluttered, buzzing in the warm air.

They recognized the house by an old walnut-tree which shaded it.

Low and covered with brown tiles, there hung outside it, beneath the
dormer-window of the garret, a string of onions. Faggots upright
against a thorn fence surrounded a bed of lettuce, a few square feet of
lavender, and sweet peas strung on sticks. Dirty water was running here
and there on the grass, and all round were several indefinite rags,
knitted stockings, a red calico jacket, and a large sheet of coarse
linen spread over the hedge. At the noise of the gate the nurse appeared
with a baby she was suckling on one arm. With her other hand she was
pulling along a poor puny little fellow, his face covered with scrofula,
the son of a Rouen hosier, whom his parents, too taken up with their
business, left in the country.

"Go in," she said; "your little one is there asleep."

The room on the ground-floor, the only one in the dwelling, had at its
farther end, against the wall, a large bed without curtains, while a
kneading-trough took up the side by the window, one pane of which
was mended with a piece of blue paper. In the corner behind the door,
shining hob-nailed shoes stood in a row under the slab of the washstand,
near a bottle of oil with a feather stuck in its mouth; a Matthieu
Laensberg lay on the dusty mantelpiece amid gunflints, candle-ends, and
bits of amadou.

Finally, the last luxury in the apartment was a "Fame" blowing her
trumpets, a picture cut out, no doubt, from some perfumer's prospectus
and nailed to the wall with six wooden shoe-pegs.

Emma's child was asleep in a wicker-cradle. She took it up in the
wrapping that enveloped it and began singing softly as she rocked
herself to and fro.

Leon walked up and down the room; it seemed strange to him to see this
beautiful woman in her nankeen dress in the midst of all this poverty.
Madam Bovary reddened; he turned away, thinking perhaps there had been
an impertinent look in his eyes. Then she put back the little girl, who
had just been sick over her collar.

The nurse at once came to dry her, protesting that it wouldn't show.

"She gives me other doses," she said: "I am always a-washing of her. If
you would have the goodness to order Camus, the grocer, to let me have
a little soap, it would really be more convenient for you, as I needn't
trouble you then."

"Very well! very well!" said Emma. "Good morning, Madame Rollet," and
she went out, wiping her shoes at the door.

The good woman accompanied her to the end of the garden, talking all the
time of the trouble she had getting up of nights.

"I'm that worn out sometimes as I drop asleep on my chair. I'm sure you
might at least give me just a pound of ground coffee; that'd last me a
month, and I'd take it of a morning with some milk."

After having submitted to her thanks, Madam Bovary left. She had gone a
little way down the path when, at the sound of wooden shoes, she turned
round. It was the nurse.

"What is it?"

Then the peasant woman, taking her aside behind an elm tree, began
talking to her of her husband, who with his trade and six francs a year
that the captain--

"Oh, be quick!" said Emma.

"Well," the nurse went on, heaving sighs between each word, "I'm afraid
he'll be put out seeing me have coffee alone, you know men--"

"But you are to have some," Emma repeated; "I will give you some. You
bother me!"

"Oh, dear! my poor, dear lady! you see in consequence of his wounds he
has terrible cramps in the chest. He even says that cider weakens him."

"Do make haste, Mere Rollet!"

"Well," the latter continued, making a curtsey, "if it weren't asking
too much," and she curtsied once more, "if you would"--and her eyes
begged--"a jar of brandy," she said at last, "and I'd rub your little
one's feet with it; they're as tender as one's tongue."

Once rid of the nurse, Emma again took Monsieur Leon's arm. She walked
fast for some time, then more slowly, and looking straight in front of
her, her eyes rested on the shoulder of the young man, whose frock-coat
had a black-velvety collar. His brown hair fell over it, straight and
carefully arranged. She noticed his nails which were longer than one
wore them at Yonville. It was one of the clerk's chief occupations to
trim them, and for this purpose he kept a special knife in his writing
desk.

They returned to Yonville by the water-side. In the warm season the
bank, wider than at other times, showed to their foot the garden walls
whence a few steps led to the river. It flowed noiselessly, swift,
and cold to the eye; long, thin grasses huddled together in it as the
current drove them, and spread themselves upon the limpid water like
streaming hair; sometimes at the tip of the reeds or on the leaf of a
water-lily an insect with fine legs crawled or rested. The sun pierced
with a ray the small blue bubbles of the waves that, breaking, followed
each other; branchless old willows mirrored their grey backs in
the water; beyond, all around, the meadows seemed empty. It was the
dinner-hour at the farms, and the young woman and her companion heard
nothing as they walked but the fall of their steps on the earth of the
path, the words they spoke, and the sound of Emma's dress rustling round
her.

The walls of the gardens with pieces of bottle on their coping were
hot as the glass windows of a conservatory. Wallflowers had sprung up
between the bricks, and with the tip of her open sunshade Madame Bovary,
as she passed, made some of their faded flowers crumble into a yellow
dust, or a spray of overhanging honeysuckle and clematis caught in its
fringe and dangled for a moment over the silk.

They were talking of a troupe of Spanish dancers who were expected
shortly at the Rouen theatre.

"Are you going?" she asked.

"If I can," he answered.

Had they nothing else to say to one another? Yet their eyes were full
of more serious speech, and while they forced themselves to find trivial
phrases, they felt the same languor stealing over them both. It was the
whisper of the soul, deep, continuous, dominating that of their voices.
Surprised with wonder at this strange sweetness, they did not think of
speaking of the sensation or of seeking its cause. Coming joys, like
tropical shores, throw over the immensity before them their inborn
softness, an odorous wind, and we are lulled by this intoxication
without a thought of the horizon that we do not even know.

In one place the ground had been trodden down by the cattle; they had to
step on large green stones put here and there in the mud.

She often stopped a moment to look where to place her foot, and
tottering on a stone that shook, her arms outspread, her form bent
forward with a look of indecision, she would laugh, afraid of falling
into the puddles of water.

When they arrived in front of her garden, Madame Bovary opened the
little gate, ran up the steps and disappeared.

Leon returned to his office. His chief was away; he just glanced at the
briefs, then cut himself a pen, and at last took up his hat and went
out.

He went to La Pature at the top of the Argueil hills at the beginning of
the forest; he threw himself upon the ground under the pines and watched
the sky through his fingers.

"How bored I am!" he said to himself, "how bored I am!"

He thought he was to be pitied for living in this village, with Homais
for a friend and Monsieru Guillaumin for master. The latter, entirely
absorbed by his business, wearing gold-rimmed spectacles and red
whiskers over a white cravat, understood nothing of mental refinements,
although he affected a stiff English manner, which in the beginning had
impressed the clerk.

As to the chemist's spouse, she was the best wife in Normandy, gentle
as a sheep, loving her children, her father, her mother, her cousins,
weeping for other's woes, letting everything go in her household, and
detesting corsets; but so slow of movement, such a bore to listen to, so
common in appearance, and of such restricted conversation, that although
she was thirty, he only twenty, although they slept in rooms next each
other and he spoke to her daily, he never thought that she might be a
woman for another, or that she possessed anything else of her sex than
the gown.

And what else was there? Binet, a few shopkeepers, two or three
publicans, the cure, and finally, Monsieur Tuvache, the mayor, with his
two sons, rich, crabbed, obtuse persons, who farmed their own lands
and had feasts among themselves, bigoted to boot, and quite unbearable
companions.

But from the general background of all these human faces Emma's stood
out isolated and yet farthest off; for between her and him he seemed to
see a vague abyss.

In the beginning he had called on her several times along with the
druggist. Charles had not appeared particularly anxious to see him
again, and Leon did not know what to do between his fear of being
indiscreet and the desire for an intimacy that seemed almost impossible.



Chapter Four

When the first cold days set in Emma left her bedroom for the
sitting-room, a long apartment with a low ceiling, in which there was
on the mantelpiece a large bunch of coral spread out against the
looking-glass. Seated in her arm chair near the window, she could see
the villagers pass along the pavement.

Twice a day Leon went from his office to the Lion d'Or. Emma could hear
him coming from afar; she leant forward listening, and the young man
glided past the curtain, always dressed in the same way, and without
turning his head. But in the twilight, when, her chin resting on her
left hand, she let the embroidery she had begun fall on her knees, she
often shuddered at the apparition of this shadow suddenly gliding past.
She would get up and order the table to be laid.

Monsieur Homais called at dinner-time. Skull-cap in hand, he came in on
tiptoe, in order to disturb no one, always repeating the same phrase,
"Good evening, everybody." Then, when he had taken his seat at the table
between the pair, he asked the doctor about his patients, and the latter
consulted his as to the probability of their payment. Next they talked
of "what was in the paper."

Homais by this hour knew it almost by heart, and he repeated it from end
to end, with the reflections of the penny-a-liners, and all the stories
of individual catastrophes that had occurred in France or abroad. But
the subject becoming exhausted, he was not slow in throwing out some
remarks on the dishes before him.

Sometimes even, half-rising, he delicately pointed out to madame the
tenderest morsel, or turning to the servant, gave her some advice on the
manipulation of stews and the hygiene of seasoning.

He talked aroma, osmazome, juices, and gelatine in a bewildering manner.
Moreover, Homais, with his head fuller of recipes than his shop of jars,
excelled in making all kinds of preserves, vinegars, and sweet liqueurs;
he knew also all the latest inventions in economic stoves, together with
the art of preserving cheese and of curing sick wines.

At eight o'clock Justin came to fetch him to shut up the shop.

Then Monsieur Homais gave him a sly look, especially if Felicite was
there, for he half noticed that his apprentice was fond of the doctor's
house.

"The young dog," he said, "is beginning to have ideas, and the devil
take me if I don't believe he's in love with your servant!"

But a more serious fault with which he reproached Justin was his
constantly listening to conversation. On Sunday, for example, one could
not get him out of the drawing-room, whither Madame Homais had called
him to fetch the children, who were falling asleep in the arm-chairs,
and dragging down with their backs calico chair-covers that were too
large.

Not many people came to these soirees at the chemist's, his
scandal-mongering and political opinions having successfully alienated
various respectable persons from him. The clerk never failed to be
there. As soon as he heard the bell he ran to meet Madame Bovary, took
her shawl, and put away under the shop-counter the thick list shoes that
she wore over her boots when there was snow.

First they played some hands at trente-et-un; next Monsieur Homais
played ecarte with Emma; Leon behind her gave her advice.

Standing up with his hands on the back of her chair he saw the teeth of
her comb that bit into her chignon. With every movement that she made
to throw her cards the right side of her dress was drawn up. From her
turned-up hair a dark colour fell over her back, and growing gradually
paler, lost itself little by little in the shade. Then her dress fell
on both sides of her chair, puffing out full of folds, and reached the
ground. When Leon occasionally felt the sole of his boot resting on it,
he drew back as if he had trodden upon some one.

When the game of cards was over, the druggist and the Doctor played
dominoes, and Emma, changing her place, leant her elbow on the table,
turning over the leaves of "L'Illustration". She had brought her ladies'
journal with her. Leon sat down near her; they looked at the engravings
together, and waited for one another at the bottom of the pages. She
often begged him to read her the verses; Leon declaimed them in a
languid voice, to which he carefully gave a dying fall in the love
passages. But the noise of the dominoes annoyed him. Monsieur Homais
was strong at the game; he could beat Charles and give him a double-six.
Then the three hundred finished, they both stretched themselves out in
front of the fire, and were soon asleep. The fire was dying out in the
cinders; the teapot was empty, Leon was still reading.

Emma listened to him, mechanically turning around the lampshade, on the
gauze of which were painted clowns in carriages, and tight-rope dances
with their balancing-poles. Leon stopped, pointing with a gesture to his
sleeping audience; then they talked in low tones, and their conversation
seemed the more sweet to them because it was unheard.

Thus a kind of bond was established between them, a constant commerce
of books and of romances. Monsieur Bovary, little given to jealousy, did
not trouble himself about it.

On his birthday he received a beautiful phrenological head, all marked
with figures to the thorax and painted blue. This was an attention of
the clerk's. He showed him many others, even to doing errands for him
at Rouen; and the book of a novelist having made the mania for cactuses
fashionable, Leon bought some for Madame Bovary, bringing them back on
his knees in the "Hirondelle," pricking his fingers on their hard hairs.

She had a board with a balustrade fixed against her window to hold the
pots. The clerk, too, had his small hanging garden; they saw each other
tending their flowers at their windows.

Of the windows of the village there was one yet more often occupied; for
on Sundays from morning to night, and every morning when the weather was
bright, one could see at the dormer-window of the garret the profile of
Monsieur Binet bending over his lathe, whose monotonous humming could be
heard at the Lion d'Or.

One evening on coming home Leon found in his room a rug in velvet and
wool with leaves on a pale ground. He called Madame Homais, Monsieur
Homais, Justin, the children, the cook; he spoke of it to his chief;
every one wanted to see this rug. Why did the doctor's wife give the
clerk presents? It looked queer. They decided that she must be his
lover.

He made this seem likely, so ceaselessly did he talk of her charms and
of her wit; so much so, that Binet once roughly answered him--

"What does it matter to me since I'm not in her set?"

He tortured himself to find out how he could make his declaration to
her, and always halting between the fear of displeasing her and the
shame of being such a coward, he wept with discouragement and desire.
Then he took energetic resolutions, wrote letters that he tore up, put
it off to times that he again deferred.

Often he set out with the determination to dare all; but this resolution
soon deserted him in Emma's presence, and when Charles, dropping in,
invited him to jump into his chaise to go with him to see some patient
in the neighbourhood, he at once accepted, bowed to madame, and went
out. Her husband, was he not something belonging to her? As to Emma,
she did not ask herself whether she loved. Love, she thought, must come
suddenly, with great outbursts and lightnings--a hurricane of the skies,
which falls upon life, revolutionises it, roots up the will like a leaf,
and sweeps the whole heart into the abyss. She did not know that on
the terrace of houses it makes lakes when the pipes are choked, and she
would thus have remained in her security when she suddenly discovered a
rent in the wall of it.



Chapter Five

It was a Sunday in February, an afternoon when the snow was falling.

They had all, Monsieur and Madame Bovary, Homais, and Monsieur Leon,
gone to see a yarn-mill that was being built in the valley a mile and a
half from Yonville. The druggist had taken Napoleon and Athalie to give
them some exercise, and Justin accompanied them, carrying the umbrellas
on his shoulder.

Nothing, however, could be less curious than this curiosity. A great
piece of waste ground, on which pell-mell, amid a mass of sand and
stones, were a few break-wheels, already rusty, surrounded by a
quadrangular building pierced by a number of little windows. The
building was unfinished; the sky could be seen through the joists of the
roofing. Attached to the stop-plank of the gable a bunch of straw mixed
with corn-ears fluttered its tricoloured ribbons in the wind.

Homais was talking. He explained to the company the future importance
of this establishment, computed the strength of the floorings, the
thickness of the walls, and regretted extremely not having a yard-stick
such as Monsieur Binet possessed for his own special use.

Emma, who had taken his arm, bent lightly against his shoulder, and
she looked at the sun's disc shedding afar through the mist his pale
splendour. She turned. Charles was there. His cap was drawn down over
his eyebrows, and his two thick lips were trembling, which added a look
of stupidity to his face; his very back, his calm back, was irritating
to behold, and she saw written upon his coat all the platitude of the
bearer.

While she was considering him thus, tasting in her irritation a sort of
depraved pleasure, Leon made a step forward. The cold that made him pale
seemed to add a more gentle languor to his face; between his cravat and
his neck the somewhat loose collar of his shirt showed the skin; the
lobe of his ear looked out from beneath a lock of hair, and his large
blue eyes, raised to the clouds, seemed to Emma more limpid and more
beautiful than those mountain-lakes where the heavens are mirrored.

"Wretched boy!" suddenly cried the chemist.

And he ran to his son, who had just precipitated himself into a heap of
lime in order to whiten his boots. At the reproaches with which he was
being overwhelmed Napoleon began to roar, while Justin dried his shoes
with a wisp of straw. But a knife was wanted; Charles offered his.

"Ah!" she said to herself, "he carried a knife in his pocket like a
peasant."

The hoar-frost was falling, and they turned back to Yonville.

In the evening Madame Bovary did not go to her neighbour's, and when
Charles had left and she felt herself alone, the comparison re-began
with the clearness of a sensation almost actual, and with that
lengthening of perspective which memory gives to things. Looking from
her bed at the clean fire that was burning, she still saw, as she had
down there, Leon standing up with one hand behind his cane, and with
the other holding Athalie, who was quietly sucking a piece of ice. She
thought him charming; she could not tear herself away from him; she
recalled his other attitudes on other days, the words he had spoken, the
sound of his voice, his whole person; and she repeated, pouting out her
lips as if for a kiss--

"Yes, charming! charming! Is he not in love?" she asked herself; "but
with whom? With me?"

All the proofs arose before her at once; her heart leapt. The flame of
the fire threw a joyous light upon the ceiling; she turned on her back,
stretching out her arms.

Then began the eternal lamentation: "Oh, if Heaven had out willed it!
And why not? What prevented it?"

When Charles came home at midnight, she seemed to have just awakened,
and as he made a noise undressing, she complained of a headache, then
asked carelessly what had happened that evening.

"Monsieur Leon," he said, "went to his room early."

She could not help smiling, and she fell asleep, her soul filled with a
new delight.

The next day, at dusk, she received a visit from Monsieur Lherueux, the
draper. He was a man of ability, was this shopkeeper. Born a Gascon but
bred a Norman, he grafted upon his southern volubility the cunning of
the Cauchois. His fat, flabby, beardless face seemed dyed by a
decoction of liquorice, and his white hair made even more vivid the
keen brilliance of his small black eyes. No one knew what he had been
formerly; a pedlar said some, a banker at Routot according to others.
What was certain was that he made complex calculations in his head that
would have frightened Binet himself. Polite to obsequiousness, he always
held himself with his back bent in the position of one who bows or who
invites.

After leaving at the door his hat surrounded with crape, he put down
a green bandbox on the table, and began by complaining to madame, with
many civilities, that he should have remained till that day without
gaining her confidence. A poor shop like his was not made to attract
a "fashionable lady"; he emphasized the words; yet she had only to
command, and he would undertake to provide her with anything she might
wish, either in haberdashery or linen, millinery or fancy goods, for
he went to town regularly four times a month. He was connected with the
best houses. You could speak of him at the "Trois Freres," at the "Barbe
d'Or," or at the "Grand Sauvage"; all these gentlemen knew him as
well as the insides of their pockets. To-day, then he had come to show
madame, in passing, various articles he happened to have, thanks to
the most rare opportunity. And he pulled out half-a-dozen embroidered
collars from the box.

Madame Bovary examined them. "I do not require anything," she said.

Then Monsieur Lheureux delicately exhibited three Algerian scarves,
several packets of English needles, a pair of straw slippers, and
finally, four eggcups in cocoanut wood, carved in open work by convicts.
Then, with both hands on the table, his neck stretched out, his figure
bent forward, open-mouthed, he watched Emma's look, who was walking up
and down undecided amid these goods. From time to time, as if to remove
some dust, he filliped with his nail the silk of the scarves spread
out at full length, and they rustled with a little noise, making in the
green twilight the gold spangles of their tissue scintillate like little
stars.

"How much are they?"

"A mere nothing," he replied, "a mere nothing. But there's no hurry;
whenever it's convenient. We are not Jews."

She reflected for a few moments, and ended by again declining Monsieur
Lheureux's offer. He replied quite unconcernedly--

"Very well. We shall understand one another by and by. I have always got
on with ladies--if I didn't with my own!"

Emma smiled.

"I wanted to tell you," he went on good-naturedly, after his joke, "that
it isn't the money I should trouble about. Why, I could give you some,
if need be."

She made a gesture of surprise.

"Ah!" said he quickly and in a low voice, "I shouldn't have to go far to
find you some, rely on that."

And he began asking after Pere Tellier, the proprietor of the "Cafe
Francais," whom Monsieur Bovary was then attending.

"What's the matter with Pere Tellier? He coughs so that he shakes his
whole house, and I'm afraid he'll soon want a deal covering rather than
a flannel vest. He was such a rake as a young man! Those sort of people,
madame, have not the least regularity; he's burnt up with brandy. Still
it's sad, all the same, to see an acquaintance go off."

And while he fastened up his box he discoursed about the doctor's
patients.

"It's the weather, no doubt," he said, looking frowningly at the floor,
"that causes these illnesses. I, too, don't feel the thing. One of these
days I shall even have to consult the doctor for a pain I have in my
back. Well, good-bye, Madame Bovary. At your service; your very humble
servant." And he closed the door gently.

Emma had her dinner served in her bedroom on a tray by the fireside; she
was a long time over it; everything was well with her.

"How good I was!" she said to herself, thinking of the scarves.

She heard some steps on the stairs. It was Leon. She got up and took
from the chest of drawers the first pile of dusters to be hemmed. When
he came in she seemed very busy.

The conversation languished; Madame Bovary gave it up every few minutes,
whilst he himself seemed quite embarrassed. Seated on a low chair near
the fire, he turned round in his fingers the ivory thimble-case. She
stitched on, or from time to time turned down the hem of the cloth with
her nail. She did not speak; he was silent, captivated by her silence,
as he would have been by her speech.

"Poor fellow!" she thought.

"How have I displeased her?" he asked himself.

At last, however, Leon said that he should have, one of these days, to
go to Rouen on some office business.

"Your music subscription is out; am I to renew it?"

"No," she replied.

"Why?"

"Because--"

And pursing her lips she slowly drew a long stitch of grey thread.

This work irritated Leon. It seemed to roughen the ends of her fingers.
A gallant phrase came into his head, but he did not risk it.

"Then you are giving it up?" he went on.

"What?" she asked hurriedly. "Music? Ah! yes! Have I not my house to
look after, my husband to attend to, a thousand things, in fact, many
duties that must be considered first?"

She looked at the clock. Charles was late. Then, she affected anxiety.
Two or three times she even repeated, "He is so good!"

The clerk was fond of Monsieur Bovary. But this tenderness on his behalf
astonished him unpleasantly; nevertheless he took up on his praises,
which he said everyone was singing, especially the chemist.

"Ah! he is a good fellow," continued Emma.

"Certainly," replied the clerk.

And he began talking of Madame Homais, whose very untidy appearance
generally made them laugh.

"What does it matter?" interrupted Emma. "A good housewife does not
trouble about her appearance."

Then she relapsed into silence.

It was the same on the following days; her talks, her manners,
everything changed. She took interest in the housework, went to church
regularly, and looked after her servant with more severity.

She took Berthe from nurse. When visitors called, Felicite brought her
in, and Madame Bovary undressed her to show off her limbs. She declared
she adored children; this was her consolation, her joy, her passion,
and she accompanied her caresses with lyrical outburst which would have
reminded anyone but the Yonville people of Sachette in "Notre Dame de
Paris."

When Charles came home he found his slippers put to warm near the fire.
His waistcoat now never wanted lining, nor his shirt buttons, and it was
quite a pleasure to see in the cupboard the night-caps arranged in piles
of the same height. She no longer grumbled as formerly at taking a turn
in the garden; what he proposed was always done, although she did not
understand the wishes to which she submitted without a murmur; and when
Leon saw him by his fireside after dinner, his two hands on his stomach,
his two feet on the fender, his two cheeks red with feeding, his eyes
moist with happiness, the child crawling along the carpet, and this
woman with the slender waist who came behind his arm-chair to kiss his
forehead: "What madness!" he said to himself. "And how to reach her!"

And thus she seemed so virtuous and inaccessible to him that he lost all
hope, even the faintest. But by this renunciation he placed her on
an extraordinary pinnacle. To him she stood outside those fleshly
attributes from which he had nothing to obtain, and in his heart she
rose ever, and became farther removed from him after the magnificent
manner of an apotheosis that is taking wing. It was one of those pure
feelings that do not interfere with life, that are cultivated because
they are rare, and whose loss would afflict more than their passion
rejoices.

Emma grew thinner, her cheeks paler, her face longer. With her black
hair, her large eyes, her aquiline nose, her birdlike walk, and always
silent now, did she not seem to be passing through life scarcely
touching it, and to bear on her brow the vague impress of some divine
destiny? She was so sad and so calm, at once so gentle and so reserved,
that near her one felt oneself seized by an icy charm, as we shudder
in churches at the perfume of the flowers mingling with the cold of the
marble. The others even did not escape from this seduction. The chemist
said--

"She is a woman of great parts, who wouldn't be misplaced in a
sub-prefecture."

The housewives admired her economy, the patients her politeness, the
poor her charity.

But she was eaten up with desires, with rage, with hate. That dress with
the narrow folds hid a distracted fear, of whose torment those chaste
lips said nothing. She was in love with Leon, and sought solitude that
she might with the more ease delight in his image. The sight of his
form troubled the voluptuousness of this mediation. Emma thrilled at
the sound of his step; then in his presence the emotion subsided, and
afterwards there remained to her only an immense astonishment that ended
in sorrow.

Leon did not know that when he left her in despair she rose after he had
gone to see him in the street. She concerned herself about his comings
and goings; she watched his face; she invented quite a history to find
an excuse for going to his room. The chemist's wife seemed happy to her
to sleep under the same roof, and her thoughts constantly centered upon
this house, like the "Lion d'Or" pigeons, who came there to dip their
red feet and white wings in its gutters. But the more Emma recognised
her love, the more she crushed it down, that it might not be evident,
that she might make it less. She would have liked Leon to guess it, and
she imagined chances, catastrophes that should facilitate this.

What restrained her was, no doubt, idleness and fear, and a sense of
shame also. She thought she had repulsed him too much, that the time was
past, that all was lost. Then, pride, and joy of being able to say to
herself, "I am virtuous," and to look at herself in the glass taking
resigned poses, consoled her a little for the sacrifice she believed she
was making.

Then the lusts of the flesh, the longing for money, and the melancholy
of passion all blended themselves into one suffering, and instead of
turning her thoughts from it, she clave to it the more, urging herself
to pain, and seeking everywhere occasion for it. She was irritated by
an ill-served dish or by a half-open door; bewailed the velvets she had
not, the happiness she had missed, her too exalted dreams, her narrow
home.

What exasperated her was that Charles did not seem to notice her
anguish. His conviction that he was making her happy seemed to her an
imbecile insult, and his sureness on this point ingratitude. For whose
sake, then was she virtuous? Was it not for him, the obstacle to all
felicity, the cause of all misery, and, as it were, the sharp clasp of
that complex strap that bucked her in on all sides.

On him alone, then, she concentrated all the various hatreds that
resulted from her boredom, and every effort to diminish only augmented
it; for this useless trouble was added to the other reasons for despair,
and contributed still more to the separation between them. Her own
gentleness to herself made her rebel against him. Domestic mediocrity
drove her to lewd fancies, marriage tenderness to adulterous desires.
She would have liked Charles to beat her, that she might have a better
right to hate him, to revenge herself upon him. She was surprised
sometimes at the atrocious conjectures that came into her thoughts, and
she had to go on smiling, to hear repeated to her at all hours that she
was happy, to pretend to be happy, to let it be believed.

Yet she had loathing of this hypocrisy. She was seized with the
temptation to flee somewhere with Leon to try a new life; but at once a
vague chasm full of darkness opened within her soul.

"Besides, he no longer loves me," she thought. "What is to become of me?
What help is to be hoped for, what consolation, what solace?"

She was left broken, breathless, inert, sobbing in a low voice, with
flowing tears.

"Why don't you tell master?" the servant asked her when she came in
during these crises.

"It is the nerves," said Emma. "Do not speak to him of it; it would
worry him."

"Ah! yes," Felicite went on, "you are just like La Guerine, Pere
Guerin's daughter, the fisherman at Pollet, that I used to know at
Dieppe before I came to you. She was so sad, so sad, to see her
standing upright on the threshold of her house, she seemed to you like a
winding-sheet spread out before the door. Her illness, it appears, was
a kind of fog that she had in her head, and the doctors could not do
anything, nor the priest either. When she was taken too bad she went
off quite alone to the sea-shore, so that the customs officer, going his
rounds, often found her lying flat on her face, crying on the shingle.
Then, after her marriage, it went off, they say."

"But with me," replied Emma, "it was after marriage that it began."



Chapter Six

One evening when the window was open, and she, sitting by it, had been
watching Lestiboudois, the beadle, trimming the box, she suddenly heard
the Angelus ringing.

It was the beginning of April, when the primroses are in bloom, and a
warm wind blows over the flower-beds newly turned, and the gardens, like
women, seem to be getting ready for the summer fetes. Through the bars
of the arbour and away beyond, the river seen in the fields, meandering
through the grass in wandering curves. The evening vapours rose between
the leafless poplars, touching their outlines with a violet tint, paler
and more transparent than a subtle gauze caught athwart their branches.
In the distance cattle moved about; neither their steps nor their lowing
could be heard; and the bell, still ringing through the air, kept up its
peaceful lamentation.

With this repeated tinkling the thoughts of the young woman lost
themselves in old memories of her youth and school-days. She remembered
the great candlesticks that rose above the vases full of flowers on the
altar, and the tabernacle with its small columns. She would have liked
to be once more lost in the long line of white veils, marked off here
and there by the stuff black hoods of the good sisters bending over
their prie-Dieu. At mass on Sundays, when she looked up, she saw the
gentle face of the Virgin amid the blue smoke of the rising incense.
Then she was moved; she felt herself weak and quite deserted, like the
down of a bird whirled by the tempest, and it was unconsciously that she
went towards the church, included to no matter what devotions, so that
her soul was absorbed and all existence lost in it.

On the Place she met Lestivoudois on his way back, for, in order not
to shorten his day's labour, he preferred interrupting his work,
then beginning it again, so that he rang the Angelus to suit his own
convenience. Besides, the ringing over a little earlier warned the lads
of catechism hour.

Already a few who had arrived were playing marbles on the stones of the
cemetery. Others, astride the wall, swung their legs, kicking with their
clogs the large nettles growing between the little enclosure and the
newest graves. This was the only green spot. All the rest was but
stones, always covered with a fine powder, despite the vestry-broom.

The children in list shoes ran about there as if it were an enclosure
made for them. The shouts of their voices could be heard through the
humming of the bell. This grew less and less with the swinging of the
great rope that, hanging from the top of the belfry, dragged its end on
the ground. Swallows flitted to and fro uttering little cries, cut the
air with the edge of their wings, and swiftly returned to their yellow
nests under the tiles of the coping. At the end of the church a lamp was
burning, the wick of a night-light in a glass hung up. Its light from a
distance looked like a white stain trembling in the oil. A long ray of
the sun fell across the nave and seemed to darken the lower sides and
the corners.

"Where is the cure?" asked Madame Bovary of one of the lads, who was
amusing himself by shaking a swivel in a hole too large for it.

"He is just coming," he answered.

And in fact the door of the presbytery grated; Abbe Bournisien appeared;
the children, pell-mell, fled into the church.

"These young scamps!" murmured the priest, "always the same!"

Then, picking up a catechism all in rags that he had struck with is
foot, "They respect nothing!" But as soon as he caught sight of Madame
Bovary, "Excuse me," he said; "I did not recognise you."

He thrust the catechism into his pocket, and stopped short, balancing
the heavy vestry key between his two fingers.

The light of the setting sun that fell full upon his face paled the
lasting of his cassock, shiny at the elbows, unravelled at the hem.
Grease and tobacco stains followed along his broad chest the lines
of the buttons, and grew more numerous the farther they were from his
neckcloth, in which the massive folds of his red chin rested; this was
dotted with yellow spots, that disappeared beneath the coarse hair of
his greyish beard. He had just dined and was breathing noisily.

"How are you?" he added.

"Not well," replied Emma; "I am ill."

"Well, and so am I," answered the priest. "These first warm days weaken
one most remarkably, don't they? But, after all, we are born to suffer,
as St. Paul says. But what does Monsieur Bovary think of it?"

"He!" she said with a gesture of contempt.

"What!" replied the good fellow, quite astonished, "doesn't he prescribe
something for you?"

"Ah!" said Emma, "it is no earthly remedy I need."

But the cure from time to time looked into the church, where the
kneeling boys were shouldering one another, and tumbling over like packs
of cards.

"I should like to know--" she went on.

"You look out, Riboudet," cried the priest in an angry voice; "I'll warm
your ears, you imp!" Then turning to Emma, "He's Boudet the carpenter's
son; his parents are well off, and let him do just as he pleases. Yet he
could learn quickly if he would, for he is very sharp. And so sometimes
for a joke I call him Riboudet (like the road one takes to go to
Maromme) and I even say 'Mon Riboudet.' Ha! Ha! 'Mont Riboudet.' The
other day I repeated that just to Monsignor, and he laughed at it; he
condescended to laugh at it. And how is Monsieur Bovary?"

She seemed not to hear him. And he went on--

"Always very busy, no doubt; for he and I are certainly the busiest
people in the parish. But he is doctor of the body," he added with a
thick laugh, "and I of the soul."

She fixed her pleading eyes upon the priest. "Yes," she said, "you
solace all sorrows."

"Ah! don't talk to me of it, Madame Bovary. This morning I had to go to
Bas-Diauville for a cow that was ill; they thought it was under a spell.
All their cows, I don't know how it is--But pardon me! Longuemarre and
Boudet! Bless me! Will you leave off?"

And with a bound he ran into the church.

The boys were just then clustering round the large desk, climbing over
the precentor's footstool, opening the missal; and others on tiptoe were
just about to venture into the confessional. But the priest suddenly
distributed a shower of cuffs among them. Seizing them by the collars of
their coats, he lifted them from the ground, and deposited them on their
knees on the stones of the choir, firmly, as if he meant planting them
there.

"Yes," said he, when he returned to Emma, unfolding his large cotton
handkerchief, one corner of which he put between his teeth, "farmers are
much to be pitied."

"Others, too," she replied.

"Assuredly. Town-labourers, for example."

"It is not they--"

"Pardon! I've there known poor mothers of families, virtuous women, I
assure you, real saints, who wanted even bread."

"But those," replied Emma, and the corners of her mouth twitched as she
spoke, "those, Monsieur le Cure, who have bread and have no--"

"Fire in the winter," said the priest.

"Oh, what does that matter?"

"What! What does it matter? It seems to me that when one has firing and
food--for, after all--"

"My God! my God!" she sighed.

"It is indigestion, no doubt? You must get home, Madame Bovary; drink
a little tea, that will strengthen you, or else a glass of fresh water
with a little moist sugar."

"Why?" And she looked like one awaking from a dream.

"Well, you see, you were putting your hand to your forehead. I thought
you felt faint." Then, bethinking himself, "But you were asking me
something? What was it? I really don't remember."

"I? Nothing! nothing!" repeated Emma.

And the glance she cast round her slowly fell upon the old man in the
cassock. They looked at one another face to face without speaking.

"Then, Madame Bovary," he said at last, "excuse me, but duty first, you
know; I must look after my good-for-nothings. The first communion will
soon be upon us, and I fear we shall be behind after all. So after
Ascension Day I keep them recta* an extra hour every Wednesday. Poor
children! One cannot lead them too soon into the path of the Lord, as,
moreover, he has himself recommended us to do by the mouth of his Divine
Son. Good health to you, madame; my respects to your husband."

     *On the straight and narrow path.

And he went into the church making a genuflexion as soon as he reached
the door.

Emma saw him disappear between the double row of forms, walking with a
heavy tread, his head a little bent over his shoulder, and with his two
hands half-open behind him.

Then she turned on her heel all of one piece, like a statue on a pivot,
and went homewards. But the loud voice of the priest, the clear voices
of the boys still reached her ears, and went on behind her.

"Are you a Christian?"

"Yes, I am a Christian."

"What is a Christian?"

"He who, being baptized-baptized-baptized--"

She went up the steps of the staircase holding on to the banisters, and
when she was in her room threw herself into an arm-chair.

The whitish light of the window-panes fell with soft undulations.

The furniture in its place seemed to have become more immobile, and to
lose itself in the shadow as in an ocean of darkness. The fire was out,
the clock went on ticking, and Emma vaguely marvelled at this calm of
all things while within herself was such tumult. But little Berthe was
there, between the window and the work-table, tottering on her knitted
shoes, and trying to come to her mother to catch hold of the ends of her
apron-strings.

"Leave me alone," said the latter, putting her from her with her hand.

The little girl soon came up closer against her knees, and leaning on
them with her arms, she looked up with her large blue eyes, while a
small thread of pure saliva dribbled from her lips on to the silk apron.

"Leave me alone," repeated the young woman quite irritably.

Her face frightened the child, who began to scream.

"Will you leave me alone?" she said, pushing her with her elbow.

Berthe fell at the foot of the drawers against the brass handle, cutting
her cheek, which began to bleed, against it. Madame Bovary sprang to
lift her up, broke the bell-rope, called for the servant with all her
might, and she was just going to curse herself when Charles appeared. It
was the dinner-hour; he had come home.

"Look, dear!" said Emma, in a calm voice, "the little one fell down
while she was playing, and has hurt herself."

Charles reassured her; the case was not a serious one, and he went for
some sticking plaster.

Madame Bovary did not go downstairs to the dining-room; she wished
to remain alone to look after the child. Then watching her sleep, the
little anxiety she felt gradually wore off, and she seemed very stupid
to herself, and very good to have been so worried just now at so little.
Berthe, in fact, no longer sobbed.

Her breathing now imperceptibly raised the cotton covering. Big tears
lay in the corner of the half-closed eyelids, through whose lashes one
could see two pale sunken pupils; the plaster stuck on her cheek drew
the skin obliquely.

"It is very strange," thought Emma, "how ugly this child is!"

When at eleven o'clock Charles came back from the chemist's shop,
whither he had gone after dinner to return the remainder of the
sticking-plaster, he found his wife standing by the cradle.

"I assure you it's nothing." he said, kissing her on the forehead.
"Don't worry, my poor darling; you will make yourself ill."

He had stayed a long time at the chemist's. Although he had not seemed
much moved, Homais, nevertheless, had exerted himself to buoy him up, to
"keep up his spirits." Then they had talked of the various dangers that
threaten childhood, of the carelessness of servants. Madame Homais knew
something of it, having still upon her chest the marks left by a basin
full of soup that a cook had formerly dropped on her pinafore, and
her good parents took no end of trouble for her. The knives were not
sharpened, nor the floors waxed; there were iron gratings to the windows
and strong bars across the fireplace; the little Homais, in spite of
their spirit, could not stir without someone watching them; at the
slightest cold their father stuffed them with pectorals; and until
they were turned four they all, without pity, had to wear wadded
head-protectors. This, it is true, was a fancy of Madame Homais'; her
husband was inwardly afflicted at it. Fearing the possible consequences
of such compression to the intellectual organs. He even went so far as
to say to her, "Do you want to make Caribs or Botocudos of them?"

Charles, however, had several times tried to interrupt the conversation.
"I should like to speak to you," he had whispered in the clerk's ear,
who went upstairs in front of him.

"Can he suspect anything?" Leon asked himself. His heart beat, and he
racked his brain with surmises.

At last, Charles, having shut the door, asked him to see himself
what would be the price at Rouen of a fine daguerreotypes. It was a
sentimental surprise he intended for his wife, a delicate attention--his
portrait in a frock-coat. But he wanted first to know "how much it would
be." The inquiries would not put Monsieur Leon out, since he went to
town almost every week.

Why? Monsieur Homais suspected some "young man's affair" at the bottom
of it, an intrigue. But he was mistaken. Leon was after no love-making.
He was sadder than ever, as Madame Lefrancois saw from the amount of
food he left on his plate. To find out more about it she questioned
the tax-collector. Binet answered roughly that he "wasn't paid by the
police."

All the same, his companion seemed very strange to him, for Leon often
threw himself back in his chair, and stretching out his arms. Complained
vaguely of life.

"It's because you don't take enough recreation," said the collector.

"What recreation?"

"If I were you I'd have a lathe."

"But I don't know how to turn," answered the clerk.

"Ah! that's true," said the other, rubbing his chin with an air of
mingled contempt and satisfaction.

Leon was weary of loving without any result; moreover he was beginning
to feel that depression caused by the repetition of the same kind of
life, when no interest inspires and no hope sustains it. He was so bored
with Yonville and its inhabitants, that the sight of certain persons,
of certain houses, irritated him beyond endurance; and the chemist, good
fellow though he was, was becoming absolutely unbearable to him. Yet
the prospect of a new condition of life frightened as much as it seduced
him.

This apprehension soon changed into impatience, and then Paris from afar
sounded its fanfare of masked balls with the laugh of grisettes. As he
was to finish reading there, why not set out at once? What prevented
him? And he began making home-preparations; he arranged his occupations
beforehand. He furnished in his head an apartment. He would lead an
artist's life there! He would take lessons on the guitar! He would have
a dressing-gown, a Basque cap, blue velvet slippers! He even already was
admiring two crossed foils over his chimney-piece, with a death's head
on the guitar above them.

The difficulty was the consent of his mother; nothing, however, seemed
more reasonable. Even his employer advised him to go to some other
chambers where he could advance more rapidly. Taking a middle course,
then, Leon looked for some place as second clerk at Rouen; found none,
and at last wrote his mother a long letter full of details, in which
he set forth the reasons for going to live at Paris immediately. She
consented.

He did not hurry. Every day for a month Hivert carried boxes, valises,
parcels for him from Yonville to Rouen and from Rouen to Yonville;
and when Leon had packed up his wardrobe, had his three arm-chairs
restuffed, bought a stock of neckties, in a word, had made more
preparations than for a voyage around the world, he put it off from week
to week, until he received a second letter from his mother urging him to
leave, since he wanted to pass his examination before the vacation.

When the moment for the farewells had come, Madame Homais wept, Justin
sobbed; Homais, as a man of nerve, concealed his emotion; he wished to
carry his friend's overcoat himself as far as the gate of the notary,
who was taking Leon to Rouen in his carriage.

The latter had just time to bid farewell to Monsieur Bovary.

When he reached the head of the stairs, he stopped, he was so out of
breath. As he came in, Madame Bovary arose hurriedly.

"It is I again!" said Leon.

"I was sure of it!"

She bit her lips, and a rush of blood flowing under her skin made her
red from the roots of her hair to the top of her collar. She remained
standing, leaning with her shoulder against the wainscot.

"The doctor is not here?" he went on.

"He is out." She repeated, "He is out."

Then there was silence. They looked at one another and their thoughts,
confounded in the same agony, clung close together like two throbbing
breasts.

"I should like to kiss Berthe," said Leon.

Emma went down a few steps and called Felicite.

He threw one long look around him that took in the walls, the
decorations, the fireplace, as if to penetrate everything, carry away
everything. But she returned, and the servant brought Berthe, who was
swinging a windmill roof downwards at the end of a string. Leon kissed
her several times on the neck.

"Good-bye, poor child! good-bye, dear little one! good-bye!" And he gave
her back to her mother.

"Take her away," she said.

They remained alone--Madame Bovary, her back turned, her face pressed
against a window-pane; Leon held his cap in his hand, knocking it softly
against his thigh.

"It is going to rain," said Emma.

"I have a cloak," he answered.

"Ah!"

She turned around, her chin lowered, her forehead bent forward.

The light fell on it as on a piece of marble, to the curve of the
eyebrows, without one's being able to guess what Emma was seeing on the
horizon or what she was thinking within herself.

"Well, good-bye," he sighed.

She raised her head with a quick movement.

"Yes, good-bye--go!"

They advanced towards each other; he held out his hand; she hesitated.

"In the English fashion, then," she said, giving her own hand wholly to
him, and forcing a laugh.

Leon felt it between his fingers, and the very essence of all his being
seemed to pass down into that moist palm. Then he opened his hand; their
eyes met again, and he disappeared.

When he reached the market-place, he stopped and hid behind a pillar to
look for the last time at this white house with the four green blinds.
He thought he saw a shadow behind the window in the room; but the
curtain, sliding along the pole as though no one were touching it,
slowly opened its long oblique folds that spread out with a single
movement, and thus hung straight and motionless as a plaster wall. Leon
set off running.

From afar he saw his employer's gig in the road, and by it a man in
a coarse apron holding the horse. Homais and Monsieur Guillaumin were
talking. They were waiting for him.

"Embrace me," said the druggist with tears in his eyes. "Here is your
coat, my good friend. Mind the cold; take care of yourself; look after
yourself."

"Come, Leon, jump in," said the notary.

Homais bend over the splash-board, and in a voice broken by sobs uttered
these three sad words--

"A pleasant journey!"

"Good-night," said Monsieur Guillaumin. "Give him his head." They set
out, and Homais went back.

Madame Bovary had opened her window overlooking the garden and watched
the clouds. They gathered around the sunset on the side of Rouen and
then swiftly rolled back their black columns, behind which the great
rays of the sun looked out like the golden arrows of a suspended trophy,
while the rest of the empty heavens was white as porcelain. But a gust
of wind bowed the poplars, and suddenly the rain fell; it pattered
against the green leaves.

Then the sun reappeared, the hens clucked, sparrows shook their wings in
the damp thickets, and the pools of water on the gravel as they flowed
away carried off the pink flowers of an acacia.

"Ah! how far off he must be already!" she thought.

Monsieur Homais, as usual, came at half-past six during dinner.

"Well," said he, "so we've sent off our young friend!"

"So it seems," replied the doctor. Then turning on his chair; "Any news
at home?"

"Nothing much. Only my wife was a little moved this afternoon. You know
women--a nothing upsets them, especially my wife. And we should be
wrong to object to that, since their nervous organization is much more
malleable than ours."

"Poor Leon!" said Charles. "How will he live at Paris? Will he get used
to it?"

Madame Bovary sighed.

"Get along!" said the chemist, smacking his lips. "The outings at
restaurants, the masked balls, the champagne--all that'll be jolly
enough, I assure you."

"I don't think he'll go wrong," objected Bovary.

"Nor do I," said Monsieur Homais quickly; "although he'll have to do
like the rest for fear of passing for a Jesuit. And you don't know what
a life those dogs lead in the Latin quarter with actresses. Besides,
students are thought a great deal of in Paris. Provided they have a few
accomplishments, they are received in the best society; there are even
ladies of the Faubourg Saint-Germain who fall in love with them, which
subsequently furnishes them opportunities for making very good matches."

"But," said the doctor, "I fear for him that down there--"

"You are right," interrupted the chemist; "that is the reverse of the
medal. And one is constantly obliged to keep one's hand in one's pocket
there. Thus, we will suppose you are in a public garden. An individual
presents himself, well dressed, even wearing an order, and whom one
would take for a diplomatist. He approaches you, he insinuates himself;
offers you a pinch of snuff, or picks up your hat. Then you become more
intimate; he takes you to a cafe, invites you to his country-house,
introduces you, between two drinks, to all sorts of people; and
three-fourths of the time it's only to plunder your watch or lead you
into some pernicious step.

"That is true," said Charles; "but I was thinking especially of
illnesses--of typhoid fever, for example, that attacks students from the
provinces."

Emma shuddered.

"Because of the change of regimen," continued the chemist, "and of the
perturbation that results therefrom in the whole system. And then the
water at Paris, don't you know! The dishes at restaurants, all the
spiced food, end by heating the blood, and are not worth, whatever
people may say of them, a good soup. For my own part, I have always
preferred plain living; it is more healthy. So when I was studying
pharmacy at Rouen, I boarded in a boarding house; I dined with the
professors."

And thus he went on, expounding his opinions generally and his personal
likings, until Justin came to fetch him for a mulled egg that was
wanted.

"Not a moment's peace!" he cried; "always at it! I can't go out for a
minute! Like a plough-horse, I have always to be moiling and toiling.
What drudgery!" Then, when he was at the door, "By the way, do you know
the news?"

"What news?"

"That it is very likely," Homais went on, raising his eyebrows and
assuming one of his most serious expression, "that the agricultural
meeting of the Seine-Inferieure will be held this year at
Yonville-l'Abbaye. The rumour, at all events, is going the round. This
morning the paper alluded to it. It would be of the utmost importance
for our district. But we'll talk it over later on. I can see, thank you;
Justin has the lantern."



Chapter Seven

The next day was a dreary one for Emma. Everything seemed to her
enveloped in a black atmosphere floating confusedly over the exterior of
things, and sorrow was engulfed within her soul with soft shrieks such
as the winter wind makes in ruined castles. It was that reverie which we
give to things that will not return, the lassitude that seizes you after
everything was done; that pain, in fine, that the interruption of every
wonted movement, the sudden cessation of any prolonged vibration, brings
on.

As on the return from Vaubyessard, when the quadrilles were running in
her head, she was full of a gloomy melancholy, of a numb despair.
Leon reappeared, taller, handsomer, more charming, more vague. Though
separated from her, he had not left her; he was there, and the walls of
the house seemed to hold his shadow.

She could not detach her eyes from the carpet where he had walked, from
those empty chairs where he had sat. The river still flowed on, and
slowly drove its ripples along the slippery banks.

They had often walked there to the murmur of the waves over the
moss-covered pebbles. How bright the sun had been! What happy afternoons
they had seen alone in the shade at the end of the garden! He read
aloud, bareheaded, sitting on a footstool of dry sticks; the fresh wind
of the meadow set trembling the leaves of the book and the nasturtiums
of the arbour. Ah! he was gone, the only charm of her life, the only
possible hope of joy. Why had she not seized this happiness when it came
to her? Why not have kept hold of it with both hands, with both knees,
when it was about to flee from her? And she cursed herself for not
having loved Leon. She thirsted for his lips. The wish took possession
of her to run after and rejoin him, throw herself into his arms and
say to him, "It is I; I am yours." But Emma recoiled beforehand at the
difficulties of the enterprise, and her desires, increased by regret,
became only the more acute.

Henceforth the memory of Leon was the centre of her boredom; it burnt
there more brightly than the fire travellers have left on the snow of
a Russian steppe. She sprang towards him, she pressed against him, she
stirred carefully the dying embers, sought all around her anything
that could revive it; and the most distant reminiscences, like the most
immediate occasions, what she experienced as well as what she imagined,
her voluptuous desires that were unsatisfied, her projects of happiness
that crackled in the wind like dead boughs, her sterile virtue, her
lost hopes, the domestic tete-a-tete--she gathered it all up, took
everything, and made it all serve as fuel for her melancholy.

The flames, however, subsided, either because the supply had exhausted
itself, or because it had been piled up too much. Love, little by
little, was quelled by absence; regret stifled beneath habit; and this
incendiary light that had empurpled her pale sky was overspread and
faded by degrees. In the supineness of her conscience she even took her
repugnance towards her husband for aspirations towards her lover, the
burning of hate for the warmth of tenderness; but as the tempest still
raged, and as passion burnt itself down to the very cinders, and no help
came, no sun rose, there was night on all sides, and she was lost in the
terrible cold that pierced her.

Then the evil days of Tostes began again. She thought herself now far
more unhappy; for she had the experience of grief, with the certainty
that it would not end.

A woman who had laid on herself such sacrifices could well allow herself
certain whims. She bought a Gothic prie-dieu, and in a month spent
fourteen francs on lemons for polishing her nails; she wrote to Rouen
for a blue cashmere gown; she chose one of Lheureux's finest scarves,
and wore it knotted around her waist over her dressing-gown; and, with
closed blinds and a book in her hand, she lay stretched out on a couch
in this garb.

She often changed her coiffure; she did her hair a la Chinoise, in
flowing curls, in plaited coils; she parted in on one side and rolled it
under like a man's.

She wanted to learn Italian; she bought dictionaries, a grammar, and
a supply of white paper. She tried serious reading, history, and
philosophy. Sometimes in the night Charles woke up with a start,
thinking he was being called to a patient. "I'm coming," he stammered;
and it was the noise of a match Emma had struck to relight the lamp. But
her reading fared like her piece of embroidery, all of which, only just
begun, filled her cupboard; she took it up, left it, passed on to other
books.

She had attacks in which she could easily have been driven to commit any
folly. She maintained one day, in opposition to her husband, that she
could drink off a large glass of brandy, and, as Charles was stupid
enough to dare her to, she swallowed the brandy to the last drop.

In spite of her vapourish airs (as the housewives of Yonville called
them), Emma, all the same, never seemed gay, and usually she had at the
corners of her mouth that immobile contraction that puckers the faces of
old maids, and those of men whose ambition has failed. She was pale all
over, white as a sheet; the skin of her nose was drawn at the nostrils,
her eyes looked at you vaguely. After discovering three grey hairs on
her temples, she talked much of her old age.

She often fainted. One day she even spat blood, and, as Charles fussed
around her showing his anxiety--

"Bah!" she answered, "what does it matter?"

Charles fled to his study and wept there, both his elbows on the table,
sitting in an arm-chair at his bureau under the phrenological head.

Then he wrote to his mother begging her to come, and they had many long
consultations together on the subject of Emma.

What should they decide? What was to be done since she rejected all
medical treatment? "Do you know what your wife wants?" replied Madame
Bovary senior.

"She wants to be forced to occupy herself with some manual work. If she
were obliged, like so many others, to earn her living, she wouldn't have
these vapours, that come to her from a lot of ideas she stuffs into her
head, and from the idleness in which she lives."

"Yet she is always busy," said Charles.

"Ah! always busy at what? Reading novels, bad books, works against
religion, and in which they mock at priests in speeches taken from
Voltaire. But all that leads you far astray, my poor child. Anyone who
has no religion always ends by turning out badly."

So it was decided to stop Emma reading novels. The enterprise did not
seem easy. The good lady undertook it. She was, when she passed through
Rouen, to go herself to the lending-library and represent that Emma had
discontinued her subscription. Would they not have a right to apply
to the police if the librarian persisted all the same in his poisonous
trade? The farewells of mother and daughter-in-law were cold. During
the three weeks that they had been together they had not exchanged
half-a-dozen words apart from the inquiries and phrases when they met at
table and in the evening before going to bed.

Madame Bovary left on a Wednesday, the market-day at Yonville.

The Place since morning had been blocked by a row of carts, which, on
end and their shafts in the air, spread all along the line of houses
from the church to the inn. On the other side there were canvas booths,
where cotton checks, blankets, and woollen stockings were sold,
together with harness for horses, and packets of blue ribbon, whose ends
fluttered in the wind. The coarse hardware was spread out on the ground
between pyramids of eggs and hampers of cheeses, from which sticky straw
stuck out.

Near the corn-machines clucking hens passed their necks through the bars
of flat cages. The people, crowding in the same place and unwilling
to move thence, sometimes threatened to smash the shop front of the
chemist. On Wednesdays his shop was never empty, and the people pushed
in less to buy drugs than for consultations. So great was Homais'
reputation in the neighbouring villages. His robust aplomb had
fascinated the rustics. They considered him a greater doctor than all
the doctors.

Emma was leaning out at the window; she was often there. The window in
the provinces replaces the theatre and the promenade, she was amusing
herself with watching the crowd of boors when she saw a gentleman in
a green velvet coat. He had on yellow gloves, although he wore heavy
gaiters; he was coming towards the doctor's house, followed by a peasant
walking with a bent head and quite a thoughtful air.

"Can I see the doctor?" he asked Justin, who was talking on the
doorsteps with Felicite, and, taking him for a servant of the
house--"Tell him that Monsieur Rodolphe Boulanger of La Huchette is
here."

It was not from territorial vanity that the new arrival added "of La
Huchette" to his name, but to make himself the better known.

La Huchette, in fact, was an estate near Yonville, where he had just
bought the chateau and two farms that he cultivated himself, without,
however, troubling very much about them. He lived as a bachelor, and was
supposed to have "at least fifteen thousand francs a year."

Charles came into the room. Monsieur Boulanger introduced his man, who
wanted to be bled because he felt "a tingling all over."

"That'll purge me," he urged as an objection to all reasoning.

So Bovary ordered a bandage and a basin, and asked Justin to hold it.
Then addressing the peasant, who was already pale--

"Don't be afraid, my lad."

"No, no, sir," said the other; "get on."

And with an air of bravado he held out his great arm. At the prick of
the lancet the blood spurted out, splashing against the looking-glass.

"Hold the basin nearer," exclaimed Charles.

"Lor!" said the peasant, "one would swear it was a little fountain
flowing. How red my blood is! That's a good sign, isn't it?"

"Sometimes," answered the doctor, "one feels nothing at first, and then
syncope sets in, and more especially with people of strong constitution
like this man."

At these words the rustic let go the lancet-case he was twisting between
his fingers. A shudder of his shoulders made the chair-back creak. His
hat fell off.

"I thought as much," said Bovary, pressing his finger on the vein.

The basin was beginning to tremble in Justin's hands; his knees shook,
he turned pale.

"Emma! Emma!" called Charles.

With one bound she came down the staircase.

"Some vinegar," he cried. "O dear! two at once!"

And in his emotion he could hardly put on the compress.

"It is nothing," said Monsieur Boulanger quietly, taking Justin in his
arms. He seated him on the table with his back resting against the wall.

Madame Bovary began taking off his cravat. The strings of his shirt had
got into a knot, and she was for some minutes moving her light fingers
about the young fellow's neck. Then she poured some vinegar on her
cambric handkerchief; she moistened his temples with little dabs, and
then blew upon them softly. The ploughman revived, but Justin's syncope
still lasted, and his eyeballs disappeared in the pale sclerotics like
blue flowers in milk.

"We must hide this from him," said Charles.

Madame Bovary took the basin to put it under the table. With the
movement she made in bending down, her dress (it was a summer dress with
four flounces, yellow, long in the waist and wide in the skirt) spread
out around her on the flags of the room; and as Emma stooping, staggered
a little as she stretched out her arms.

The stuff here and there gave with the inflections of her bust.

Then she went to fetch a bottle of water, and she was melting some
pieces of sugar when the chemist arrived. The servant had been to
fetch him in the tumult. Seeing his pupil's eyes staring he drew a long
breath; then going around him he looked at him from head to foot.

"Fool!" he said, "really a little fool! A fool in four letters! A
phlebotomy's a big affair, isn't it! And a fellow who isn't afraid of
anything; a kind of squirrel, just as he is who climbs to vertiginous
heights to shake down nuts. Oh, yes! you just talk to me, boast about
yourself! Here's a fine fitness for practising pharmacy later on; for
under serious circumstances you may be called before the tribunals in
order to enlighten the minds of the magistrates, and you would have to
keep your head then, to reason, show yourself a man, or else pass for an
imbecile."

Justin did not answer. The chemist went on--

"Who asked you to come? You are always pestering the doctor and madame.
On Wednesday, moreover, your presence is indispensable to me. There are
now twenty people in the shop. I left everything because of the interest
I take in you. Come, get along! Sharp! Wait for me, and keep an eye on
the jars."

When Justin, who was rearranging his dress, had gone, they talked for a
little while about fainting-fits. Madame Bovary had never fainted.

"That is extraordinary for a lady," said Monsieur Boulanger; "but some
people are very susceptible. Thus in a duel, I have seen a second lose
consciousness at the mere sound of the loading of pistols."

"For my part," said the chemist, "the sight of other people's blood
doesn't affect me at all, but the mere thought of my own flowing would
make me faint if I reflected upon it too much."

Monsieur Boulanger, however, dismissed his servant, advising him to calm
himself, since his fancy was over.

"It procured me the advantage of making your acquaintance," he added,
and he looked at Emma as he said this. Then he put three francs on the
corner of the table, bowed negligently, and went out.

He was soon on the other side of the river (this was his way back to La
Huchette), and Emma saw him in the meadow, walking under the poplars,
slackening his pace now and then as one who reflects.

"She is very pretty," he said to himself; "she is very pretty, this
doctor's wife. Fine teeth, black eyes, a dainty foot, a figure like a
Parisienne's. Where the devil does she come from? Wherever did that fat
fellow pick her up?"

Monsieur Rodolphe Boulanger was thirty-four; he was of brutal
temperament and intelligent perspicacity, having, moreover, had much to
do with women, and knowing them well. This one had seemed pretty to him;
so he was thinking about her and her husband.

"I think he is very stupid. She is tired of him, no doubt. He has dirty
nails, and hasn't shaved for three days. While he is trotting after his
patients, she sits there botching socks. And she gets bored! She would
like to live in town and dance polkas every evening. Poor little woman!
She is gaping after love like a carp after water on a kitchen-table.
With three words of gallantry she'd adore one, I'm sure of it. She'd be
tender, charming. Yes; but how to get rid of her afterwards?"

Then the difficulties of love-making seen in the distance made him by
contrast think of his mistress. She was an actress at Rouen, whom he
kept; and when he had pondered over this image, with which, even in
remembrance, he was satiated--

"Ah! Madame Bovary," he thought, "is much prettier, especially fresher.
Virginie is decidedly beginning to grow fat. She is so finiky about her
pleasures; and, besides, she has a mania for prawns."

The fields were empty, and around him Rodolphe only heard the regular
beating of the grass striking against his boots, with a cry of the
grasshopper hidden at a distance among the oats. He again saw Emma in
her room, dressed as he had seen her, and he undressed her.

"Oh, I will have her," he cried, striking a blow with his stick at a
clod in front of him. And he at once began to consider the political
part of the enterprise. He asked himself--

"Where shall we meet? By what means? We shall always be having the brat
on our hands, and the servant, the neighbours, and husband, all sorts of
worries. Pshaw! one would lose too much time over it."

Then he resumed, "She really has eyes that pierce one's heart like a
gimlet. And that pale complexion! I adore pale women!"

When he reached the top of the Arguiel hills he had made up his mind.
"It's only finding the opportunities. Well, I will call in now and then.
I'll send them venison, poultry; I'll have myself bled, if need be. We
shall become friends; I'll invite them to my place. By Jove!" added he,
"there's the agricultural show coming on. She'll be there. I shall see
her. We'll begin boldly, for that's the surest way."



Chapter Eight

At last it came, the famous agricultural show. On the morning of the
solemnity all the inhabitants at their doors were chatting over the
preparations. The pediment of the town hall had been hung with garlands
of ivy; a tent had been erected in a meadow for the banquet; and in the
middle of the Place, in front of the church, a kind of bombarde was
to announce the arrival of the prefect and the names of the successful
farmers who had obtained prizes. The National Guard of Buchy (there was
none at Yonville) had come to join the corps of firemen, of whom Binet
was captain. On that day he wore a collar even higher than usual; and,
tightly buttoned in his tunic, his figure was so stiff and motionless
that the whole vital portion of his person seemed to have descended into
his legs, which rose in a cadence of set steps with a single movement.
As there was some rivalry between the tax-collector and the colonel,
both, to show off their talents, drilled their men separately. One
saw the red epaulettes and the black breastplates pass and re-pass
alternately; there was no end to it, and it constantly began again.
There had never been such a display of pomp. Several citizens had
scoured their houses the evening before; tri-coloured flags hung from
half-open windows; all the public-houses were full; and in the lovely
weather the starched caps, the golden crosses, and the coloured
neckerchiefs seemed whiter than snow, shone in the sun, and relieved
with the motley colours the sombre monotony of the frock-coats and blue
smocks. The neighbouring farmers' wives, when they got off their horses,
pulled out the long pins that fastened around them their dresses, turned
up for fear of mud; and the husbands, for their part, in order to save
their hats, kept their handkerchiefs around them, holding one corner
between their teeth.

The crowd came into the main street from both ends of the village.
People poured in from the lanes, the alleys, the houses; and from time
to time one heard knockers banging against doors closing behind women
with their gloves, who were going out to see the fete. What was most
admired were two long lamp-stands covered with lanterns, that flanked a
platform on which the authorities were to sit. Besides this there were
against the four columns of the town hall four kinds of poles,
each bearing a small standard of greenish cloth, embellished with
inscriptions in gold letters.

On one was written, "To Commerce"; on the other, "To Agriculture"; on
the third, "To Industry"; and on the fourth, "To the Fine Arts."

But the jubilation that brightened all faces seemed to darken that of
Madame Lefrancois, the innkeeper. Standing on her kitchen-steps she
muttered to herself, "What rubbish! what rubbish! With their canvas
booth! Do they think the prefect will be glad to dine down there under
a tent like a gipsy? They call all this fussing doing good to the place!
Then it wasn't worth while sending to Neufchatel for the keeper of a
cookshop! And for whom? For cowherds! tatterdemalions!"

The druggist was passing. He had on a frock-coat, nankeen trousers,
beaver shoes, and, for a wonder, a hat with a low crown.

"Your servant! Excuse me, I am in a hurry." And as the fat widow asked
where he was going--

"It seems odd to you, doesn't it, I who am always more cooped up in my
laboratory than the man's rat in his cheese."

"What cheese?" asked the landlady.

"Oh, nothing! nothing!" Homais continued. "I merely wished to convey
to you, Madame Lefrancois, that I usually live at home like a recluse.
To-day, however, considering the circumstances, it is necessary--"

"Oh, you're going down there!" she said contemptuously.

"Yes, I am going," replied the druggist, astonished. "Am I not a member
of the consulting commission?"

Mere Lefrancois looked at him for a few moments, and ended by saying
with a smile--

"That's another pair of shoes! But what does agriculture matter to you?
Do you understand anything about it?"

"Certainly I understand it, since I am a druggist--that is to say,
a chemist. And the object of chemistry, Madame Lefrancois, being the
knowledge of the reciprocal and molecular action of all natural bodies,
it follows that agriculture is comprised within its domain. And, in
fact, the composition of the manure, the fermentation of liquids, the
analyses of gases, and the influence of miasmata, what, I ask you, is
all this, if it isn't chemistry, pure and simple?"

The landlady did not answer. Homais went on--

"Do you think that to be an agriculturist it is necessary to have tilled
the earth or fattened fowls oneself? It is necessary rather to know the
composition of the substances in question--the geological strata, the
atmospheric actions, the quality of the soil, the minerals, the waters,
the density of the different bodies, their capillarity, and what not.
And one must be master of all the principles of hygiene in order to
direct, criticize the construction of buildings, the feeding of animals,
the diet of domestics. And, moreover, Madame Lefrancois, one must know
botany, be able to distinguish between plants, you understand, which are
the wholesome and those that are deleterious, which are unproductive
and which nutritive, if it is well to pull them up here and re-sow them
there, to propagate some, destroy others; in brief, one must keep pace
with science by means of pamphlets and public papers, be always on the
alert to find out improvements."

The landlady never took her eyes off the "Cafe Francois" and the chemist
went on--

"Would to God our agriculturists were chemists, or that at least they
would pay more attention to the counsels of science. Thus lately I
myself wrote a considerable tract, a memoir of over seventy-two pages,
entitled, 'Cider, its Manufacture and its Effects, together with some
New Reflections on the Subject,' that I sent to the Agricultural Society
of Rouen, and which even procured me the honour of being received among
its members--Section, Agriculture; Class, Pomological. Well, if my
work had been given to the public--" But the druggist stopped, Madame
Lefrancois seemed so preoccupied.

"Just look at them!" she said. "It's past comprehension! Such a cookshop
as that!" And with a shrug of the shoulders that stretched out over her
breast the stitches of her knitted bodice, she pointed with both hands
at her rival's inn, whence songs were heard issuing. "Well, it won't
last long," she added. "It'll be over before a week."

Homais drew back with stupefaction. She came down three steps and
whispered in his ear--

"What! you didn't know it? There is to be an execution in next week.
It's Lheureux who is selling him out; he has killed him with bills."

"What a terrible catastrophe!" cried the druggist, who always found
expressions in harmony with all imaginable circumstances.

Then the landlady began telling him the story that she had heard from
Theodore, Monsieur Guillaumin's servant, and although she detested
Tellier, she blamed Lheureux. He was "a wheedler, a sneak."

"There!" she said. "Look at him! he is in the market; he is bowing to
Madame Bovary, who's got on a green bonnet. Why, she's taking Monsieur
Boulanger's arm."

"Madame Bovary!" exclaimed Homais. "I must go at once and pay her my
respects. Perhaps she'll be very glad to have a seat in the enclosure
under the peristyle." And, without heeding Madame Lefrancois, who was
calling him back to tell him more about it, the druggist walked off
rapidly with a smile on his lips, with straight knees, bowing copiously
to right and left, and taking up much room with the large tails of his
frock-coat that fluttered behind him in the wind.

Rodolphe, having caught sight of him from afar, hurried on, but Madame
Bovary lost her breath; so he walked more slowly, and, smiling at her,
said in a rough tone--

"It's only to get away from that fat fellow, you know, the druggist."
She pressed his elbow.

"What's the meaning of that?" he asked himself. And he looked at her out
of the corner of his eyes.

Her profile was so calm that one could guess nothing from it. It stood
out in the light from the oval of her bonnet, with pale ribbons on it
like the leaves of weeds. Her eyes with their long curved lashes looked
straight before her, and though wide open, they seemed slightly puckered
by the cheek-bones, because of the blood pulsing gently under the
delicate skin. A pink line ran along the partition between her nostrils.
Her head was bent upon her shoulder, and the pearl tips of her white
teeth were seen between her lips.

"Is she making fun of me?" thought Rodolphe.

Emma's gesture, however, had only been meant for a warning; for Monsieur
Lheureux was accompanying them, and spoke now and again as if to enter
into the conversation.

"What a superb day! Everybody is out! The wind is east!"

And neither Madame Bovary nor Rodolphe answered him, whilst at the
slightest movement made by them he drew near, saying, "I beg your
pardon!" and raised his hat.

When they reached the farrier's house, instead of following the road
up to the fence, Rodolphe suddenly turned down a path, drawing with him
Madame Bovary. He called out--

"Good evening, Monsieur Lheureux! See you again presently."

"How you got rid of him!" she said, laughing.

"Why," he went on, "allow oneself to be intruded upon by others? And as
to-day I have the happiness of being with you--"

Emma blushed. He did not finish his sentence. Then he talked of the fine
weather and of the pleasure of walking on the grass. A few daisies had
sprung up again.

"Here are some pretty Easter daisies," he said, "and enough of them to
furnish oracles to all the amorous maids in the place."

He added, "Shall I pick some? What do you think?"

"Are you in love?" she asked, coughing a little.

"H'm, h'm! who knows?" answered Rodolphe.

The meadow began to fill, and the housewives hustled you with their
great umbrellas, their baskets, and their babies. One had often to get
out of the way of a long file of country folk, servant-maids with blue
stockings, flat shoes, silver rings, and who smelt of milk, when one
passed close to them. They walked along holding one another by the hand,
and thus they spread over the whole field from the row of open trees to
the banquet tent.

But this was the examination time, and the farmers one after the other
entered a kind of enclosure formed by a long cord supported on sticks.

The beasts were there, their noses towards the cord, and making a
confused line with their unequal rumps. Drowsy pigs were burrowing in
the earth with their snouts, calves were bleating, lambs baaing; the
cows, on knees folded in, were stretching their bellies on the grass,
slowly chewing the cud, and blinking their heavy eyelids at the gnats
that buzzed round them. Plough-men with bare arms were holding by the
halter prancing stallions that neighed with dilated nostrils looking
towards the mares. These stood quietly, stretching out their heads and
flowing manes, while their foals rested in their shadow, or now and then
came and sucked them. And above the long undulation of these crowded
animals one saw some white mane rising in the wind like a wave, or some
sharp horns sticking out, and the heads of men running about. Apart,
outside the enclosure, a hundred paces off, was a large black bull,
muzzled, with an iron ring in its nostrils, and who moved no more than
if he had been in bronze. A child in rags was holding him by a rope.

Between the two lines the committee-men were walking with heavy steps,
examining each animal, then consulting one another in a low voice. One
who seemed of more importance now and then took notes in a book as he
walked along. This was the president of the jury, Monsieur Derozerays de
la Panville. As soon as he recognised Rodolphe he came forward quickly,
and smiling amiably, said--

"What! Monsieur Boulanger, you are deserting us?"

Rodolphe protested that he was just coming. But when the president had
disappeared--

"Ma foi!*" said he, "I shall not go. Your company is better than his."

     *Upon my word!

And while poking fun at the show, Rodolphe, to move about more easily,
showed the gendarme his blue card, and even stopped now and then in
front of some fine beast, which Madame Bovary did not at all admire.
He noticed this, and began jeering at the Yonville ladies and their
dresses; then he apologised for the negligence of his own. He had that
incongruity of common and elegant in which the habitually vulgar think
they see the revelation of an eccentric existence, of the perturbations
of sentiment, the tyrannies of art, and always a certain contempt for
social conventions, that seduces or exasperates them. Thus his cambric
shirt with plaited cuffs was blown out by the wind in the opening of his
waistcoat of grey ticking, and his broad-striped trousers disclosed at
the ankle nankeen boots with patent leather gaiters.

These were so polished that they reflected the grass. He trampled on
horses's dung with them, one hand in the pocket of his jacket and his
straw hat on one side.

"Besides," added he, "when one lives in the country--"

"It's waste of time," said Emma.

"That is true," replied Rodolphe. "To think that not one of these people
is capable of understanding even the cut of a coat!"

Then they talked about provincial mediocrity, of the lives it crushed,
the illusions lost there.

"And I too," said Rodolphe, "am drifting into depression."

"You!" she said in astonishment; "I thought you very light-hearted."

"Ah! yes. I seem so, because in the midst of the world I know how to
wear the mask of a scoffer upon my face; and yet, how many a time at the
sight of a cemetery by moonlight have I not asked myself whether it were
not better to join those sleeping there!"

"Oh! and your friends?" she said. "You do not think of them."

"My friends! What friends? Have I any? Who cares for me?" And he
accompanied the last words with a kind of whistling of the lips.

But they were obliged to separate from each other because of a great
pile of chairs that a man was carrying behind them. He was so overladen
with them that one could only see the tips of his wooden shoes and the
ends of his two outstretched arms. It was Lestiboudois, the gravedigger,
who was carrying the church chairs about amongst the people. Alive to
all that concerned his interests, he had hit upon this means of turning
the show to account; and his idea was succeeding, for he no longer knew
which way to turn. In fact, the villagers, who were hot, quarreled for
these seats, whose straw smelt of incense, and they leant against the
thick backs, stained with the wax of candles, with a certain veneration.

Madame Bovary again took Rodolphe's arm; he went on as if speaking to
himself--

"Yes, I have missed so many things. Always alone! Ah! if I had some aim
in life, if I had met some love, if I had found someone! Oh, how I would
have spent all the energy of which I am capable, surmounted everything,
overcome everything!"

"Yet it seems to me," said Emma, "that you are not to be pitied."

"Ah! you think so?" said Rodolphe.

"For, after all," she went on, "you are free--" she hesitated, "rich--"

"Do not mock me," he replied.

And she protested that she was not mocking him, when the report of a
cannon resounded. Immediately all began hustling one another pell-mell
towards the village.

It was a false alarm. The prefect seemed not to be coming, and the
members of the jury felt much embarrassed, not knowing if they ought to
begin the meeting or still wait.

At last at the end of the Place a large hired landau appeared, drawn by
two thin horses, which a coachman in a white hat was whipping lustily.
Binet had only just time to shout, "Present arms!" and the colonel to
imitate him. All ran towards the enclosure; everyone pushed forward. A
few even forgot their collars; but the equipage of the prefect seemed
to anticipate the crowd, and the two yoked jades, trapesing in their
harness, came up at a little trot in front of the peristyle of the town
hall at the very moment when the National Guard and firemen deployed,
beating drums and marking time.

"Present!" shouted Binet.

"Halt!" shouted the colonel. "Left about, march."

And after presenting arms, during which the clang of the band, letting
loose, rang out like a brass kettle rolling downstairs, all the guns
were lowered. Then was seen stepping down from the carriage a gentleman
in a short coat with silver braiding, with bald brow, and wearing a tuft
of hair at the back of his head, of a sallow complexion and the most
benign appearance. His eyes, very large and covered by heavy lids, were
half-closed to look at the crowd, while at the same time he raised his
sharp nose, and forced a smile upon his sunken mouth. He recognised the
mayor by his scarf, and explained to him that the prefect was not able
to come. He himself was a councillor at the prefecture; then he added
a few apologies. Monsieur Tuvache answered them with compliments; the
other confessed himself nervous; and they remained thus, face to face,
their foreheads almost touching, with the members of the jury all round,
the municipal council, the notable personages, the National Guard and
the crowd. The councillor pressing his little cocked hat to his
breast repeated his bows, while Tuvache, bent like a bow, also smiled,
stammered, tried to say something, protested his devotion to the
monarchy and the honour that was being done to Yonville.

Hippolyte, the groom from the inn, took the head of the horses from the
coachman, and, limping along with his club-foot, led them to the door
of the "Lion d'Or", where a number of peasants collected to look at the
carriage. The drum beat, the howitzer thundered, and the gentlemen one
by one mounted the platform, where they sat down in red utrecht velvet
arm-chairs that had been lent by Madame Tuvache.

All these people looked alike. Their fair flabby faces, somewhat tanned
by the sun, were the colour of sweet cider, and their puffy whiskers
emerged from stiff collars, kept up by white cravats with broad bows.
All the waist-coats were of velvet, double-breasted; all the watches
had, at the end of a long ribbon, an oval cornelian seal; everyone
rested his two hands on his thighs, carefully stretching the stride of
their trousers, whose unsponged glossy cloth shone more brilliantly than
the leather of their heavy boots.

The ladies of the company stood at the back under the vestibule between
the pillars while the common herd was opposite, standing up or sitting
on chairs. As a matter of fact, Lestiboudois had brought thither all
those that he had moved from the field, and he even kept running back
every minute to fetch others from the church. He caused such confusion
with this piece of business that one had great difficulty in getting to
the small steps of the platform.

"I think," said Monsieur Lheureux to the chemist, who was passing to his
place, "that they ought to have put up two Venetian masts with something
rather severe and rich for ornaments; it would have been a very pretty
effect."

"To be sure," replied Homais; "but what can you expect? The mayor took
everything on his own shoulders. He hasn't much taste. Poor Tuvache! and
he is even completely destitute of what is called the genius of art."

Rodolphe, meanwhile, with Madame Bovary, had gone up to the first
floor of the town hall, to the "council-room," and, as it was empty,
he declared that they could enjoy the sight there more comfortably. He
fetched three stools from the round table under the bust of the monarch,
and having carried them to one of the windows, they sat down by each
other.

There was commotion on the platform, long whisperings, much parleying.
At last the councillor got up. They knew now that his name was Lieuvain,
and in the crowd the name was passed from one to the other. After he had
collated a few pages, and bent over them to see better, he began--

"Gentlemen! May I be permitted first of all (before addressing you on
the object of our meeting to-day, and this sentiment will, I am sure, be
shared by you all), may I be permitted, I say, to pay a tribute to the
higher administration, to the government to the monarch, gentle men, our
sovereign, to that beloved king, to whom no branch of public or private
prosperity is a matter of indifference, and who directs with a hand at
once so firm and wise the chariot of the state amid the incessant perils
of a stormy sea, knowing, moreover, how to make peace respected as well
as war, industry, commerce, agriculture, and the fine arts?"

"I ought," said Rodolphe, "to get back a little further."

"Why?" said Emma.

But at this moment the voice of the councillor rose to an extraordinary
pitch. He declaimed--

"This is no longer the time, gentlemen, when civil discord ensanguined
our public places, when the landlord, the business-man, the working-man
himself, falling asleep at night, lying down to peaceful sleep, trembled
lest he should be awakened suddenly by the noise of incendiary tocsins,
when the most subversive doctrines audaciously sapped foundations."

"Well, someone down there might see me," Rodolphe resumed, "then
I should have to invent excuses for a fortnight; and with my bad
reputation--"

"Oh, you are slandering yourself," said Emma.

"No! It is dreadful, I assure you."

"But, gentlemen," continued the councillor, "if, banishing from my
memory the remembrance of these sad pictures, I carry my eyes back
to the actual situation of our dear country, what do I see there?
Everywhere commerce and the arts are flourishing; everywhere new means
of communication, like so many new arteries in the body of the state,
establish within it new relations. Our great industrial centres have
recovered all their activity; religion, more consolidated, smiles in
all hearts; our ports are full, confidence is born again, and France
breathes once more!"

"Besides," added Rodolphe, "perhaps from the world's point of view they
are right."

"How so?" she asked.

"What!" said he. "Do you not know that there are souls constantly
tormented? They need by turns to dream and to act, the purest passions
and the most turbulent joys, and thus they fling themselves into all
sorts of fantasies, of follies."

Then she looked at him as one looks at a traveller who has voyaged over
strange lands, and went on--

"We have not even this distraction, we poor women!"

"A sad distraction, for happiness isn't found in it."

"But is it ever found?" she asked.

"Yes; one day it comes," he answered.

"And this is what you have understood," said the councillor.

"You, farmers, agricultural labourers! you pacific pioneers of a work
that belongs wholly to civilization! you, men of progress and morality,
you have understood, I say, that political storms are even more
redoubtable than atmospheric disturbances!"

"It comes one day," repeated Rodolphe, "one day suddenly, and when
one is despairing of it. Then the horizon expands; it is as if a voice
cried, 'It is here!' You feel the need of confiding the whole of your
life, of giving everything, sacrificing everything to this being. There
is no need for explanations; they understand one another. They have seen
each other in dreams!"

(And he looked at her.) "In fine, here it is, this treasure so sought
after, here before you. It glitters, it flashes; yet one still doubts,
one does not believe it; one remains dazzled, as if one went out iron
darkness into light."

And as he ended Rodolphe suited the action to the word. He passed his
hand over his face, like a man seized with giddiness. Then he let it
fall on Emma's. She took hers away.

"And who would be surprised at it, gentlemen? He only who is so blind,
so plunged (I do not fear to say it), so plunged in the prejudices
of another age as still to misunderstand the spirit of agricultural
populations. Where, indeed, is to be found more patriotism than in the
country, greater devotion to the public welfare, more intelligence, in a
word? And, gentlemen, I do not mean that superficial intelligence,
vain ornament of idle minds, but rather that profound and balanced
intelligence that applies itself above all else to useful objects, thus
contributing to the good of all, to the common amelioration and to
the support of the state, born of respect for law and the practice of
duty--"

"Ah! again!" said Rodolphe. "Always 'duty.' I am sick of the word.
They are a lot of old blockheads in flannel vests and of old women with
foot-warmers and rosaries who constantly drone into our ears 'Duty,
duty!' Ah! by Jove! one's duty is to feel what is great, cherish the
beautiful, and not accept all the conventions of society with the
ignominy that it imposes upon us."

"Yet--yet--" objected Madame Bovary.

"No, no! Why cry out against the passions? Are they not the one
beautiful thing on the earth, the source of heroism, of enthusiasm, of
poetry, music, the arts, of everything, in a word?"

"But one must," said Emma, "to some extent bow to the opinion of the
world and accept its moral code."

"Ah! but there are two," he replied. "The small, the conventional, that
of men, that which constantly changes, that brays out so loudly, that
makes such a commotion here below, of the earth earthly, like the mass
of imbeciles you see down there. But the other, the eternal, that is
about us and above, like the landscape that surrounds us, and the blue
heavens that give us light."

Monsieur Lieuvain had just wiped his mouth with a pocket-handkerchief.
He continued--

"And what should I do here gentlemen, pointing out to you the uses
of agriculture? Who supplies our wants? Who provides our means of
subsistence? Is it not the agriculturist? The agriculturist, gentlemen,
who, sowing with laborious hand the fertile furrows of the country,
brings forth the corn, which, being ground, is made into a powder by
means of ingenious machinery, comes out thence under the name of flour,
and from there, transported to our cities, is soon delivered at the
baker's, who makes it into food for poor and rich alike. Again, is it
not the agriculturist who fattens, for our clothes, his abundant
flocks in the pastures? For how should we clothe ourselves, how nourish
ourselves, without the agriculturist? And, gentlemen, is it even
necessary to go so far for examples? Who has not frequently reflected
on all the momentous things that we get out of that modest animal, the
ornament of poultry-yards, that provides us at once with a soft pillow
for our bed, with succulent flesh for our tables, and eggs? But I should
never end if I were to enumerate one after the other all the different
products which the earth, well cultivated, like a generous mother,
lavishes upon her children. Here it is the vine, elsewhere the apple
tree for cider, there colza, farther on cheeses and flax. Gentlemen, let
us not forget flax, which has made such great strides of late years, and
to which I will more particularly call your attention."

He had no need to call it, for all the mouths of the multitude were wide
open, as if to drink in his words. Tuvache by his side listened to him
with staring eyes. Monsieur Derozerays from time to time softly closed
his eyelids, and farther on the chemist, with his son Napoleon between
his knees, put his hand behind his ear in order not to lose a syllable.
The chins of the other members of the jury went slowly up and down in
their waistcoats in sign of approval. The firemen at the foot of the
platform rested on their bayonets; and Binet, motionless, stood with
out-turned elbows, the point of his sabre in the air. Perhaps he could
hear, but certainly he could see nothing, because of the visor of his
helmet, that fell down on his nose. His lieutenant, the youngest son of
Monsieur Tuvache, had a bigger one, for his was enormous, and shook on
his head, and from it an end of his cotton scarf peeped out. He smiled
beneath it with a perfectly infantine sweetness, and his pale little
face, whence drops were running, wore an expression of enjoyment and
sleepiness.

The square as far as the houses was crowded with people. One saw folk
leaning on their elbows at all the windows, others standing at doors,
and Justin, in front of the chemist's shop, seemed quite transfixed by
the sight of what he was looking at. In spite of the silence Monsieur
Lieuvain's voice was lost in the air. It reached you in fragments of
phrases, and interrupted here and there by the creaking of chairs in the
crowd; then you suddenly heard the long bellowing of an ox, or else the
bleating of the lambs, who answered one another at street corners. In
fact, the cowherds and shepherds had driven their beasts thus far, and
these lowed from time to time, while with their tongues they tore down
some scrap of foliage that hung above their mouths.

Rodolphe had drawn nearer to Emma, and said to her in a low voice,
speaking rapidly--

"Does not this conspiracy of the world revolt you? Is there a single
sentiment it does not condemn? The noblest instincts, the purest
sympathies are persecuted, slandered; and if at length two poor souls do
meet, all is so organised that they cannot blend together. Yet they will
make the attempt; they will flutter their wings; they will call upon
each other. Oh! no matter. Sooner or later, in six months, ten years,
they will come together, will love; for fate has decreed it, and they
are born one for the other."

His arms were folded across his knees, and thus lifting his face towards
Emma, close by her, he looked fixedly at her. She noticed in his eyes
small golden lines radiating from black pupils; she even smelt the
perfume of the pomade that made his hair glossy.

Then a faintness came over her; she recalled the Viscount who had
waltzed with her at Vaubyessard, and his beard exhaled like this air an
odour of vanilla and citron, and mechanically she half-closed her eyes
the better to breathe it in. But in making this movement, as she leant
back in her chair, she saw in the distance, right on the line of the
horizon, the old diligence, the "Hirondelle," that was slowly descending
the hill of Leux, dragging after it a long trail of dust. It was in this
yellow carriage that Leon had so often come back to her, and by this
route down there that he had gone for ever. She fancied she saw him
opposite at his windows; then all grew confused; clouds gathered; it
seemed to her that she was again turning in the waltz under the light of
the lustres on the arm of the Viscount, and that Leon was not far away,
that he was coming; and yet all the time she was conscious of the scent
of Rodolphe's head by her side. This sweetness of sensation pierced
through her old desires, and these, like grains of sand under a gust
of wind, eddied to and fro in the subtle breath of the perfume which
suffused her soul. She opened wide her nostrils several times to drink
in the freshness of the ivy round the capitals. She took off her gloves,
she wiped her hands, then fanned her face with her handkerchief, while
athwart the throbbing of her temples she heard the murmur of the
crowd and the voice of the councillor intoning his phrases. He
said--"Continue, persevere; listen neither to the suggestions of
routine, nor to the over-hasty councils of a rash empiricism.

"Apply yourselves, above all, to the amelioration of the soil, to good
manures, to the development of the equine, bovine, ovine, and porcine
races. Let these shows be to you pacific arenas, where the victor in
leaving it will hold forth a hand to the vanquished, and will fraternise
with him in the hope of better success. And you, aged servants, humble
domestics, whose hard labour no Government up to this day has taken into
consideration, come hither to receive the reward of your silent virtues,
and be assured that the state henceforward has its eye upon you; that it
encourages you, protects you; that it will accede to your just
demands, and alleviate as much as in it lies the burden of your painful
sacrifices."

Monsieur Lieuvain then sat down; Monsieur Derozerays got up, beginning
another speech. His was not perhaps so florid as that of the councillor,
but it recommended itself by a more direct style, that is to say, by
more special knowledge and more elevated considerations. Thus the praise
of the Government took up less space in it; religion and agriculture
more. He showed in it the relations of these two, and how they had
always contributed to civilisation. Rodolphe with Madame Bovary was
talking dreams, presentiments, magnetism. Going back to the cradle of
society, the orator painted those fierce times when men lived on acorns
in the heart of woods. Then they had left off the skins of beasts, had
put on cloth, tilled the soil, planted the vine. Was this a good, and
in this discovery was there not more of injury than of gain? Monsieur
Derozerays set himself this problem. From magnetism little by little
Rodolphe had come to affinities, and while the president was citing
Cincinnatus and his plough, Diocletian, planting his cabbages, and the
Emperors of China inaugurating the year by the sowing of seed, the
young man was explaining to the young woman that these irresistible
attractions find their cause in some previous state of existence.

"Thus we," he said, "why did we come to know one another? What chance
willed it? It was because across the infinite, like two streams that
flow but to unite; our special bents of mind had driven us towards each
other."

And he seized her hand; she did not withdraw it.

"For good farming generally!" cried the president.

"Just now, for example, when I went to your house."

"To Monsieur Bizat of Quincampoix."

"Did I know I should accompany you?"

"Seventy francs."

"A hundred times I wished to go; and I followed you--I remained."

"Manures!"

"And I shall remain to-night, to-morrow, all other days, all my life!"

"To Monsieur Caron of Argueil, a gold medal!"

"For I have never in the society of any other person found so complete a
charm."

"To Monsieur Bain of Givry-Saint-Martin."

"And I shall carry away with me the remembrance of you."

"For a merino ram!"

"But you will forget me; I shall pass away like a shadow."

"To Monsieur Belot of Notre-Dame."

"Oh, no! I shall be something in your thought, in your life, shall I
not?"

"Porcine race; prizes--equal, to Messrs. Leherisse and Cullembourg,
sixty francs!"

Rodolphe was pressing her hand, and he felt it all warm and quivering
like a captive dove that wants to fly away; but, whether she was trying
to take it away or whether she was answering his pressure; she made a
movement with her fingers. He exclaimed--

"Oh, I thank you! You do not repulse me! You are good! You understand
that I am yours! Let me look at you; let me contemplate you!"

A gust of wind that blew in at the window ruffled the cloth on the
table, and in the square below all the great caps of the peasant women
were uplifted by it like the wings of white butterflies fluttering.

"Use of oil-cakes," continued the president. He was hurrying on:
"Flemish manure-flax-growing-drainage-long leases-domestic service."

Rodolphe was no longer speaking. They looked at one another. A supreme
desire made their dry lips tremble, and wearily, without an effort,
their fingers intertwined.

"Catherine Nicaise Elizabeth Leroux, of Sassetot-la-Guerriere, for
fifty-four years of service at the same farm, a silver medal--value,
twenty-five francs!"

"Where is Catherine Leroux?" repeated the councillor.

She did not present herself, and one could hear voices whispering--

"Go up!"

"Don't be afraid!"

"Oh, how stupid she is!"

"Well, is she there?" cried Tuvache.

"Yes; here she is."

"Then let her come up!"

Then there came forward on the platform a little old woman with timid
bearing, who seemed to shrink within her poor clothes. On her feet she
wore heavy wooden clogs, and from her hips hung a large blue apron. Her
pale face framed in a borderless cap was more wrinkled than a withered
russet apple. And from the sleeves of her red jacket looked out two
large hands with knotty joints, the dust of barns, the potash of washing
the grease of wools had so encrusted, roughened, hardened these that
they seemed dirty, although they had been rinsed in clear water; and
by dint of long service they remained half open, as if to bear humble
witness for themselves of so much suffering endured. Something of
monastic rigidity dignified her face. Nothing of sadness or of emotion
weakened that pale look. In her constant living with animals she had
caught their dumbness and their calm. It was the first time that she
found herself in the midst of so large a company, and inwardly scared by
the flags, the drums, the gentlemen in frock-coats, and the order of the
councillor, she stood motionless, not knowing whether to advance or run
away, nor why the crowd was pushing her and the jury were smiling at
her.

Thus stood before these radiant bourgeois this half-century of
servitude.

"Approach, venerable Catherine Nicaise Elizabeth Leroux!" said the
councillor, who had taken the list of prize-winners from the president;
and, looking at the piece of paper and the old woman by turns, he
repeated in a fatherly tone--"Approach! approach!"

"Are you deaf?" said Tuvache, fidgeting in his armchair; and he began
shouting in her ear, "Fifty-four years of service. A silver medal!
Twenty-five francs! For you!"

Then, when she had her medal, she looked at it, and a smile of beatitude
spread over her face; and as she walked away they could hear her
muttering "I'll give it to our cure up home, to say some masses for me!"

"What fanaticism!" exclaimed the chemist, leaning across to the notary.

The meeting was over, the crowd dispersed, and now that the speeches had
been read, each one fell back into his place again, and everything into
the old grooves; the masters bullied the servants, and these struck the
animals, indolent victors, going back to the stalls, a green-crown on
their horns.

The National Guards, however, had gone up to the first floor of the
town hall with buns spitted on their bayonets, and the drummer of the
battalion carried a basket with bottles. Madame Bovary took Rodolphe's
arm; he saw her home; they separated at her door; then he walked about
alone in the meadow while he waited for the time of the banquet.

The feast was long, noisy, ill served; the guests were so crowded that
they could hardly move their elbows; and the narrow planks used for
forms almost broke down under their weight. They ate hugely. Each one
stuffed himself on his own account. Sweat stood on every brow, and a
whitish steam, like the vapour of a stream on an autumn morning, floated
above the table between the hanging lamps. Rodolphe, leaning against
the calico of the tent was thinking so earnestly of Emma that he heard
nothing. Behind him on the grass the servants were piling up the dirty
plates, his neighbours were talking; he did not answer them; they filled
his glass, and there was silence in his thoughts in spite of the growing
noise. He was dreaming of what she had said, of the line of her lips;
her face, as in a magic mirror, shone on the plates of the shakos, the
folds of her gown fell along the walls, and days of love unrolled to all
infinity before him in the vistas of the future.

He saw her again in the evening during the fireworks, but she was with
her husband, Madame Homais, and the druggist, who was worrying about the
danger of stray rockets, and every moment he left the company to go and
give some advice to Binet.

The pyrotechnic pieces sent to Monsieur Tuvache had, through an excess
of caution, been shut up in his cellar, and so the damp powder would
not light, and the principal set piece, that was to represent a dragon
biting his tail, failed completely. Now and then a meagre Roman-candle
went off; then the gaping crowd sent up a shout that mingled with the
cry of the women, whose waists were being squeezed in the darkness. Emma
silently nestled against Charles's shoulder; then, raising her chin, she
watched the luminous rays of the rockets against the dark sky. Rodolphe
gazed at her in the light of the burning lanterns.

They went out one by one. The stars shone out. A few crops of rain began
to fall. She knotted her fichu round her bare head.

At this moment the councillor's carriage came out from the inn.

His coachman, who was drunk, suddenly dozed off, and one could see from
the distance, above the hood, between the two lanterns, the mass of his
body, that swayed from right to left with the giving of the traces.

"Truly," said the druggist, "one ought to proceed most rigorously
against drunkenness! I should like to see written up weekly at the door
of the town hall on a board ad hoc* the names of all those who during
the week got intoxicated on alcohol. Besides, with regard to statistics,
one would thus have, as it were, public records that one could refer to
in case of need. But excuse me!"

     *Specifically for that.

And he once more ran off to the captain. The latter was going back to
see his lathe again.

"Perhaps you would not do ill," Homais said to him, "to send one of your
men, or to go yourself--"

"Leave me alone!" answered the tax-collector. "It's all right!"

"Do not be uneasy," said the druggist, when he returned to his friends.
"Monsieur Binet has assured me that all precautions have been taken. No
sparks have fallen; the pumps are full. Let us go to rest."

"Ma foi! I want it," said Madame Homais, yawning at large. "But never
mind; we've had a beautiful day for our fete."

Rodolphe repeated in a low voice, and with a tender look, "Oh, yes! very
beautiful!"

And having bowed to one another, they separated.

Two days later, in the "Final de Rouen," there was a long article on the
show. Homais had composed it with verve the very next morning.

"Why these festoons, these flowers, these garlands? Whither hurries this
crowd like the waves of a furious sea under the torrents of a tropical
sun pouring its heat upon our heads?"

Then he spoke of the condition of the peasants. Certainly the Government
was doing much, but not enough. "Courage!" he cried to it; "a thousand
reforms are indispensable; let us accomplish them!" Then touching on
the entry of the councillor, he did not forget "the martial air of our
militia;" nor "our most merry village maidens;" nor the "bald-headed old
men like patriarchs who were there, and of whom some, the remnants of
our phalanxes, still felt their hearts beat at the manly sound of the
drums." He cited himself among the first of the members of the jury,
and he even called attention in a note to the fact that Monsieur Homais,
chemist, had sent a memoir on cider to the agricultural society.

When he came to the distribution of the prizes, he painted the joy of
the prize-winners in dithyrambic strophes. "The father embraced the son,
the brother the brother, the husband his consort. More than one showed
his humble medal with pride; and no doubt when he got home to his good
housewife, he hung it up weeping on the modest walls of his cot.

"About six o'clock a banquet prepared in the meadow of Monsieur Leigeard
brought together the principal personages of the fete. The greatest
cordiality reigned here. Divers toasts were proposed: Monsieur
Lieuvain, the King; Monsieur Tuvache, the Prefect; Monsieur Derozerays,
Agriculture; Monsieur Homais, Industry and the Fine Arts, those twin
sisters; Monsieur Leplichey, Progress. In the evening some brilliant
fireworks on a sudden illumined the air. One would have called it a
veritable kaleidoscope, a real operatic scene; and for a moment our
little locality might have thought itself transported into the midst of
a dream of the 'Thousand and One Nights.' Let us state that no untoward
event disturbed this family meeting." And he added "Only the absence
of the clergy was remarked. No doubt the priests understand progress in
another fashion. Just as you please, messieurs the followers of Loyola!"



Chapter Nine

Six weeks passed. Rodolphe did not come again. At last one evening he
appeared.

The day after the show he had said to himself--"We mustn't go back too
soon; that would be a mistake."

And at the end of a week he had gone off hunting. After the hunting he
had thought it was too late, and then he reasoned thus--

"If from the first day she loved me, she must from impatience to see me
again love me more. Let's go on with it!"

And he knew that his calculation had been right when, on entering the
room, he saw Emma turn pale.

She was alone. The day was drawing in. The small muslin curtain along
the windows deepened the twilight, and the gilding of the barometer, on
which the rays of the sun fell, shone in the looking-glass between the
meshes of the coral.

Rodolphe remained standing, and Emma hardly answered his first
conventional phrases.

"I," he said, "have been busy. I have been ill."

"Seriously?" she cried.

"Well," said Rodolphe, sitting down at her side on a footstool, "no; it
was because I did not want to come back."

"Why?"

"Can you not guess?"

He looked at her again, but so hard that she lowered her head, blushing.
He went on--

"Emma!"

"Sir," she said, drawing back a little.

"Ah! you see," replied he in a melancholy voice, "that I was right not
to come back; for this name, this name that fills my whole soul, and
that escaped me, you forbid me to use! Madame Bovary! why all the
world calls you thus! Besides, it is not your name; it is the name of
another!"

He repeated, "of another!" And he hid his face in his hands.

"Yes, I think of you constantly. The memory of you drives me to despair.
Ah! forgive me! I will leave you! Farewell! I will go far away, so far
that you will never hear of me again; and yet--to-day--I know not what
force impelled me towards you. For one does not struggle against Heaven;
one cannot resist the smile of angels; one is carried away by that which
is beautiful, charming, adorable."

It was the first time that Emma had heard such words spoken to herself,
and her pride, like one who reposes bathed in warmth, expanded softly
and fully at this glowing language.

"But if I did not come," he continued, "if I could not see you, at least
I have gazed long on all that surrounds you. At night-every night-I
arose; I came hither; I watched your house, its glimmering in the moon,
the trees in the garden swaying before your window, and the little lamp,
a gleam shining through the window-panes in the darkness. Ah! you never
knew that there, so near you, so far from you, was a poor wretch!"

She turned towards him with a sob.

"Oh, you are good!" she said.

"No, I love you, that is all! You do not doubt that! Tell me--one
word--only one word!"

And Rodolphe imperceptibly glided from the footstool to the ground; but
a sound of wooden shoes was heard in the kitchen, and he noticed the
door of the room was not closed.

"How kind it would be of you," he went on, rising, "if you would humour
a whim of mine." It was to go over her house; he wanted to know it; and
Madame Bovary seeing no objection to this, they both rose, when Charles
came in.

"Good morning, doctor," Rodolphe said to him.

The doctor, flattered at this unexpected title, launched out into
obsequious phrases. Of this the other took advantage to pull himself
together a little.

"Madame was speaking to me," he then said, "about her health."

Charles interrupted him; he had indeed a thousand anxieties; his wife's
palpitations of the heart were beginning again. Then Rodolphe asked if
riding would not be good.

"Certainly! excellent! just the thing! There's an idea! You ought to
follow it up."

And as she objected that she had no horse, Monsieur Rodolphe offered
one. She refused his offer; he did not insist. Then to explain his visit
he said that his ploughman, the man of the blood-letting, still suffered
from giddiness.

"I'll call around," said Bovary.

"No, no! I'll send him to you; we'll come; that will be more convenient
for you."

"Ah! very good! I thank you."

And as soon as they were alone, "Why don't you accept Monsieur
Boulanger's kind offer?"

She assumed a sulky air, invented a thousand excuses, and finally
declared that perhaps it would look odd.

"Well, what the deuce do I care for that?" said Charles, making a
pirouette. "Health before everything! You are wrong."

"And how do you think I can ride when I haven't got a habit?"

"You must order one," he answered.

The riding-habit decided her.

When the habit was ready, Charles wrote to Monsieur Boulanger that his
wife was at his command, and that they counted on his good-nature.

The next day at noon Rodolphe appeared at Charles's door with two
saddle-horses. One had pink rosettes at his ears and a deerskin
side-saddle.

Rodolphe had put on high soft boots, saying to himself that no doubt she
had never seen anything like them. In fact, Emma was charmed with his
appearance as he stood on the landing in his great velvet coat and white
corduroy breeches. She was ready; she was waiting for him.

Justin escaped from the chemist's to see her start, and the chemist also
came out. He was giving Monsieur Boulanger a little good advice.

"An accident happens so easily. Be careful! Your horses perhaps are
mettlesome."

She heard a noise above her; it was Felicite drumming on the windowpanes
to amuse little Berthe. The child blew her a kiss; her mother answered
with a wave of her whip.

"A pleasant ride!" cried Monsieur Homais. "Prudence! above all,
prudence!" And he flourished his newspaper as he saw them disappear.

As soon as he felt the ground, Emma's horse set off at a gallop.

Rodolphe galloped by her side. Now and then they exchanged a word. Her
figure slightly bent, her hand well up, and her right arm stretched out,
she gave herself up to the cadence of the movement that rocked her in
her saddle. At the bottom of the hill Rodolphe gave his horse its head;
they started together at a bound, then at the top suddenly the horses
stopped, and her large blue veil fell about her.

It was early in October. There was fog over the land. Hazy clouds
hovered on the horizon between the outlines of the hills; others, rent
asunder, floated up and disappeared. Sometimes through a rift in the
clouds, beneath a ray of sunshine, gleamed from afar the roots of
Yonville, with the gardens at the water's edge, the yards, the walls and
the church steeple. Emma half closed her eyes to pick out her house, and
never had this poor village where she lived appeared so small. From the
height on which they were the whole valley seemed an immense pale lake
sending off its vapour into the air. Clumps of trees here and there
stood out like black rocks, and the tall lines of the poplars that rose
above the mist were like a beach stirred by the wind.

By the side, on the turf between the pines, a brown light shimmered
in the warm atmosphere. The earth, ruddy like the powder of tobacco,
deadened the noise of their steps, and with the edge of their shoes the
horses as they walked kicked the fallen fir cones in front of them.

Rodolphe and Emma thus went along the skirt of the wood. She turned
away from time to time to avoid his look, and then she saw only the pine
trunks in lines, whose monotonous succession made her a little giddy.
The horses were panting; the leather of the saddles creaked.

Just as they were entering the forest the sun shone out.

"God protects us!" said Rodolphe.

"Do you think so?" she said.

"Forward! forward!" he continued.

He "tchk'd" with his tongue. The two beasts set off at a trot.

Long ferns by the roadside caught in Emma's stirrup.

Rodolphe leant forward and removed them as they rode along. At other
times, to turn aside the branches, he passed close to her, and Emma felt
his knee brushing against her leg. The sky was now blue, the leaves no
longer stirred. There were spaces full of heather in flower, and plots
of violets alternated with the confused patches of the trees that were
grey, fawn, or golden coloured, according to the nature of their leaves.
Often in the thicket was heard the fluttering of wings, or else the
hoarse, soft cry of the ravens flying off amidst the oaks.

They dismounted. Rodolphe fastened up the horses. She walked on in
front on the moss between the paths. But her long habit got in her way,
although she held it up by the skirt; and Rodolphe, walking behind her,
saw between the black cloth and the black shoe the fineness of her white
stocking, that seemed to him as if it were a part of her nakedness.

She stopped. "I am tired," she said.

"Come, try again," he went on. "Courage!"

Then some hundred paces farther on she again stopped, and through her
veil, that fell sideways from her man's hat over her hips, her face
appeared in a bluish transparency as if she were floating under azure
waves.

"But where are we going?"

He did not answer. She was breathing irregularly. Rodolphe looked round
him biting his moustache. They came to a larger space where the coppice
had been cut. They sat down on the trunk of a fallen tree, and Rodolphe
began speaking to her of his love. He did not begin by frightening her
with compliments. He was calm, serious, melancholy.

Emma listened to him with bowed head, and stirred the bits of wood on
the ground with the tip of her foot. But at the words, "Are not our
destinies now one?"

"Oh, no!" she replied. "You know that well. It is impossible!" She rose
to go. He seized her by the wrist. She stopped. Then, having gazed
at him for a few moments with an amorous and humid look, she said
hurriedly--

"Ah! do not speak of it again! Where are the horses? Let us go back."

He made a gesture of anger and annoyance. She repeated:

"Where are the horses? Where are the horses?"

Then smiling a strange smile, his pupil fixed, his teeth set, he
advanced with outstretched arms. She recoiled trembling. She stammered:

"Oh, you frighten me! You hurt me! Let me go!"

"If it must be," he went on, his face changing; and he again became
respectful, caressing, timid. She gave him her arm. They went back. He
said--

"What was the matter with you? Why? I do not understand. You were
mistaken, no doubt. In my soul you are as a Madonna on a pedestal, in
a place lofty, secure, immaculate. But I need you to live! I must have
your eyes, your voice, your thought! Be my friend, my sister, my angel!"

And he put out his arm round her waist. She feebly tried to disengage
herself. He supported her thus as they walked along.

But they heard the two horses browsing on the leaves.

"Oh! one moment!" said Rodolphe. "Do not let us go! Stay!"

He drew her farther on to a small pool where duckweeds made a greenness
on the water. Faded water lilies lay motionless between the reeds.
At the noise of their steps in the grass, frogs jumped away to hide
themselves.

"I am wrong! I am wrong!" she said. "I am mad to listen to you!"

"Why? Emma! Emma!"

"Oh, Rodolphe!" said the young woman slowly, leaning on his shoulder.

The cloth of her habit caught against the velvet of his coat. She threw
back her white neck, swelling with a sigh, and faltering, in tears, with
a long shudder and hiding her face, she gave herself up to him--

The shades of night were falling; the horizontal sun passing between the
branches dazzled the eyes. Here and there around her, in the leaves
or on the ground, trembled luminous patches, as it hummingbirds flying
about had scattered their feathers. Silence was everywhere; something
sweet seemed to come forth from the trees; she felt her heart, whose
beating had begun again, and the blood coursing through her flesh like a
stream of milk. Then far away, beyond the wood, on the other hills, she
heard a vague prolonged cry, a voice which lingered, and in silence she
heard it mingling like music with the last pulsations of her throbbing
nerves. Rodolphe, a cigar between his lips, was mending with his
penknife one of the two broken bridles.

They returned to Yonville by the same road. On the mud they saw again
the traces of their horses side by side, the same thickets, the same
stones to the grass; nothing around them seemed changed; and yet for her
something had happened more stupendous than if the mountains had moved
in their places. Rodolphe now and again bent forward and took her hand
to kiss it.

She was charming on horseback--upright, with her slender waist, her knee
bent on the mane of her horse, her face somewhat flushed by the fresh
air in the red of the evening.

On entering Yonville she made her horse prance in the road. People
looked at her from the windows.

At dinner her husband thought she looked well, but she pretended not to
hear him when he inquired about her ride, and she remained sitting there
with her elbow at the side of her plate between the two lighted candles.

"Emma!" he said.

"What?"

"Well, I spent the afternoon at Monsieur Alexandre's. He has an old cob,
still very fine, only a little broken-kneed, and that could be bought; I
am sure, for a hundred crowns." He added, "And thinking it might please
you, I have bespoken it--bought it. Have I done right? Do tell me?"

She nodded her head in assent; then a quarter of an hour later--

"Are you going out to-night?" she asked.

"Yes. Why?"

"Oh, nothing, nothing, my dear!"

And as soon as she had got rid of Charles she went and shut herself up
in her room.

At first she felt stunned; she saw the trees, the paths, the ditches,
Rodolphe, and she again felt the pressure of his arm, while the leaves
rustled and the reeds whistled.

But when she saw herself in the glass she wondered at her face. Never
had her eyes been so large, so black, of so profound a depth. Something
subtle about her being transfigured her. She repeated, "I have a lover!
a lover!" delighting at the idea as if a second puberty had come to her.
So at last she was to know those joys of love, that fever of happiness
of which she had despaired! She was entering upon marvels where all
would be passion, ecstasy, delirium. An azure infinity encompassed
her, the heights of sentiment sparkled under her thought, and ordinary
existence appeared only afar off, down below in the shade, through the
interspaces of these heights.

Then she recalled the heroines of the books that she had read, and the
lyric legion of these adulterous women began to sing in her memory with
the voice of sisters that charmed her. She became herself, as it were,
an actual part of these imaginings, and realised the love-dream of her
youth as she saw herself in this type of amorous women whom she had
so envied. Besides, Emma felt a satisfaction of revenge. Had she not
suffered enough? But now she triumphed, and the love so long pent up
burst forth in full joyous bubblings. She tasted it without remorse,
without anxiety, without trouble.

The day following passed with a new sweetness. They made vows to one
another She told him of her sorrows. Rodolphe interrupted her with
kisses; and she looking at him through half-closed eyes, asked him to
call her again by her name--to say that he loved her They were in the
forest, as yesterday, in the shed of some woodenshoe maker. The walls
were of straw, and the roof so low they had to stoop. They were seated
side by side on a bed of dry leaves.

From that day forth they wrote to one another regularly every evening.
Emma placed her letter at the end of the garden, by the river, in a
fissure of the wall. Rodolphe came to fetch it, and put another there,
that she always found fault with as too short.

One morning, when Charles had gone out before day break, she was seized
with the fancy to see Rodolphe at once. She would go quickly to La
Huchette, stay there an hour, and be back again at Yonville while
everyone was still asleep. This idea made her pant with desire, and she
soon found herself in the middle of the field, walking with rapid steps,
without looking behind her.

Day was just breaking. Emma from afar recognised her lover's house. Its
two dove-tailed weathercocks stood out black against the pale dawn.

Beyond the farmyard there was a detached building that she thought must
be the chateau She entered--it was if the doors at her approach had
opened wide of their own accord. A large straight staircase led up to
the corridor. Emma raised the latch of a door, and suddenly at the end
of the room she saw a man sleeping. It was Rodolphe. She uttered a cry.

"You here? You here?" he repeated. "How did you manage to come? Ah! your
dress is damp."

"I love you," she answered, throwing her arms about his neck.

This first piece of daring successful, now every time Charles went out
early Emma dressed quickly and slipped on tiptoe down the steps that led
to the waterside.

But when the plank for the cows was taken up, she had to go by the walls
alongside of the river; the bank was slippery; in order not to fall
she caught hold of the tufts of faded wallflowers. Then she went across
ploughed fields, in which she sank, stumbling; and clogging her thin
shoes. Her scarf, knotted round her head, fluttered to the wind in the
meadows. She was afraid of the oxen; she began to run; she arrived out
of breath, with rosy cheeks, and breathing out from her whole person a
fresh perfume of sap, of verdure, of the open air. At this hour Rodolphe
still slept. It was like a spring morning coming into his room.

The yellow curtains along the windows let a heavy, whitish light enter
softly. Emma felt about, opening and closing her eyes, while the drops
of dew hanging from her hair formed, as it were, a topaz aureole around
her face. Rodolphe, laughing, drew her to him, and pressed her to his
breast.

Then she examined the apartment, opened the drawers of the tables,
combed her hair with his comb, and looked at herself in his
shaving-glass. Often she even put between her teeth the big pipe that
lay on the table by the bed, amongst lemons and pieces of sugar near a
bottle of water.

It took them a good quarter of an hour to say goodbye. Then Emma cried.
She would have wished never to leave Rodolphe. Something stronger than
herself forced her to him; so much so, that one day, seeing her come
unexpectedly, he frowned as one put out.

"What is the matter with you?" she said. "Are you ill? Tell me!"

At last he declared with a serious air that her visits were becoming
imprudent--that she was compromising herself.



Chapter Ten

Gradually Rodolphe's fears took possession of her. At first, love had
intoxicated her; and she had thought of nothing beyond. But now that he
was indispensable to her life, she feared to lose anything of this, or
even that it should be disturbed. When she came back from his house she
looked all about her, anxiously watching every form that passed in the
horizon, and every village window from which she could be seen. She
listened for steps, cries, the noise of the ploughs, and she stopped
short, white, and trembling more than the aspen leaves swaying overhead.

One morning as she was thus returning, she suddenly thought she saw the
long barrel of a carbine that seemed to be aimed at her. It stuck out
sideways from the end of a small tub half-buried in the grass on the
edge of a ditch. Emma, half-fainting with terror, nevertheless walked
on, and a man stepped out of the tub like a Jack-in-the-box. He had
gaiters buckled up to the knees, his cap pulled down over his eyes,
trembling lips, and a red nose. It was Captain Binet lying in ambush for
wild ducks.

"You ought to have called out long ago!" he exclaimed; "When one sees a
gun, one should always give warning."

The tax-collector was thus trying to hide the fright he had had, for
a prefectorial order having prohibited duckhunting except in boats,
Monsieur Binet, despite his respect for the laws, was infringing them,
and so he every moment expected to see the rural guard turn up. But
this anxiety whetted his pleasure, and, all alone in his tub, he
congratulated himself on his luck and on his cuteness. At sight of
Emma he seemed relieved from a great weight, and at once entered upon a
conversation.

"It isn't warm; it's nipping."

Emma answered nothing. He went on--

"And you're out so early?"

"Yes," she said stammering; "I am just coming from the nurse where my
child is."

"Ah! very good! very good! For myself, I am here, just as you see me,
since break of day; but the weather is so muggy, that unless one had the
bird at the mouth of the gun--"

"Good evening, Monsieur Binet," she interrupted him, turning on her
heel.

"Your servant, madame," he replied drily; and he went back into his tub.

Emma regretted having left the tax-collector so abruptly. No doubt he
would form unfavourable conjectures. The story about the nurse was the
worst possible excuse, everyone at Yonville knowing that the little
Bovary had been at home with her parents for a year. Besides, no one
was living in this direction; this path led only to La Huchette. Binet,
then, would guess whence she came, and he would not keep silence; he
would talk, that was certain. She remained until evening racking her
brain with every conceivable lying project, and had constantly before
her eyes that imbecile with the game-bag.

Charles after dinner, seeing her gloomy, proposed, by way of
distraction, to take her to the chemist's, and the first person she
caught sight of in the shop was the taxcollector again. He was standing
in front of the counter, lit up by the gleams of the red bottle, and was
saying--

"Please give me half an ounce of vitriol."

"Justin," cried the druggist, "bring us the sulphuric acid." Then to
Emma, who was going up to Madame Homais' room, "No, stay here; it isn't
worth while going up; she is just coming down. Warm yourself at the
stove in the meantime. Excuse me. Good-day, doctor," (for the chemist
much enjoyed pronouncing the word "doctor," as if addressing another by
it reflected on himself some of the grandeur that he found in it). "Now,
take care not to upset the mortars! You'd better fetch some chairs from
the little room; you know very well that the arm-chairs are not to be
taken out of the drawing-room."

And to put his arm-chair back in its place he was darting away from the
counter, when Binet asked him for half an ounce of sugar acid.

"Sugar acid!" said the chemist contemptuously, "don't know it; I'm
ignorant of it! But perhaps you want oxalic acid. It is oxalic acid,
isn't it?"

Binet explained that he wanted a corrosive to make himself some
copperwater with which to remove rust from his hunting things.

Emma shuddered. The chemist began saying--

"Indeed the weather is not propitious on account of the damp."

"Nevertheless," replied the tax-collector, with a sly look, "there are
people who like it."

She was stifling.

"And give me--"

"Will he never go?" thought she.

"Half an ounce of resin and turpentine, four ounces of yellow wax,
and three half ounces of animal charcoal, if you please, to clean the
varnished leather of my togs."

The druggist was beginning to cut the wax when Madame Homais appeared,
Irma in her arms, Napoleon by her side, and Athalie following. She sat
down on the velvet seat by the window, and the lad squatted down on a
footstool, while his eldest sister hovered round the jujube box near
her papa. The latter was filling funnels and corking phials, sticking on
labels, making up parcels. Around him all were silent; only from time
to time, were heard the weights jingling in the balance, and a few low
words from the chemist giving directions to his pupil.

"And how's the little woman?" suddenly asked Madame Homais.

"Silence!" exclaimed her husband, who was writing down some figures in
his waste-book.

"Why didn't you bring her?" she went on in a low voice.

"Hush! hush!" said Emma, pointing with her finger to the druggist.

But Binet, quite absorbed in looking over his bill, had probably heard
nothing. At last he went out. Then Emma, relieved, uttered a deep sigh.

"How hard you are breathing!" said Madame Homais.

"Well, you see, it's rather warm," she replied.

So the next day they talked over how to arrange their rendezvous. Emma
wanted to bribe her servant with a present, but it would be better to
find some safe house at Yonville. Rodolphe promised to look for one.

All through the winter, three or four times a week, in the dead of night
he came to the garden. Emma had on purpose taken away the key of the
gate, which Charles thought lost.

To call her, Rodolphe threw a sprinkle of sand at the shutters. She
jumped up with a start; but sometimes he had to wait, for Charles had a
mania for chatting by the fireside, and he would not stop. She was wild
with impatience; if her eyes could have done it, she would have hurled
him out at the window. At last she would begin to undress, then take up
a book, and go on reading very quietly as if the book amused her. But
Charles, who was in bed, called to her to come too.

"Come, now, Emma," he said, "it is time."

"Yes, I am coming," she answered.

Then, as the candles dazzled him; he turned to the wall and fell asleep.
She escaped, smiling, palpitating, undressed. Rodolphe had a large
cloak; he wrapped her in it, and putting his arm round her waist, he
drew her without a word to the end of the garden.

It was in the arbour, on the same seat of old sticks where formerly Leon
had looked at her so amorously on the summer evenings. She never thought
of him now.

The stars shone through the leafless jasmine branches. Behind them they
heard the river flowing, and now and again on the bank the rustling
of the dry reeds. Masses of shadow here and there loomed out in the
darkness, and sometimes, vibrating with one movement, they rose up and
swayed like immense black waves pressing forward to engulf them. The
cold of the nights made them clasp closer; the sighs of their lips
seemed to them deeper; their eyes that they could hardly see, larger;
and in the midst of the silence low words were spoken that fell on
their souls sonorous, crystalline, and that reverberated in multiplied
vibrations.

When the night was rainy, they took refuge in the consulting-room
between the cart-shed and the stable. She lighted one of the kitchen
candles that she had hidden behind the books. Rodolphe settled down
there as if at home. The sight of the library, of the bureau, of the
whole apartment, in fine, excited his merriment, and he could not
refrain from making jokes about Charles, which rather embarrassed Emma.
She would have liked to see him more serious, and even on occasions
more dramatic; as, for example, when she thought she heard a noise of
approaching steps in the alley.

"Someone is coming!" she said.

He blew out the light.

"Have you your pistols?"

"Why?"

"Why, to defend yourself," replied Emma.

"From your husband? Oh, poor devil!" And Rodolphe finished his sentence
with a gesture that said, "I could crush him with a flip of my finger."

She was wonder-stricken at his bravery, although she felt in it a sort
of indecency and a naive coarseness that scandalised her.

Rodolphe reflected a good deal on the affair of the pistols. If she had
spoken seriously, it was very ridiculous, he thought, even odious; for
he had no reason to hate the good Charles, not being what is called
devoured by jealousy; and on this subject Emma had taken a great vow
that he did not think in the best of taste.

Besides, she was growing very sentimental. She had insisted on
exchanging miniatures; they had cut off handfuls of hair, and now she
was asking for a ring--a real wedding-ring, in sign of an eternal union.
She often spoke to him of the evening chimes, of the voices of nature.
Then she talked to him of her mother--hers! and of his mother--his!
Rodolphe had lost his twenty years ago. Emma none the less consoled
him with caressing words as one would have done a lost child, and she
sometimes even said to him, gazing at the moon--

"I am sure that above there together they approve of our love."

But she was so pretty. He had possessed so few women of such
ingenuousness. This love without debauchery was a new experience for
him, and, drawing him out of his lazy habits, caressed at once his pride
and his sensuality. Emma's enthusiasm, which his bourgeois good sense
disdained, seemed to him in his heart of hearts charming, since it
was lavished on him. Then, sure of being loved, he no longer kept up
appearances, and insensibly his ways changed.

He had no longer, as formerly, words so gentle that they made her cry,
nor passionate caresses that made her mad, so that their great love,
which engrossed her life, seemed to lessen beneath her like the water of
a stream absorbed into its channel, and she could see the bed of it.
She would not believe it; she redoubled in tenderness, and Rodolphe
concealed his indifference less and less.

She did not know if she regretted having yielded to him, or whether she
did not wish, on the contrary, to enjoy him the more. The humiliation
of feeling herself weak was turning to rancour, tempered by their
voluptuous pleasures. It was not affection; it was like a continual
seduction. He subjugated her; she almost feared him.

Appearances, nevertheless, were calmer than ever, Rodolphe having
succeeded in carrying out the adultery after his own fancy; and at the
end of six months, when the spring-time came, they were to one another
like a married couple, tranquilly keeping up a domestic flame.

It was the time of year when old Rouault sent his turkey in remembrance
of the setting of his leg. The present always arrived with a letter.
Emma cut the string that tied it to the basket, and read the following
lines:--

"My Dear Children--I hope this will find you well, and that this one
will be as good as the others. For it seems to me a little more tender,
if I may venture to say so, and heavier. But next time, for a change,
I'll give you a turkeycock, unless you have a preference for some dabs;
and send me back the hamper, if you please, with the two old ones. I
have had an accident with my cart-sheds, whose covering flew off one
windy night among the trees. The harvest has not been overgood either.
Finally, I don't know when I shall come to see you. It is so difficult
now to leave the house since I am alone, my poor Emma."

Here there was a break in the lines, as if the old fellow had dropped
his pen to dream a little while.

"For myself, I am very well, except for a cold I caught the other day at
the fair at Yvetot, where I had gone to hire a shepherd, having turned
away mine because he was too dainty. How we are to be pitied with such
a lot of thieves! Besides, he was also rude. I heard from a pedlar, who,
travelling through your part of the country this winter, had a tooth
drawn, that Bovary was as usual working hard. That doesn't surprise me;
and he showed me his tooth; we had some coffee together. I asked him if
he had seen you, and he said not, but that he had seen two horses in the
stables, from which I conclude that business is looking up. So much
the better, my dear children, and may God send you every imaginable
happiness! It grieves me not yet to have seen my dear little
grand-daughter, Berthe Bovary. I have planted an Orleans plum-tree for
her in the garden under your room, and I won't have it touched unless it
is to have jam made for her by and bye, that I will keep in the cupboard
for her when she comes.

"Good-bye, my dear children. I kiss you, my girl, you too, my
son-in-law, and the little one on both cheeks. I am, with best
compliments, your loving father.

"Theodore Rouault."

She held the coarse paper in her fingers for some minutes. The spelling
mistakes were interwoven one with the other, and Emma followed the
kindly thought that cackled right through it like a hen half hidden
in the hedge of thorns. The writing had been dried with ashes from
the hearth, for a little grey powder slipped from the letter on to her
dress, and she almost thought she saw her father bending over the hearth
to take up the tongs. How long since she had been with him, sitting on
the footstool in the chimney-corner, where she used to burn the end of
a bit of wood in the great flame of the sea-sedges! She remembered the
summer evenings all full of sunshine. The colts neighed when anyone
passed by, and galloped, galloped. Under her window there was a beehive,
and sometimes the bees wheeling round in the light struck against her
window like rebounding balls of gold. What happiness there had been
at that time, what freedom, what hope! What an abundance of illusions!
Nothing was left of them now. She had got rid of them all in her soul's
life, in all her successive conditions of life, maidenhood, her marriage,
and her love--thus constantly losing them all her life through, like
a traveller who leaves something of his wealth at every inn along his
road.

But what then, made her so unhappy? What was the extraordinary
catastrophe that had transformed her? And she raised her head, looking
round as if to seek the cause of that which made her suffer.

An April ray was dancing on the china of the whatnot; the fire burned;
beneath her slippers she felt the softness of the carpet; the day was
bright, the air warm, and she heard her child shouting with laughter.

In fact, the little girl was just then rolling on the lawn in the midst
of the grass that was being turned. She was lying flat on her stomach
at the top of a rick. The servant was holding her by her skirt.
Lestiboudois was raking by her side, and every time he came near she
lent forward, beating the air with both her arms.

"Bring her to me," said her mother, rushing to embrace her. "How I love
you, my poor child! How I love you!"

Then noticing that the tips of her ears were rather dirty, she rang at
once for warm water, and washed her, changed her linen, her stockings,
her shoes, asked a thousand questions about her health, as if on the
return from a long journey, and finally, kissing her again and crying
a little, she gave her back to the servant, who stood quite
thunderstricken at this excess of tenderness.

That evening Rodolphe found her more serious than usual.

"That will pass over," he concluded; "it's a whim:"

And he missed three rendezvous running. When he did come, she showed
herself cold and almost contemptuous.

"Ah! you're losing your time, my lady!"

And he pretended not to notice her melancholy sighs, nor the
handkerchief she took out.

Then Emma repented. She even asked herself why she detested Charles; if
it had not been better to have been able to love him? But he gave her
no opportunities for such a revival of sentiment, so that she was much
embarrassed by her desire for sacrifice, when the druggist came just in
time to provide her with an opportunity.



Chapter Eleven

He had recently read a eulogy on a new method for curing club-foot, and
as he was a partisan of progress, he conceived the patriotic idea that
Yonville, in order to keep to the fore, ought to have some operations
for strephopody or club-foot.

"For," said he to Emma, "what risk is there? See--" (and he enumerated
on his fingers the advantages of the attempt), "success, almost certain
relief and beautifying of the patient, celebrity acquired by the
operator. Why, for example, should not your husband relieve poor
Hippolyte of the 'Lion d'Or'? Note that he would not fail to tell about
his cure to all the travellers, and then" (Homais lowered his voice and
looked round him) "who is to prevent me from sending a short paragraph
on the subject to the paper? Eh! goodness me! an article gets about; it
is talked of; it ends by making a snowball! And who knows? who knows?"

In fact, Bovary might succeed. Nothing proved to Emma that he was not
clever; and what a satisfaction for her to have urged him to a step by
which his reputation and fortune would be increased! She only wished to
lean on something more solid than love.

Charles, urged by the druggist and by her, allowed himself to be
persuaded. He sent to Rouen for Dr. Duval's volume, and every evening,
holding his head between both hands, plunged into the reading of it.

While he was studying equinus, varus, and valgus, that is to say,
katastrephopody, endostrephopody, and exostrephopody (or better, the
various turnings of the foot downwards, inwards, and outwards, with the
hypostrephopody and anastrephopody), otherwise torsion downwards and
upwards, Monsier Homais, with all sorts of arguments, was exhorting the
lad at the inn to submit to the operation.

"You will scarcely feel, probably, a slight pain; it is a simple prick,
like a little blood-letting, less than the extraction of certain corns."

Hippolyte, reflecting, rolled his stupid eyes.

"However," continued the chemist, "it doesn't concern me. It's for your
sake, for pure humanity! I should like to see you, my friend, rid of
your hideous caudication, together with that waddling of the lumbar
regions which, whatever you say, must considerably interfere with you in
the exercise of your calling."

Then Homais represented to him how much jollier and brisker he would
feel afterwards, and even gave him to understand that he would be more
likely to please the women; and the stable-boy began to smile heavily.
Then he attacked him through his vanity:

"Aren't you a man? Hang it! what would you have done if you had had to
go into the army, to go and fight beneath the standard? Ah! Hippolyte!"

And Homais retired, declaring that he could not understand this
obstinacy, this blindness in refusing the benefactions of science.

The poor fellow gave way, for it was like a conspiracy. Binet, who never
interfered with other people's business, Madame Lefrancois, Artemise,
the neighbours, even the mayor, Monsieur Tuvache--everyone persuaded
him, lectured him, shamed him; but what finally decided him was that it
would cost him nothing. Bovary even undertook to provide the machine
for the operation. This generosity was an idea of Emma's, and Charles
consented to it, thinking in his heart of hearts that his wife was an
angel.

So by the advice of the chemist, and after three fresh starts, he had a
kind of box made by the carpenter, with the aid of the locksmith,
that weighed about eight pounds, and in which iron, wood, sheer-iron,
leather, screws, and nuts had not been spared.

But to know which of Hippolyte's tendons to cut, it was necessary first
of all to find out what kind of club-foot he had.

He had a foot forming almost a straight line with the leg, which,
however, did not prevent it from being turned in, so that it was an
equinus together with something of a varus, or else a slight varus with
a strong tendency to equinus. But with this equinus, wide in foot like
a horse's hoof, with rugose skin, dry tendons, and large toes, on which
the black nails looked as if made of iron, the clubfoot ran about like
a deer from morn till night. He was constantly to be seen on the Place,
jumping round the carts, thrusting his limping foot forwards. He seemed
even stronger on that leg than the other. By dint of hard service it had
acquired, as it were, moral qualities of patience and energy; and
when he was given some heavy work, he stood on it in preference to its
fellow.

Now, as it was an equinus, it was necessary to cut the tendon of
Achilles, and, if need were, the anterior tibial muscle could be seen to
afterwards for getting rid of the varus; for the doctor did not dare to
risk both operations at once; he was even trembling already for fear of
injuring some important region that he did not know.

Neither Ambrose Pare, applying for the first time since Celsus, after an
interval of fifteen centuries, a ligature to an artery, nor Dupuytren,
about to open an abscess in the brain, nor Gensoul when he first took
away the superior maxilla, had hearts that trembled, hands that shook,
minds so strained as Monsieur Bovary when he approached Hippolyte, his
tenotome between his fingers. And as at hospitals, near by on a table
lay a heap of lint, with waxed thread, many bandages--a pyramid of
bandages--every bandage to be found at the druggist's. It was Monsieur
Homais who since morning had been organising all these preparations,
as much to dazzle the multitude as to keep up his illusions. Charles
pierced the skin; a dry crackling was heard. The tendon was cut, the
operation over. Hippolyte could not get over his surprise, but bent over
Bovary's hands to cover them with kisses.

"Come, be calm," said the druggist; "later on you will show your
gratitude to your benefactor."

And he went down to tell the result to five or six inquirers who were
waiting in the yard, and who fancied that Hippolyte would reappear
walking properly. Then Charles, having buckled his patient into the
machine, went home, where Emma, all anxiety, awaited him at the door.
She threw herself on his neck; they sat down to table; he ate much,
and at dessert he even wanted to take a cup of coffee, a luxury he only
permitted himself on Sundays when there was company.

The evening was charming, full of prattle, of dreams together. They
talked about their future fortune, of the improvements to be made in
their house; he saw people's estimation of him growing, his comforts
increasing, his wife always loving him; and she was happy to refresh
herself with a new sentiment, healthier, better, to feel at last some
tenderness for this poor fellow who adored her. The thought of Rodolphe
for one moment passed through her mind, but her eyes turned again to
Charles; she even noticed with surprise that he had not bad teeth.

They were in bed when Monsieur Homais, in spite of the servant, suddenly
entered the room, holding in his hand a sheet of paper just written. It
was the paragraph he intended for the "Fanal de Rouen." He brought it
for them to read.

"Read it yourself," said Bovary.

He read--

"'Despite the prejudices that still invest a part of the face of Europe
like a net, the light nevertheless begins to penetrate our country
places. Thus on Tuesday our little town of Yonville found itself the
scene of a surgical operation which is at the same time an act of
loftiest philanthropy. Monsieur Bovary, one of our most distinguished
practitioners--'"

"Oh, that is too much! too much!" said Charles, choking with emotion.

"No, no! not at all! What next!"

"'--Performed an operation on a club-footed man.' I have not used the
scientific term, because you know in a newspaper everyone would not
perhaps understand. The masses must--'"

"No doubt," said Bovary; "go on!"

"I proceed," said the chemist. "'Monsieur Bovary, one of our most
distinguished practitioners, performed an operation on a club-footed man
called Hippolyte Tautain, stableman for the last twenty-five years at
the hotel of the "Lion d'Or," kept by Widow Lefrancois, at the Place
d'Armes. The novelty of the attempt, and the interest incident to the
subject, had attracted such a concourse of persons that there was
a veritable obstruction on the threshold of the establishment. The
operation, moreover, was performed as if by magic, and barely a
few drops of blood appeared on the skin, as though to say that the
rebellious tendon had at last given way beneath the efforts of art. The
patient, strangely enough--we affirm it as an eye-witness--complained
of no pain. His condition up to the present time leaves nothing to be
desired. Everything tends to show that his convelescence will be brief;
and who knows even if at our next village festivity we shall not see our
good Hippolyte figuring in the bacchic dance in the midst of a chorus
of joyous boon-companions, and thus proving to all eyes by his verve
and his capers his complete cure? Honour, then, to the generous savants!
Honour to those indefatigable spirits who consecrate their vigils to the
amelioration or to the alleviation of their kind! Honour, thrice honour!
Is it not time to cry that the blind shall see, the deaf hear, the lame
walk? But that which fanaticism formerly promised to its elect, science
now accomplishes for all men. We shall keep our readers informed as to
the successive phases of this remarkable cure.'"

This did not prevent Mere Lefrancois, from coming five days after,
scared, and crying out--

"Help! he is dying! I am going crazy!"

Charles rushed to the "Lion d'Or," and the chemist, who caught sight
of him passing along the Place hatless, abandoned his shop. He appeared
himself breathless, red, anxious, and asking everyone who was going up
the stairs--

"Why, what's the matter with our interesting strephopode?"

The strephopode was writhing in hideous convulsions, so that the machine
in which his leg was enclosed was knocked against the wall enough to
break it.

With many precautions, in order not to disturb the position of the limb,
the box was removed, and an awful sight presented itself. The outlines
of the foot disappeared in such a swelling that the entire skin seemed
about to burst, and it was covered with ecchymosis, caused by the famous
machine. Hippolyte had already complained of suffering from it. No
attention had been paid to him; they had to acknowledge that he had not
been altogether wrong, and he was freed for a few hours. But, hardly had
the oedema gone down to some extent, than the two savants thought fit
to put back the limb in the apparatus, strapping it tighter to hasten
matters. At last, three days after, Hippolyte being unable to endure it
any longer, they once more removed the machine, and were much surprised
at the result they saw. The livid tumefaction spread over the leg, with
blisters here and there, whence there oozed a black liquid. Matters
were taking a serious turn. Hippolyte began to worry himself, and Mere
Lefrancois, had him installed in the little room near the kitchen, so
that he might at least have some distraction.

But the tax-collector, who dined there every day, complained bitterly of
such companionship. Then Hippolyte was removed to the billiard-room.
He lay there moaning under his heavy coverings, pale with long beard,
sunken eyes, and from time to time turning his perspiring head on the
dirty pillow, where the flies alighted. Madame Bovary went to see him.
She brought him linen for his poultices; she comforted, and encouraged
him. Besides, he did not want for company, especially on market-days,
when the peasants were knocking about the billiard-balls round him,
fenced with the cues, smoked, drank, sang, and brawled.

"How are you?" they said, clapping him on the shoulder. "Ah! you're not
up to much, it seems, but it's your own fault. You should do this! do
that!" And then they told him stories of people who had all been cured
by other remedies than his. Then by way of consolation they added--

"You give way too much! Get up! You coddle yourself like a king! All the
same, old chap, you don't smell nice!"

Gangrene, in fact, was spreading more and more. Bovary himself turned
sick at it. He came every hour, every moment. Hippolyte looked at him
with eyes full of terror, sobbing--

"When shall I get well? Oh, save me! How unfortunate I am! How
unfortunate I am!"

And the doctor left, always recommending him to diet himself.

"Don't listen to him, my lad," said Mere Lefrancois, "Haven't they
tortured you enough already? You'll grow still weaker. Here! swallow
this."

And she gave him some good beef-tea, a slice of mutton, a piece of
bacon, and sometimes small glasses of brandy, that he had not the
strength to put to his lips.

Abbe Bournisien, hearing that he was growing worse, asked to see him.
He began by pitying his sufferings, declaring at the same time that he
ought to rejoice at them since it was the will of the Lord, and take
advantage of the occasion to reconcile himself to Heaven.

"For," said the ecclesiastic in a paternal tone, "you rather neglected
your duties; you were rarely seen at divine worship. How many years is
it since you approached the holy table? I understand that your work,
that the whirl of the world may have kept you from care for your
salvation. But now is the time to reflect. Yet don't despair. I have
known great sinners, who, about to appear before God (you are not yet
at this point I know), had implored His mercy, and who certainly died in
the best frame of mind. Let us hope that, like them, you will set us a
good example. Thus, as a precaution, what is to prevent you from saying
morning and evening a 'Hail Mary, full of grace,' and 'Our Father which
art in heaven'? Yes, do that, for my sake, to oblige me. That won't cost
you anything. Will you promise me?"

The poor devil promised. The cure came back day after day. He chatted
with the landlady; and even told anecdotes interspersed with jokes and
puns that Hippolyte did not understand. Then, as soon as he could, he
fell back upon matters of religion, putting on an appropriate expression
of face.

His zeal seemed successful, for the club-foot soon manifested a desire
to go on a pilgrimage to Bon-Secours if he were cured; to which Monsieur
Bournisien replied that he saw no objection; two precautions were better
than one; it was no risk anyhow.

The druggist was indignant at what he called the manoeuvres of the
priest; they were prejudicial, he said, to Hippolyte's convalescence,
and he kept repeating to Madame Lefrancois, "Leave him alone! leave him
alone! You perturb his morals with your mysticism." But the good woman
would no longer listen to him; he was the cause of it all. From a spirit
of contradiction she hung up near the bedside of the patient a basin
filled with holy-water and a branch of box.

Religion, however, seemed no more able to succour him than surgery, and
the invincible gangrene still spread from the extremities towards
the stomach. It was all very well to vary the potions and change the
poultices; the muscles each day rotted more and more; and at last
Charles replied by an affirmative nod of the head when Mere Lefrancois,
asked him if she could not, as a forlorn hope, send for Monsieur Canivet
of Neufchatel, who was a celebrity.

A doctor of medicine, fifty years of age, enjoying a good position
and self-possessed, Charles's colleague did not refrain from laughing
disdainfully when he had uncovered the leg, mortified to the knee. Then
having flatly declared that it must be amputated, he went off to the
chemist's to rail at the asses who could have reduced a poor man to such
a state. Shaking Monsieur Homais by the button of his coat, he shouted
out in the shop--

"These are the inventions of Paris! These are the ideas of those gentry
of the capital! It is like strabismus, chloroform, lithotrity, a heap of
monstrosities that the Government ought to prohibit. But they want to do
the clever, and they cram you with remedies without, troubling about
the consequences. We are not so clever, not we! We are not savants,
coxcombs, fops! We are practitioners; we cure people, and we should
not dream of operating on anyone who is in perfect health. Straighten
club-feet! As if one could straighten club-feet! It is as if one wished,
for example, to make a hunchback straight!"

Homais suffered as he listened to this discourse, and he concealed his
discomfort beneath a courtier's smile; for he needed to humour Monsier
Canivet, whose prescriptions sometimes came as far as Yonville. So he
did not take up the defence of Bovary; he did not even make a single
remark, and, renouncing his principles, he sacrificed his dignity to the
more serious interests of his business.

This amputation of the thigh by Doctor Canivet was a great event in the
village. On that day all the inhabitants got up earlier, and the Grande
Rue, although full of people, had something lugubrious about it, as
if an execution had been expected. At the grocer's they discussed
Hippolyte's illness; the shops did no business, and Madame Tuvache, the
mayor's wife, did not stir from her window, such was her impatience to
see the operator arrive.

He came in his gig, which he drove himself. But the springs of the right
side having at length given way beneath the weight of his corpulence, it
happened that the carriage as it rolled along leaned over a little, and
on the other cushion near him could be seen a large box covered in red
sheep-leather, whose three brass clasps shone grandly.

After he had entered like a whirlwind the porch of the "Lion d'Or," the
doctor, shouting very loud, ordered them to unharness his horse. Then he
went into the stable to see that he was eating his oats all right; for
on arriving at a patient's he first of all looked after his mare and his
gig. People even said about this--

"Ah! Monsieur Canivet's a character!"

And he was the more esteemed for this imperturbable coolness. The
universe to the last man might have died, and he would not have missed
the smallest of his habits.

Homais presented himself.

"I count on you," said the doctor. "Are we ready? Come along!"

But the druggist, turning red, confessed that he was too sensitive to
assist at such an operation.

"When one is a simple spectator," he said, "the imagination, you know,
is impressed. And then I have such a nervous system!"

"Pshaw!" interrupted Canivet; "on the contrary, you seem to me inclined
to apoplexy. Besides, that doesn't astonish me, for you chemist fellows
are always poking about your kitchens, which must end by spoiling your
constitutions. Now just look at me. I get up every day at four o'clock;
I shave with cold water (and am never cold). I don't wear flannels, and
I never catch cold; my carcass is good enough! I live now in one way,
now in another, like a philosopher, taking pot-luck; that is why I
am not squeamish like you, and it is as indifferent to me to carve a
Christian as the first fowl that turns up. Then, perhaps, you will say,
habit! habit!"

Then, without any consideration for Hippolyte, who was sweating with
agony between his sheets, these gentlemen entered into a conversation,
in which the druggist compared the coolness of a surgeon to that of a
general; and this comparison was pleasing to Canivet, who launched out
on the exigencies of his art. He looked upon, it as a sacred office,
although the ordinary practitioners dishonoured it. At last, coming back
to the patient, he examined the bandages brought by Homais, the same
that had appeared for the club-foot, and asked for someone to hold the
limb for him. Lestiboudois was sent for, and Monsieur Canivet having
turned up his sleeves, passed into the billiard-room, while the druggist
stayed with Artemise and the landlady, both whiter than their aprons,
and with ears strained towards the door.

Bovary during this time did not dare to stir from his house.

He kept downstairs in the sitting-room by the side of the fireless
chimney, his chin on his breast, his hands clasped, his eyes staring.
"What a mishap!" he thought, "what a mishap!" Perhaps, after all, he had
made some slip. He thought it over, but could hit upon nothing. But the
most famous surgeons also made mistakes; and that is what no one would
ever believe! People, on the contrary, would laugh, jeer! It would
spread as far as Forges, as Neufchatel, as Rouen, everywhere! Who could
say if his colleagues would not write against him. Polemics would ensue;
he would have to answer in the papers. Hippolyte might even prosecute
him. He saw himself dishonoured, ruined, lost; and his imagination,
assailed by a world of hypotheses, tossed amongst them like an empty
cask borne by the sea and floating upon the waves.

Emma, opposite, watched him; she did not share his humiliation; she felt
another--that of having supposed such a man was worth anything. As if
twenty times already she had not sufficiently perceived his mediocrity.

Charles was walking up and down the room; his boots creaked on the
floor.

"Sit down," she said; "you fidget me."

He sat down again.

How was it that she--she, who was so intelligent--could have allowed
herself to be deceived again? and through what deplorable madness had
she thus ruined her life by continual sacrifices? She recalled all her
instincts of luxury, all the privations of her soul, the sordidness of
marriage, of the household, her dream sinking into the mire like wounded
swallows; all that she had longed for, all that she had denied herself,
all that she might have had! And for what? for what?

In the midst of the silence that hung over the village a heart-rending
cry rose on the air. Bovary turned white to fainting. She knit her
brows with a nervous gesture, then went on. And it was for him, for this
creature, for this man, who understood nothing, who felt nothing! For he
was there quite quiet, not even suspecting that the ridicule of his name
would henceforth sully hers as well as his. She had made efforts to love
him, and she had repented with tears for having yielded to another!

"But it was perhaps a valgus!" suddenly exclaimed Bovary, who was
meditating.

At the unexpected shock of this phrase falling on her thought like a
leaden bullet on a silver plate, Emma, shuddering, raised her head in
order to find out what he meant to say; and they looked at the other in
silence, almost amazed to see each other, so far sundered were they
by their inner thoughts. Charles gazed at her with the dull look of
a drunken man, while he listened motionless to the last cries of the
sufferer, that followed each other in long-drawn modulations, broken by
sharp spasms like the far-off howling of some beast being slaughtered.
Emma bit her wan lips, and rolling between her fingers a piece of coral
that she had broken, fixed on Charles the burning glance of her eyes
like two arrows of fire about to dart forth. Everything in him irritated
her now; his face, his dress, what he did not say, his whole person, his
existence, in fine. She repented of her past virtue as of a crime, and
what still remained of it rumbled away beneath the furious blows of her
pride. She revelled in all the evil ironies of triumphant adultery.
The memory of her lover came back to her with dazzling attractions; she
threw her whole soul into it, borne away towards this image with a fresh
enthusiasm; and Charles seemed to her as much removed from her life, as
absent forever, as impossible and annihilated, as if he had been about
to die and were passing under her eyes.

There was a sound of steps on the pavement. Charles looked up, and
through the lowered blinds he saw at the corner of the market in
the broad sunshine Dr. Canivet, who was wiping his brow with his
handkerchief. Homais, behind him, was carrying a large red box in his
hand, and both were going towards the chemist's.

Then with a feeling of sudden tenderness and discouragement Charles
turned to his wife saying to her--

"Oh, kiss me, my own!"

"Leave me!" she said, red with anger.

"What is the matter?" he asked, stupefied. "Be calm; compose yourself.
You know well enough that I love you. Come!"

"Enough!" she cried with a terrible look.

And escaping from the room, Emma closed the door so violently that the
barometer fell from the wall and smashed on the floor.

Charles sank back into his arm-chair overwhelmed, trying to discover
what could be wrong with her, fancying some nervous illness, weeping,
and vaguely feeling something fatal and incomprehensible whirling round
him.

When Rodolphe came to the garden that evening, he found his mistress
waiting for him at the foot of the steps on the lowest stair. They threw
their arms round one another, and all their rancour melted like snow
beneath the warmth of that kiss.



\chapter{Chapter Twelve}

They began to love one another again. Often, even in the middle of the
day, Emma suddenly wrote to him, then from the window made a sign to
Justin, who, taking his apron off, quickly ran to La Huchette. Rodolphe
would come; she had sent for him to tell him that she was bored, that
her husband was odious, her life frightful.

"But what can I do?" he cried one day impatiently.

"Ah! if you would--"

She was sitting on the floor between his knees, her hair loose, her look
lost.

"Why, what?" said Rodolphe.

She sighed.

"We would go and live elsewhere--somewhere!"

"You are really mad!" he said laughing. "How could that be possible?"

She returned to the subject; he pretended not to understand, and turned
the conversation.

What he did not understand was all this worry about so simple an affair
as love. She had a motive, a reason, and, as it were, a pendant to her
affection.

Her tenderness, in fact, grew each day with her repulsion to her
husband. The more she gave up herself to the one, the more she loathed
the other. Never had Charles seemed to her so disagreeable, to have
such stodgy fingers, such vulgar ways, to be so dull as when they found
themselves together after her meeting with Rodolphe. Then, while playing
the spouse and virtue, she was burning at the thought of that head whose
black hair fell in a curl over the sunburnt brow, of that form at once
so strong and elegant, of that man, in a word, who had such experience
in his reasoning, such passion in his desires. It was for him that she
filed her nails with the care of a chaser, and that there was never
enough cold-cream for her skin, nor of patchouli for her handkerchiefs.
She loaded herself with bracelets, rings, and necklaces. When he
was coming she filled the two large blue glass vases with roses, and
prepared her room and her person like a courtesan expecting a prince.
The servant had to be constantly washing linen, and all day Felicite
did not stir from the kitchen, where little Justin, who often kept her
company, watched her at work.

With his elbows on the long board on which she was ironing, he
greedily watched all these women's clothes spread about him, the dimity
petticoats, the fichus, the collars, and the drawers with running
strings, wide at the hips and growing narrower below.

"What is that for?" asked the young fellow, passing his hand over the
crinoline or the hooks and eyes.

"Why, haven't you ever seen anything?" Felicite answered laughing. "As
if your mistress, Madame Homais, didn't wear the same."

"Oh, I daresay! Madame Homais!" And he added with a meditative air, "As
if she were a lady like madame!"

But Felicite grew impatient of seeing him hanging round her. She was six
years older than he, and Theodore, Monsieur Guillaumin's servant, was
beginning to pay court to her.

"Let me alone," she said, moving her pot of starch. "You'd better be
off and pound almonds; you are always dangling about women. Before you
meddle with such things, bad boy, wait till you've got a beard to your
chin."

"Oh, don't be cross! I'll go and clean her boots."

And he at once took down from the shelf Emma's boots, all coated with
mud, the mud of the rendezvous, that crumbled into powder beneath his
fingers, and that he watched as it gently rose in a ray of sunlight.

"How afraid you are of spoiling them!" said the servant, who wasn't so
particular when she cleaned them herself, because as soon as the stuff
of the boots was no longer fresh madame handed them over to her.

Emma had a number in her cupboard that she squandered one after the
other, without Charles allowing himself the slightest observation. So
also he disbursed three hundred francs for a wooden leg that she thought
proper to make a present of to Hippolyte. Its top was covered with cork,
and it had spring joints, a complicated mechanism, covered over by black
trousers ending in a patent-leather boot. But Hippolyte, not daring
to use such a handsome leg every day, begged Madame Bovary to get him
another more convenient one. The doctor, of course, had again to defray
the expense of this purchase.

So little by little the stable-man took up his work again. One saw him
running about the village as before, and when Charles heard from afar
the sharp noise of the wooden leg, he at once went in another direction.

It was Monsieur Lheureux, the shopkeeper, who had undertaken the order;
this provided him with an excuse for visiting Emma. He chatted with her
about the new goods from Paris, about a thousand feminine trifles, made
himself very obliging, and never asked for his money. Emma yielded to
this lazy mode of satisfying all her caprices. Thus she wanted to have
a very handsome ridding-whip that was at an umbrella-maker's at Rouen
to give to Rodolphe. The week after Monsieur Lheureux placed it on her
table.

But the next day he called on her with a bill for two hundred and
seventy francs, not counting the centimes. Emma was much embarrassed;
all the drawers of the writing-table were empty; they owed over a
fortnight's wages to Lestiboudois, two quarters to the servant, for any
quantity of other things, and Bovary was impatiently expecting Monsieur
Derozeray's account, which he was in the habit of paying every year
about Midsummer.

She succeeded at first in putting off Lheureux. At last he lost
patience; he was being sued; his capital was out, and unless he got some
in he should be forced to take back all the goods she had received.

"Oh, very well, take them!" said Emma.

"I was only joking," he replied; "the only thing I regret is the whip.
My word! I'll ask monsieur to return it to me."

"No, no!" she said.

"Ah! I've got you!" thought Lheureux.

And, certain of his discovery, he went out repeating to himself in an
undertone, and with his usual low whistle--

"Good! we shall see! we shall see!"

She was thinking how to get out of this when the servant coming in
put on the mantelpiece a small roll of blue paper "from Monsieur
Derozeray's." Emma pounced upon and opened it. It contained fifteen
napoleons; it was the account. She heard Charles on the stairs; threw
the gold to the back of her drawer, and took out the key.

Three days after Lheureux reappeared.

"I have an arrangement to suggest to you," he said. "If, instead of the
sum agreed on, you would take--"

"Here it is," she said placing fourteen napoleons in his hand.

The tradesman was dumfounded. Then, to conceal his disappointment, he
was profuse in apologies and proffers of service, all of which Emma
declined; then she remained a few moments fingering in the pocket of
her apron the two five-franc pieces that he had given her in change.
She promised herself she would economise in order to pay back later on.
"Pshaw!" she thought, "he won't think about it again."

Besides the riding-whip with its silver-gilt handle, Rodolphe had
received a seal with the motto Amor nel cor* furthermore, a scarf for
a muffler, and, finally, a cigar-case exactly like the Viscount's, that
Charles had formerly picked up in the road, and that Emma had kept.
These presents, however, humiliated him; he refused several; she
insisted, and he ended by obeying, thinking her tyrannical and
overexacting.

     *A loving heart.

Then she had strange ideas.

"When midnight strikes," she said, "you must think of me."

And if he confessed that he had not thought of her, there were floods of
reproaches that always ended with the eternal question--

"Do you love me?"

"Why, of course I love you," he answered.

"A great deal?"

"Certainly!"

"You haven't loved any others?"

"Did you think you'd got a virgin?" he exclaimed laughing.

Emma cried, and he tried to console her, adorning his protestations with
puns.

"Oh," she went on, "I love you! I love you so that I could not live
without you, do you see? There are times when I long to see you again,
when I am torn by all the anger of love. I ask myself, Where is
he? Perhaps he is talking to other women. They smile upon him; he
approaches. Oh no; no one else pleases you. There are some more
beautiful, but I love you best. I know how to love best. I am your
servant, your concubine! You are my king, my idol! You are good, you are
beautiful, you are clever, you are strong!"

He had so often heard these things said that they did not strike him as
original. Emma was like all his mistresses; and the charm of novelty,
gradually falling away like a garment, laid bare the eternal monotony
of passion, that has always the same forms and the same language. He
did not distinguish, this man of so much experience, the difference of
sentiment beneath the sameness of expression. Because lips libertine
and venal had murmured such words to him, he believed but little in the
candour of hers; exaggerated speeches hiding mediocre affections must be
discounted; as if the fullness of the soul did not sometimes overflow in
the emptiest metaphors, since no one can ever give the exact measure of
his needs, nor of his conceptions, nor of his sorrows; and since human
speech is like a cracked tin kettle, on which we hammer out tunes to
make bears dance when we long to move the stars.

But with that superior critical judgment that belongs to him who, in no
matter what circumstance, holds back, Rodolphe saw other delights to be
got out of this love. He thought all modesty in the way. He treated her
quite sans facon.* He made of her something supple and corrupt. Hers
was an idiotic sort of attachment, full of admiration for him, of
voluptuousness for her, a beatitude that benumbed her; her soul sank
into this drunkenness, shrivelled up, drowned in it, like Clarence in
his butt of Malmsey.

     *Off-handedly.


By the mere effect of her love Madame Bovary's manners changed.
Her looks grew bolder, her speech more free; she even committed the
impropriety of walking out with Monsieur Rodolphe, a cigarette in her
mouth, "as if to defy the people." At last, those who still doubted
doubted no longer when one day they saw her getting out of the
"Hirondelle," her waist squeezed into a waistcoat like a man; and Madame
Bovary senior, who, after a fearful scene with her husband, had taken
refuge at her son's, was not the least scandalised of the women-folk.
Many other things displeased her. First, Charles had not attended to
her advice about the forbidding of novels; then the "ways of the house"
annoyed her; she allowed herself to make some remarks, and there were
quarrels, especially one on account of Felicite.

Madame Bovary senior, the evening before, passing along the passage,
had surprised her in company of a man--a man with a brown collar, about
forty years old, who, at the sound of her step, had quickly escaped
through the kitchen. Then Emma began to laugh, but the good lady grew
angry, declaring that unless morals were to be laughed at one ought to
look after those of one's servants.

"Where were you brought up?" asked the daughter-in-law, with so
impertinent a look that Madame Bovary asked her if she were not perhaps
defending her own case.

"Leave the room!" said the young woman, springing up with a bound.

"Emma! Mamma!" cried Charles, trying to reconcile them.

But both had fled in their exasperation. Emma was stamping her feet as
she repeated--

"Oh! what manners! What a peasant!"

He ran to his mother; she was beside herself. She stammered

"She is an insolent, giddy-headed thing, or perhaps worse!"

And she was for leaving at once if the other did not apologise. So
Charles went back again to his wife and implored her to give way; he
knelt to her; she ended by saying--

"Very well! I'll go to her."

And in fact she held out her hand to her mother-in-law with the dignity
of a marchioness as she said--

"Excuse me, madame."

Then, having gone up again to her room, she threw herself flat on her
bed and cried there like a child, her face buried in the pillow.

She and Rodolphe had agreed that in the event of anything extraordinary
occurring, she should fasten a small piece of white paper to the blind,
so that if by chance he happened to be in Yonville, he could hurry to
the lane behind the house. Emma made the signal; she had been waiting
three-quarters of an hour when she suddenly caught sight of Rodolphe at
the corner of the market. She felt tempted to open the window and call
him, but he had already disappeared. She fell back in despair.

Soon, however, it seemed to her that someone was walking on the
pavement. It was he, no doubt. She went downstairs, crossed the yard. He
was there outside. She threw herself into his arms.

"Do take care!" he said.

"Ah! if you knew!" she replied.

And she began telling him everything, hurriedly, disjointedly,
exaggerating the facts, inventing many, and so prodigal of parentheses
that he understood nothing of it.

"Come, my poor angel, courage! Be comforted! be patient!"

"But I have been patient; I have suffered for four years. A love like
ours ought to show itself in the face of heaven. They torture me! I can
bear it no longer! Save me!"

She clung to Rodolphe. Her eyes, full of tears, flashed like flames
beneath a wave; her breast heaved; he had never loved her so much, so
that he lost his head and said "What is, it? What do you wish?"

"Take me away," she cried, "carry me off! Oh, I pray you!"

And she threw herself upon his mouth, as if to seize there the
unexpected consent if breathed forth in a kiss.

"But--" Rodolphe resumed.

"What?"

"Your little girl!"

She reflected a few moments, then replied--

"We will take her! It can't be helped!"

"What a woman!" he said to himself, watching her as she went. For she
had run into the garden. Someone was calling her.

On the following days Madame Bovary senior was much surprised at the
change in her daughter-in-law. Emma, in fact, was showing herself more
docile, and even carried her deference so far as to ask for a recipe for
pickling gherkins.

Was it the better to deceive them both? Or did she wish by a sort of
voluptuous stoicism to feel the more profoundly the bitterness of the
things she was about to leave?

But she paid no heed to them; on the contrary, she lived as lost in the
anticipated delight of her coming happiness.

It was an eternal subject for conversation with Rodolphe. She leant on
his shoulder murmuring--

"Ah! when we are in the mail-coach! Do you think about it? Can it be? It
seems to me that the moment I feel the carriage start, it will be as if
we were rising in a balloon, as if we were setting out for the clouds.
Do you know that I count the hours? And you?"

Never had Madame Bovary been so beautiful as at this period; she had
that indefinable beauty that results from joy, from enthusiasm, from
success, and that is only the harmony of temperament with circumstances.
Her desires, her sorrows, the experience of pleasure, and her ever-young
illusions, that had, as soil and rain and winds and the sun make flowers
grow, gradually developed her, and she at length blossomed forth in all
the plenitude of her nature. Her eyelids seemed chiselled expressly for
her long amorous looks in which the pupil disappeared, while a strong
inspiration expanded her delicate nostrils and raised the fleshy corner
of her lips, shaded in the light by a little black down. One would have
thought that an artist apt in conception had arranged the curls of hair
upon her neck; they fell in a thick mass, negligently, and with the
changing chances of their adultery, that unbound them every day. Her
voice now took more mellow infections, her figure also; something subtle
and penetrating escaped even from the folds of her gown and from the
line of her foot. Charles, as when they were first married, thought her
delicious and quite irresistible.

When he came home in the middle of the night, he did not dare to wake
her. The porcelain night-light threw a round trembling gleam upon the
ceiling, and the drawn curtains of the little cot formed as it were a
white hut standing out in the shade, and by the bedside Charles looked
at them. He seemed to hear the light breathing of his child. She would
grow big now; every season would bring rapid progress. He already saw
her coming from school as the day drew in, laughing, with ink-stains on
her jacket, and carrying her basket on her arm. Then she would have to
be sent to the boarding-school; that would cost much; how was it to
be done? Then he reflected. He thought of hiring a small farm in the
neighbourhood, that he would superintend every morning on his way to his
patients. He would save up what he brought in; he would put it in the
savings-bank. Then he would buy shares somewhere, no matter where;
besides, his practice would increase; he counted upon that, for he
wanted Berthe to be well-educated, to be accomplished, to learn to play
the piano. Ah! how pretty she would be later on when she was fifteen,
when, resembling her mother, she would, like her, wear large straw hats
in the summer-time; from a distance they would be taken for two sisters.
He pictured her to himself working in the evening by their side beneath
the light of the lamp; she would embroider him slippers; she would look
after the house; she would fill all the home with her charm and her
gaiety. At last, they would think of her marriage; they would find her
some good young fellow with a steady business; he would make her happy;
this would last for ever.

Emma was not asleep; she pretended to be; and while he dozed off by her
side she awakened to other dreams.

To the gallop of four horses she was carried away for a week towards a
new land, whence they would return no more. They went on and on, their
arms entwined, without a word. Often from the top of a mountain there
suddenly glimpsed some splendid city with domes, and bridges, and
ships, forests of citron trees, and cathedrals of white marble, on whose
pointed steeples were storks' nests. They went at a walking-pace because
of the great flag-stones, and on the ground there were bouquets of
flowers, offered you by women dressed in red bodices. They heard the
chiming of bells, the neighing of mules, together with the murmur of
guitars and the noise of fountains, whose rising spray refreshed heaps
of fruit arranged like a pyramid at the foot of pale statues that smiled
beneath playing waters. And then, one night they came to a fishing
village, where brown nets were drying in the wind along the cliffs and
in front of the huts. It was there that they would stay; they would live
in a low, flat-roofed house, shaded by a palm-tree, in the heart of a
gulf, by the sea. They would row in gondolas, swing in hammocks, and
their existence would be easy and large as their silk gowns, warm and
star-spangled as the nights they would contemplate. However, in the
immensity of this future that she conjured up, nothing special stood
forth; the days, all magnificent, resembled each other like waves; and
it swayed in the horizon, infinite, harmonised, azure, and bathed in
sunshine. But the child began to cough in her cot or Bovary snored
more loudly, and Emma did not fall asleep till morning, when the dawn
whitened the windows, and when little Justin was already in the square
taking down the shutters of the chemist's shop.

She had sent for Monsieur Lheureux, and had said to him--

"I want a cloak--a large lined cloak with a deep collar."

"You are going on a journey?" he asked.

"No; but--never mind. I may count on you, may I not, and quickly?"

He bowed.

"Besides, I shall want," she went on, "a trunk--not too heavy--handy."

"Yes, yes, I understand. About three feet by a foot and a half, as they
are being made just now."

"And a travelling bag."

"Decidedly," thought Lheureux, "there's a row on here."

"And," said Madame Bovary, taking her watch from her belt, "take this;
you can pay yourself out of it."

But the tradesman cried out that she was wrong; they knew one another;
did he doubt her? What childishness!

She insisted, however, on his taking at least the chain, and Lheureux
had already put it in his pocket and was going, when she called him
back.

"You will leave everything at your place. As to the cloak"--she seemed
to be reflecting--"do not bring it either; you can give me the maker's
address, and tell him to have it ready for me."

It was the next month that they were to run away. She was to leave
Yonville as if she was going on some business to Rouen. Rodolphe would
have booked the seats, procured the passports, and even have written to
Paris in order to have the whole mail-coach reserved for them as far as
Marseilles, where they would buy a carriage, and go on thence without
stopping to Genoa. She would take care to send her luggage to Lheureux
whence it would be taken direct to the "Hirondelle," so that no one
would have any suspicion. And in all this there never was any allusion
to the child. Rodolphe avoided speaking of her; perhaps he no longer
thought about it.

He wished to have two more weeks before him to arrange some affairs;
then at the end of a week he wanted two more; then he said he was ill;
next he went on a journey. The month of August passed, and, after all
these delays, they decided that it was to be irrevocably fixed for the
4th September--a Monday.

At length the Saturday before arrived.

Rodolphe came in the evening earlier than usual.

"Everything is ready?" she asked him.

"Yes."

Then they walked round a garden-bed, and went to sit down near the
terrace on the kerb-stone of the wall.

"You are sad," said Emma.

"No; why?"

And yet he looked at her strangely in a tender fashion.

"It is because you are going away?" she went on; "because you are
leaving what is dear to you--your life? Ah! I understand. I have nothing
in the world! you are all to me; so shall I be to you. I will be your
people, your country; I will tend, I will love you!"

"How sweet you are!" he said, seizing her in his arms.

"Really!" she said with a voluptuous laugh. "Do you love me? Swear it
then!"

"Do I love you--love you? I adore you, my love."

The moon, full and purple-coloured, was rising right out of the earth
at the end of the meadow. She rose quickly between the branches of the
poplars, that hid her here and there like a black curtain pierced with
holes. Then she appeared dazzling with whiteness in the empty heavens
that she lit up, and now sailing more slowly along, let fall upon the
river a great stain that broke up into an infinity of stars; and the
silver sheen seemed to writhe through the very depths like a heedless
serpent covered with luminous scales; it also resembled some monster
candelabra all along which sparkled drops of diamonds running together.
The soft night was about them; masses of shadow filled the branches.
Emma, her eyes half closed, breathed in with deep sighs the fresh wind
that was blowing. They did not speak, lost as they were in the rush of
their reverie. The tenderness of the old days came back to their hearts,
full and silent as the flowing river, with the softness of the perfume
of the syringas, and threw across their memories shadows more immense
and more sombre than those of the still willows that lengthened out over
the grass. Often some night-animal, hedgehog or weasel, setting out on
the hunt, disturbed the lovers, or sometimes they heard a ripe peach
falling all alone from the espalier.

"Ah! what a lovely night!" said Rodolphe.

"We shall have others," replied Emma; and, as if speaking to herself:
"Yet, it will be good to travel. And yet, why should my heart be
so heavy? Is it dread of the unknown? The effect of habits left? Or
rather--? No; it is the excess of happiness. How weak I am, am I not?
Forgive me!"

"There is still time!" he cried. "Reflect! perhaps you may repent!"

"Never!" she cried impetuously. And coming closer to him: "What ill
could come to me? There is no desert, no precipice, no ocean I would not
traverse with you. The longer we live together the more it will be like
an embrace, every day closer, more heart to heart. There will be
nothing to trouble us, no cares, no obstacle. We shall be alone, all to
ourselves eternally. Oh, speak! Answer me!"

At regular intervals he answered, "Yes--Yes--" She had passed her hands
through his hair, and she repeated in a childlike voice, despite the big
tears which were falling, "Rodolphe! Rodolphe! Ah! Rodolphe! dear little
Rodolphe!"

Midnight struck.

"Midnight!" said she. "Come, it is to-morrow. One day more!"

He rose to go; and as if the movement he made had been the signal for
their flight, Emma said, suddenly assuming a gay air--

"You have the passports?"

"Yes."

"You are forgetting nothing?"

"No."

"Are you sure?"

"Certainly."

"It is at the Hotel de Provence, is it not, that you will wait for me at
midday?"

He nodded.

"Till to-morrow then!" said Emma in a last caress; and she watched him
go.

He did not turn round. She ran after him, and, leaning over the water's
edge between the bulrushes--

"To-morrow!" she cried.

He was already on the other side of the river and walking fast across
the meadow.

After a few moments Rodolphe stopped; and when he saw her with her white
gown gradually fade away in the shade like a ghost, he was seized with
such a beating of the heart that he leant against a tree lest he should
fall.

"What an imbecile I am!" he said with a fearful oath. "No matter! She
was a pretty mistress!"

And immediately Emma's beauty, with all the pleasures of their love,
came back to him. For a moment he softened; then he rebelled against
her.

"For, after all," he exclaimed, gesticulating, "I can't exile
myself--have a child on my hands."

He was saying these things to give himself firmness.

"And besides, the worry, the expense! Ah! no, no, no, no! a thousand
times no! That would be too stupid."



\chapter{Chapter Thirteen}

No sooner was Rodolphe at home than he sat down quickly at his bureau
under the stag's head that hung as a trophy on the wall. But when he had
the pen between his fingers, he could think of nothing, so that, resting
on his elbows, he began to reflect. Emma seemed to him to have receded
into a far-off past, as if the resolution he had taken had suddenly
placed a distance between them.

To get back something of her, he fetched from the cupboard at the
bedside an old Rheims biscuit-box, in which he usually kept his letters
from women, and from it came an odour of dry dust and withered
roses. First he saw a handkerchief with pale little spots. It was a
handkerchief of hers. Once when they were walking her nose had bled; he
had forgotten it. Near it, chipped at all the corners, was a miniature
given him by Emma: her toilette seemed to him pretentious, and her
languishing look in the worst possible taste. Then, from looking at this
image and recalling the memory of its original, Emma's features little
by little grew confused in his remembrance, as if the living and the
painted face, rubbing one against the other, had effaced each other.
Finally, he read some of her letters; they were full of explanations
relating to their journey, short, technical, and urgent, like business
notes. He wanted to see the long ones again, those of old times. In
order to find them at the bottom of the box, Rodolphe disturbed all the
others, and mechanically began rummaging amidst this mass of papers and
things, finding pell-mell bouquets, garters, a black mask, pins, and
hair--hair! dark and fair, some even, catching in the hinges of the box,
broke when it was opened.

Thus dallying with his souvenirs, he examined the writing and the style
of the letters, as varied as their orthography. They were tender or
jovial, facetious, melancholy; there were some that asked for love,
others that asked for money. A word recalled faces to him, certain
gestures, the sound of a voice; sometimes, however, he remembered
nothing at all.

In fact, these women, rushing at once into his thoughts, cramped each
other and lessened, as reduced to a uniform level of love that equalised
them all. So taking handfuls of the mixed-up letters, he amused himself
for some moments with letting them fall in cascades from his right into
his left hand. At last, bored and weary, Rodolphe took back the box to
the cupboard, saying to himself, "What a lot of rubbish!" Which summed
up his opinion; for pleasures, like schoolboys in a school courtyard,
had so trampled upon his heart that no green thing grew there, and that
which passed through it, more heedless than children, did not even, like
them, leave a name carved upon the wall.

"Come," said he, "let's begin."

He wrote--

"Courage, Emma! courage! I would not bring misery into your life."

"After all, that's true," thought Rodolphe. "I am acting in her
interest; I am honest."

"Have you carefully weighed your resolution? Do you know to what an
abyss I was dragging you, poor angel? No, you do not, do you? You were
coming confident and fearless, believing in happiness in the future. Ah!
unhappy that we are--insensate!"

Rodolphe stopped here to think of some good excuse.

"If I told her all my fortune is lost? No! Besides, that would stop
nothing. It would all have to be begun over again later on. As if one
could make women like that listen to reason!" He reflected, then went
on--

"I shall not forget you, oh believe it; and I shall ever have a profound
devotion for you; but some day, sooner or later, this ardour (such is
the fate of human things) would have grown less, no doubt. Lassitude
would have come to us, and who knows if I should not even have had the
atrocious pain of witnessing your remorse, of sharing it myself, since
I should have been its cause? The mere idea of the grief that would come
to you tortures me, Emma. Forget me! Why did I ever know you? Why were
you so beautiful? Is it my fault? O my God! No, no! Accuse only fate."

"That's a word that always tells," he said to himself.

"Ah, if you had been one of those frivolous women that one sees,
certainly I might, through egotism, have tried an experiment, in that
case without danger for you. But that delicious exaltation, at once your
charm and your torment, has prevented you from understanding, adorable
woman that you are, the falseness of our future position. Nor had I
reflected upon this at first, and I rested in the shade of that ideal
happiness as beneath that of the manchineel tree, without foreseeing the
consequences."

"Perhaps she'll think I'm giving it up from avarice. Ah, well! so much
the worse; it must be stopped!"

"The world is cruel, Emma. Wherever we might have gone, it would have
persecuted us. You would have had to put up with indiscreet questions,
calumny, contempt, insult perhaps. Insult to you! Oh! And I, who would
place you on a throne! I who bear with me your memory as a talisman! For
I am going to punish myself by exile for all the ill I have done you.
I am going away. Whither I know not. I am mad. Adieu! Be good always.
Preserve the memory of the unfortunate who has lost you. Teach my name
to your child; let her repeat it in her prayers."

The wicks of the candles flickered. Rodolphe got up to, shut the window,
and when he had sat down again--

"I think it's all right. Ah! and this for fear she should come and hunt
me up."

"I shall be far away when you read these sad lines, for I have wished to
flee as quickly as possible to shun the temptation of seeing you again.
No weakness! I shall return, and perhaps later on we shall talk together
very coldly of our old love. Adieu!"

And there was a last "adieu" divided into two words! "A Dieu!" which he
thought in very excellent taste.

"Now how am I to sign?" he said to himself. "'Yours devotedly?' No!
'Your friend?' Yes, that's it."

"Your friend."

He re-read his letter. He considered it very good.

"Poor little woman!" he thought with emotion. "She'll think me harder
than a rock. There ought to have been some tears on this; but I can't
cry; it isn't my fault." Then, having emptied some water into a glass,
Rodolphe dipped his finger into it, and let a big drop fall on the
paper, that made a pale stain on the ink. Then looking for a seal, he
came upon the one "Amor nel cor."

"That doesn't at all fit in with the circumstances. Pshaw! never mind!"

After which he smoked three pipes and went to bed.

The next day when he was up (at about two o'clock--he had slept late),
Rodolphe had a basket of apricots picked. He put his letter at
the bottom under some vine leaves, and at once ordered Girard, his
ploughman, to take it with care to Madame Bovary. He made use of this
means for corresponding with her, sending according to the season fruits
or game.

"If she asks after me," he said, "you will tell her that I have gone on
a journey. You must give the basket to her herself, into her own hands.
Get along and take care!"

Girard put on his new blouse, knotted his handkerchief round the
apricots, and walking with great heavy steps in his thick iron-bound
galoshes, made his way to Yonville.

Madame Bovary, when he got to her house, was arranging a bundle of linen
on the kitchen-table with Felicite.

"Here," said the ploughboy, "is something for you--from the master."

She was seized with apprehension, and as she sought in her pocket for
some coppers, she looked at the peasant with haggard eyes, while he
himself looked at her with amazement, not understanding how such a
present could so move anyone. At last he went out. Felicite remained.
She could bear it no longer; she ran into the sitting room as if to take
the apricots there, overturned the basket, tore away the leaves, found
the letter, opened it, and, as if some fearful fire were behind her,
Emma flew to her room terrified.

Charles was there; she saw him; he spoke to her; she heard nothing, and
she went on quickly up the stairs, breathless, distraught, dumb, and
ever holding this horrible piece of paper, that crackled between her
fingers like a plate of sheet-iron. On the second floor she stopped
before the attic door, which was closed.

Then she tried to calm herself; she recalled the letter; she must finish
it; she did not dare to. And where? How? She would be seen! "Ah, no!
here," she thought, "I shall be all right."

Emma pushed open the door and went in.

The slates threw straight down a heavy heat that gripped her temples,
stifled her; she dragged herself to the closed garret-window. She drew
back the bolt, and the dazzling light burst in with a leap.

Opposite, beyond the roofs, stretched the open country till it was lost
to sight. Down below, underneath her, the village square was empty; the
stones of the pavement glittered, the weathercocks on the houses were
motionless. At the corner of the street, from a lower storey, rose a
kind of humming with strident modulations. It was Binet turning.

She leant against the embrasure of the window, and reread the letter
with angry sneers. But the more she fixed her attention upon it, the
more confused were her ideas. She saw him again, heard him, encircled
him with her arms, and throbs of her heart, that beat against her breast
like blows of a sledge-hammer, grew faster and faster, with uneven
intervals. She looked about her with the wish that the earth might
crumble into pieces. Why not end it all? What restrained her? She was
free. She advanced, looking at the paving-stones, saying to herself,
"Come! come!"

The luminous ray that came straight up from below drew the weight of
her body towards the abyss. It seemed to her that the ground of the
oscillating square went up the walls and that the floor dipped on
end like a tossing boat. She was right at the edge, almost hanging,
surrounded by vast space. The blue of the heavens suffused her, the air
was whirling in her hollow head; she had but to yield, to let herself
be taken; and the humming of the lathe never ceased, like an angry voice
calling her.

"Emma! Emma!" cried Charles.

She stopped.

"Wherever are you? Come!"

The thought that she had just escaped from death almost made her faint
with terror. She closed her eyes; then she shivered at the touch of a
hand on her sleeve; it was Felicite.

"Master is waiting for you, madame; the soup is on the table."

And she had to go down to sit at table.

She tried to eat. The food choked her. Then she unfolded her napkin as
if to examine the darns, and she really thought of applying herself to
this work, counting the threads in the linen. Suddenly the remembrance
of the letter returned to her. How had she lost it? Where could she find
it? But she felt such weariness of spirit that she could not even invent
a pretext for leaving the table. Then she became a coward; she was
afraid of Charles; he knew all, that was certain! Indeed he pronounced
these words in a strange manner:

"We are not likely to see Monsieur Rodolphe soon again, it seems."

"Who told you?" she said, shuddering.

"Who told me!" he replied, rather astonished at her abrupt tone. "Why,
Girard, whom I met just now at the door of the Cafe Francais. He has
gone on a journey, or is to go."

She gave a sob.

"What surprises you in that? He absents himself like that from time
to time for a change, and, ma foi, I think he's right, when one has a
fortune and is a bachelor. Besides, he has jolly times, has our friend.
He's a bit of a rake. Monsieur Langlois told me--"

He stopped for propriety's sake because the servant came in. She put
back into the basket the apricots scattered on the sideboard. Charles,
without noticing his wife's colour, had them brought to him, took one,
and bit into it.

"Ah! perfect!" said he; "just taste!"

And he handed her the basket, which she put away from her gently.

"Do just smell! What an odour!" he remarked, passing it under her nose
several times.

"I am choking," she cried, leaping up. But by an effort of will the
spasm passed; then--

"It is nothing," she said, "it is nothing! It is nervousness. Sit down
and go on eating." For she dreaded lest he should begin questioning her,
attending to her, that she should not be left alone.

Charles, to obey her, sat down again, and he spat the stones of the
apricots into his hands, afterwards putting them on his plate.

Suddenly a blue tilbury passed across the square at a rapid trot. Emma
uttered a cry and fell back rigid to the ground.

In fact, Rodolphe, after many reflections, had decided to set out for
Rouen. Now, as from La Huchette to Buchy there is no other way than by
Yonville, he had to go through the village, and Emma had recognised him
by the rays of the lanterns, which like lightning flashed through the
twilight.

The chemist, at the tumult which broke out in the house ran thither. The
table with all the plates was upset; sauce, meat, knives, the salt, and
cruet-stand were strewn over the room; Charles was calling for help;
Berthe, scared, was crying; and Felicite, whose hands trembled, was
unlacing her mistress, whose whole body shivered convulsively.

"I'll run to my laboratory for some aromatic vinegar," said the
druggist.

Then as she opened her eyes on smelling the bottle--

"I was sure of it," he remarked; "that would wake any dead person for
you!"

"Speak to us," said Charles; "collect yourself; it is your Charles, who
loves you. Do you know me? See! here is your little girl! Oh, kiss her!"

The child stretched out her arms to her mother to cling to her neck. But
turning away her head, Emma said in a broken voice "No, no! no one!"

She fainted again. They carried her to her bed. She lay there stretched
at full length, her lips apart, her eyelids closed, her hands open,
motionless, and white as a waxen image. Two streams of tears flowed from
her eyes and fell slowly upon the pillow.

Charles, standing up, was at the back of the alcove, and the chemist,
near him, maintained that meditative silence that is becoming on the
serious occasions of life.

"Do not be uneasy," he said, touching his elbow; "I think the paroxysm
is past."

"Yes, she is resting a little now," answered Charles, watching her
sleep. "Poor girl! poor girl! She had gone off now!"

Then Homais asked how the accident had come about. Charles answered that
she had been taken ill suddenly while she was eating some apricots.

"Extraordinary!" continued the chemist. "But it might be that the
apricots had brought on the syncope. Some natures are so sensitive to
certain smells; and it would even be a very fine question to study both
in its pathological and physiological relation. The priests know the
importance of it, they who have introduced aromatics into all their
ceremonies. It is to stupefy the senses and to bring on ecstasies--a
thing, moreover, very easy in persons of the weaker sex, who are more
delicate than the other. Some are cited who faint at the smell of burnt
hartshorn, of new bread--"

"Take care; you'll wake her!" said Bovary in a low voice.

"And not only," the druggist went on, "are human beings subject to such
anomalies, but animals also. Thus you are not ignorant of the singularly
aphrodisiac effect produced by the Nepeta cataria, vulgarly called
catmint, on the feline race; and, on the other hand, to quote an example
whose authenticity I can answer for. Bridaux (one of my old comrades, at
present established in the Rue Malpalu) possesses a dog that falls into
convulsions as soon as you hold out a snuff-box to him. He often even
makes the experiment before his friends at his summer-house at Guillaume
Wood. Would anyone believe that a simple sternutation could produce such
ravages on a quadrupedal organism? It is extremely curious, is it not?"

"Yes," said Charles, who was not listening to him.

"This shows us," went on the other, smiling with benign
self-sufficiency, "the innumerable irregularities of the nervous system.
With regard to madame, she has always seemed to me, I confess, very
susceptible. And so I should by no means recommend to you, my dear
friend, any of those so-called remedies that, under the pretence
of attacking the symptoms, attack the constitution. No; no useless
physicking! Diet, that is all; sedatives, emollients, dulcification.
Then, don't you think that perhaps her imagination should be worked
upon?"

"In what way? How?" said Bovary.

"Ah! that is it. Such is indeed the question. 'That is the question,' as
I lately read in a newspaper."

But Emma, awaking, cried out--

"The letter! the letter!"

They thought she was delirious; and she was by midnight. Brain-fever had
set in.

For forty-three days Charles did not leave her. He gave up all his
patients; he no longer went to bed; he was constantly feeling her pulse,
putting on sinapisms and cold-water compresses. He sent Justin as far as
Neufchatel for ice; the ice melted on the way; he sent him back again.
He called Monsieur Canivet into consultation; he sent for Dr. Lariviere,
his old master, from Rouen; he was in despair. What alarmed him most was
Emma's prostration, for she did not speak, did not listen, did not even
seem to suffer, as if her body and soul were both resting together after
all their troubles.

About the middle of October she could sit up in bed supported by
pillows. Charles wept when he saw her eat her first bread-and-jelly. Her
strength returned to her; she got up for a few hours of an afternoon,
and one day, when she felt better, he tried to take her, leaning on his
arm, for a walk round the garden. The sand of the paths was disappearing
beneath the dead leaves; she walked slowly, dragging along her slippers,
and leaning against Charles's shoulder. She smiled all the time.

They went thus to the bottom of the garden near the terrace. She drew
herself up slowly, shading her eyes with her hand to look. She looked
far off, as far as she could, but on the horizon were only great
bonfires of grass smoking on the hills.

"You will tire yourself, my darling!" said Bovary. And, pushing her
gently to make her go into the arbour, "Sit down on this seat; you'll be
comfortable."

"Oh! no; not there!" she said in a faltering voice.

She was seized with giddiness, and from that evening her illness
recommenced, with a more uncertain character, it is true, and more
complex symptoms. Now she suffered in her heart, then in the chest, the
head, the limbs; she had vomitings, in which Charles thought he saw the
first signs of cancer.

And besides this, the poor fellow was worried about money matters.



\chapter{Chapter Fourteen}

To begin with, he did not know how he could pay Monsieur Homais for all
the physic supplied by him, and though, as a medical man, he was not
obliged to pay for it, he nevertheless blushed a little at such an
obligation. Then the expenses of the household, now that the servant was
mistress, became terrible. Bills rained in upon the house; the tradesmen
grumbled; Monsieur Lheureux especially harassed him. In fact, at
the height of Emma's illness, the latter, taking advantage of the
circumstances to make his bill larger, had hurriedly brought the cloak,
the travelling-bag, two trunks instead of one, and a number of other
things. It was very well for Charles to say he did not want them. The
tradesman answered arrogantly that these articles had been ordered, and
that he would not take them back; besides, it would vex madame in her
convalescence; the doctor had better think it over; in short, he was
resolved to sue him rather than give up his rights and take back his
goods. Charles subsequently ordered them to be sent back to the shop.
Felicite forgot; he had other things to attend to; then thought no more
about them. Monsieur Lheureux returned to the charge, and, by turns
threatening and whining, so managed that Bovary ended by signing a
bill at six months. But hardly had he signed this bill than a bold idea
occurred to him: it was to borrow a thousand francs from Lheureux.
So, with an embarrassed air, he asked if it were possible to get them,
adding that it would be for a year, at any interest he wished. Lheureux
ran off to his shop, brought back the money, and dictated another bill,
by which Bovary undertook to pay to his order on the 1st of September
next the sum of one thousand and seventy francs, which, with the hundred
and eighty already agreed to, made just twelve hundred and fifty, thus
lending at six per cent in addition to one-fourth for commission: and
the things bringing him in a good third at the least, this ought in
twelve months to give him a profit of a hundred and thirty francs. He
hoped that the business would not stop there; that the bills would not
be paid; that they would be renewed; and that his poor little money,
having thriven at the doctor's as at a hospital, would come back to him
one day considerably more plump, and fat enough to burst his bag.

Everything, moreover, succeeded with him. He was adjudicator for a
supply of cider to the hospital at Neufchatel; Monsieur Guillaumin
promised him some shares in the turf-pits of Gaumesnil, and he dreamt of
establishing a new diligence service between Arcueil and Rouen, which
no doubt would not be long in ruining the ramshackle van of the "Lion
d'Or," and that, travelling faster, at a cheaper rate, and carrying more
luggage, would thus put into his hands the whole commerce of Yonville.

Charles several times asked himself by what means he should next year be
able to pay back so much money. He reflected, imagined expedients, such
as applying to his father or selling something. But his father would be
deaf, and he--he had nothing to sell. Then he foresaw such worries that
he quickly dismissed so disagreeable a subject of meditation from
his mind. He reproached himself with forgetting Emma, as if, all his
thoughts belonging to this woman, it was robbing her of something not to
be constantly thinking of her.

The winter was severe, Madame Bovary's convalescence slow. When it
was fine they wheeled her arm-chair to the window that overlooked the
square, for she now had an antipathy to the garden, and the blinds on
that side were always down. She wished the horse to be sold; what she
formerly liked now displeased her. All her ideas seemed to be limited to
the care of herself. She stayed in bed taking little meals, rang for the
servant to inquire about her gruel or to chat with her. The snow on
the market-roof threw a white, still light into the room; then the rain
began to fall; and Emma waited daily with a mind full of eagerness for
the inevitable return of some trifling events which nevertheless had no
relation to her. The most important was the arrival of the "Hirondelle"
in the evening. Then the landlady shouted out, and other voices
answered, while Hippolyte's lantern, as he fetched the boxes from the
boot, was like a star in the darkness. At mid-day Charles came in;
then he went out again; next she took some beef-tea, and towards five
o'clock, as the day drew in, the children coming back from school,
dragging their wooden shoes along the pavement, knocked the clapper of
the shutters with their rulers one after the other.

It was at this hour that Monsieur Bournisien came to see her. He
inquired after her health, gave her news, exhorted her to religion, in a
coaxing little prattle that was not without its charm. The mere thought
of his cassock comforted her.

One day, when at the height of her illness, she had thought herself
dying, and had asked for the communion; and, while they were making the
preparations in her room for the sacrament, while they were turning the
night table covered with syrups into an altar, and while Felicite was
strewing dahlia flowers on the floor, Emma felt some power passing
over her that freed her from her pains, from all perception, from
all feeling. Her body, relieved, no longer thought; another life was
beginning; it seemed to her that her being, mounting toward God, would
be annihilated in that love like a burning incense that melts into
vapour. The bed-clothes were sprinkled with holy water, the priest drew
from the holy pyx the white wafer; and it was fainting with a celestial
joy that she put out her lips to accept the body of the Saviour
presented to her. The curtains of the alcove floated gently round her
like clouds, and the rays of the two tapers burning on the night-table
seemed to shine like dazzling halos. Then she let her head fall back,
fancying she heard in space the music of seraphic harps, and perceived
in an azure sky, on a golden throne in the midst of saints holding green
palms, God the Father, resplendent with majesty, who with a sign sent to
earth angels with wings of fire to carry her away in their arms.

This splendid vision dwelt in her memory as the most beautiful thing
that it was possible to dream, so that now she strove to recall her
sensation. That still lasted, however, but in a less exclusive fashion
and with a deeper sweetness. Her soul, tortured by pride, at length
found rest in Christian humility, and, tasting the joy of weakness, she
saw within herself the destruction of her will, that must have left a
wide entrance for the inroads of heavenly grace. There existed, then,
in the place of happiness, still greater joys--another love beyond all
loves, without pause and without end, one that would grow eternally! She
saw amid the illusions of her hope a state of purity floating above the
earth mingling with heaven, to which she aspired. She wanted to become
a saint. She bought chaplets and wore amulets; she wished to have in her
room, by the side of her bed, a reliquary set in emeralds that she might
kiss it every evening.

The cure marvelled at this humour, although Emma's religion, he thought,
might, from its fervour, end by touching on heresy, extravagance. But
not being much versed in these matters, as soon as they went beyond a
certain limit he wrote to Monsieur Boulard, bookseller to Monsignor,
to send him "something good for a lady who was very clever." The
bookseller, with as much indifference as if he had been sending off
hardware to niggers, packed up, pellmell, everything that was then the
fashion in the pious book trade. There were little manuals in questions
and answers, pamphlets of aggressive tone after the manner of Monsieur
de Maistre, and certain novels in rose-coloured bindings and with
a honied style, manufactured by troubadour seminarists or penitent
blue-stockings. There were the "Think of it; the Man of the World at
Mary's Feet, by Monsieur de ***, decorated with many Orders"; "The
Errors of Voltaire, for the Use of the Young," etc.

Madame Bovary's mind was not yet sufficiently clear to apply herself
seriously to anything; moreover, she began this reading in too much
hurry. She grew provoked at the doctrines of religion; the arrogance
of the polemic writings displeased her by their inveteracy in attacking
people she did not know; and the secular stories, relieved with
religion, seemed to her written in such ignorance of the world, that
they insensibly estranged her from the truths for whose proof she was
looking. Nevertheless, she persevered; and when the volume slipped
from her hands, she fancied herself seized with the finest Catholic
melancholy that an ethereal soul could conceive.

As for the memory of Rodolphe, she had thrust it back to the bottom of
her heart, and it remained there more solemn and more motionless than
a king's mummy in a catacomb. An exhalation escaped from this embalmed
love, that, penetrating through everything, perfumed with tenderness the
immaculate atmosphere in which she longed to live. When she knelt on her
Gothic prie-Dieu, she addressed to the Lord the same suave words that
she had murmured formerly to her lover in the outpourings of adultery.
It was to make faith come; but no delights descended from the heavens,
and she arose with tired limbs and with a vague feeling of a gigantic
dupery.

This searching after faith, she thought, was only one merit the more,
and in the pride of her devoutness Emma compared herself to those grand
ladies of long ago whose glory she, had dreamed of over a portrait of La
Valliere, and who, trailing with so much majesty the lace-trimmed trains
of their long gowns, retired into solitudes to shed at the feet of
Christ all the tears of hearts that life had wounded.

Then she gave herself up to excessive charity. She sewed clothes for the
poor, she sent wood to women in childbed; and Charles one day, on coming
home, found three good-for-nothings in the kitchen seated at the table
eating soup. She had her little girl, whom during her illness her
husband had sent back to the nurse, brought home. She wanted to teach
her to read; even when Berthe cried, she was not vexed. She had made
up her mind to resignation, to universal indulgence. Her language about
everything was full of ideal expressions. She said to her child, "Is
your stomach-ache better, my angel?"

Madame Bovary senior found nothing to censure except perhaps this mania
of knitting jackets for orphans instead of mending her own house-linen;
but, harassed with domestic quarrels, the good woman took pleasure in
this quiet house, and she even stayed there till after Easter, to escape
the sarcasms of old Bovary, who never failed on Good Friday to order
chitterlings.

Besides the companionship of her mother-in-law, who strengthened her a
little by the rectitude of her judgment and her grave ways, Emma almost
every day had other visitors. These were Madame Langlois, Madame Caron,
Madame Dubreuil, Madame Tuvache, and regularly from two to five o'clock
the excellent Madame Homais, who, for her part, had never believed any
of the tittle-tattle about her neighbour. The little Homais also came to
see her; Justin accompanied them. He went up with them to her bedroom,
and remained standing near the door, motionless and mute. Often even
Madame Bovary; taking no heed of him, began her toilette. She began by
taking out her comb, shaking her head with a quick movement, and when
he for the first time saw all this mass of hair that fell to her knees
unrolling in black ringlets, it was to him, poor child! like a sudden
entrance into something new and strange, whose splendour terrified him.

Emma, no doubt, did not notice his silent attentions or his timidity.
She had no suspicion that the love vanished from her life was there,
palpitating by her side, beneath that coarse holland shirt, in that
youthful heart open to the emanations of her beauty. Besides, she
now enveloped all things with such indifference, she had words so
affectionate with looks so haughty, such contradictory ways, that one
could no longer distinguish egotism from charity, or corruption from
virtue. One evening, for example, she was angry with the servant, who
had asked to go out, and stammered as she tried to find some pretext.
Then suddenly--

"So you love him?" she said.

And without waiting for any answer from Felicite, who was blushing, she
added, "There! run along; enjoy yourself!"

In the beginning of spring she had the garden turned up from end to end,
despite Bovary's remonstrances. However, he was glad to see her at last
manifest a wish of any kind. As she grew stronger she displayed more
wilfulness. First, she found occasion to expel Mere Rollet, the nurse,
who during her convalescence had contracted the habit of coming too
often to the kitchen with her two nurslings and her boarder, better
off for teeth than a cannibal. Then she got rid of the Homais family,
successively dismissed all the other visitors, and even frequented
church less assiduously, to the great approval of the druggist, who said
to her in a friendly way--

"You were going in a bit for the cassock!"

As formerly, Monsieur Bournisien dropped in every day when he came out
after catechism class. He preferred staying out of doors to taking the
air "in the grove," as he called the arbour. This was the time when
Charles came home. They were hot; some sweet cider was brought out, and
they drank together to madame's complete restoration.

Binet was there; that is to say, a little lower down against the terrace
wall, fishing for crayfish. Bovary invited him to have a drink, and he
thoroughly understood the uncorking of the stone bottles.

"You must," he said, throwing a satisfied glance all round him, even to
the very extremity of the landscape, "hold the bottle perpendicularly on
the table, and after the strings are cut, press up the cork with
little thrusts, gently, gently, as indeed they do seltzer-water at
restaurants."

But during his demonstration the cider often spurted right into their
faces, and then the ecclesiastic, with a thick laugh, never missed this
joke--

"Its goodness strikes the eye!"

He was, in fact, a good fellow and one day he was not even scandalised
at the chemist, who advised Charles to give madame some distraction
by taking her to the theatre at Rouen to hear the illustrious tenor,
Lagardy. Homais, surprised at this silence, wanted to know his opinion,
and the priest declared that he considered music less dangerous for
morals than literature.

But the chemist took up the defence of letters. The theatre, he
contended, served for railing at prejudices, and, beneath a mask of
pleasure, taught virtue.

"'Castigat ridendo mores,'* Monsieur Bournisien! Thus consider the
greater part of Voltaire's tragedies; they are cleverly strewn with
philosophical reflections, that made them a vast school of morals and
diplomacy for the people."

     *It corrects customs through laughter.


"I," said Binet, "once saw a piece called the 'Gamin de Paris,' in which
there was the character of an old general that is really hit off to a
T. He sets down a young swell who had seduced a working girl, who at the
ending--"

"Certainly," continued Homais, "there is bad literature as there is bad
pharmacy, but to condemn in a lump the most important of the fine arts
seems to me a stupidity, a Gothic idea, worthy of the abominable times
that imprisoned Galileo."

"I know very well," objected the cure, "that there are good works,
good authors. However, if it were only those persons of different sexes
united in a bewitching apartment, decorated rouge, those lights, those
effeminate voices, all this must, in the long-run, engender a
certain mental libertinage, give rise to immodest thoughts and impure
temptations. Such, at any rate, is the opinion of all the Fathers.
Finally," he added, suddenly assuming a mystic tone of voice while
he rolled a pinch of snuff between his fingers, "if the Church has
condemned the theatre, she must be right; we must submit to her
decrees."

"Why," asked the druggist, "should she excommunicate actors? For
formerly they openly took part in religious ceremonies. Yes, in the
middle of the chancel they acted; they performed a kind of farce called
'Mysteries,' which often offended against the laws of decency."

The ecclesiastic contented himself with uttering a groan, and the
chemist went on--

"It's like it is in the Bible; there there are, you know, more than one
piquant detail, matters really libidinous!"

And on a gesture of irritation from Monsieur Bournisien--

"Ah! you'll admit that it is not a book to place in the hands of a young
girl, and I should be sorry if Athalie--"

"But it is the Protestants, and not we," cried the other impatiently,
"who recommend the Bible."

"No matter," said Homais. "I am surprised that in our days, in this
century of enlightenment, anyone should still persist in proscribing an
intellectual relaxation that is inoffensive, moralising, and sometimes
even hygienic; is it not, doctor?"

"No doubt," replied the doctor carelessly, either because, sharing the
same ideas, he wished to offend no one, or else because he had not any
ideas.

The conversation seemed at an end when the chemist thought fit to shoot
a Parthian arrow.

"I've known priests who put on ordinary clothes to go and see dancers
kicking about."

"Come, come!" said the cure.

"Ah! I've known some!" And separating the words of his sentence, Homais
repeated, "I--have--known--some!"

"Well, they were wrong," said Bournisien, resigned to anything.

"By Jove! they go in for more than that," exclaimed the druggist.

"Sir!" replied the ecclesiastic, with such angry eyes that the druggist
was intimidated by them.

"I only mean to say," he replied in less brutal a tone, "that toleration
is the surest way to draw people to religion."

"That is true! that is true!" agreed the good fellow, sitting down again
on his chair. But he stayed only a few moments.

Then, as soon as he had gone, Monsieur Homais said to the doctor--

"That's what I call a cock-fight. I beat him, did you see, in a
way!--Now take my advice. Take madame to the theatre, if it were only
for once in your life, to enrage one of these ravens, hang it! If anyone
could take my place, I would accompany you myself. Be quick about it.
Lagardy is only going to give one performance; he's engaged to go to
England at a high salary. From what I hear, he's a regular dog; he's
rolling in money; he's taking three mistresses and a cook along with
him. All these great artists burn the candle at both ends; they require
a dissolute life, that suits the imagination to some extent. But they
die at the hospital, because they haven't the sense when young to lay
by. Well, a pleasant dinner! Goodbye till to-morrow."

The idea of the theatre quickly germinated in Bovary's head, for he at
once communicated it to his wife, who at first refused, alleging the
fatigue, the worry, the expense; but, for a wonder, Charles did not give
in, so sure was he that this recreation would be good for her. He saw
nothing to prevent it: his mother had sent them three hundred francs
which he had no longer expected; the current debts were not very large,
and the falling in of Lheureux's bills was still so far off that
there was no need to think about them. Besides, imagining that she
was refusing from delicacy, he insisted the more; so that by dint of
worrying her she at last made up her mind, and the next day at eight
o'clock they set out in the "Hirondelle."

The druggist, whom nothing whatever kept at Yonville, but who thought
himself bound not to budge from it, sighed as he saw them go.

"Well, a pleasant journey!" he said to them; "happy mortals that you
are!"

Then addressing himself to Emma, who was wearing a blue silk gown with
four flounces--

"You are as lovely as a Venus. You'll cut a figure at Rouen."

The diligence stopped at the "Croix-Rouge" in the Place Beauvoisine. It
was the inn that is in every provincial faubourg, with large stables
and small bedrooms, where one sees in the middle of the court chickens
pilfering the oats under the muddy gigs of the commercial travellers--a
good old house, with worm-eaten balconies that creak in the wind on
winter nights, always full of people, noise, and feeding, whose black
tables are sticky with coffee and brandy, the thick windows made yellow
by the flies, the damp napkins stained with cheap wine, and that always
smells of the village, like ploughboys dressed in Sundayclothes, has
a cafe on the street, and towards the countryside a kitchen-garden.
Charles at once set out. He muddled up the stage-boxes with the gallery,
the pit with the boxes; asked for explanations, did not understand them;
was sent from the box-office to the acting-manager; came back to the
inn, returned to the theatre, and thus several times traversed the whole
length of the town from the theatre to the boulevard.

Madame Bovary bought a bonnet, gloves, and a bouquet. The doctor was
much afraid of missing the beginning, and, without having had time to
swallow a plate of soup, they presented themselves at the doors of the
theatre, which were still closed.




\chapter{Chapter Fifteen}

The crowd was waiting against the wall, symmetrically enclosed between
the balustrades. At the corner of the neighbouring streets huge bills
repeated in quaint letters "Lucie de Lammermoor-Lagardy-Opera-etc." The
weather was fine, the people were hot, perspiration trickled amid the
curls, and handkerchiefs taken from pockets were mopping red foreheads;
and now and then a warm wind that blew from the river gently stirred the
border of the tick awnings hanging from the doors of the public-houses.
A little lower down, however, one was refreshed by a current of icy air
that smelt of tallow, leather, and oil. This was an exhalation from
the Rue des Charrettes, full of large black warehouses where they made
casks.

For fear of seeming ridiculous, Emma before going in wished to have a
little stroll in the harbour, and Bovary prudently kept his tickets in
his hand, in the pocket of his trousers, which he pressed against his
stomach.

Her heart began to beat as soon as she reached the vestibule. She
involuntarily smiled with vanity on seeing the crowd rushing to the
right by the other corridor while she went up the staircase to the
reserved seats. She was as pleased as a child to push with her finger
the large tapestried door. She breathed in with all her might the
dusty smell of the lobbies, and when she was seated in her box she bent
forward with the air of a duchess.

The theatre was beginning to fill; opera-glasses were taken from their
cases, and the subscribers, catching sight of one another, were bowing.
They came to seek relaxation in the fine arts after the anxieties of
business; but "business" was not forgotten; they still talked cottons,
spirits of wine, or indigo. The heads of old men were to be seen,
inexpressive and peaceful, with their hair and complexions looking like
silver medals tarnished by steam of lead. The young beaux were strutting
about in the pit, showing in the opening of their waistcoats their pink
or applegreen cravats, and Madame Bovary from above admired them leaning
on their canes with golden knobs in the open palm of their yellow
gloves.

Now the lights of the orchestra were lit, the lustre, let down from the
ceiling, throwing by the glimmering of its facets a sudden gaiety over
the theatre; then the musicians came in one after the other; and
first there was the protracted hubbub of the basses grumbling, violins
squeaking, cornets trumpeting, flutes and flageolets fifing. But three
knocks were heard on the stage, a rolling of drums began, the brass
instruments played some chords, and the curtain rising, discovered a
country-scene.

It was the cross-roads of a wood, with a fountain shaded by an oak to
the left. Peasants and lords with plaids on their shoulders were singing
a hunting-song together; then a captain suddenly came on, who evoked
the spirit of evil by lifting both his arms to heaven. Another appeared;
they went away, and the hunters started afresh. She felt herself
transported to the reading of her youth, into the midst of Walter Scott.
She seemed to hear through the mist the sound of the Scotch bagpipes
re-echoing over the heather. Then her remembrance of the novel helping
her to understand the libretto, she followed the story phrase by phrase,
while vague thoughts that came back to her dispersed at once again with
the bursts of music. She gave herself up to the lullaby of the melodies,
and felt all her being vibrate as if the violin bows were drawn over her
nerves. She had not eyes enough to look at the costumes, the scenery,
the actors, the painted trees that shook when anyone walked, and the
velvet caps, cloaks, swords--all those imaginary things that floated
amid the harmony as in the atmosphere of another world. But a young
woman stepped forward, throwing a purse to a squire in green. She was
left alone, and the flute was heard like the murmur of a fountain or the
warbling of birds. Lucie attacked her cavatina in G major bravely. She
plained of love; she longed for wings. Emma, too, fleeing from life,
would have liked to fly away in an embrace. Suddenly Edgar-Lagardy
appeared.

He had that splendid pallor that gives something of the majesty of
marble to the ardent races of the South. His vigorous form was tightly
clad in a brown-coloured doublet; a small chiselled poniard hung against
his left thigh, and he cast round laughing looks showing his white
teeth. They said that a Polish princess having heard him sing one night
on the beach at Biarritz, where he mended boats, had fallen in love
with him. She had ruined herself for him. He had deserted her for
other women, and this sentimental celebrity did not fail to enhance his
artistic reputation. The diplomatic mummer took care always to slip into
his advertisements some poetic phrase on the fascination of his
person and the susceptibility of his soul. A fine organ, imperturbable
coolness, more temperament than intelligence, more power of emphasis
than of real singing, made up the charm of this admirable charlatan
nature, in which there was something of the hairdresser and the
toreador.

From the first scene he evoked enthusiasm. He pressed Lucy in his arms,
he left her, he came back, he seemed desperate; he had outbursts of
rage, then elegiac gurglings of infinite sweetness, and the notes
escaped from his bare neck full of sobs and kisses. Emma leant forward
to see him, clutching the velvet of the box with her nails. She was
filling her heart with these melodious lamentations that were drawn
out to the accompaniment of the double-basses, like the cries of the
drowning in the tumult of a tempest. She recognised all the intoxication
and the anguish that had almost killed her. The voice of a prima donna
seemed to her to be but echoes of her conscience, and this illusion that
charmed her as some very thing of her own life. But no one on earth had
loved her with such love. He had not wept like Edgar that last moonlit
night when they said, "To-morrow! to-morrow!" The theatre rang with
cheers; they recommenced the entire movement; the lovers spoke of
the flowers on their tomb, of vows, exile, fate, hopes; and when they
uttered the final adieu, Emma gave a sharp cry that mingled with the
vibrations of the last chords.

"But why," asked Bovary, "does that gentleman persecute her?"

"No, no!" she answered; "he is her lover!"

"Yet he vows vengeance on her family, while the other one who came on
before said, 'I love Lucie and she loves me!' Besides, he went off with
her father arm in arm. For he certainly is her father, isn't he--the
ugly little man with a cock's feather in his hat?"

Despite Emma's explanations, as soon as the recitative duet began
in which Gilbert lays bare his abominable machinations to his master
Ashton, Charles, seeing the false troth-ring that is to deceive Lucie,
thought it was a love-gift sent by Edgar. He confessed, moreover, that
he did not understand the story because of the music, which interfered
very much with the words.

"What does it matter?" said Emma. "Do be quiet!"

"Yes, but you know," he went on, leaning against her shoulder, "I like
to understand things."

"Be quiet! be quiet!" she cried impatiently.

Lucie advanced, half supported by her women, a wreath of orange blossoms
in her hair, and paler than the white satin of her gown. Emma dreamed
of her marriage day; she saw herself at home again amid the corn in the
little path as they walked to the church. Oh, why had not she, like
this woman, resisted, implored? She, on the contrary, had been joyous,
without seeing the abyss into which she was throwing herself. Ah! if
in the freshness of her beauty, before the soiling of marriage and the
disillusions of adultery, she could have anchored her life upon some
great, strong heart, then virtue, tenderness, voluptuousness, and duty
blending, she would never have fallen from so high a happiness. But that
happiness, no doubt, was a lie invented for the despair of all desire.
She now knew the smallness of the passions that art exaggerated. So,
striving to divert her thoughts, Emma determined now to see in this
reproduction of her sorrows only a plastic fantasy, well enough to
please the eye, and she even smiled internally with disdainful pity when
at the back of the stage under the velvet hangings a man appeared in a
black cloak.

His large Spanish hat fell at a gesture he made, and immediately the
instruments and the singers began the sextet. Edgar, flashing with fury,
dominated all the others with his clearer voice; Ashton hurled homicidal
provocations at him in deep notes; Lucie uttered her shrill plaint,
Arthur at one side, his modulated tones in the middle register, and the
bass of the minister pealed forth like an organ, while the voices of the
women repeating his words took them up in chorus delightfully. They were
all in a row gesticulating, and anger, vengeance, jealousy, terror, and
stupefaction breathed forth at once from their half-opened mouths. The
outraged lover brandished his naked sword; his guipure ruffle rose with
jerks to the movements of his chest, and he walked from right to left
with long strides, clanking against the boards the silver-gilt spurs of
his soft boots, widening out at the ankles. He, she thought must have an
inexhaustible love to lavish it upon the crowd with such effusion.
All her small fault-findings faded before the poetry of the part
that absorbed her; and, drawn towards this man by the illusion of the
character, she tried to imagine to herself his life--that life resonant,
extraordinary, splendid, and that might have been hers if fate had
willed it. They would have known one another, loved one another. With
him, through all the kingdoms of Europe she would have travelled from
capital to capital, sharing his fatigues and his pride, picking up the
flowers thrown to him, herself embroidering his costumes. Then each
evening, at the back of a box, behind the golden trellis-work she would
have drunk in eagerly the expansions of this soul that would have sung
for her alone; from the stage, even as he acted, he would have looked
at her. But the mad idea seized her that he was looking at her; it was
certain. She longed to run to his arms, to take refuge in his strength,
as in the incarnation of love itself, and to say to him, to cry out,
"Take me away! carry me with you! let us go! Thine, thine! all my ardour
and all my dreams!"

The curtain fell.

The smell of the gas mingled with that of the breaths, the waving of the
fans, made the air more suffocating. Emma wanted to go out; the
crowd filled the corridors, and she fell back in her arm-chair with
palpitations that choked her. Charles, fearing that she would faint, ran
to the refreshment-room to get a glass of barley-water.

He had great difficulty in getting back to his seat, for his elbows were
jerked at every step because of the glass he held in his hands, and
he even spilt three-fourths on the shoulders of a Rouen lady in short
sleeves, who feeling the cold liquid running down to her loins, uttered
cries like a peacock, as if she were being assassinated. Her husband,
who was a millowner, railed at the clumsy fellow, and while she was with
her handkerchief wiping up the stains from her handsome cherry-coloured
taffeta gown, he angrily muttered about indemnity, costs, reimbursement.
At last Charles reached his wife, saying to her, quite out of breath--

"Ma foi! I thought I should have had to stay there. There is such a
crowd--SUCH a crowd!"

He added--

"Just guess whom I met up there! Monsieur Leon!"

"Leon?"

"Himself! He's coming along to pay his respects." And as he finished
these words the ex-clerk of Yonville entered the box.

He held out his hand with the ease of a gentleman; and Madame Bovary
extended hers, without doubt obeying the attraction of a stronger will.
She had not felt it since that spring evening when the rain fell upon
the green leaves, and they had said good-bye standing at the window.
But soon recalling herself to the necessities of the situation, with an
effort she shook off the torpor of her memories, and began stammering a
few hurried words.

"Ah, good-day! What! you here?"

"Silence!" cried a voice from the pit, for the third act was beginning.

"So you are at Rouen?"

"Yes."

"And since when?"

"Turn them out! turn them out!" People were looking at them. They were
silent.

But from that moment she listened no more; and the chorus of the guests,
the scene between Ashton and his servant, the grand duet in D major, all
were for her as far off as if the instruments had grown less sonorous
and the characters more remote. She remembered the games at cards at the
druggist's, and the walk to the nurse's, the reading in the arbour,
the tete-a-tete by the fireside--all that poor love, so calm and so
protracted, so discreet, so tender, and that she had nevertheless
forgotten. And why had he come back? What combination of circumstances
had brought him back into her life? He was standing behind her, leaning
with his shoulder against the wall of the box; now and again she felt
herself shuddering beneath the hot breath from his nostrils falling upon
her hair.

"Does this amuse you?" said he, bending over her so closely that the end
of his moustache brushed her cheek. She replied carelessly--

"Oh, dear me, no, not much."

Then he proposed that they should leave the theatre and go and take an
ice somewhere.

"Oh, not yet; let us stay," said Bovary. "Her hair's undone; this is
going to be tragic."

But the mad scene did not at all interest Emma, and the acting of the
singer seemed to her exaggerated.

"She screams too loud," said she, turning to Charles, who was listening.

"Yes--a little," he replied, undecided between the frankness of his
pleasure and his respect for his wife's opinion.

Then with a sigh Leon said--

"The heat is--"

"Unbearable! Yes!"

"Do you feel unwell?" asked Bovary.

"Yes, I am stifling; let us go."

Monsieur Leon put her long lace shawl carefully about her shoulders, and
all three went off to sit down in the harbour, in the open air, outside
the windows of a cafe.

First they spoke of her illness, although Emma interrupted Charles
from time to time, for fear, she said, of boring Monsieur Leon; and the
latter told them that he had come to spend two years at Rouen in a large
office, in order to get practice in his profession, which was different
in Normandy and Paris. Then he inquired after Berthe, the Homais, Mere
Lefrancois, and as they had, in the husband's presence, nothing more to
say to one another, the conversation soon came to an end.

People coming out of the theatre passed along the pavement, humming or
shouting at the top of their voices, "O bel ange, ma Lucie!*" Then Leon,
playing the dilettante, began to talk music. He had seen Tambourini,
Rubini, Persiani, Grisi, and, compared with them, Lagardy, despite his
grand outbursts, was nowhere.

     *Oh beautiful angel, my Lucie.


"Yet," interrupted Charles, who was slowly sipping his rum-sherbet,
"they say that he is quite admirable in the last act. I regret leaving
before the end, because it was beginning to amuse me."

"Why," said the clerk, "he will soon give another performance."

But Charles replied that they were going back next day. "Unless," he
added, turning to his wife, "you would like to stay alone, kitten?"

And changing his tactics at this unexpected opportunity that presented
itself to his hopes, the young man sang the praises of Lagardy in the
last number. It was really superb, sublime. Then Charles insisted--

"You would get back on Sunday. Come, make up your mind. You are wrong if
you feel that this is doing you the least good."

The tables round them, however, were emptying; a waiter came and stood
discreetly near them. Charles, who understood, took out his purse; the
clerk held back his arm, and did not forget to leave two more pieces of
silver that he made chink on the marble.

"I am really sorry," said Bovary, "about the money which you are--"

The other made a careless gesture full of cordiality, and taking his hat
said--

"It is settled, isn't it? To-morrow at six o'clock?"

Charles explained once more that he could not absent himself longer, but
that nothing prevented Emma--

"But," she stammered, with a strange smile, "I am not sure--"

"Well, you must think it over. We'll see. Night brings counsel." Then to
Leon, who was walking along with them, "Now that you are in our part of
the world, I hope you'll come and ask us for some dinner now and then."

The clerk declared he would not fail to do so, being obliged, moreover,
to go to Yonville on some business for his office. And they parted
before the Saint-Herbland Passage just as the clock in the cathedral
struck half-past eleven.




Part III



Chapter One

Monsieur Leon, while studying law, had gone pretty often to the
dancing-rooms, where he was even a great success amongst the grisettes,
who thought he had a distinguished air. He was the best-mannered of the
students; he wore his hair neither too long nor too short, didn't spend
all his quarter's money on the first day of the month, and kept on good
terms with his professors. As for excesses, he had always abstained from
them, as much from cowardice as from refinement.

Often when he stayed in his room to read, or else when sitting of an
evening under the lime-trees of the Luxembourg, he let his Code fall to
the ground, and the memory of Emma came back to him. But gradually this
feeling grew weaker, and other desires gathered over it, although it
still persisted through them all. For Leon did not lose all hope; there
was for him, as it were, a vague promise floating in the future, like a
golden fruit suspended from some fantastic tree.

Then, seeing her again after three years of absence his passion
reawakened. He must, he thought, at last make up his mind to possess
her. Moreover, his timidity had worn off by contact with his gay
companions, and he returned to the provinces despising everyone who had
not with varnished shoes trodden the asphalt of the boulevards. By
the side of a Parisienne in her laces, in the drawing-room of some
illustrious physician, a person driving his carriage and wearing many
orders, the poor clerk would no doubt have trembled like a child; but
here, at Rouen, on the harbour, with the wife of this small doctor
he felt at his ease, sure beforehand he would shine. Self-possession
depends on its environment. We don't speak on the first floor as on the
fourth; and the wealthy woman seems to have, about her, to guard her
virtue, all her banknotes, like a cuirass in the lining of her corset.

On leaving the Bovarys the night before, Leon had followed them
through the streets at a distance; then having seen them stop at the
"Croix-Rouge," he turned on his heel, and spent the night meditating a
plan.

So the next day about five o'clock he walked into the kitchen of the
inn, with a choking sensation in his throat, pale cheeks, and that
resolution of cowards that stops at nothing.

"The gentleman isn't in," answered a servant.

This seemed to him a good omen. He went upstairs.

She was not disturbed at his approach; on the contrary, she apologised
for having neglected to tell him where they were staying.

"Oh, I divined it!" said Leon.

He pretended he had been guided towards her by chance, by, instinct. She
began to smile; and at once, to repair his folly, Leon told her that he
had spent his morning in looking for her in all the hotels in the town
one after the other.

"So you have made up your mind to stay?" he added.

"Yes," she said, "and I am wrong. One ought not to accustom oneself to
impossible pleasures when there are a thousand demands upon one."

"Oh, I can imagine!"

"Ah! no; for you, you are a man!"

But men too had had their trials, and the conversation went off into
certain philosophical reflections. Emma expatiated much on the misery of
earthly affections, and the eternal isolation in which the heart remains
entombed.

To show off, or from a naive imitation of this melancholy which called
forth his, the young man declared that he had been awfully bored during
the whole course of his studies. The law irritated him, other vocations
attracted him, and his mother never ceased worrying him in every one
of her letters. As they talked they explained more and more fully the
motives of their sadness, working themselves up in their progressive
confidence. But they sometimes stopped short of the complete exposition
of their thought, and then sought to invent a phrase that might express
it all the same. She did not confess her passion for another; he did not
say that he had forgotten her.

Perhaps he no longer remembered his suppers with girls after masked
balls; and no doubt she did not recollect the rendezvous of old when she
ran across the fields in the morning to her lover's house. The noises
of the town hardly reached them, and the room seemed small, as if
on purpose to hem in their solitude more closely. Emma, in a dimity
dressing-gown, leant her head against the back of the old arm-chair; the
yellow wall-paper formed, as it were, a golden background behind her,
and her bare head was mirrored in the glass with the white parting in
the middle, and the tip of her ears peeping out from the folds of her
hair.

"But pardon me!" she said. "It is wrong of me. I weary you with my
eternal complaints."

"No, never, never!"

"If you knew," she went on, raising to the ceiling her beautiful eyes,
in which a tear was trembling, "all that I had dreamed!"

"And I! Oh, I too have suffered! Often I went out; I went away. I
dragged myself along the quays, seeking distraction amid the din of the
crowd without being able to banish the heaviness that weighed upon me.
In an engraver's shop on the boulevard there is an Italian print of one
of the Muses. She is draped in a tunic, and she is looking at the
moon, with forget-me-nots in her flowing hair. Something drove me there
continually; I stayed there hours together." Then in a trembling voice,
"She resembled you a little."

Madame Bovary turned away her head that he might not see the
irrepressible smile she felt rising to her lips.

"Often," he went on, "I wrote you letters that I tore up."

She did not answer. He continued--

"I sometimes fancied that some chance would bring you. I thought I
recognised you at street-corners, and I ran after all the carriages
through whose windows I saw a shawl fluttering, a veil like yours."

She seemed resolved to let him go on speaking without interruption.
Crossing her arms and bending down her face, she looked at the rosettes
on her slippers, and at intervals made little movements inside the satin
of them with her toes.

At last she sighed.

"But the most wretched thing, is it not--is to drag out, as I do, a
useless existence. If our pains were only of some use to someone, we
should find consolation in the thought of the sacrifice."

He started off in praise of virtue, duty, and silent immolation, having
himself an incredible longing for self-sacrifice that he could not
satisfy.

"I should much like," she said, "to be a nurse at a hospital."

"Alas! men have none of these holy missions, and I see nowhere any
calling--unless perhaps that of a doctor."

With a slight shrug of her shoulders, Emma interrupted him to speak of
her illness, which had almost killed her. What a pity! She should not be
suffering now! Leon at once envied the calm of the tomb, and one evening
he had even made his will, asking to be buried in that beautiful rug
with velvet stripes he had received from her. For this was how they
would have wished to be, each setting up an ideal to which they were now
adapting their past life. Besides, speech is a rolling-mill that always
thins out the sentiment.

But at this invention of the rug she asked, "But why?"

"Why?" He hesitated. "Because I loved you so!" And congratulating
himself at having surmounted the difficulty, Leon watched her face out
of the corner of his eyes.

It was like the sky when a gust of wind drives the clouds across. The
mass of sad thoughts that darkened them seemed to be lifted from her
blue eyes; her whole face shone. He waited. At last she replied--

"I always suspected it."

Then they went over all the trifling events of that far-off existence,
whose joys and sorrows they had just summed up in one word. They
recalled the arbour with clematis, the dresses she had worn, the
furniture of her room, the whole of her house.

"And our poor cactuses, where are they?"

"The cold killed them this winter."

"Ah! how I have thought of them, do you know? I often saw them again as
of yore, when on the summer mornings the sun beat down upon your blinds,
and I saw your two bare arms passing out amongst the flowers."

"Poor friend!" she said, holding out her hand to him.

Leon swiftly pressed his lips to it. Then, when he had taken a deep
breath--

"At that time you were to me I know not what incomprehensible force that
took captive my life. Once, for instance, I went to see you; but you, no
doubt, do not remember it."

"I do," she said; "go on."

"You were downstairs in the ante-room, ready to go out, standing on
the last stair; you were wearing a bonnet with small blue flowers; and
without any invitation from you, in spite of myself, I went with you.
Every moment, however, I grew more and more conscious of my folly, and
I went on walking by you, not daring to follow you completely, and
unwilling to leave you. When you went into a shop, I waited in the
street, and I watched you through the window taking off your gloves and
counting the change on the counter. Then you rang at Madame Tuvache's;
you were let in, and I stood like an idiot in front of the great heavy
door that had closed after you."

Madame Bovary, as she listened to him, wondered that she was so old. All
these things reappearing before her seemed to widen out her life; it was
like some sentimental immensity to which she returned; and from time to
time she said in a low voice, her eyes half closed--

"Yes, it is true--true--true!"

They heard eight strike on the different clocks of the Beauvoisine
quarter, which is full of schools, churches, and large empty hotels.
They no longer spoke, but they felt as they looked upon each other a
buzzing in their heads, as if something sonorous had escaped from the
fixed eyes of each of them. They were hand in hand now, and the past,
the future, reminiscences and dreams, all were confounded in the
sweetness of this ecstasy. Night was darkening over the walls, on which
still shone, half hidden in the shade, the coarse colours of four bills
representing four scenes from the "Tour de Nesle," with a motto in
Spanish and French at the bottom. Through the sash-window a patch of
dark sky was seen between the pointed roofs.

She rose to light two wax-candles on the drawers, then she sat down
again.

"Well!" said Leon.

"Well!" she replied.

He was thinking how to resume the interrupted conversation, when she
said to him--

"How is it that no one until now has ever expressed such sentiments to
me?"

The clerk said that ideal natures were difficult to understand. He from
the first moment had loved her, and he despaired when he thought of the
happiness that would have been theirs, if thanks to fortune, meeting her
earlier, they had been indissolubly bound to one another.

"I have sometimes thought of it," she went on.

"What a dream!" murmured Leon. And fingering gently the blue binding of
her long white sash, he added, "And who prevents us from beginning now?"

"No, my friend," she replied; "I am too old; you are too young. Forget
me! Others will love you; you will love them."

"Not as you!" he cried.

"What a child you are! Come, let us be sensible. I wish it."

She showed him the impossibility of their love, and that they must
remain, as formerly, on the simple terms of a fraternal friendship.

Was she speaking thus seriously? No doubt Emma did not herself know,
quite absorbed as she was by the charm of the seduction, and the
necessity of defending herself from it; and contemplating the young
man with a moved look, she gently repulsed the timid caresses that his
trembling hands attempted.

"Ah! forgive me!" he cried, drawing back.

Emma was seized with a vague fear at this shyness, more dangerous to her
than the boldness of Rodolphe when he advanced to her open-armed. No man
had ever seemed to her so beautiful. An exquisite candour emanated from
his being. He lowered his long fine eyelashes, that curled upwards.
His cheek, with the soft skin reddened, she thought, with desire of her
person, and Emma felt an invincible longing to press her lips to it.
Then, leaning towards the clock as if to see the time--

"Ah! how late it is!" she said; "how we do chatter!"

He understood the hint and took up his hat.

"It has even made me forget the theatre. And poor Bovary has left me
here especially for that. Monsieur Lormeaux, of the Rue Grand-Pont, was
to take me and his wife."

And the opportunity was lost, as she was to leave the next day.

"Really!" said Leon.

"Yes."

"But I must see you again," he went on. "I wanted to tell you--"

"What?"

"Something--important--serious. Oh, no! Besides, you will not go; it is
impossible. If you should--listen to me. Then you have not understood
me; you have not guessed--"

"Yet you speak plainly," said Emma.

"Ah! you can jest. Enough! enough! Oh, for pity's sake, let me see you
once--only once!"

"Well--" She stopped; then, as if thinking better of it, "Oh, not here!"

"Where you will."

"Will you--" She seemed to reflect; then abruptly, "To-morrow at eleven
o'clock in the cathedral."

"I shall be there," he cried, seizing her hands, which she disengaged.

And as they were both standing up, he behind her, and Emma with her head
bent, he stooped over her and pressed long kisses on her neck.

"You are mad! Ah! you are mad!" she said, with sounding little laughs,
while the kisses multiplied.

Then bending his head over her shoulder, he seemed to beg the consent of
her eyes. They fell upon him full of an icy dignity.

Leon stepped back to go out. He stopped on the threshold; then he
whispered with a trembling voice, "Tomorrow!"

She answered with a nod, and disappeared like a bird into the next room.

In the evening Emma wrote the clerk an interminable letter, in which she
cancelled the rendezvous; all was over; they must not, for the sake of
their happiness, meet again. But when the letter was finished, as she
did not know Leon's address, she was puzzled.

"I'll give it to him myself," she said; "he will come."

The next morning, at the open window, and humming on his balcony, Leon
himself varnished his pumps with several coatings. He put on white
trousers, fine socks, a green coat, emptied all the scent he had into
his handkerchief, then having had his hair curled, he uncurled it again,
in order to give it a more natural elegance.

"It is still too early," he thought, looking at the hairdresser's
cuckoo-clock, that pointed to the hour of nine. He read an old fashion
journal, went out, smoked a cigar, walked up three streets, thought it
was time, and went slowly towards the porch of Notre Dame.

It was a beautiful summer morning. Silver plate sparkled in the
jeweller's windows, and the light falling obliquely on the cathedral
made mirrors of the corners of the grey stones; a flock of birds
fluttered in the grey sky round the trefoil bell-turrets; the square,
resounding with cries, was fragrant with the flowers that bordered its
pavement, roses, jasmines, pinks, narcissi, and tube-roses, unevenly
spaced out between moist grasses, catmint, and chickweed for the birds;
the fountains gurgled in the centre, and under large umbrellas, amidst
melons, piled up in heaps, flower-women, bare-headed, were twisting
paper round bunches of violets.

The young man took one. It was the first time that he had bought flowers
for a woman, and his breast, as he smelt them, swelled with pride, as if
this homage that he meant for another had recoiled upon himself.

But he was afraid of being seen; he resolutely entered the church. The
beadle, who was just then standing on the threshold in the middle of the
left doorway, under the "Dancing Marianne," with feather cap, and rapier
dangling against his calves, came in, more majestic than a cardinal, and
as shining as a saint on a holy pyx.

He came towards Leon, and, with that smile of wheedling benignity
assumed by ecclesiastics when they question children--

"The gentleman, no doubt, does not belong to these parts? The gentleman
would like to see the curiosities of the church?"

"No!" said the other.

And he first went round the lower aisles. Then he went out to look at
the Place. Emma was not coming yet. He went up again to the choir.

The nave was reflected in the full fonts with the beginning of the
arches and some portions of the glass windows. But the reflections of
the paintings, broken by the marble rim, were continued farther on upon
the flag-stones, like a many-coloured carpet. The broad daylight from
without streamed into the church in three enormous rays from the three
opened portals. From time to time at the upper end a sacristan passed,
making the oblique genuflexion of devout persons in a hurry. The crystal
lustres hung motionless. In the choir a silver lamp was burning, and
from the side chapels and dark places of the church sometimes rose
sounds like sighs, with the clang of a closing grating, its echo
reverberating under the lofty vault.

Leon with solemn steps walked along by the walls. Life had never seemed
so good to him. She would come directly, charming, agitated, looking
back at the glances that followed her, and with her flounced dress, her
gold eyeglass, her thin shoes, with all sorts of elegant trifles that he
had never enjoyed, and with the ineffable seduction of yielding virtue.
The church like a huge boudoir spread around her; the arches bent down
to gather in the shade the confession of her love; the windows shone
resplendent to illumine her face, and the censers would burn that she
might appear like an angel amid the fumes of the sweet-smelling odours.

But she did not come. He sat down on a chair, and his eyes fell upon a
blue stained window representing boatmen carrying baskets. He looked at
it long, attentively, and he counted the scales of the fishes and the
button-holes of the doublets, while his thoughts wandered off towards
Emma.

The beadle, standing aloof, was inwardly angry at this individual who
took the liberty of admiring the cathedral by himself. He seemed to him
to be conducting himself in a monstrous fashion, to be robbing him in a
sort, and almost committing sacrilege.

But a rustle of silk on the flags, the tip of a bonnet, a lined
cloak--it was she! Leon rose and ran to meet her.

Emma was pale. She walked fast.

"Read!" she said, holding out a paper to him. "Oh, no!"

And she abruptly withdrew her hand to enter the chapel of the Virgin,
where, kneeling on a chair, she began to pray.

The young man was irritated at this bigot fancy; then he nevertheless
experienced a certain charm in seeing her, in the middle of a
rendezvous, thus lost in her devotions, like an Andalusian marchioness;
then he grew bored, for she seemed never coming to an end.

Emma prayed, or rather strove to pray, hoping that some sudden
resolution might descend to her from heaven; and to draw down divine
aid she filled full her eyes with the splendours of the tabernacle. She
breathed in the perfumes of the full-blown flowers in the large vases,
and listened to the stillness of the church, that only heightened the
tumult of her heart.

She rose, and they were about to leave, when the beadle came forward,
hurriedly saying--

"Madame, no doubt, does not belong to these parts? Madame would like to
see the curiosities of the church?"

"Oh, no!" cried the clerk.

"Why not?" said she. For she clung with her expiring virtue to the
Virgin, the sculptures, the tombs--anything.

Then, in order to proceed "by rule," the beadle conducted them right to
the entrance near the square, where, pointing out with his cane a large
circle of block-stones without inscription or carving--

"This," he said majestically, "is the circumference of the beautiful
bell of Ambroise. It weighed forty thousand pounds. There was not its
equal in all Europe. The workman who cast it died of the joy--"

"Let us go on," said Leon.

The old fellow started off again; then, having got back to the chapel of
the Virgin, he stretched forth his arm with an all-embracing gesture
of demonstration, and, prouder than a country squire showing you his
espaliers, went on--

"This simple stone covers Pierre de Breze, lord of Varenne and of
Brissac, grand marshal of Poitou, and governor of Normandy, who died at
the battle of Montlhery on the 16th of July, 1465."

Leon bit his lips, fuming.

"And on the right, this gentleman all encased in iron, on the
prancing horse, is his grandson, Louis de Breze, lord of Breval and of
Montchauvet, Count de Maulevrier, Baron de Mauny, chamberlain to the
king, Knight of the Order, and also governor of Normandy; died on the
23rd of July, 1531--a Sunday, as the inscription specifies; and below,
this figure, about to descend into the tomb, portrays the same person.
It is not possible, is it, to see a more perfect representation of
annihilation?"

Madame Bovary put up her eyeglasses. Leon, motionless, looked at her,
no longer even attempting to speak a single word, to make a gesture,
so discouraged was he at this two-fold obstinacy of gossip and
indifference.

The everlasting guide went on--

"Near him, this kneeling woman who weeps is his spouse, Diane de
Poitiers, Countess de Breze, Duchess de Valentinois, born in 1499, died
in 1566, and to the left, the one with the child is the Holy Virgin. Now
turn to this side; here are the tombs of the Ambroise. They were both
cardinals and archbishops of Rouen. That one was minister under Louis
XII. He did a great deal for the cathedral. In his will he left thirty
thousand gold crowns for the poor."

And without stopping, still talking, he pushed them into a chapel
full of balustrades, some put away, and disclosed a kind of block that
certainly might once have been an ill-made statue.

"Truly," he said with a groan, "it adorned the tomb of Richard Coeur de
Lion, King of England and Duke of Normandy. It was the Calvinists, sir,
who reduced it to this condition. They had buried it for spite in the
earth, under the episcopal seat of Monsignor. See! this is the door by
which Monsignor passes to his house. Let us pass on quickly to see the
gargoyle windows."

But Leon hastily took some silver from his pocket and seized Emma's
arm. The beadle stood dumfounded, not able to understand this untimely
munificence when there were still so many things for the stranger to
see. So calling him back, he cried--

"Sir! sir! The steeple! the steeple!"

"No, thank you!" said Leon.

"You are wrong, sir! It is four hundred and forty feet high, nine less
than the great pyramid of Egypt. It is all cast; it--"

Leon was fleeing, for it seemed to him that his love, that for nearly
two hours now had become petrified in the church like the stones, would
vanish like a vapour through that sort of truncated funnel, of oblong
cage, of open chimney that rises so grotesquely from the cathedral like
the extravagant attempt of some fantastic brazier.

"But where are we going?" she said.

Making no answer, he walked on with a rapid step; and Madame Bovary
was already, dipping her finger in the holy water when behind them they
heard a panting breath interrupted by the regular sound of a cane. Leon
turned back.

"Sir!"

"What is it?"

And he recognised the beadle, holding under his arms and balancing
against his stomach some twenty large sewn volumes. They were works
"which treated of the cathedral."

"Idiot!" growled Leon, rushing out of the church.

A lad was playing about the close.

"Go and get me a cab!"

The child bounded off like a ball by the Rue Quatre-Vents; then they
were alone a few minutes, face to face, and a little embarrassed.

"Ah! Leon! Really--I don't know--if I ought," she whispered. Then with a
more serious air, "Do you know, it is very improper--"

"How so?" replied the clerk. "It is done at Paris."

And that, as an irresistible argument, decided her.

Still the cab did not come. Leon was afraid she might go back into the
church. At last the cab appeared.

"At all events, go out by the north porch," cried the beadle, who was
left alone on the threshold, "so as to see the Resurrection, the Last
Judgment, Paradise, King David, and the Condemned in Hell-flames."

"Where to, sir?" asked the coachman.

"Where you like," said Leon, forcing Emma into the cab.

And the lumbering machine set out. It went down the Rue Grand-Pont,
crossed the Place des Arts, the Quai Napoleon, the Pont Neuf, and
stopped short before the statue of Pierre Corneille.

"Go on," cried a voice that came from within.

The cab went on again, and as soon as it reached the Carrefour
Lafayette, set off down-hill, and entered the station at a gallop.

"No, straight on!" cried the same voice.

The cab came out by the gate, and soon having reached the Cours, trotted
quietly beneath the elm-trees. The coachman wiped his brow, put his
leather hat between his knees, and drove his carriage beyond the side
alley by the meadow to the margin of the waters.

It went along by the river, along the towing-path paved with sharp
pebbles, and for a long while in the direction of Oyssel, beyond the
isles.

But suddenly it turned with a dash across Quatremares, Sotteville, La
Grande-Chaussee, the Rue d'Elbeuf, and made its third halt in front of
the Jardin des Plantes.

"Get on, will you?" cried the voice more furiously.

And at once resuming its course, it passed by Saint-Sever, by the
Quai'des Curandiers, the Quai aux Meules, once more over the bridge, by
the Place du Champ de Mars, and behind the hospital gardens, where old
men in black coats were walking in the sun along the terrace all green
with ivy. It went up the Boulevard Bouvreuil, along the Boulevard
Cauchoise, then the whole of Mont-Riboudet to the Deville hills.

It came back; and then, without any fixed plan or direction, wandered
about at hazard. The cab was seen at Saint-Pol, at Lescure, at Mont
Gargan, at La Rougue-Marc and Place du Gaillardbois; in the Rue
Maladrerie, Rue Dinanderie, before Saint-Romain, Saint-Vivien,
Saint-Maclou, Saint-Nicaise--in front of the Customs, at the "Vieille
Tour," the "Trois Pipes," and the Monumental Cemetery. From time to time
the coachman, on his box cast despairing eyes at the public-houses.
He could not understand what furious desire for locomotion urged these
individuals never to wish to stop. He tried to now and then, and at
once exclamations of anger burst forth behind him. Then he lashed his
perspiring jades afresh, but indifferent to their jolting, running up
against things here and there, not caring if he did, demoralised, and
almost weeping with thirst, fatigue, and depression.

And on the harbour, in the midst of the drays and casks, and in the
streets, at the corners, the good folk opened large wonder-stricken
eyes at this sight, so extraordinary in the provinces, a cab with blinds
drawn, and which appeared thus constantly shut more closely than a tomb,
and tossing about like a vessel.

Once in the middle of the day, in the open country, just as the sun
beat most fiercely against the old plated lanterns, a bared hand passed
beneath the small blinds of yellow canvas, and threw out some scraps
of paper that scattered in the wind, and farther off lighted like white
butterflies on a field of red clover all in bloom.

At about six o'clock the carriage stopped in a back street of the
Beauvoisine Quarter, and a woman got out, who walked with her veil down,
and without turning her head.



Chapter Two

On reaching the inn, Madame Bovary was surprised not to see the
diligence. Hivert, who had waited for her fifty-three minutes, had at
last started.

Yet nothing forced her to go; but she had given her word that she would
return that same evening. Moreover, Charles expected her, and in her
heart she felt already that cowardly docility that is for some women at
once the chastisement and atonement of adultery.

She packed her box quickly, paid her bill, took a cab in the yard,
hurrying on the driver, urging him on, every moment inquiring about
the time and the miles traversed. He succeeded in catching up the
"Hirondelle" as it neared the first houses of Quincampoix.

Hardly was she seated in her corner than she closed her eyes, and opened
them at the foot of the hill, when from afar she recognised Felicite,
who was on the lookout in front of the farrier's shop. Hivert pulled
in his horses and, the servant, climbing up to the window, said
mysteriously--

"Madame, you must go at once to Monsieur Homais. It's for something
important."

The village was silent as usual. At the corner of the streets were small
pink heaps that smoked in the air, for this was the time for jam-making,
and everyone at Yonville prepared his supply on the same day. But in
front of the chemist's shop one might admire a far larger heap, and that
surpassed the others with the superiority that a laboratory must have
over ordinary stores, a general need over individual fancy.

She went in. The large arm-chair was upset, and even the "Fanal de
Rouen" lay on the ground, outspread between two pestles. She pushed open
the lobby door, and in the middle of the kitchen, amid brown jars full
of picked currants, of powdered sugar and lump sugar, of the scales on
the table, and of the pans on the fire, she saw all the Homais, small
and large, with aprons reaching to their chins, and with forks in their
hands. Justin was standing up with bowed head, and the chemist was
screaming--

"Who told you to go and fetch it in the Capharnaum."

"What is it? What is the matter?"

"What is it?" replied the druggist. "We are making preserves; they are
simmering; but they were about to boil over, because there is too
much juice, and I ordered another pan. Then he, from indolence, from
laziness, went and took, hanging on its nail in my laboratory, the key
of the Capharnaum."

It was thus the druggist called a small room under the leads, full of
the utensils and the goods of his trade. He often spent long hours there
alone, labelling, decanting, and doing up again; and he looked upon
it not as a simple store, but as a veritable sanctuary, whence there
afterwards issued, elaborated by his hands, all sorts of pills, boluses,
infusions, lotions, and potions, that would bear far and wide his
celebrity. No one in the world set foot there, and he respected it so,
that he swept it himself. Finally, if the pharmacy, open to all comers,
was the spot where he displayed his pride, the Capharnaum was the refuge
where, egoistically concentrating himself, Homais delighted in the
exercise of his predilections, so that Justin's thoughtlessness seemed
to him a monstrous piece of irreverence, and, redder than the currants,
he repeated--

"Yes, from the Capharnaum! The key that locks up the acids and caustic
alkalies! To go and get a spare pan! a pan with a lid! and that I
shall perhaps never use! Everything is of importance in the delicate
operations of our art! But, devil take it! one must make distinctions,
and not employ for almost domestic purposes that which is meant for
pharmaceutical! It is as if one were to carve a fowl with a scalpel; as
if a magistrate--"

"Now be calm," said Madame Homais.

And Athalie, pulling at his coat, cried "Papa! papa!"

"No, let me alone," went on the druggist "let me alone, hang it! My
word! One might as well set up for a grocer. That's it! go it! respect
nothing! break, smash, let loose the leeches, burn the mallow-paste,
pickle the gherkins in the window jars, tear up the bandages!"

"I thought you had--" said Emma.

"Presently! Do you know to what you exposed yourself? Didn't you see
anything in the corner, on the left, on the third shelf? Speak, answer,
articulate something."

"I--don't--know," stammered the young fellow.

"Ah! you don't know! Well, then, I do know! You saw a bottle of blue
glass, sealed with yellow wax, that contains a white powder, on which I
have even written 'Dangerous!' And do you know what is in it? Arsenic!
And you go and touch it! You take a pan that was next to it!"

"Next to it!" cried Madame Homais, clasping her hands. "Arsenic! You
might have poisoned us all."

And the children began howling as if they already had frightful pains in
their entrails.

"Or poison a patient!" continued the druggist. "Do you want to see me
in the prisoner's dock with criminals, in a court of justice? To see
me dragged to the scaffold? Don't you know what care I take in managing
things, although I am so thoroughly used to it? Often I am horrified
myself when I think of my responsibility; for the Government persecutes
us, and the absurd legislation that rules us is a veritable Damocles'
sword over our heads."

Emma no longer dreamed of asking what they wanted her for, and the
druggist went on in breathless phrases--

"That is your return for all the kindness we have shown you! That is how
you recompense me for the really paternal care that I lavish on you! For
without me where would you be? What would you be doing? Who provides
you with food, education, clothes, and all the means of figuring one day
with honour in the ranks of society? But you must pull hard at the oar
if you're to do that, and get, as, people say, callosities upon your
hands. Fabricando fit faber, age quod agis.*"

     * The worker lives by working, do what he will.


He was so exasperated he quoted Latin. He would have quoted Chinese
or Greenlandish had he known those two languages, for he was in one
of those crises in which the whole soul shows indistinctly what it
contains, like the ocean, which, in the storm, opens itself from the
seaweeds on its shores down to the sands of its abysses.

And he went on--

"I am beginning to repent terribly of having taken you up! I should
certainly have done better to have left you to rot in your poverty and
the dirt in which you were born. Oh, you'll never be fit for anything
but to herd animals with horns! You have no aptitude for science! You
hardly know how to stick on a label! And there you are, dwelling with me
snug as a parson, living in clover, taking your ease!"

But Emma, turning to Madame Homais, "I was told to come here--"

"Oh, dear me!" interrupted the good woman, with a sad air, "how am I to
tell you? It is a misfortune!"

She could not finish, the druggist was thundering--"Empty it! Clean it!
Take it back! Be quick!"

And seizing Justin by the collar of his blouse, he shook a book out of
his pocket. The lad stooped, but Homais was the quicker, and, having
picked up the volume, contemplated it with staring eyes and open mouth.

"CONJUGAL--LOVE!" he said, slowly separating the two words. "Ah! very
good! very good! very pretty! And illustrations! Oh, this is too much!"

Madame Homais came forward.

"No, do not touch it!"

The children wanted to look at the pictures.

"Leave the room," he said imperiously; and they went out.

First he walked up and down with the open volume in his hand, rolling
his eyes, choking, tumid, apoplectic. Then he came straight to his
pupil, and, planting himself in front of him with crossed arms--

"Have you every vice, then, little wretch? Take care! you are on a
downward path. Did not you reflect that this infamous book might fall
in the hands of my children, kindle a spark in their minds, tarnish the
purity of Athalie, corrupt Napoleon. He is already formed like a man.
Are you quite sure, anyhow, that they have not read it? Can you certify
to me--"

"But really, sir," said Emma, "you wished to tell me--"

"Ah, yes! madame. Your father-in-law is dead."

In fact, Monsieur Bovary senior had expired the evening before suddenly
from an attack of apoplexy as he got up from table, and by way of
greater precaution, on account of Emma's sensibility, Charles had begged
Homais to break the horrible news to her gradually. Homais had thought
over his speech; he had rounded, polished it, made it rhythmical; it was
a masterpiece of prudence and transitions, of subtle turns and delicacy;
but anger had got the better of rhetoric.

Emma, giving up all chance of hearing any details, left the pharmacy;
for Monsieur Homais had taken up the thread of his vituperations.
However, he was growing calmer, and was now grumbling in a paternal tone
whilst he fanned himself with his skull-cap.

"It is not that I entirely disapprove of the work. Its author was a
doctor! There are certain scientific points in it that it is not ill a
man should know, and I would even venture to say that a man must know.
But later--later! At any rate, not till you are man yourself and your
temperament is formed."

When Emma knocked at the door. Charles, who was waiting for her, came
forward with open arms and said to her with tears in his voice--

"Ah! my dear!"

And he bent over her gently to kiss her. But at the contact of his lips
the memory of the other seized her, and she passed her hand over her
face shuddering.

But she made answer, "Yes, I know, I know!"

He showed her the letter in which his mother told the event without any
sentimental hypocrisy. She only regretted her husband had not received
the consolations of religion, as he had died at Daudeville, in the
street, at the door of a cafe after a patriotic dinner with some
ex-officers.

Emma gave him back the letter; then at dinner, for appearance's sake,
she affected a certain repugnance. But as he urged her to try, she
resolutely began eating, while Charles opposite her sat motionless in a
dejected attitude.

Now and then he raised his head and gave her a long look full of
distress. Once he sighed, "I should have liked to see him again!"

She was silent. At last, understanding that she must say something, "How
old was your father?" she asked.

"Fifty-eight."

"Ah!"

And that was all.

A quarter of an hour after he added, "My poor mother! what will become
of her now?"

She made a gesture that signified she did not know. Seeing her so
taciturn, Charles imagined her much affected, and forced himself to say
nothing, not to reawaken this sorrow which moved him. And, shaking off
his own--

"Did you enjoy yourself yesterday?" he asked.

"Yes."

When the cloth was removed, Bovary did not rise, nor did Emma; and as
she looked at him, the monotony of the spectacle drove little by little
all pity from her heart. He seemed to her paltry, weak, a cipher--in
a word, a poor thing in every way. How to get rid of him? What an
interminable evening! Something stupefying like the fumes of opium
seized her.

They heard in the passage the sharp noise of a wooden leg on the boards.
It was Hippolyte bringing back Emma's luggage. In order to put it down
he described painfully a quarter of a circle with his stump.

"He doesn't even remember any more about it," she thought, looking at
the poor devil, whose coarse red hair was wet with perspiration.

Bovary was searching at the bottom of his purse for a centime, and
without appearing to understand all there was of humiliation for him
in the mere presence of this man, who stood there like a personified
reproach to his incurable incapacity.

"Hallo! you've a pretty bouquet," he said, noticing Leon's violets on
the chimney.

"Yes," she replied indifferently; "it's a bouquet I bought just now from
a beggar."

Charles picked up the flowers, and freshening his eyes, red with tears,
against them, smelt them delicately.

She took them quickly from his hand and put them in a glass of water.

The next day Madame Bovary senior arrived. She and her son wept much.
Emma, on the pretext of giving orders, disappeared. The following day
they had a talk over the mourning. They went and sat down with their
workboxes by the waterside under the arbour.

Charles was thinking of his father, and was surprised to feel so much
affection for this man, whom till then he had thought he cared little
about. Madame Bovary senior was thinking of her husband. The worst
days of the past seemed enviable to her. All was forgotten beneath the
instinctive regret of such a long habit, and from time to time whilst
she sewed, a big tear rolled along her nose and hung suspended there a
moment. Emma was thinking that it was scarcely forty-eight hours since
they had been together, far from the world, all in a frenzy of joy, and
not having eyes enough to gaze upon each other. She tried to recall the
slightest details of that past day. But the presence of her husband and
mother-in-law worried her. She would have liked to hear nothing, to see
nothing, so as not to disturb the meditation on her love, that, do what
she would, became lost in external sensations.

She was unpicking the lining of a dress, and the strips were scattered
around her. Madame Bovary senior was plying her scissor without looking
up, and Charles, in his list slippers and his old brown surtout that he
used as a dressing-gown, sat with both hands in his pockets, and did not
speak either; near them Berthe, in a little white pinafore, was raking
sand in the walks with her spade. Suddenly she saw Monsieur Lheureux,
the linendraper, come in through the gate.

He came to offer his services "under the sad circumstances." Emma
answered that she thought she could do without. The shopkeeper was not
to be beaten.

"I beg your pardon," he said, "but I should like to have a private talk
with you." Then in a low voice, "It's about that affair--you know."

Charles crimsoned to his ears. "Oh, yes! certainly." And in his
confusion, turning to his wife, "Couldn't you, my darling?"

She seemed to understand him, for she rose; and Charles said to his
mother, "It is nothing particular. No doubt, some household trifle." He
did not want her to know the story of the bill, fearing her reproaches.

As soon as they were alone, Monsieur Lheureux in sufficiently clear
terms began to congratulate Emma on the inheritance, then to talk of
indifferent matters, of the espaliers, of the harvest, and of his own
health, which was always so-so, always having ups and downs. In fact, he
had to work devilish hard, although he didn't make enough, in spite of
all people said, to find butter for his bread.

Emma let him talk on. She had bored herself so prodigiously the last two
days.

"And so you're quite well again?" he went on. "Ma foi! I saw your
husband in a sad state. He's a good fellow, though we did have a little
misunderstanding."

She asked what misunderstanding, for Charles had said nothing of the
dispute about the goods supplied to her.

"Why, you know well enough," cried Lheureux. "It was about your little
fancies--the travelling trunks."

He had drawn his hat over his eyes, and, with his hands behind his
back, smiling and whistling, he looked straight at her in an unbearable
manner. Did he suspect anything?

She was lost in all kinds of apprehensions. At last, however, he went
on--

"We made it up, all the same, and I've come again to propose another
arrangement."

This was to renew the bill Bovary had signed. The doctor, of course,
would do as he pleased; he was not to trouble himself, especially just
now, when he would have a lot of worry. "And he would do better to give
it over to someone else--to you, for example. With a power of attorney
it could be easily managed, and then we (you and I) would have our
little business transactions together."

She did not understand. He was silent. Then, passing to his trade,
Lheureux declared that madame must require something. He would send her
a black barege, twelve yards, just enough to make a gown.

"The one you've on is good enough for the house, but you want another
for calls. I saw that the very moment that I came in. I've the eye of an
American!"

He did not send the stuff; he brought it. Then he came again to measure
it; he came again on other pretexts, always trying to make himself
agreeable, useful, "enfeoffing himself," as Homais would have said, and
always dropping some hint to Emma about the power of attorney. He never
mentioned the bill; she did not think of it. Charles, at the beginning
of her convalescence, had certainly said something about it to her,
but so many emotions had passed through her head that she no longer
remembered it. Besides, she took care not to talk of any money
questions. Madame Bovary seemed surprised at this, and attributed the
change in her ways to the religious sentiments she had contracted during
her illness.

But as soon as she was gone, Emma greatly astounded Bovary by her
practical good sense. It would be necessary to make inquiries, to look
into mortgages, and see if there were any occasion for a sale by auction
or a liquidation. She quoted technical terms casually, pronounced the
grand words of order, the future, foresight, and constantly exaggerated
the difficulties of settling his father's affairs so much, that at last
one day she showed him the rough draft of a power of attorney to manage
and administer his business, arrange all loans, sign and endorse all
bills, pay all sums, etc. She had profited by Lheureux's lessons.
Charles naively asked her where this paper came from.

"Monsieur Guillaumin"; and with the utmost coolness she added, "I don't
trust him overmuch. Notaries have such a bad reputation. Perhaps we
ought to consult--we only know--no one."

"Unless Leon--" replied Charles, who was reflecting. But it was
difficult to explain matters by letter. Then she offered to make the
journey, but he thanked her. She insisted. It was quite a contest of
mutual consideration. At last she cried with affected waywardness--

"No, I will go!"

"How good you are!" he said, kissing her forehead.

The next morning she set out in the "Hirondelle" to go to Rouen to
consult Monsieur Leon, and she stayed there three days.



Chapter Three

They were three full, exquisite days--a true honeymoon. They were at
the Hotel-de-Boulogne, on the harbour; and they lived there, with drawn
blinds and closed doors, with flowers on the floor, and iced syrups were
brought them early in the morning.

Towards evening they took a covered boat and went to dine on one of the
islands. It was the time when one hears by the side of the dockyard the
caulking-mallets sounding against the hull of vessels. The smoke of
the tar rose up between the trees; there were large fatty drops on the
water, undulating in the purple colour of the sun, like floating plaques
of Florentine bronze.

They rowed down in the midst of moored boats, whose long oblique cables
grazed lightly against the bottom of the boat. The din of the town
gradually grew distant; the rolling of carriages, the tumult of voices,
the yelping of dogs on the decks of vessels. She took off her bonnet,
and they landed on their island.

They sat down in the low-ceilinged room of a tavern, at whose door hung
black nets. They ate fried smelts, cream and cherries. They lay down
upon the grass; they kissed behind the poplars; and they would fain,
like two Robinsons, have lived for ever in this little place, which
seemed to them in their beatitude the most magnificent on earth. It was
not the first time that they had seen trees, a blue sky, meadows; that
they had heard the water flowing and the wind blowing in the leaves;
but, no doubt, they had never admired all this, as if Nature had
not existed before, or had only begun to be beautiful since the
gratification of their desires.

At night they returned. The boat glided along the shores of the islands.
They sat at the bottom, both hidden by the shade, in silence. The square
oars rang in the iron thwarts, and, in the stillness, seemed to mark
time, like the beating of a metronome, while at the stern the rudder
that trailed behind never ceased its gentle splash against the water.

Once the moon rose; they did not fail to make fine phrases, finding the
orb melancholy and full of poetry. She even began to sing--

"One night, do you remember, we were sailing," etc.

Her musical but weak voice died away along the waves, and the winds
carried off the trills that Leon heard pass like the flapping of wings
about him.

She was opposite him, leaning against the partition of the shallop,
through one of whose raised blinds the moon streamed in. Her black
dress, whose drapery spread out like a fan, made her seem more slender,
taller. Her head was raised, her hands clasped, her eyes turned towards
heaven. At times the shadow of the willows hid her completely; then she
reappeared suddenly, like a vision in the moonlight.

Leon, on the floor by her side, found under his hand a ribbon of scarlet
silk. The boatman looked at it, and at last said--

"Perhaps it belongs to the party I took out the other day. A lot
of jolly folk, gentlemen and ladies, with cakes, champagne,
cornets--everything in style! There was one especially, a tall handsome
man with small moustaches, who was that funny! And they all kept saying,
'Now tell us something, Adolphe--Dolpe,' I think."

She shivered.

"You are in pain?" asked Leon, coming closer to her.

"Oh, it's nothing! No doubt, it is only the night air."

"And who doesn't want for women, either," softly added the sailor,
thinking he was paying the stranger a compliment.

Then, spitting on his hands, he took the oars again.

Yet they had to part. The adieux were sad. He was to send his letters to
Mere Rollet, and she gave him such precise instructions about a double
envelope that he admired greatly her amorous astuteness.

"So you can assure me it is all right?" she said with her last kiss.

"Yes, certainly."

"But why," he thought afterwards as he came back through the streets
alone, "is she so very anxious to get this power of attorney?"



Chapter Four

Leon soon put on an air of superiority before his comrades, avoided
their company, and completely neglected his work.

He waited for her letters; he re-read them; he wrote to her. He called
her to mind with all the strength of his desires and of his memories.
Instead of lessening with absence, this longing to see her again grew,
so that at last on Saturday morning he escaped from his office.

When, from the summit of the hill, he saw in the valley below the
church-spire with its tin flag swinging in the wind, he felt that
delight mingled with triumphant vanity and egoistic tenderness that
millionaires must experience when they come back to their native
village.

He went rambling round her house. A light was burning in the kitchen. He
watched for her shadow behind the curtains, but nothing appeared.

Mere Lefrancois, when she saw him, uttered many exclamations. She
thought he "had grown and was thinner," while Artemise, on the contrary,
thought him stouter and darker.

He dined in the little room as of yore, but alone, without the
tax-gatherer; for Binet, tired of waiting for the "Hirondelle," had
definitely put forward his meal one hour, and now he dined punctually at
five, and yet he declared usually the rickety old concern "was late."

Leon, however, made up his mind, and knocked at the doctor's door.
Madame was in her room, and did not come down for a quarter of an hour.
The doctor seemed delighted to see him, but he never stirred out that
evening, nor all the next day.

He saw her alone in the evening, very late, behind the garden in the
lane; in the lane, as she had the other one! It was a stormy night, and
they talked under an umbrella by lightning flashes.

Their separation was becoming intolerable. "I would rather die!" said
Emma. She was writhing in his arms, weeping. "Adieu! adieu! When shall I
see you again?"

They came back again to embrace once more, and it was then that
she promised him to find soon, by no matter what means, a regular
opportunity for seeing one another in freedom at least once a week. Emma
never doubted she should be able to do this. Besides, she was full of
hope. Some money was coming to her.

On the strength of it she bought a pair of yellow curtains with large
stripes for her room, whose cheapness Monsieur Lheureux had commended;
she dreamed of getting a carpet, and Lheureux, declaring that it wasn't
"drinking the sea," politely undertook to supply her with one. She could
no longer do without his services. Twenty times a day she sent for him,
and he at once put by his business without a murmur. People could not
understand either why Mere Rollet breakfasted with her every day, and
even paid her private visits.

It was about this time, that is to say, the beginning of winter, that
she seemed seized with great musical fervour.

One evening when Charles was listening to her, she began the same piece
four times over, each time with much vexation, while he, not noticing
any difference, cried--

"Bravo! very goodl You are wrong to stop. Go on!"

"Oh, no; it is execrable! My fingers are quite rusty."

The next day he begged her to play him something again.

"Very well; to please you!"

And Charles confessed she had gone off a little. She played wrong notes
and blundered; then, stopping short--

"Ah! it is no use. I ought to take some lessons; but--" She bit her lips
and added, "Twenty francs a lesson, that's too dear!"

"Yes, so it is--rather," said Charles, giggling stupidly. "But it seems
to me that one might be able to do it for less; for there are artists of
no reputation, and who are often better than the celebrities."

"Find them!" said Emma.

The next day when he came home he looked at her shyly, and at last could
no longer keep back the words.

"How obstinate you are sometimes! I went to Barfucheres to-day. Well,
Madame Liegard assured me that her three young ladies who are at
La Misericorde have lessons at fifty sous apiece, and that from an
excellent mistress!"

She shrugged her shoulders and did not open her piano again. But when
she passed by it (if Bovary were there), she sighed--

"Ah! my poor piano!"

And when anyone came to see her, she did not fail to inform them she
had given up music, and could not begin again now for important reasons.
Then people commiserated her--

"What a pity! she had so much talent!"

They even spoke to Bovary about it. They put him to shame, and
especially the chemist.

"You are wrong. One should never let any of the faculties of nature lie
fallow. Besides, just think, my good friend, that by inducing madame to
study; you are economising on the subsequent musical education of
your child. For my own part, I think that mothers ought themselves to
instruct their children. That is an idea of Rousseau's, still rather
new perhaps, but that will end by triumphing, I am certain of it, like
mothers nursing their own children and vaccination."

So Charles returned once more to this question of the piano. Emma
replied bitterly that it would be better to sell it. This poor piano,
that had given her vanity so much satisfaction--to see it go was to
Bovary like the indefinable suicide of a part of herself.

"If you liked," he said, "a lesson from time to time, that wouldn't
after all be very ruinous."

"But lessons," she replied, "are only of use when followed up."

And thus it was she set about obtaining her husband's permission to go
to town once a week to see her lover. At the end of a month she was even
considered to have made considerable progress.



Chapter Five

She went on Thursdays. She got up and dressed silently, in order not to
awaken Charles, who would have made remarks about her getting ready too
early. Next she walked up and down, went to the windows, and looked out
at the Place. The early dawn was broadening between the pillars of the
market, and the chemist's shop, with the shutters still up, showed in
the pale light of the dawn the large letters of his signboard.

When the clock pointed to a quarter past seven, she went off to the
"Lion d'Or," whose door Artemise opened yawning. The girl then made
up the coals covered by the cinders, and Emma remained alone in the
kitchen. Now and again she went out. Hivert was leisurely harnessing his
horses, listening, moreover, to Mere Lefrancois, who, passing her head
and nightcap through a grating, was charging him with commissions and
giving him explanations that would have confused anyone else. Emma kept
beating the soles of her boots against the pavement of the yard.

At last, when he had eaten his soup, put on his cloak, lighted his pipe,
and grasped his whip, he calmly installed himself on his seat.

The "Hirondelle" started at a slow trot, and for about a mile stopped
here and there to pick up passengers who waited for it, standing at the
border of the road, in front of their yard gates.

Those who had secured seats the evening before kept it waiting; some
even were still in bed in their houses. Hivert called, shouted, swore;
then he got down from his seat and went and knocked loudly at the doors.
The wind blew through the cracked windows.

The four seats, however, filled up. The carriage rolled off; rows of
apple-trees followed one upon another, and the road between its two long
ditches, full of yellow water, rose, constantly narrowing towards the
horizon.

Emma knew it from end to end; she knew that after a meadow there was
a sign-post, next an elm, a barn, or the hut of a lime-kiln tender.
Sometimes even, in the hope of getting some surprise, she shut her eyes,
but she never lost the clear perception of the distance to be traversed.

At last the brick houses began to follow one another more closely, the
earth resounded beneath the wheels, the "Hirondelle" glided between the
gardens, where through an opening one saw statues, a periwinkle plant,
clipped yews, and a swing. Then on a sudden the town appeared. Sloping
down like an amphitheatre, and drowned in the fog, it widened out
beyond the bridges confusedly. Then the open country spread away with
a monotonous movement till it touched in the distance the vague line of
the pale sky. Seen thus from above, the whole landscape looked immovable
as a picture; the anchored ships were massed in one corner, the river
curved round the foot of the green hills, and the isles, oblique in
shape, lay on the water, like large, motionless, black fishes. The
factory chimneys belched forth immense brown fumes that were blown away
at the top. One heard the rumbling of the foundries, together with the
clear chimes of the churches that stood out in the mist. The leafless
trees on the boulevards made violet thickets in the midst of the
houses, and the roofs, all shining with the rain, threw back unequal
reflections, according to the height of the quarters in which they were.
Sometimes a gust of wind drove the clouds towards the Saint Catherine
hills, like aerial waves that broke silently against a cliff.

A giddiness seemed to her to detach itself from this mass of existence,
and her heart swelled as if the hundred and twenty thousand souls that
palpitated there had all at once sent into it the vapour of the passions
she fancied theirs. Her love grew in the presence of this vastness, and
expanded with tumult to the vague murmurings that rose towards her. She
poured it out upon the square, on the walks, on the streets, and the
old Norman city outspread before her eyes as an enormous capital, as a
Babylon into which she was entering. She leant with both hands against
the window, drinking in the breeze; the three horses galloped, the
stones grated in the mud, the diligence rocked, and Hivert, from afar,
hailed the carts on the road, while the bourgeois who had spent the
night at the Guillaume woods came quietly down the hill in their little
family carriages.

They stopped at the barrier; Emma undid her overshoes, put on other
gloves, rearranged her shawl, and some twenty paces farther she got down
from the "Hirondelle."

The town was then awakening. Shop-boys in caps were cleaning up the
shop-fronts, and women with baskets against their hips, at intervals
uttered sonorous cries at the corners of streets. She walked with
downcast eyes, close to the walls, and smiling with pleasure under her
lowered black veil.

For fear of being seen, she did not usually take the most direct road.
She plunged into dark alleys, and, all perspiring, reached the bottom
of the Rue Nationale, near the fountain that stands there. It, is the
quarter for theatres, public-houses, and whores. Often a cart would
pass near her, bearing some shaking scenery. Waiters in aprons were
sprinkling sand on the flagstones between green shrubs. It all smelt of
absinthe, cigars, and oysters.

She turned down a street; she recognised him by his curling hair that
escaped from beneath his hat.

Leon walked along the pavement. She followed him to the hotel. He went
up, opened the door, entered--What an embrace!

Then, after the kisses, the words gushed forth. They told each other the
sorrows of the week, the presentiments, the anxiety for the letters; but
now everything was forgotten; they gazed into each other's faces with
voluptuous laughs, and tender names.

The bed was large, of mahogany, in the shape of a boat. The curtains
were in red levantine, that hung from the ceiling and bulged out too
much towards the bell-shaped bedside; and nothing in the world was so
lovely as her brown head and white skin standing out against this purple
colour, when, with a movement of shame, she crossed her bare arms,
hiding her face in her hands.

The warm room, with its discreet carpet, its gay ornaments, and its
calm light, seemed made for the intimacies of passion. The curtain-rods,
ending in arrows, their brass pegs, and the great balls of the fire-dogs
shone suddenly when the sun came in. On the chimney between the
candelabra there were two of those pink shells in which one hears the
murmur of the sea if one holds them to the ear.

How they loved that dear room, so full of gaiety, despite its rather
faded splendour! They always found the furniture in the same place, and
sometimes hairpins, that she had forgotten the Thursday before, under
the pedestal of the clock. They lunched by the fireside on a little
round table, inlaid with rosewood. Emma carved, put bits on his plate
with all sorts of coquettish ways, and she laughed with a sonorous and
libertine laugh when the froth of the champagne ran over from the
glass to the rings on her fingers. They were so completely lost in
the possession of each other that they thought themselves in their
own house, and that they would live there till death, like two spouses
eternally young. They said "our room," "our carpet," she even said "my
slippers," a gift of Leon's, a whim she had had. They were pink satin,
bordered with swansdown. When she sat on his knees, her leg, then too
short, hung in the air, and the dainty shoe, that had no back to it, was
held only by the toes to her bare foot.

He for the first time enjoyed the inexpressible delicacy of feminine
refinements. He had never met this grace of language, this reserve of
clothing, these poses of the weary dove. He admired the exaltation of
her soul and the lace on her petticoat. Besides, was she not "a lady"
and a married woman--a real mistress, in fine?

By the diversity of her humour, in turn mystical or mirthful, talkative,
taciturn, passionate, careless, she awakened in him a thousand desires,
called up instincts or memories. She was the mistress of all the novels,
the heroine of all the dramas, the vague "she" of all the volumes
of verse. He found again on her shoulder the amber colouring of the
"Odalisque Bathing"; she had the long waist of feudal chatelaines, and
she resembled the "Pale Woman of Barcelona." But above all she was the
Angel!

Often looking at her, it seemed to him that his soul, escaping towards
her, spread like a wave about the outline of her head, and descended
drawn down into the whiteness of her breast. He knelt on the ground
before her, and with both elbows on her knees looked at her with a
smile, his face upturned.

She bent over him, and murmured, as if choking with intoxication--

"Oh, do not move! do not speak! look at me! Something so sweet comes
from your eyes that helps me so much!"

She called him "child." "Child, do you love me?"

And she did not listen for his answer in the haste of her lips that
fastened to his mouth.

On the clock there was a bronze cupid, who smirked as he bent his arm
beneath a golden garland. They had laughed at it many a time, but when
they had to part everything seemed serious to them.

Motionless in front of each other, they kept repeating, "Till Thursday,
till Thursday."

Suddenly she seized his head between her hands, kissed him hurriedly on
the forehead, crying, "Adieu!" and rushed down the stairs.

She went to a hairdresser's in the Rue de la Comedie to have her hair
arranged. Night fell; the gas was lighted in the shop. She heard the
bell at the theatre calling the mummers to the performance, and she saw,
passing opposite, men with white faces and women in faded gowns going in
at the stage-door.

It was hot in the room, small, and too low where the stove was hissing
in the midst of wigs and pomades. The smell of the tongs, together with
the greasy hands that handled her head, soon stunned her, and she dozed
a little in her wrapper. Often, as he did her hair, the man offered her
tickets for a masked ball.

Then she went away. She went up the streets; reached the Croix-Rouge,
put on her overshoes, that she had hidden in the morning under the seat,
and sank into her place among the impatient passengers. Some got out
at the foot of the hill. She remained alone in the carriage. At every
turning all the lights of the town were seen more and more completely,
making a great luminous vapour about the dim houses. Emma knelt on the
cushions and her eyes wandered over the dazzling light. She sobbed;
called on Leon, sent him tender words and kisses lost in the wind.

On the hillside a poor devil wandered about with his stick in the midst
of the diligences. A mass of rags covered his shoulders, and an old
staved-in beaver, turned out like a basin, hid his face; but when he
took it off he discovered in the place of eyelids empty and bloody
orbits. The flesh hung in red shreds, and there flowed from it liquids
that congealed into green scale down to the nose, whose black nostrils
sniffed convulsively. To speak to you he threw back his head with an
idiotic laugh; then his bluish eyeballs, rolling constantly, at the
temples beat against the edge of the open wound. He sang a little song
as he followed the carriages--

"Maids an the warmth of a summer day Dream of love, and of love always"

And all the rest was about birds and sunshine and green leaves.

Sometimes he appeared suddenly behind Emma, bareheaded, and she drew
back with a cry. Hivert made fun of him. He would advise him to get a
booth at the Saint Romain fair, or else ask him, laughing, how his young
woman was.

Often they had started when, with a sudden movement, his hat entered the
diligence through the small window, while he clung with his other arm
to the footboard, between the wheels splashing mud. His voice, feeble
at first and quavering, grew sharp; it resounded in the night like the
indistinct moan of a vague distress; and through the ringing of the
bells, the murmur of the trees, and the rumbling of the empty vehicle,
it had a far-off sound that disturbed Emma. It went to the bottom of
her soul, like a whirlwind in an abyss, and carried her away into the
distances of a boundless melancholy. But Hivert, noticing a weight
behind, gave the blind man sharp cuts with his whip. The thong lashed
his wounds, and he fell back into the mud with a yell. Then the
passengers in the "Hirondelle" ended by falling asleep, some with open
mouths, others with lowered chins, leaning against their neighbour's
shoulder, or with their arm passed through the strap, oscillating
regularly with the jolting of the carriage; and the reflection of the
lantern swinging without, on the crupper of the wheeler; penetrating
into the interior through the chocolate calico curtains, threw
sanguineous shadows over all these motionless people. Emma, drunk with
grief, shivered in her clothes, feeling her feet grow colder and colder,
and death in her soul.

Charles at home was waiting for her; the "Hirondelle" was always late
on Thursdays. Madame arrived at last, and scarcely kissed the child. The
dinner was not ready. No matter! She excused the servant. This girl now
seemed allowed to do just as she liked.

Often her husband, noting her pallor, asked if she were unwell.

"No," said Emma.

"But," he replied, "you seem so strange this evening."

"Oh, it's nothing! nothing!"

There were even days when she had no sooner come in than she went up to
her room; and Justin, happening to be there, moved about noiselessly,
quicker at helping her than the best of maids. He put the matches
ready, the candlestick, a book, arranged her nightgown, turned back the
bedclothes.

"Come!" said she, "that will do. Now you can go."

For he stood there, his hands hanging down and his eyes wide open, as if
enmeshed in the innumerable threads of a sudden reverie.

The following day was frightful, and those that came after still more
unbearable, because of her impatience to once again seize her happiness;
an ardent lust, inflamed by the images of past experience, and that
burst forth freely on the seventh day beneath Leon's caresses. His
ardours were hidden beneath outbursts of wonder and gratitude. Emma
tasted this love in a discreet, absorbed fashion, maintained it by all
the artifices of her tenderness, and trembled a little lest it should be
lost later on.

She often said to him, with her sweet, melancholy voice--

"Ah! you too, you will leave me! You will marry! You will be like all
the others."

He asked, "What others?"

"Why, like all men," she replied. Then added, repulsing him with a
languid movement--

"You are all evil!"

One day, as they were talking philosophically of earthly disillusions,
to experiment on his jealousy, or yielding, perhaps, to an over-strong
need to pour out her heart, she told him that formerly, before him, she
had loved someone.

"Not like you," she went on quickly, protesting by the head of her child
that "nothing had passed between them."

The young man believed her, but none the less questioned her to find out
what he was.

"He was a ship's captain, my dear."

Was this not preventing any inquiry, and, at the same time, assuming a
higher ground through this pretended fascination exercised over a man
who must have been of warlike nature and accustomed to receive homage?

The clerk then felt the lowliness of his position; he longed for
epaulettes, crosses, titles. All that would please her--he gathered that
from her spendthrift habits.

Emma nevertheless concealed many of these extravagant fancies, such as
her wish to have a blue tilbury to drive into Rouen, drawn by an English
horse and driven by a groom in top-boots. It was Justin who had inspired
her with this whim, by begging her to take him into her service as
valet-de-chambre*, and if the privation of it did not lessen the
pleasure of her arrival at each rendezvous, it certainly augmented the
bitterness of the return.

     * Manservant.


Often, when they talked together of Paris, she ended by murmuring, "Ah!
how happy we should be there!"

"Are we not happy?" gently answered the young man passing his hands over
her hair.

"Yes, that is true," she said. "I am mad. Kiss me!"

To her husband she was more charming than ever. She made him
pistachio-creams, and played him waltzes after dinner. So he thought
himself the most fortunate of men and Emma was without uneasiness, when,
one evening suddenly he said--

"It is Mademoiselle Lempereur, isn't it, who gives you lessons?"

"Yes."

"Well, I saw her just now," Charles went on, "at Madame Liegeard's. I
spoke to her about you, and she doesn't know you."

This was like a thunderclap. However, she replied quite naturally--

"Ah! no doubt she forgot my name."

"But perhaps," said the doctor, "there are several Demoiselles Lempereur
at Rouen who are music-mistresses."

"Possibly!" Then quickly--"But I have my receipts here. See!"

And she went to the writing-table, ransacked all the drawers, rummaged
the papers, and at last lost her head so completely that Charles
earnestly begged her not to take so much trouble about those wretched
receipts.

"Oh, I will find them," she said.

And, in fact, on the following Friday, as Charles was putting on one
of his boots in the dark cabinet where his clothes were kept, he felt
a piece of paper between the leather and his sock. He took it out and
read--

"Received, for three months' lessons and several pieces of music, the
sum of sixty-three francs.--Felicie Lempereur, professor of music."

"How the devil did it get into my boots?"

"It must," she replied, "have fallen from the old box of bills that is
on the edge of the shelf."

From that moment her existence was but one long tissue of lies, in which
she enveloped her love as in veils to hide it. It was a want, a mania,
a pleasure carried to such an extent that if she said she had the day
before walked on the right side of a road, one might know she had taken
the left.

One morning, when she had gone, as usual, rather lightly clothed, it
suddenly began to snow, and as Charles was watching the weather from the
window, he caught sight of Monsieur Bournisien in the chaise of Monsieur
Tuvache, who was driving him to Rouen. Then he went down to give the
priest a thick shawl that he was to hand over to Emma as soon as he
reached the "Croix-Rouge." When he got to the inn, Monsieur Bournisien
asked for the wife of the Yonville doctor. The landlady replied that
she very rarely came to her establishment. So that evening, when he
recognised Madame Bovary in the "Hirondelle," the cure told her his
dilemma, without, however, appearing to attach much importance to it,
for he began praising a preacher who was doing wonders at the Cathedral,
and whom all the ladies were rushing to hear.

Still, if he did not ask for any explanation, others, later on, might
prove less discreet. So she thought well to get down each time at the
"Croix-Rouge," so that the good folk of her village who saw her on the
stairs should suspect nothing.

One day, however, Monsieur Lheureux met her coming out of the Hotel
de Boulogne on Leon's arm; and she was frightened, thinking he would
gossip. He was not such a fool. But three days after he came to her
room, shut the door, and said, "I must have some money."

She declared she could not give him any. Lheureux burst into
lamentations and reminded her of all the kindnesses he had shown her.

In fact, of the two bills signed by Charles, Emma up to the present had
paid only one. As to the second, the shopkeeper, at her request, had
consented to replace it by another, which again had been renewed for a
long date. Then he drew from his pocket a list of goods not paid for; to
wit, the curtains, the carpet, the material for the armchairs, several
dresses, and divers articles of dress, the bills for which amounted to
about two thousand francs.

She bowed her head. He went on--

"But if you haven't any ready money, you have an estate." And he
reminded her of a miserable little hovel situated at Barneville, near
Aumale, that brought in almost nothing. It had formerly been part of a
small farm sold by Monsieur Bovary senior; for Lheureux knew everything,
even to the number of acres and the names of the neighbours.

"If I were in your place," he said, "I should clear myself of my debts,
and have money left over."

She pointed out the difficulty of getting a purchaser. He held out the
hope of finding one; but she asked him how she should manage to sell it.

"Haven't you your power of attorney?" he replied.

The phrase came to her like a breath of fresh air. "Leave me the bill,"
said Emma.

"Oh, it isn't worth while," answered Lheureux.

He came back the following week and boasted of having, after much
trouble, at last discovered a certain Langlois, who, for a long time,
had had an eye on the property, but without mentioning his price.

"Never mind the price!" she cried.

But they would, on the contrary, have to wait, to sound the fellow.
The thing was worth a journey, and, as she could not undertake it, he
offered to go to the place to have an interview with Langlois. On his
return he announced that the purchaser proposed four thousand francs.

Emma was radiant at this news.

"Frankly," he added, "that's a good price."

She drew half the sum at once, and when she was about to pay her account
the shopkeeper said--

"It really grieves me, on my word! to see you depriving yourself all at
once of such a big sum as that."

Then she looked at the bank-notes, and dreaming of the unlimited number
of rendezvous represented by those two thousand francs, she stammered--

"What! what!"

"Oh!" he went on, laughing good-naturedly, "one puts anything one likes
on receipts. Don't you think I know what household affairs are?" And he
looked at her fixedly, while in his hand he held two long papers that he
slid between his nails. At last, opening his pocket-book, he spread out
on the table four bills to order, each for a thousand francs.

"Sign these," he said, "and keep it all!"

She cried out, scandalised.

"But if I give you the surplus," replied Monsieur Lheureux impudently,
"is that not helping you?"

And taking a pen he wrote at the bottom of the account, "Received of
Madame Bovary four thousand francs."

"Now who can trouble you, since in six months you'll draw the arrears
for your cottage, and I don't make the last bill due till after you've
been paid?"

Emma grew rather confused in her calculations, and her ears tingled
as if gold pieces, bursting from their bags, rang all round her on
the floor. At last Lheureux explained that he had a very good friend,
Vincart, a broker at Rouen, who would discount these four bills. Then
he himself would hand over to madame the remainder after the actual debt
was paid.

But instead of two thousand francs he brought only eighteen hundred, for
the friend Vincart (which was only fair) had deducted two hundred francs
for commission and discount. Then he carelessly asked for a receipt.

"You understand--in business--sometimes. And with the date, if you
please, with the date."

A horizon of realisable whims opened out before Emma. She was prudent
enough to lay by a thousand crowns, with which the first three bills
were paid when they fell due; but the fourth, by chance, came to the
house on a Thursday, and Charles, quite upset, patiently awaited his
wife's return for an explanation.

If she had not told him about this bill, it was only to spare him such
domestic worries; she sat on his knees, caressed him, cooed to him, gave
him a long enumeration of all the indispensable things that had been got
on credit.

"Really, you must confess, considering the quantity, it isn't too dear."

Charles, at his wit's end, soon had recourse to the eternal Lheureux,
who swore he would arrange matters if the doctor would sign him two
bills, one of which was for seven hundred francs, payable in three
months. In order to arrange for this he wrote his mother a pathetic
letter. Instead of sending a reply she came herself; and when Emma
wanted to know whether he had got anything out of her, "Yes," he
replied; "but she wants to see the account." The next morning at
daybreak Emma ran to Lheureux to beg him to make out another account for
not more than a thousand francs, for to show the one for four thousand
it would be necessary to say that she had paid two-thirds, and confess,
consequently, the sale of the estate--a negotiation admirably carried
out by the shopkeeper, and which, in fact, was only actually known later
on.

Despite the low price of each article, Madame Bovary senior, of course,
thought the expenditure extravagant.

"Couldn't you do without a carpet? Why have recovered the arm-chairs? In
my time there was a single arm-chair in a house, for elderly persons--at
any rate it was so at my mother's, who was a good woman, I can tell you.
Everybody can't be rich! No fortune can hold out against waste! I should
be ashamed to coddle myself as you do! And yet I am old. I need looking
after. And there! there! fitting up gowns! fallals! What! silk for
lining at two francs, when you can get jaconet for ten sous, or even for
eight, that would do well enough!"

Emma, lying on a lounge, replied as quietly as possible--"Ah! Madame,
enough! enough!"

The other went on lecturing her, predicting they would end in the
workhouse. But it was Bovary's fault. Luckily he had promised to destroy
that power of attorney.

"What?"

"Ah! he swore he would," went on the good woman.

Emma opened the window, called Charles, and the poor fellow was obliged
to confess the promise torn from him by his mother.

Emma disappeared, then came back quickly, and majestically handed her a
thick piece of paper.

"Thank you," said the old woman. And she threw the power of attorney
into the fire.

Emma began to laugh, a strident, piercing, continuous laugh; she had an
attack of hysterics.

"Oh, my God!" cried Charles. "Ah! you really are wrong! You come here
and make scenes with her!"

His mother, shrugging her shoulders, declared it was "all put on."

But Charles, rebelling for the first time, took his wife's part, so that
Madame Bovary, senior, said she would leave. She went the very next day,
and on the threshold, as he was trying to detain her, she replied--

"No, no! You love her better than me, and you are right. It is natural.
For the rest, so much the worse! You will see. Good day--for I am not
likely to come soon again, as you say, to make scenes."

Charles nevertheless was very crestfallen before Emma, who did not hide
the resentment she still felt at his want of confidence, and it needed
many prayers before she would consent to have another power of attorney.
He even accompanied her to Monsieur Guillaumin to have a second one,
just like the other, drawn up.

"I understand," said the notary; "a man of science can't be worried with
the practical details of life."

And Charles felt relieved by this comfortable reflection, which gave his
weakness the flattering appearance of higher pre-occupation.

And what an outburst the next Thursday at the hotel in their room with
Leon! She laughed, cried, sang, sent for sherbets, wanted to smoke
cigarettes, seemed to him wild and extravagant, but adorable, superb.

He did not know what recreation of her whole being drove her more and
more to plunge into the pleasures of life. She was becoming irritable,
greedy, voluptuous; and she walked about the streets with him carrying
her head high, without fear, so she said, of compromising herself.
At times, however, Emma shuddered at the sudden thought of meeting
Rodolphe, for it seemed to her that, although they were separated
forever, she was not completely free from her subjugation to him.

One night she did not return to Yonville at all. Charles lost his head
with anxiety, and little Berthe would not go to bed without her mamma,
and sobbed enough to break her heart. Justin had gone out searching the
road at random. Monsieur Homais even had left his pharmacy.

At last, at eleven o'clock, able to bear it no longer, Charles
harnessed his chaise, jumped in, whipped up his horse, and reached the
"Croix-Rouge" about two o'clock in the morning. No one there! He thought
that the clerk had perhaps seen her; but where did he live? Happily,
Charles remembered his employer's address, and rushed off there.

Day was breaking, and he could distinguish the escutcheons over the
door, and knocked. Someone, without opening the door, shouted out the
required information, adding a few insults to those who disturb people
in the middle of the night.

The house inhabited by the clerk had neither bell, knocker, nor porter.
Charles knocked loudly at the shutters with his hands. A policeman
happened to pass by. Then he was frightened, and went away.

"I am mad," he said; "no doubt they kept her to dinner at Monsieur
Lormeaux'." But the Lormeaux no longer lived at Rouen.

"She probably stayed to look after Madame Dubreuil. Why, Madame Dubreuil
has been dead these ten months! Where can she be?"

An idea occurred to him. At a cafe he asked for a Directory, and
hurriedly looked for the name of Mademoiselle Lempereur, who lived at
No. 74 Rue de la Renelle-des-Maroquiniers.

As he was turning into the street, Emma herself appeared at the other
end of it. He threw himself upon her rather than embraced her, crying--

"What kept you yesterday?"

"I was not well."

"What was it? Where? How?"

She passed her hand over her forehead and answered, "At Mademoiselle
Lempereur's."

"I was sure of it! I was going there."

"Oh, it isn't worth while," said Emma. "She went out just now; but for
the future don't worry. I do not feel free, you see, if I know that the
least delay upsets you like this."

This was a sort of permission that she gave herself, so as to get
perfect freedom in her escapades. And she profited by it freely, fully.
When she was seized with the desire to see Leon, she set out upon any
pretext; and as he was not expecting her on that day, she went to fetch
him at his office.

It was a great delight at first, but soon he no longer concealed the
truth, which was, that his master complained very much about these
interruptions.

"Pshaw! come along," she said.

And he slipped out.

She wanted him to dress all in black, and grow a pointed beard, to
look like the portraits of Louis XIII. She wanted to see his lodgings;
thought them poor. He blushed at them, but she did not notice this, then
advised him to buy some curtains like hers, and as he objected to the
expense--

"Ah! ah! you care for your money," she said laughing.

Each time Leon had to tell her everything that he had done since their
last meeting. She asked him for some verses--some verses "for herself,"
a "love poem" in honour of her. But he never succeeded in getting a
rhyme for the second verse; and at last ended by copying a sonnet in
a "Keepsake." This was less from vanity than from the one desire of
pleasing her. He did not question her ideas; he accepted all her tastes;
he was rather becoming her mistress than she his. She had tender words
and kisses that thrilled his soul. Where could she have learnt this
corruption almost incorporeal in the strength of its profanity and
dissimulation?



Chapter Six

During the journeys he made to see her, Leon had often dined at the
chemist's, and he felt obliged from politeness to invite him in turn.

"With pleasure!" Monsieur Homais replied; "besides, I must invigorate
my mind, for I am getting rusty here. We'll go to the theatre, to the
restaurant; we'll make a night of it."

"Oh, my dear!" tenderly murmured Madame Homais, alarmed at the vague
perils he was preparing to brave.

"Well, what? Do you think I'm not sufficiently ruining my health living
here amid the continual emanations of the pharmacy? But there! that is
the way with women! They are jealous of science, and then are opposed to
our taking the most legitimate distractions. No matter! Count upon
me. One of these days I shall turn up at Rouen, and we'll go the pace
together."

The druggist would formerly have taken good care not to use such an
expression, but he was cultivating a gay Parisian style, which he
thought in the best taste; and, like his neighbour, Madame Bovary, he
questioned the clerk curiously about the customs of the capital; he
even talked slang to dazzle the bourgeois, saying bender, crummy, dandy,
macaroni, the cheese, cut my stick and "I'll hook it," for "I am going."

So one Thursday Emma was surprised to meet Monsieur Homais in the
kitchen of the "Lion d'Or," wearing a traveller's costume, that is to
say, wrapped in an old cloak which no one knew he had, while he carried
a valise in one hand and the foot-warmer of his establishment in the
other. He had confided his intentions to no one, for fear of causing the
public anxiety by his absence.

The idea of seeing again the place where his youth had been spent no
doubt excited him, for during the whole journey he never ceased talking,
and as soon as he had arrived, he jumped quickly out of the diligence
to go in search of Leon. In vain the clerk tried to get rid of him.
Monsieur Homais dragged him off to the large Cafe de la Normandie,
which he entered majestically, not raising his hat, thinking it very
provincial to uncover in any public place.

Emma waited for Leon three quarters of an hour. At last she ran to
his office; and, lost in all sorts of conjectures, accusing him of
indifference, and reproaching herself for her weakness, she spent the
afternoon, her face pressed against the window-panes.

At two o'clock they were still at a table opposite each other. The large
room was emptying; the stove-pipe, in the shape of a palm-tree, spread
its gilt leaves over the white ceiling, and near them, outside the
window, in the bright sunshine, a little fountain gurgled in a white
basin, where; in the midst of watercress and asparagus, three torpid
lobsters stretched across to some quails that lay heaped up in a pile on
their sides.

Homais was enjoying himself. Although he was even more intoxicated with
the luxury than the rich fare, the Pommard wine all the same rather
excited his faculties; and when the omelette au rhum* appeared, he began
propounding immoral theories about women. What seduced him above all
else was chic. He admired an elegant toilette in a well-furnished
apartment, and as to bodily qualities, he didn't dislike a young girl.

     * In rum.


Leon watched the clock in despair. The druggist went on drinking,
eating, and talking.

"You must be very lonely," he said suddenly, "here at Rouen. To be sure
your lady-love doesn't live far away."

And the other blushed--

"Come now, be frank. Can you deny that at Yonville--"

The young man stammered something.

"At Madame Bovary's, you're not making love to--"

"To whom?"

"The servant!"

He was not joking; but vanity getting the better of all prudence, Leon,
in spite of himself protested. Besides, he only liked dark women.

"I approve of that," said the chemist; "they have more passion."

And whispering into his friend's ear, he pointed out the symptoms by
which one could find out if a woman had passion. He even launched into
an ethnographic digression: the German was vapourish, the French woman
licentious, the Italian passionate.

"And negresses?" asked the clerk.

"They are an artistic taste!" said Homais. "Waiter! two cups of coffee!"

"Are we going?" at last asked Leon impatiently.

"Ja!"

But before leaving he wanted to see the proprietor of the establishment
and made him a few compliments. Then the young man, to be alone, alleged
he had some business engagement.

"Ah! I will escort you," said Homais.

And all the while he was walking through the streets with him he talked
of his wife, his children; of their future, and of his business; told
him in what a decayed condition it had formerly been, and to what a
degree of perfection he had raised it.

Arrived in front of the Hotel de Boulogne, Leon left him abruptly, ran
up the stairs, and found his mistress in great excitement. At mention of
the chemist she flew into a passion. He, however, piled up good reasons;
it wasn't his fault; didn't she know Homais--did she believe that he
would prefer his company? But she turned away; he drew her back, and,
sinking on his knees, clasped her waist with his arms in a languorous
pose, full of concupiscence and supplication.

She was standing up, her large flashing eyes looked at him seriously,
almost terribly. Then tears obscured them, her red eyelids were lowered,
she gave him her hands, and Leon was pressing them to his lips when a
servant appeared to tell the gentleman that he was wanted.

"You will come back?" she said.

"Yes."

"But when?"

"Immediately."

"It's a trick," said the chemist, when he saw Leon. "I wanted to
interrupt this visit, that seemed to me to annoy you. Let's go and have
a glass of garus at Bridoux'."

Leon vowed that he must get back to his office. Then the druggist joked
him about quill-drivers and the law.

"Leave Cujas and Barthole alone a bit. Who the devil prevents you? Be a
man! Let's go to Bridoux'. You'll see his dog. It's very interesting."

And as the clerk still insisted--

"I'll go with you. I'll read a paper while I wait for you, or turn over
the leaves of a 'Code.'"

Leon, bewildered by Emma's anger, Monsieur Homais' chatter, and,
perhaps, by the heaviness of the luncheon, was undecided, and, as it
were, fascinated by the chemist, who kept repeating--

"Let's go to Bridoux'. It's just by here, in the Rue Malpalu."

Then, through cowardice, through stupidity, through that indefinable
feeling that drags us into the most distasteful acts, he allowed
himself to be led off to Bridoux', whom they found in his small yard,
superintending three workmen, who panted as they turned the large
wheel of a machine for making seltzer-water. Homais gave them some good
advice. He embraced Bridoux; they took some garus. Twenty times Leon
tried to escape, but the other seized him by the arm saying--

"Presently! I'm coming! We'll go to the 'Fanal de Rouen' to see the
fellows there. I'll introduce you to Thornassin."

At last he managed to get rid of him, and rushed straight to the hotel.
Emma was no longer there. She had just gone in a fit of anger. She
detested him now. This failing to keep their rendezvous seemed to her an
insult, and she tried to rake up other reasons to separate herself from
him. He was incapable of heroism, weak, banal, more spiritless than a
woman, avaricious too, and cowardly.

Then, growing calmer, she at length discovered that she had, no doubt,
calumniated him. But the disparaging of those we love always alienates
us from them to some extent. We must not touch our idols; the gilt
sticks to our fingers.

They gradually came to talking more frequently of matters outside their
love, and in the letters that Emma wrote him she spoke of flowers,
verses, the moon and the stars, naive resources of a waning passion
striving to keep itself alive by all external aids. She was constantly
promising herself a profound felicity on her next journey. Then
she confessed to herself that she felt nothing extraordinary. This
disappointment quickly gave way to a new hope, and Emma returned to him
more inflamed, more eager than ever. She undressed brutally, tearing off
the thin laces of her corset that nestled around her hips like a gliding
snake. She went on tiptoe, barefooted, to see once more that the
door was closed, then, pale, serious, and, without speaking, with one
movement, she threw herself upon his breast with a long shudder.

Yet there was upon that brow covered with cold drops, on those quivering
lips, in those wild eyes, in the strain of those arms, something vague
and dreary that seemed to Leon to glide between them subtly as if to
separate them.

He did not dare to question her; but, seeing her so skilled, she must
have passed, he thought, through every experience of suffering and of
pleasure. What had once charmed now frightened him a little. Besides, he
rebelled against his absorption, daily more marked, by her personality.
He begrudged Emma this constant victory. He even strove not to love her;
then, when he heard the creaking of her boots, he turned coward, like
drunkards at the sight of strong drinks.

She did not fail, in truth, to lavish all sorts of attentions upon him,
from the delicacies of food to the coquettries of dress and languishing
looks. She brought roses to her breast from Yonville, which she threw
into his face; was anxious about his health, gave him advice as to his
conduct; and, in order the more surely to keep her hold on him, hoping
perhaps that heaven would take her part, she tied a medal of the
Virgin round his neck. She inquired like a virtuous mother about his
companions. She said to him--

"Don't see them; don't go out; think only of ourselves; love me!"

She would have liked to be able to watch over his life; and the idea
occurred to her of having him followed in the streets. Near the hotel
there was always a kind of loafer who accosted travellers, and who would
not refuse. But her pride revolted at this.

"Bah! so much the worse. Let him deceive me! What does it matter to me?
As If I cared for him!"

One day, when they had parted early and she was returning alone along
the boulevard, she saw the walls of her convent; then she sat down on a
form in the shade of the elm-trees. How calm that time had been! How she
longed for the ineffable sentiments of love that she had tried to figure
to herself out of books! The first month of her marriage, her rides in
the wood, the viscount that waltzed, and Lagardy singing, all repassed
before her eyes. And Leon suddenly appeared to her as far off as the
others.

"Yet I love him," she said to herself.

No matter! She was not happy--she never had been. Whence came this
insufficiency in life--this instantaneous turning to decay of everything
on which she leant? But if there were somewhere a being strong and
beautiful, a valiant nature, full at once of exaltation and refinement,
a poet's heart in an angel's form, a lyre with sounding chords ringing
out elegiac epithalamia to heaven, why, perchance, should she not find
him? Ah! how impossible! Besides, nothing was worth the trouble of
seeking it; everything was a lie. Every smile hid a yawn of boredom,
every joy a curse, all pleasure satiety, and the sweetest kisses left
upon your lips only the unattainable desire for a greater delight.

A metallic clang droned through the air, and four strokes were heard
from the convent-clock. Four o'clock! And it seemed to her that she had
been there on that form an eternity. But an infinity of passions may be
contained in a minute, like a crowd in a small space.

Emma lived all absorbed in hers, and troubled no more about money
matters than an archduchess.

Once, however, a wretched-looking man, rubicund and bald, came to her
house, saying he had been sent by Monsieur Vincart of Rouen. He took out
the pins that held together the side-pockets of his long green overcoat,
stuck them into his sleeve, and politely handed her a paper.

It was a bill for seven hundred francs, signed by her, and which
Lheureux, in spite of all his professions, had paid away to Vincart. She
sent her servant for him. He could not come. Then the stranger, who
had remained standing, casting right and left curious glances, that his
thick, fair eyebrows hid, asked with a naive air--

"What answer am I to take Monsieur Vincart?"

"Oh," said Emma, "tell him that I haven't it. I will send next week; he
must wait; yes, till next week."

And the fellow went without another word.

But the next day at twelve o'clock she received a summons, and the sight
of the stamped paper, on which appeared several times in large letters,
"Maitre Hareng, bailiff at Buchy," so frightened her that she rushed in
hot haste to the linendraper's. She found him in his shop, doing up a
parcel.

"Your obedient!" he said; "I am at your service."

But Lheureux, all the same, went on with his work, helped by a young
girl of about thirteen, somewhat hunch-backed, who was at once his clerk
and his servant.

Then, his clogs clattering on the shop-boards, he went up in front
of Madame Bovary to the first door, and introduced her into a narrow
closet, where, in a large bureau in sapon-wood, lay some ledgers,
protected by a horizontal padlocked iron bar. Against the wall, under
some remnants of calico, one glimpsed a safe, but of such dimensions
that it must contain something besides bills and money. Monsieur
Lheureux, in fact, went in for pawnbroking, and it was there that he had
put Madame Bovary's gold chain, together with the earrings of poor old
Tellier, who, at last forced to sell out, had bought a meagre store
of grocery at Quincampoix, where he was dying of catarrh amongst his
candles, that were less yellow than his face.

Lheureux sat down in a large cane arm-chair, saying: "What news?"

"See!"

And she showed him the paper.

"Well how can I help it?"

Then she grew angry, reminding him of the promise he had given not to
pay away her bills. He acknowledged it.

"But I was pressed myself; the knife was at my own throat."

"And what will happen now?" she went on.

"Oh, it's very simple; a judgment and then a distraint--that's about
it!"

Emma kept down a desire to strike him, and asked gently if there was no
way of quieting Monsieur Vincart.

"I dare say! Quiet Vincart! You don't know him; he's more ferocious than
an Arab!"

Still Monsieur Lheureux must interfere.

"Well, listen. It seems to me so far I've been very good to you." And
opening one of his ledgers, "See," he said. Then running up the page
with his finger, "Let's see! let's see! August 3d, two hundred francs;
June 17th, a hundred and fifty; March 23d, forty-six. In April--"

He stopped, as if afraid of making some mistake.

"Not to speak of the bills signed by Monsieur Bovary, one for seven
hundred francs, and another for three hundred. As to your little
installments, with the interest, why, there's no end to 'em; one gets
quite muddled over 'em. I'll have nothing more to do with it."

She wept; she even called him "her good Monsieur Lheureux." But he
always fell back upon "that rascal Vincart." Besides, he hadn't a brass
farthing; no one was paying him now-a-days; they were eating his coat
off his back; a poor shopkeeper like him couldn't advance money.

Emma was silent, and Monsieur Lheureux, who was biting the feathers of a
quill, no doubt became uneasy at her silence, for he went on--

"Unless one of these days I have something coming in, I might--"

"Besides," said she, "as soon as the balance of Barneville--"

"What!"

And on hearing that Langlois had not yet paid he seemed much surprised.
Then in a honied voice--

"And we agree, you say?"

"Oh! to anything you like."

On this he closed his eyes to reflect, wrote down a few figures, and
declaring it would be very difficult for him, that the affair was shady,
and that he was being bled, he wrote out four bills for two hundred and
fifty francs each, to fall due month by month.

"Provided that Vincart will listen to me! However, it's settled. I don't
play the fool; I'm straight enough."

Next he carelessly showed her several new goods, not one of which,
however, was in his opinion worthy of madame.

"When I think that there's a dress at threepence-halfpenny a yard, and
warranted fast colours! And yet they actually swallow it! Of course you
understand one doesn't tell them what it really is!" He hoped by this
confession of dishonesty to others to quite convince her of his probity
to her.

Then he called her back to show her three yards of guipure that he had
lately picked up "at a sale."

"Isn't it lovely?" said Lheureux. "It is very much used now for the
backs of arm-chairs. It's quite the rage."

And, more ready than a juggler, he wrapped up the guipure in some blue
paper and put it in Emma's hands.

"But at least let me know--"

"Yes, another time," he replied, turning on his heel.

That same evening she urged Bovary to write to his mother, to ask her
to send as quickly as possible the whole of the balance due from the
father's estate. The mother-in-law replied that she had nothing more,
the winding up was over, and there was due to them besides Barneville an
income of six hundred francs, that she would pay them punctually.

Then Madame Bovary sent in accounts to two or three patients, and she
made large use of this method, which was very successful. She was always
careful to add a postscript: "Do not mention this to my husband; you
know how proud he is. Excuse me. Yours obediently." There were some
complaints; she intercepted them.

To get money she began selling her old gloves, her old hats, the old
odds and ends, and she bargained rapaciously, her peasant blood standing
her in good stead. Then on her journey to town she picked up nick-nacks
secondhand, that, in default of anyone else, Monsieur Lheureux would
certainly take off her hands. She bought ostrich feathers, Chinese
porcelain, and trunks; she borrowed from Felicite, from Madame
Lefrancois, from the landlady at the Croix-Rouge, from everybody, no
matter where.

With the money she at last received from Barneville she paid two bills;
the other fifteen hundred francs fell due. She renewed the bills, and
thus it was continually.

Sometimes, it is true, she tried to make a calculation, but she
discovered things so exorbitant that she could not believe them
possible. Then she recommenced, soon got confused, gave it all up, and
thought no more about it.

The house was very dreary now. Tradesmen were seen leaving it with angry
faces. Handkerchiefs were lying about on the stoves, and little Berthe,
to the great scandal of Madame Homais, wore stockings with holes in
them. If Charles timidly ventured a remark, she answered roughly that it
wasn't her fault.

What was the meaning of all these fits of temper? He explained
everything through her old nervous illness, and reproaching himself with
having taken her infirmities for faults, accused himself of egotism, and
longed to go and take her in his arms.

"Ah, no!" he said to himself; "I should worry her."

And he did not stir.

After dinner he walked about alone in the garden; he took little Berthe
on his knees, and unfolding his medical journal, tried to teach her
to read. But the child, who never had any lessons, soon looked up with
large, sad eyes and began to cry. Then he comforted her; went to fetch
water in her can to make rivers on the sand path, or broke off branches
from the privet hedges to plant trees in the beds. This did not spoil
the garden much, all choked now with long weeds. They owed Lestiboudois
for so many days. Then the child grew cold and asked for her mother.

"Call the servant," said Charles. "You know, dearie, that mamma does not
like to be disturbed."

Autumn was setting in, and the leaves were already falling, as they did
two years ago when she was ill. Where would it all end? And he walked up
and down, his hands behind his back.

Madame was in her room, which no one entered. She stayed there all
day long, torpid, half dressed, and from time to time burning Turkish
pastilles which she had bought at Rouen in an Algerian's shop. In order
not to have at night this sleeping man stretched at her side, by dint of
manoeuvring, she at last succeeded in banishing him to the second floor,
while she read till morning extravagant books, full of pictures of
orgies and thrilling situations. Often, seized with fear, she cried out,
and Charles hurried to her.

"Oh, go away!" she would say.

Or at other times, consumed more ardently than ever by that inner flame
to which adultery added fuel, panting, tremulous, all desire, she threw
open her window, breathed in the cold air, shook loose in the wind her
masses of hair, too heavy, and, gazing upon the stars, longed for some
princely love. She thought of him, of Leon. She would then have given
anything for a single one of those meetings that surfeited her.

These were her gala days. She wanted them to be sumptuous, and when he
alone could not pay the expenses, she made up the deficit liberally,
which happened pretty well every time. He tried to make her understand
that they would be quite as comfortable somewhere else, in a smaller
hotel, but she always found some objection.

One day she drew six small silver-gilt spoons from her bag (they were
old Roualt's wedding present), begging him to pawn them at once for her,
and Leon obeyed, though the proceeding annoyed him. He was afraid of
compromising himself.

Then, on, reflection, he began to think his mistress's ways were growing
odd, and that they were perhaps not wrong in wishing to separate him
from her.

In fact someone had sent his mother a long anonymous letter to warn her
that he was "ruining himself with a married woman," and the good lady at
once conjuring up the eternal bugbear of families, the vague pernicious
creature, the siren, the monster, who dwells fantastically in depths of
love, wrote to Lawyer Dubocage, his employer, who behaved perfectly in
the affair. He kept him for three quarters of an hour trying to open
his eyes, to warn him of the abyss into which he was falling. Such
an intrigue would damage him later on, when he set up for himself. He
implored him to break with her, and, if he would not make this sacrifice
in his own interest, to do it at least for his, Dubocage's sake.

At last Leon swore he would not see Emma again, and he reproached
himself with not having kept his word, considering all the worry and
lectures this woman might still draw down upon him, without reckoning
the jokes made by his companions as they sat round the stove in the
morning. Besides, he was soon to be head clerk; it was time to settle
down. So he gave up his flute, exalted sentiments, and poetry; for every
bourgeois in the flush of his youth, were it but for a day, a moment,
has believed himself capable of immense passions, of lofty enterprises.
The most mediocre libertine has dreamed of sultanas; every notary bears
within him the debris of a poet.

He was bored now when Emma suddenly began to sob on his breast, and his
heart, like the people who can only stand a certain amount of music,
dozed to the sound of a love whose delicacies he no longer noted.

They knew one another too well for any of those surprises of possession
that increase its joys a hundred-fold. She was as sick of him as he
was weary of her. Emma found again in adultery all the platitudes of
marriage.

But how to get rid of him? Then, though she might feel humiliated at
the baseness of such enjoyment, she clung to it from habit or from
corruption, and each day she hungered after them the more, exhausting
all felicity in wishing for too much of it. She accused Leon of her
baffled hopes, as if he had betrayed her; and she even longed for some
catastrophe that would bring about their separation, since she had not
the courage to make up her mind to it herself.

She none the less went on writing him love letters, in virtue of the
notion that a woman must write to her lover.

But whilst she wrote it was another man she saw, a phantom fashioned out
of her most ardent memories, of her finest reading, her strongest
lusts, and at last he became so real, so tangible, that she palpitated
wondering, without, however, the power to imagine him clearly, so lost
was he, like a god, beneath the abundance of his attributes. He dwelt in
that azure land where silk ladders hang from balconies under the breath
of flowers, in the light of the moon. She felt him near her; he was
coming, and would carry her right away in a kiss.

Then she fell back exhausted, for these transports of vague love wearied
her more than great debauchery.

She now felt constant ache all over her. Often she even received
summonses, stamped paper that she barely looked at. She would have liked
not to be alive, or to be always asleep.

On Mid-Lent she did not return to Yonville, but in the evening went to
a masked ball. She wore velvet breeches, red stockings, a club wig, and
three-cornered hat cocked on one side. She danced all night to the wild
tones of the trombones; people gathered round her, and in the morning
she found herself on the steps of the theatre together with five or six
masks, debardeuses* and sailors, Leon's comrades, who were talking about
having supper.

     * People dressed as longshoremen.


The neighbouring cafes were full. They caught sight of one on the
harbour, a very indifferent restaurant, whose proprietor showed them to
a little room on the fourth floor.

The men were whispering in a corner, no doubt consorting about expenses.
There were a clerk, two medical students, and a shopman--what company
for her! As to the women, Emma soon perceived from the tone of their
voices that they must almost belong to the lowest class. Then she was
frightened, pushed back her chair, and cast down her eyes.

The others began to eat; she ate nothing. Her head was on fire, her eyes
smarted, and her skin was ice-cold. In her head she seemed to feel the
floor of the ball-room rebounding again beneath the rhythmical pulsation
of the thousands of dancing feet. And now the smell of the punch, the
smoke of the cigars, made her giddy. She fainted, and they carried her
to the window.

Day was breaking, and a great stain of purple colour broadened out
in the pale horizon over the St. Catherine hills. The livid river was
shivering in the wind; there was no one on the bridges; the street lamps
were going out.

She revived, and began thinking of Berthe asleep yonder in the servant's
room. Then a cart filled with long strips of iron passed by, and made a
deafening metallic vibration against the walls of the houses.

She slipped away suddenly, threw off her costume, told Leon she must get
back, and at last was alone at the Hotel de Boulogne. Everything, even
herself, was now unbearable to her. She wished that, taking wing like a
bird, she could fly somewhere, far away to regions of purity, and there
grow young again.

She went out, crossed the Boulevard, the Place Cauchoise, and the
Faubourg, as far as an open street that overlooked some gardens. She
walked rapidly; the fresh air calming her; and, little by little, the
faces of the crowd, the masks, the quadrilles, the lights, the supper,
those women, all disappeared like mists fading away. Then, reaching the
"Croix-Rouge," she threw herself on the bed in her little room on the
second floor, where there were pictures of the "Tour de Nesle." At four
o'clock Hivert awoke her.

When she got home, Felicite showed her behind the clock a grey paper.
She read--

"In virtue of the seizure in execution of a judgment."

What judgment? As a matter of fact, the evening before another paper
had been brought that she had not yet seen, and she was stunned by these
words--

"By order of the king, law, and justice, to Madame Bovary." Then,
skipping several lines, she read, "Within twenty-four hours, without
fail--" But what? "To pay the sum of eight thousand francs." And there
was even at the bottom, "She will be constrained thereto by every
form of law, and notably by a writ of distraint on her furniture and
effects."

What was to be done? In twenty-four hours--tomorrow. Lheureux, she
thought, wanted to frighten her again; for she saw through all his
devices, the object of his kindnesses. What reassured her was the very
magnitude of the sum.

However, by dint of buying and not paying, of borrowing, signing bills,
and renewing these bills that grew at each new falling-in, she had ended
by preparing a capital for Monsieur Lheureux which he was impatiently
awaiting for his speculations.

She presented herself at his place with an offhand air.

"You know what has happened to me? No doubt it's a joke!"

"How so?"

He turned away slowly, and, folding his arms, said to her--

"My good lady, did you think I should go on to all eternity being your
purveyor and banker, for the love of God? Now be just. I must get back
what I've laid out. Now be just."

She cried out against the debt.

"Ah! so much the worse. The court has admitted it. There's a judgment.
It's been notified to you. Besides, it isn't my fault. It's Vincart's."

"Could you not--?"

"Oh, nothing whatever."

"But still, now talk it over."

And she began beating about the bush; she had known nothing about it; it
was a surprise.

"Whose fault is that?" said Lheureux, bowing ironically. "While I'm
slaving like a nigger, you go gallivanting about."

"Ah! no lecturing."

"It never does any harm," he replied.

She turned coward; she implored him; she even pressed her pretty white
and slender hand against the shopkeeper's knee.

"There, that'll do! Anyone'd think you wanted to seduce me!"

"You are a wretch!" she cried.

"Oh, oh! go it! go it!"

"I will show you up. I shall tell my husband."

"All right! I too. I'll show your husband something."

And Lheureux drew from his strong box the receipt for eighteen hundred
francs that she had given him when Vincart had discounted the bills.

"Do you think," he added, "that he'll not understand your little theft,
the poor dear man?"

She collapsed, more overcome than if felled by the blow of a pole-axe.
He was walking up and down from the window to the bureau, repeating all
the while--

"Ah! I'll show him! I'll show him!" Then he approached her, and in a
soft voice said--

"It isn't pleasant, I know; but, after all, no bones are broken, and,
since that is the only way that is left for you paying back my money--"

"But where am I to get any?" said Emma, wringing her hands.

"Bah! when one has friends like you!"

And he looked at her in so keen, so terrible a fashion, that she
shuddered to her very heart.

"I promise you," she said, "to sign--"

"I've enough of your signatures."

"I will sell something."

"Get along!" he said, shrugging his shoulders; "you've not got
anything."

And he called through the peep-hole that looked down into the shop--

"Annette, don't forget the three coupons of No. 14."

The servant appeared. Emma understood, and asked how much money would be
wanted to put a stop to the proceedings.

"It is too late."

"But if I brought you several thousand francs--a quarter of the sum--a
third--perhaps the whole?"

"No; it's no use!"

And he pushed her gently towards the staircase.

"I implore you, Monsieur Lheureux, just a few days more!" She was
sobbing.

"There! tears now!"

"You are driving me to despair!"

"What do I care?" said he, shutting the door.



\chapter{Chapter Seven}

She was stoical the next day when Maitre Hareng, the bailiff, with two
assistants, presented himself at her house to draw up the inventory for
the distraint.

They began with Bovary's consulting-room, and did not write down
the phrenological head, which was considered an "instrument of his
profession"; but in the kitchen they counted the plates; the saucepans,
the chairs, the candlesticks, and in the bedroom all the nick-nacks on
the whatnot. They examined her dresses, the linen, the dressing-room;
and her whole existence to its most intimate details, was, like a corpse
on whom a post-mortem is made, outspread before the eyes of these three
men.

Maitre Hareng, buttoned up in his thin black coat, wearing a white
choker and very tight foot-straps, repeated from time to time--"Allow
me, madame. You allow me?" Often he uttered exclamations. "Charming!
very pretty." Then he began writing again, dipping his pen into the horn
inkstand in his left hand.

When they had done with the rooms they went up to the attic. She kept a
desk there in which Rodolphe's letters were locked. It had to be opened.

"Ah! a correspondence," said Maitre Hareng, with a discreet smile. "But
allow me, for I must make sure the box contains nothing else." And he
tipped up the papers lightly, as if to shake out napoleons. Then she
grew angered to see this coarse hand, with fingers red and pulpy like
slugs, touching these pages against which her heart had beaten.

They went at last. Felicite came back. Emma had sent her out to watch
for Bovary in order to keep him off, and they hurriedly installed the
man in possession under the roof, where he swore he would remain.

During the evening Charles seemed to her careworn. Emma watched him with
a look of anguish, fancying she saw an accusation in every line of his
face. Then, when her eyes wandered over the chimney-piece ornamented
with Chinese screens, over the large curtains, the armchairs, all
those things, in a word, that had, softened the bitterness of her life,
remorse seized her or rather an immense regret, that, far from crushing,
irritated her passion. Charles placidly poked the fire, both his feet on
the fire-dogs.

Once the man, no doubt bored in his hiding-place, made a slight noise.

"Is anyone walking upstairs?" said Charles.

"No," she replied; "it is a window that has been left open, and is
rattling in the wind."

The next day, Sunday, she went to Rouen to call on all the brokers whose
names she knew. They were at their country-places or on journeys. She
was not discouraged; and those whom she did manage to see she asked for
money, declaring she must have some, and that she would pay it back.
Some laughed in her face; all refused.

At two o'clock she hurried to Leon, and knocked at the door. No one
answered. At length he appeared.

"What brings you here?"

"Do I disturb you?"

"No; but--" And he admitted that his landlord didn't like his having
"women" there.

"I must speak to you," she went on.

Then he took down the key, but she stopped him.

"No, no! Down there, in our home!"

And they went to their room at the Hotel de Boulogne.

On arriving she drank off a large glass of water. She was very pale. She
said to him--

"Leon, you will do me a service?"

And, shaking him by both hands that she grasped tightly, she added--

"Listen, I want eight thousand francs."

"But you are mad!"

"Not yet."

And thereupon, telling him the story of the distraint, she explained
her distress to him; for Charles knew nothing of it; her mother-in-law
detested her; old Rouault could do nothing; but he, Leon, he would set
about finding this indispensable sum.

"How on earth can I?"

"What a coward you are!" she cried.

Then he said stupidly, "You are exaggerating the difficulty. Perhaps,
with a thousand crowns or so the fellow could be stopped."

All the greater reason to try and do something; it was impossible that
they could not find three thousand francs. Besides, Leon, could be
security instead of her.

"Go, try, try! I will love you so!"

He went out, and came back at the end of an hour, saying, with solemn
face--

"I have been to three people with no success."

Then they remained sitting face to face at the two chimney corners,
motionless, in silence. Emma shrugged her shoulders as she stamped her
feet. He heard her murmuring--

"If I were in your place _I_ should soon get some."

"But where?"

"At your office." And she looked at him.

An infernal boldness looked out from her burning eyes, and their lids
drew close together with a lascivious and encouraging look, so that the
young man felt himself growing weak beneath the mute will of this woman
who was urging him to a crime. Then he was afraid, and to avoid any
explanation he smote his forehead, crying--

"Morel is to come back to-night; he will not refuse me, I hope" (this
was one of his friends, the son of a very rich merchant); "and I will
bring it you to-morrow," he added.

Emma did not seem to welcome this hope with all the joy he had expected.
Did she suspect the lie? He went on, blushing--

"However, if you don't see me by three o'clock do not wait for me, my
darling. I must be off now; forgive me! Goodbye!"

He pressed her hand, but it felt quite lifeless. Emma had no strength
left for any sentiment.

Four o'clock struck, and she rose to return to Yonville, mechanically
obeying the force of old habits.

The weather was fine. It was one of those March days, clear and sharp,
when the sun shines in a perfectly white sky. The Rouen folk, in
Sunday-clothes, were walking about with happy looks. She reached the
Place du Parvis. People were coming out after vespers; the crowd flowed
out through the three doors like a stream through the three arches of
a bridge, and in the middle one, more motionless than a rock, stood the
beadle.

Then she remembered the day when, all anxious and full of hope, she had
entered beneath this large nave, that had opened out before her, less
profound than her love; and she walked on weeping beneath her veil,
giddy, staggering, almost fainting.

"Take care!" cried a voice issuing from the gate of a courtyard that was
thrown open.

She stopped to let pass a black horse, pawing the ground between the
shafts of a tilbury, driven by a gentleman in sable furs. Who was it?
She knew him. The carriage darted by and disappeared.

Why, it was he--the Viscount. She turned away; the street was empty. She
was so overwhelmed, so sad, that she had to lean against a wall to keep
herself from falling.

Then she thought she had been mistaken. Anyhow, she did not know. All
within her and around her was abandoning her. She felt lost, sinking
at random into indefinable abysses, and it was almost with joy that, on
reaching the "Croix-Rouge," she saw the good Homais, who was watching
a large box full of pharmaceutical stores being hoisted on to the
"Hirondelle." In his hand he held tied in a silk handkerchief six
cheminots for his wife.

Madame Homais was very fond of these small, heavy turban-shaped loaves,
that are eaten in Lent with salt butter; a last vestige of Gothic food
that goes back, perhaps, to the time of the Crusades, and with which
the robust Normans gorged themselves of yore, fancying they saw on the
table, in the light of the yellow torches, between tankards of hippocras
and huge boars' heads, the heads of Saracens to be devoured. The
druggist's wife crunched them up as they had done--heroically, despite
her wretched teeth. And so whenever Homais journeyed to town, he never
failed to bring her home some that he bought at the great baker's in the
Rue Massacre.

"Charmed to see you," he said, offering Emma a hand to help her into the
"Hirondelle." Then he hung up his cheminots to the cords of the netting,
and remained bare-headed in an attitude pensive and Napoleonic.

But when the blind man appeared as usual at the foot of the hill he
exclaimed--

"I can't understand why the authorities tolerate such culpable
industries. Such unfortunates should be locked up and forced to work.
Progress, my word! creeps at a snail's pace. We are floundering about in
mere barbarism."

The blind man held out his hat, that flapped about at the door, as if it
were a bag in the lining that had come unnailed.

"This," said the chemist, "is a scrofulous affection."

And though he knew the poor devil, he pretended to see him for the first
time, murmured something about "cornea," "opaque cornea," "sclerotic,"
"facies," then asked him in a paternal tone--

"My friend, have you long had this terrible infirmity? Instead of
getting drunk at the public, you'd do better to die yourself."

He advised him to take good wine, good beer, and good joints. The blind
man went on with his song; he seemed, moreover, almost idiotic. At last
Monsieur Homais opened his purse--

"Now there's a sou; give me back two lairds, and don't forget my advice:
you'll be the better for it."

Hivert openly cast some doubt on the efficacy of it. But the druggist
said that he would cure himself with an antiphlogistic pomade of his own
composition, and he gave his address--"Monsieur Homais, near the market,
pretty well known."

"Now," said Hivert, "for all this trouble you'll give us your
performance."

The blind man sank down on his haunches, with his head thrown back,
whilst he rolled his greenish eyes, lolled out his tongue, and rubbed
his stomach with both hands as he uttered a kind of hollow yell like a
famished dog. Emma, filled with disgust, threw him over her shoulder
a five-franc piece. It was all her fortune. It seemed to her very fine
thus to throw it away.

The coach had gone on again when suddenly Monsieur Homais leant out
through the window, crying--

"No farinaceous or milk food, wear wool next the skin, and expose the
diseased parts to the smoke of juniper berries."

The sight of the well-known objects that defiled before her eyes
gradually diverted Emma from her present trouble. An intolerable fatigue
overwhelmed her, and she reached her home stupefied, discouraged, almost
asleep.

"Come what may come!" she said to herself. "And then, who knows? Why, at
any moment could not some extraordinary event occur? Lheureux even might
die!"

At nine o'clock in the morning she was awakened by the sound of voices
in the Place. There was a crowd round the market reading a large bill
fixed to one of the posts, and she saw Justin, who was climbing on to
a stone and tearing down the bill. But at this moment the rural guard
seized him by the collar. Monsieur Homais came out of his shop, and Mere
Lefrangois, in the midst of the crowd, seemed to be perorating.

"Madame! madame!" cried Felicite, running in, "it's abominable!"

And the poor girl, deeply moved, handed her a yellow paper that she had
just torn off the door. Emma read with a glance that all her furniture
was for sale.

Then they looked at one another silently. The servant and mistress had
no secret one from the other. At last Felicite sighed--

"If I were you, madame, I should go to Monsieur Guillaumin."

"Do you think--"

And this question meant to say--

"You who know the house through the servant, has the master spoken
sometimes of me?"

"Yes, you'd do well to go there."

She dressed, put on her black gown, and her hood with jet beads, and
that she might not be seen (there was still a crowd on the Place), she
took the path by the river, outside the village.

She reached the notary's gate quite breathless. The sky was sombre, and
a little snow was falling. At the sound of the bell, Theodore in a
red waistcoat appeared on the steps; he came to open the door almost
familiarly, as to an acquaintance, and showed her into the dining-room.

A large porcelain stove crackled beneath a cactus that filled up the
niche in the wall, and in black wood frames against the oak-stained
paper hung Steuben's "Esmeralda" and Schopin's "Potiphar." The
ready-laid table, the two silver chafing-dishes, the crystal door-knobs,
the parquet and the furniture, all shone with a scrupulous, English
cleanliness; the windows were ornamented at each corner with stained
glass.

"Now this," thought Emma, "is the dining-room I ought to have."

The notary came in pressing his palm-leaf dressing-gown to his breast
with his left arm, while with the other hand he raised and quickly put
on again his brown velvet cap, pretentiously cocked on the right side,
whence looked out the ends of three fair curls drawn from the back of
the head, following the line of his bald skull.

After he had offered her a seat he sat down to breakfast, apologising
profusely for his rudeness.

"I have come," she said, "to beg you, sir--"

"What, madame? I am listening."

And she began explaining her position to him. Monsieur Guillaumin knew
it, being secretly associated with the linendraper, from whom he always
got capital for the loans on mortgages that he was asked to make.

So he knew (and better than she herself) the long story of the bills,
small at first, bearing different names as endorsers, made out at long
dates, and constantly renewed up to the day, when, gathering together
all the protested bills, the shopkeeper had bidden his friend Vincart
take in his own name all the necessary proceedings, not wishing to pass
for a tiger with his fellow-citizens.

She mingled her story with recriminations against Lheureux, to which the
notary replied from time to time with some insignificant word. Eating
his cutlet and drinking his tea, he buried his chin in his sky-blue
cravat, into which were thrust two diamond pins, held together by a
small gold chain; and he smiled a singular smile, in a sugary, ambiguous
fashion. But noticing that her feet were damp, he said--

"Do get closer to the stove; put your feet up against the porcelain."

She was afraid of dirtying it. The notary replied in a gallant tone--

"Beautiful things spoil nothing."

Then she tried to move him, and, growing moved herself, she began
telling him about the poorness of her home, her worries, her wants.
He could understand that; an elegant woman! and, without leaving off
eating, he had turned completely round towards her, so that his knee
brushed against her boot, whose sole curled round as it smoked against
the stove.

But when she asked for a thousand sous, he closed his lips, and declared
he was very sorry he had not had the management of her fortune before,
for there were hundreds of ways very convenient, even for a lady, of
turning her money to account. They might, either in the turf-peats
of Grumesnil or building-ground at Havre, almost without risk, have
ventured on some excellent speculations; and he let her consume herself
with rage at the thought of the fabulous sums that she would certainly
have made.

"How was it," he went on, "that you didn't come to me?"

"I hardly know," she said.

"Why, hey? Did I frighten you so much? It is I, on the contrary, who
ought to complain. We hardly know one another; yet I am very devoted to
you. You do not doubt that, I hope?"

He held out his hand, took hers, covered it with a greedy kiss, then
held it on his knee; and he played delicately with her fingers whilst
he murmured a thousand blandishments. His insipid voice murmured like a
running brook; a light shone in his eyes through the glimmering of his
spectacles, and his hand was advancing up Emma's sleeve to press her
arm. She felt against her cheek his panting breath. This man oppressed
her horribly.

She sprang up and said to him--

"Sir, I am waiting."

"For what?" said the notary, who suddenly became very pale.

"This money."

"But--" Then, yielding to the outburst of too powerful a desire, "Well,
yes!"

He dragged himself towards her on his knees, regardless of his
dressing-gown.

"For pity's sake, stay. I love you!"

He seized her by her waist. Madame Bovary's face flushed purple. She
recoiled with a terrible look, crying--

"You are taking a shameless advantage of my distress, sir! I am to be
pitied--not to be sold."

And she went out.

The notary remained quite stupefied, his eyes fixed on his fine
embroidered slippers. They were a love gift, and the sight of them at
last consoled him. Besides, he reflected that such an adventure might
have carried him too far.

"What a wretch! what a scoundrel! what an infamy!" she said to herself,
as she fled with nervous steps beneath the aspens of the path. The
disappointment of her failure increased the indignation of her outraged
modesty; it seemed to her that Providence pursued her implacably, and,
strengthening herself in her pride, she had never felt so much esteem
for herself nor so much contempt for others. A spirit of warfare
transformed her. She would have liked to strike all men, to spit in
their faces, to crush them, and she walked rapidly straight on, pale,
quivering, maddened, searching the empty horizon with tear-dimmed eyes,
and as it were rejoicing in the hate that was choking her.

When she saw her house a numbness came over her. She could not go on;
and yet she must. Besides, whither could she flee?

Felicite was waiting for her at the door. "Well?"

"No!" said Emma.

And for a quarter of an hour the two of them went over the various
persons in Yonville who might perhaps be inclined to help her. But each
time that Felicite named someone Emma replied--

"Impossible! they will not!"

"And the master'll soon be in."

"I know that well enough. Leave me alone."

She had tried everything; there was nothing more to be done now; and
when Charles came in she would have to say to him--

"Go away! This carpet on which you are walking is no longer ours. In
your own house you do not possess a chair, a pin, a straw, and it is I,
poor man, who have ruined you."

Then there would be a great sob; next he would weep abundantly, and at
last, the surprise past, he would forgive her.

"Yes," she murmured, grinding her teeth, "he will forgive me, he who
would give a million if I would forgive him for having known me! Never!
never!"

This thought of Bovary's superiority to her exasperated her. Then,
whether she confessed or did not confess, presently, immediately,
to-morrow, he would know the catastrophe all the same; so she must wait
for this horrible scene, and bear the weight of his magnanimity. The
desire to return to Lheureux's seized her--what would be the use? To
write to her father--it was too late; and perhaps, she began to repent
now that she had not yielded to that other, when she heard the trot of
a horse in the alley. It was he; he was opening the gate; he was whiter
than the plaster wall. Rushing to the stairs, she ran out quickly to the
square; and the wife of the mayor, who was talking to Lestiboudois in
front of the church, saw her go in to the tax-collector's.

She hurried off to tell Madame Caron, and the two ladies went up to
the attic, and, hidden by some linen spread across props, stationed
themselves comfortably for overlooking the whole of Binet's room.

He was alone in his garret, busy imitating in wood one of those
indescribable bits of ivory, composed of crescents, of spheres hollowed
out one within the other, the whole as straight as an obelisk, and of no
use whatever; and he was beginning on the last piece--he was nearing his
goal. In the twilight of the workshop the white dust was flying from his
tools like a shower of sparks under the hoofs of a galloping horse; the
two wheels were turning, droning; Binet smiled, his chin lowered, his
nostrils distended, and, in a word, seemed lost in one of those complete
happinesses that, no doubt, belong only to commonplace occupations,
which amuse the mind with facile difficulties, and satisfy by a
realisation of that beyond which such minds have not a dream.

"Ah! there she is!" exclaimed Madame Tuvache.

But it was impossible because of the lathe to hear what she was saying.

At last these ladies thought they made out the word "francs," and Madame
Tuvache whispered in a low voice--

"She is begging him to give her time for paying her taxes."

"Apparently!" replied the other.

They saw her walking up and down, examining the napkin-rings, the
candlesticks, the banister rails against the walls, while Binet stroked
his beard with satisfaction.

"Do you think she wants to order something of him?" said Madame Tuvache.

"Why, he doesn't sell anything," objected her neighbour.

The tax-collector seemed to be listening with wide-open eyes, as if he
did not understand. She went on in a tender, suppliant manner. She came
nearer to him, her breast heaving; they no longer spoke.

"Is she making him advances?" said Madame Tuvache. Binet was scarlet to
his very ears. She took hold of his hands.

"Oh, it's too much!"

And no doubt she was suggesting something abominable to him; for the
tax-collector--yet he was brave, had fought at Bautzen and at Lutzen,
had been through the French campaign, and had even been recommended for
the cross--suddenly, as at the sight of a serpent, recoiled as far as he
could from her, crying--

"Madame! what do you mean?"

"Women like that ought to be whipped," said Madame Tuvache.

"But where is she?" continued Madame Caron, for she had disappeared
whilst they spoke; then catching sight of her going up the Grande Rue,
and turning to the right as if making for the cemetery, they were lost
in conjectures.

"Nurse Rollet," she said on reaching the nurse's, "I am choking; unlace
me!" She fell on the bed sobbing. Nurse Rollet covered her with a
petticoat and remained standing by her side. Then, as she did not
answer, the good woman withdrew, took her wheel and began spinning flax.

"Oh, leave off!" she murmured, fancying she heard Binet's lathe.

"What's bothering her?" said the nurse to herself. "Why has she come
here?"

She had rushed thither; impelled by a kind of horror that drove her from
her home.

Lying on her back, motionless, and with staring eyes, she saw things but
vaguely, although she tried to with idiotic persistence. She looked
at the scales on the walls, two brands smoking end to end, and a long
spider crawling over her head in a rent in the beam. At last she began
to collect her thoughts. She remembered--one day--Leon--Oh! how long
ago that was--the sun was shining on the river, and the clematis were
perfuming the air. Then, carried away as by a rushing torrent, she soon
began to recall the day before.

"What time is it?" she asked.

Mere Rollet went out, raised the fingers of her right hand to that side
of the sky that was brightest, and came back slowly, saying--

"Nearly three."

"Ah! thanks, thanks!"

For he would come; he would have found some money. But he would,
perhaps, go down yonder, not guessing she was here, and she told the
nurse to run to her house to fetch him.

"Be quick!"

"But, my dear lady, I'm going, I'm going!"

She wondered now that she had not thought of him from the first.
Yesterday he had given his word; he would not break it. And she already
saw herself at Lheureux's spreading out her three bank-notes on his
bureau. Then she would have to invent some story to explain matters to
Bovary. What should it be?

The nurse, however, was a long while gone. But, as there was no clock
in the cot, Emma feared she was perhaps exaggerating the length of time.
She began walking round the garden, step by step; she went into the path
by the hedge, and returned quickly, hoping that the woman would have
come back by another road. At last, weary of waiting, assailed by fears
that she thrust from her, no longer conscious whether she had been here
a century or a moment, she sat down in a corner, closed her eyes, and
stopped her ears. The gate grated; she sprang up. Before she had spoken
Mere Rollet said to her--

"There is no one at your house!"

"What?"

"Oh, no one! And the doctor is crying. He is calling for you; they're
looking for you."

Emma answered nothing. She gasped as she turned her eyes about
her, while the peasant woman, frightened at her face, drew back
instinctively, thinking her mad. Suddenly she struck her brow and
uttered a cry; for the thought of Rodolphe, like a flash of lightning in
a dark night, had passed into her soul. He was so good, so delicate, so
generous! And besides, should he hesitate to do her this service, she
would know well enough how to constrain him to it by re-waking, in a
single moment, their lost love. So she set out towards La Huchette, not
seeing that she was hastening to offer herself to that which but a while
ago had so angered her, not in the least conscious of her prostitution.



\chapter{Chapter Eight}

She asked herself as she walked along, "What am I going to say? How
shall I begin?" And as she went on she recognised the thickets,
the trees, the sea-rushes on the hill, the chateau yonder. All the
sensations of her first tenderness came back to her, and her poor aching
heart opened out amorously. A warm wind blew in her face; the melting
snow fell drop by drop from the buds to the grass.

She entered, as she used to, through the small park-gate. She reached
the avenue bordered by a double row of dense lime-trees. They were
swaying their long whispering branches to and fro. The dogs in their
kennels all barked, and the noise of their voices resounded, but brought
out no one.

She went up the large straight staircase with wooden balusters that led
to the corridor paved with dusty flags, into which several doors in a
row opened, as in a monastery or an inn. His was at the top, right
at the end, on the left. When she placed her fingers on the lock her
strength suddenly deserted her. She was afraid, almost wished he
would not be there, though this was her only hope, her last chance of
salvation. She collected her thoughts for one moment, and, strengthening
herself by the feeling of present necessity, went in.

He was in front of the fire, both his feet on the mantelpiece, smoking a
pipe.

"What! it is you!" he said, getting up hurriedly.

"Yes, it is I, Rodolphe. I should like to ask your advice."

And, despite all her efforts, it was impossible for her to open her
lips.

"You have not changed; you are charming as ever!"

"Oh," she replied bitterly, "they are poor charms since you disdained
them."

Then he began a long explanation of his conduct, excusing himself in
vague terms, in default of being able to invent better.

She yielded to his words, still more to his voice and the sight of him,
so that, she pretended to believe, or perhaps believed; in the pretext
he gave for their rupture; this was a secret on which depended the
honour, the very life of a third person.

"No matter!" she said, looking at him sadly. "I have suffered much."

He replied philosophically--

"Such is life!"

"Has life," Emma went on, "been good to you at least, since our
separation?"

"Oh, neither good nor bad."

"Perhaps it would have been better never to have parted."

"Yes, perhaps."

"You think so?" she said, drawing nearer, and she sighed. "Oh, Rodolphe!
if you but knew! I loved you so!"

It was then that she took his hand, and they remained some time, their
fingers intertwined, like that first day at the Show. With a gesture of
pride he struggled against this emotion. But sinking upon his breast she
said to him--

"How did you think I could live without you? One cannot lose the habit
of happiness. I was desolate. I thought I should die. I will tell you
about all that and you will see. And you--you fled from me!"

For, all the three years, he had carefully avoided her in consequence
of that natural cowardice that characterises the stronger sex. Emma went
on, with dainty little nods, more coaxing than an amorous kitten--

"You love others, confess it! Oh, I understand them, dear! I excuse
them. You probably seduced them as you seduced me. You are indeed a man;
you have everything to make one love you. But we'll begin again, won't
we? We will love one another. See! I am laughing; I am happy! Oh,
speak!"

And she was charming to see, with her eyes, in which trembled a tear,
like the rain of a storm in a blue corolla.

He had drawn her upon his knees, and with the back of his hand was
caressing her smooth hair, where in the twilight was mirrored like a
golden arrow one last ray of the sun. She bent down her brow; at last he
kissed her on the eyelids quite gently with the tips of his lips.

"Why, you have been crying! What for?"

She burst into tears. Rodolphe thought this was an outburst of her
love. As she did not speak, he took this silence for a last remnant of
resistance, and then he cried out--

"Oh, forgive me! You are the only one who pleases me. I was imbecile and
cruel. I love you. I will love you always. What is it. Tell me!" He was
kneeling by her.

"Well, I am ruined, Rodolphe! You must lend me three thousand francs."

"But--but--" said he, getting up slowly, while his face assumed a grave
expression.

"You know," she went on quickly, "that my husband had placed his whole
fortune at a notary's. He ran away. So we borrowed; the patients don't
pay us. Moreover, the settling of the estate is not yet done; we shall
have the money later on. But to-day, for want of three thousand francs,
we are to be sold up. It is to be at once, this very moment, and,
counting upon your friendship, I have come to you."

"Ah!" thought Rodolphe, turning very pale, "that was what she came for."
At last he said with a calm air--

"Dear madame, I have not got them."

He did not lie. If he had had them, he would, no doubt, have given them,
although it is generally disagreeable to do such fine things: a demand
for money being, of all the winds that blow upon love, the coldest and
most destructive.

First she looked at him for some moments.

"You have not got them!" she repeated several times. "You have not got
them! I ought to have spared myself this last shame. You never loved me.
You are no better than the others."

She was betraying, ruining herself.

Rodolphe interrupted her, declaring he was "hard up" himself.

"Ah! I pity you," said Emma. "Yes--very much."

And fixing her eyes upon an embossed carabine, that shone against its
panoply, "But when one is so poor one doesn't have silver on the butt of
one's gun. One doesn't buy a clock inlaid with tortoise shell," she went
on, pointing to a buhl timepiece, "nor silver-gilt whistles for one's
whips," and she touched them, "nor charms for one's watch. Oh, he wants
for nothing! even to a liqueur-stand in his room! For you love yourself;
you live well. You have a chateau, farms, woods; you go hunting; you
travel to Paris. Why, if it were but that," she cried, taking up two
studs from the mantelpiece, "but the least of these trifles, one can get
money for them. Oh, I do not want them, keep them!"

And she threw the two links away from her, their gold chain breaking as
it struck against the wall.

"But I! I would have given you everything. I would have sold all, worked
for you with my hands, I would have begged on the highroads for a smile,
for a look, to hear you say 'Thanks!' And you sit there quietly in your
arm-chair, as if you had not made me suffer enough already! But for you,
and you know it, I might have lived happily. What made you do it? Was
it a bet? Yet you loved me--you said so. And but a moment since--Ah!
it would have been better to have driven me away. My hands are hot with
your kisses, and there is the spot on the carpet where at my knees you
swore an eternity of love! You made me believe you; for two years you
held me in the most magnificent, the sweetest dream! Eh! Our plans for
the journey, do you remember? Oh, your letter! your letter! it tore my
heart! And then when I come back to him--to him, rich, happy, free--to
implore the help the first stranger would give, a suppliant, and
bringing back to him all my tenderness, he repulses me because it would
cost him three thousand francs!"

"I haven't got them," replied Rodolphe, with that perfect calm with
which resigned rage covers itself as with a shield.

She went out. The walls trembled, the ceiling was crushing her, and she
passed back through the long alley, stumbling against the heaps of dead
leaves scattered by the wind. At last she reached the ha-ha hedge in
front of the gate; she broke her nails against the lock in her haste to
open it. Then a hundred steps farther on, breathless, almost falling,
she stopped. And now turning round, she once more saw the impassive
chateau, with the park, the gardens, the three courts, and all the
windows of the facade.

She remained lost in stupor, and having no more consciousness of herself
than through the beating of her arteries, that she seemed to hear
bursting forth like a deafening music filling all the fields. The earth
beneath her feet was more yielding than the sea, and the furrows seemed
to her immense brown waves breaking into foam. Everything in her
head, of memories, ideas, went off at once like a thousand pieces of
fireworks. She saw her father, Lheureux's closet, their room at home,
another landscape. Madness was coming upon her; she grew afraid, and
managed to recover herself, in a confused way, it is true, for she did
not in the least remember the cause of the terrible condition she was
in, that is to say, the question of money. She suffered only in her
love, and felt her soul passing from her in this memory; as wounded men,
dying, feel their life ebb from their bleeding wounds.

Night was falling, crows were flying about.

Suddenly it seemed to her that fiery spheres were exploding in the air
like fulminating balls when they strike, and were whirling, whirling,
to melt at last upon the snow between the branches of the trees. In the
midst of each of them appeared the face of Rodolphe. They multiplied and
drew near her, penetrating, her. It all disappeared; she recognised the
lights of the houses that shone through the fog.

Now her situation, like an abyss, rose up before her. She was panting as
if her heart would burst. Then in an ecstasy of heroism, that made
her almost joyous, she ran down the hill, crossed the cow-plank, the
foot-path, the alley, the market, and reached the chemist's shop. She
was about to enter, but at the sound of the bell someone might come, and
slipping in by the gate, holding her breath, feeling her way along the
walls, she went as far as the door of the kitchen, where a candle stuck
on the stove was burning. Justin in his shirt-sleeves was carrying out a
dish.

"Ah! they are dining; I will wait."

He returned; she tapped at the window. He went out.

"The key! the one for upstairs where he keeps the--"

"What?"

And he looked at her, astonished at the pallor of her face, that stood
out white against the black background of the night. She seemed to
him extraordinarily beautiful and majestic as a phantom. Without
understanding what she wanted, he had the presentiment of something
terrible.

But she went on quickly in a love voice; in a sweet, melting voice, "I
want it; give it to me."

As the partition wall was thin, they could hear the clatter of the forks
on the plates in the dining-room.

She pretended that she wanted to kill the rats that kept her from
sleeping.

"I must tell master."

"No, stay!" Then with an indifferent air, "Oh, it's not worth while;
I'll tell him presently. Come, light me upstairs."

She entered the corridor into which the laboratory door opened. Against
the wall was a key labelled Capharnaum.

"Justin!" called the druggist impatiently.

"Let us go up."

And he followed her. The key turned in the lock, and she went straight
to the third shelf, so well did her memory guide her, seized the blue
jar, tore out the cork, plunged in her hand, and withdrawing it full of
a white powder, she began eating it.

"Stop!" he cried, rushing at her.

"Hush! someone will come."

He was in despair, was calling out.

"Say nothing, or all the blame will fall on your master."

Then she went home, suddenly calmed, and with something of the serenity
of one that had performed a duty.

When Charles, distracted by the news of the distraint, returned home,
Emma had just gone out. He cried aloud, wept, fainted, but she did not
return. Where could she be? He sent Felicite to Homais, to Monsieur
Tuvache, to Lheureux, to the "Lion d'Or," everywhere, and in the
intervals of his agony he saw his reputation destroyed, their fortune
lost, Berthe's future ruined. By what?--Not a word! He waited till six
in the evening. At last, unable to bear it any longer, and fancying she
had gone to Rouen, he set out along the highroad, walked a mile, met no
one, again waited, and returned home. She had come back.

"What was the matter? Why? Explain to me."

She sat down at her writing-table and wrote a letter, which she sealed
slowly, adding the date and the hour. Then she said in a solemn tone:

"You are to read it to-morrow; till then, I pray you, do not ask me a
single question. No, not one!"

"But--"

"Oh, leave me!"

She lay down full length on her bed. A bitter taste that she felt in her
mouth awakened her. She saw Charles, and again closed her eyes.

She was studying herself curiously, to see if she were not suffering.
But no! nothing as yet. She heard the ticking of the clock, the
crackling of the fire, and Charles breathing as he stood upright by her
bed.

"Ah! it is but a little thing, death!" she thought. "I shall fall asleep
and all will be over."

She drank a mouthful of water and turned to the wall. The frightful
taste of ink continued.

"I am thirsty; oh! so thirsty," she sighed.

"What is it?" said Charles, who was handing her a glass.

"It is nothing! Open the window; I am choking."

She was seized with a sickness so sudden that she had hardly time to
draw out her handkerchief from under the pillow.

"Take it away," she said quickly; "throw it away."

He spoke to her; she did not answer. She lay motionless, afraid that
the slightest movement might make her vomit. But she felt an icy cold
creeping from her feet to her heart.

"Ah! it is beginning," she murmured.

"What did you say?"

She turned her head from side to side with a gentle movement full of
agony, while constantly opening her mouth as if something very heavy
were weighing upon her tongue. At eight o'clock the vomiting began
again.

Charles noticed that at the bottom of the basin there was a sort of
white sediment sticking to the sides of the porcelain.

"This is extraordinary--very singular," he repeated.

But she said in a firm voice, "No, you are mistaken."

Then gently, and almost as caressing her, he passed his hand over her
stomach. She uttered a sharp cry. He fell back terror-stricken.

Then she began to groan, faintly at first. Her shoulders were shaken by
a strong shuddering, and she was growing paler than the sheets in which
her clenched fingers buried themselves. Her unequal pulse was now almost
imperceptible.

Drops of sweat oozed from her bluish face, that seemed as if rigid in
the exhalations of a metallic vapour. Her teeth chattered, her dilated
eyes looked vaguely about her, and to all questions she replied only
with a shake of the head; she even smiled once or twice. Gradually, her
moaning grew louder; a hollow shriek burst from her; she pretended she
was better and that she would get up presently. But she was seized with
convulsions and cried out--

"Ah! my God! It is horrible!"

He threw himself on his knees by her bed.

"Tell me! what have you eaten? Answer, for heaven's sake!"

And he looked at her with a tenderness in his eyes such as she had never
seen.

"Well, there--there!" she said in a faint voice. He flew to the
writing-table, tore open the seal, and read aloud: "Accuse no one." He
stopped, passed his hands across his eyes, and read it over again.

"What! help--help!"

He could only keep repeating the word: "Poisoned! poisoned!" Felicite
ran to Homais, who proclaimed it in the market-place; Madame Lefrancois
heard it at the "Lion d'Or"; some got up to go and tell their
neighbours, and all night the village was on the alert.

Distraught, faltering, reeling, Charles wandered about the room. He
knocked against the furniture, tore his hair, and the chemist had never
believed that there could be so terrible a sight.

He went home to write to Monsieur Canivet and to Doctor Lariviere. He
lost his head, and made more than fifteen rough copies. Hippolyte went
to Neufchatel, and Justin so spurred Bovary's horse that he left it
foundered and three parts dead by the hill at Bois-Guillaume.

Charles tried to look up his medical dictionary, but could not read it;
the lines were dancing.

"Be calm," said the druggist; "we have only to administer a powerful
antidote. What is the poison?"

Charles showed him the letter. It was arsenic.

"Very well," said Homais, "we must make an analysis."

For he knew that in cases of poisoning an analysis must be made; and the
other, who did not understand, answered--

"Oh, do anything! save her!"

Then going back to her, he sank upon the carpet, and lay there with his
head leaning against the edge of her bed, sobbing.

"Don't cry," she said to him. "Soon I shall not trouble you any more."

"Why was it? Who drove you to it?"

She replied. "It had to be, my dear!"

"Weren't you happy? Is it my fault? I did all I could!"

"Yes, that is true--you are good--you."

And she passed her hand slowly over his hair. The sweetness of this
sensation deepened his sadness; he felt his whole being dissolving
in despair at the thought that he must lose her, just when she was
confessing more love for him than ever. And he could think of nothing;
he did not know, he did not dare; the urgent need for some immediate
resolution gave the finishing stroke to the turmoil of his mind.

So she had done, she thought, with all the treachery; and meanness,
and numberless desires that had tortured her. She hated no one now; a
twilight dimness was settling upon her thoughts, and, of all earthly
noises, Emma heard none but the intermittent lamentations of this poor
heart, sweet and indistinct like the echo of a symphony dying away.

"Bring me the child," she said, raising herself on her elbow.

"You are not worse, are you?" asked Charles.

"No, no!"

The child, serious, and still half-asleep, was carried in on the
servant's arm in her long white nightgown, from which her bare
feet peeped out. She looked wonderingly at the disordered room, and
half-closed her eyes, dazzled by the candles burning on the table. They
reminded her, no doubt, of the morning of New Year's day and Mid-Lent,
when thus awakened early by candle-light she came to her mother's bed to
fetch her presents, for she began saying--

"But where is it, mamma?" And as everybody was silent, "But I can't see
my little stocking."

Felicite held her over the bed while she still kept looking towards the
mantelpiece.

"Has nurse taken it?" she asked.

And at this name, that carried her back to the memory of her adulteries
and her calamities, Madame Bovary turned away her head, as at the
loathing of another bitterer poison that rose to her mouth. But Berthe
remained perched on the bed.

"Oh, how big your eyes are, mamma! How pale you are! how hot you are!"

Her mother looked at her. "I am frightened!" cried the child, recoiling.

Emma took her hand to kiss it; the child struggled.

"That will do. Take her away," cried Charles, who was sobbing in the
alcove.

Then the symptoms ceased for a moment; she seemed less agitated; and at
every insignificant word, at every respiration a little more easy, he
regained hope. At last, when Canivet came in, he threw himself into his
arms.

"Ah! it is you. Thanks! You are good! But she is better. See! look at
her."

His colleague was by no means of this opinion, and, as he said of
himself, "never beating about the bush," he prescribed, an emetic in
order to empty the stomach completely.

She soon began vomiting blood. Her lips became drawn. Her limbs were
convulsed, her whole body covered with brown spots, and her pulse
slipped beneath the fingers like a stretched thread, like a harp-string
nearly breaking.

After this she began to scream horribly. She cursed the poison, railed
at it, and implored it to be quick, and thrust away with her stiffened
arms everything that Charles, in more agony than herself, tried to make
her drink. He stood up, his handkerchief to his lips, with a rattling
sound in his throat, weeping, and choked by sobs that shook his whole
body. Felicite was running hither and thither in the room. Homais,
motionless, uttered great sighs; and Monsieur Canivet, always retaining
his self-command, nevertheless began to feel uneasy.

"The devil! yet she has been purged, and from the moment that the cause
ceases--"

"The effect must cease," said Homais, "that is evident."

"Oh, save her!" cried Bovary.

And, without listening to the chemist, who was still venturing the
hypothesis, "It is perhaps a salutary paroxysm," Canivet was about to
administer some theriac, when they heard the cracking of a whip; all the
windows rattled, and a post-chaise drawn by three horses abreast, up to
their ears in mud, drove at a gallop round the corner of the market. It
was Doctor Lariviere.

The apparition of a god would not have caused more commotion. Bovary
raised his hands; Canivet stopped short; and Homais pulled off his
skull-cap long before the doctor had come in.

He belonged to that great school of surgery begotten of Bichat, to that
generation, now extinct, of philosophical practitioners, who, loving
their art with a fanatical love, exercised it with enthusiasm and
wisdom. Everyone in his hospital trembled when he was angry; and his
students so revered him that they tried, as soon as they were themselves
in practice, to imitate him as much as possible. So that in all the
towns about they were found wearing his long wadded merino overcoat
and black frock-coat, whose buttoned cuffs slightly covered his brawny
hands--very beautiful hands, and that never knew gloves, as though to be
more ready to plunge into suffering. Disdainful of honours, of titles,
and of academies, like one of the old Knight-Hospitallers, generous,
fatherly to the poor, and practising virtue without believing in it, he
would almost have passed for a saint if the keenness of his intellect
had not caused him to be feared as a demon. His glance, more penetrating
than his bistouries, looked straight into your soul, and dissected every
lie athwart all assertions and all reticences. And thus he went along,
full of that debonair majesty that is given by the consciousness
of great talent, of fortune, and of forty years of a labourious and
irreproachable life.

He frowned as soon as he had passed the door when he saw the cadaverous
face of Emma stretched out on her back with her mouth open. Then, while
apparently listening to Canivet, he rubbed his fingers up and down
beneath his nostrils, and repeated--

"Good! good!"

But he made a slow gesture with his shoulders. Bovary watched him; they
looked at one another; and this man, accustomed as he was to the sight
of pain, could not keep back a tear that fell on his shirt-frill.

He tried to take Canivet into the next room. Charles followed him.

"She is very ill, isn't she? If we put on sinapisms? Anything! Oh, think
of something, you who have saved so many!"

Charles caught him in both his arms, and gazed at him wildly,
imploringly, half-fainting against his breast.

"Come, my poor fellow, courage! There is nothing more to be done."

And Doctor Lariviere turned away.

"You are going?"

"I will come back."

He went out only to give an order to the coachman, with Monsieur
Canivet, who did not care either to have Emma die under his hands.

The chemist rejoined them on the Place. He could not by temperament keep
away from celebrities, so he begged Monsieur Lariviere to do him the
signal honour of accepting some breakfast.

He sent quickly to the "Lion d'Or" for some pigeons; to the butcher's
for all the cutlets that were to be had; to Tuvache for cream; and
to Lestiboudois for eggs; and the druggist himself aided in the
preparations, while Madame Homais was saying as she pulled together the
strings of her jacket--

"You must excuse us, sir, for in this poor place, when one hasn't been
told the night before--"

"Wine glasses!" whispered Homais.

"If only we were in town, we could fall back upon stuffed trotters."

"Be quiet! Sit down, doctor!"

He thought fit, after the first few mouthfuls, to give some details as
to the catastrophe.

"We first had a feeling of siccity in the pharynx, then intolerable
pains at the epigastrium, super purgation, coma."

"But how did she poison herself?"

"I don't know, doctor, and I don't even know where she can have procured
the arsenious acid."

Justin, who was just bringing in a pile of plates, began to tremble.

"What's the matter?" said the chemist.

At this question the young man dropped the whole lot on the ground with
a crash.

"Imbecile!" cried Homais, "awkward lout! block-head! confounded ass!"

But suddenly controlling himself--

"I wished, doctor, to make an analysis, and primo I delicately
introduced a tube--"

"You would have done better," said the physician, "to introduce your
fingers into her throat."

His colleague was silent, having just before privately received a severe
lecture about his emetic, so that this good Canivet, so arrogant and so
verbose at the time of the clubfoot, was to-day very modest. He smiled
without ceasing in an approving manner.

Homais dilated in Amphytrionic pride, and the affecting thought of
Bovary vaguely contributed to his pleasure by a kind of egotistic
reflex upon himself. Then the presence of the doctor transported him.
He displayed his erudition, cited pell-mell cantharides, upas, the
manchineel, vipers.

"I have even read that various persons have found themselves
under toxicological symptoms, and, as it were, thunderstricken by
black-pudding that had been subjected to a too vehement fumigation.
At least, this was stated in a very fine report drawn up by one of our
pharmaceutical chiefs, one of our masters, the illustrious Cadet de
Gassicourt!"

Madame Homais reappeared, carrying one of those shaky machines that
are heated with spirits of wine; for Homais liked to make his coffee
at table, having, moreover, torrefied it, pulverised it, and mixed it
himself.

"Saccharum, doctor?" said he, offering the sugar.

Then he had all his children brought down, anxious to have the
physician's opinion on their constitutions.

At last Monsieur Lariviere was about to leave, when Madame Homais asked
for a consultation about her husband. He was making his blood too thick
by going to sleep every evening after dinner.

"Oh, it isn't his blood that's too thick," said the physician.

And, smiling a little at his unnoticed joke, the doctor opened the
door. But the chemist's shop was full of people; he had the greatest
difficulty in getting rid of Monsieur Tuvache, who feared his spouse
would get inflammation of the lungs, because she was in the habit of
spitting on the ashes; then of Monsieur Binet, who sometimes experienced
sudden attacks of great hunger; and of Madame Caron, who suffered
from tinglings; of Lheureux, who had vertigo; of Lestiboudois, who had
rheumatism; and of Madame Lefrancois, who had heartburn. At last the
three horses started; and it was the general opinion that he had not
shown himself at all obliging.

Public attention was distracted by the appearance of Monsieur
Bournisien, who was going across the market with the holy oil.

Homais, as was due to his principles, compared priests to ravens
attracted by the odour of death. The sight of an ecclesiastic was
personally disagreeable to him, for the cassock made him think of the
shroud, and he detested the one from some fear of the other.

Nevertheless, not shrinking from what he called his mission, he returned
to Bovary's in company with Canivet whom Monsieur Lariviere, before
leaving, had strongly urged to make this visit; and he would, but for
his wife's objections, have taken his two sons with him, in order
to accustom them to great occasions; that this might be a lesson, an
example, a solemn picture, that should remain in their heads later on.

The room when they went in was full of mournful solemnity. On the
work-table, covered over with a white cloth, there were five or six
small balls of cotton in a silver dish, near a large crucifix between
two lighted candles.

Emma, her chin sunken upon her breast, had her eyes inordinately wide
open, and her poor hands wandered over the sheets with that hideous
and soft movement of the dying, that seems as if they wanted already to
cover themselves with the shroud. Pale as a statue and with eyes red as
fire, Charles, not weeping, stood opposite her at the foot of the bed,
while the priest, bending one knee, was muttering words in a low voice.

She turned her face slowly, and seemed filled with joy on seeing
suddenly the violet stole, no doubt finding again, in the midst of
a temporary lull in her pain, the lost voluptuousness of her first
mystical transports, with the visions of eternal beatitude that were
beginning.

The priest rose to take the crucifix; then she stretched forward her
neck as one who is athirst, and glueing her lips to the body of the
Man-God, she pressed upon it with all her expiring strength the fullest
kiss of love that she had ever given. Then he recited the Misereatur and
the Indulgentiam, dipped his right thumb in the oil, and began to give
extreme unction. First upon the eyes, that had so coveted all worldly
pomp; then upon the nostrils, that had been greedy of the warm breeze
and amorous odours; then upon the mouth, that had uttered lies, that had
curled with pride and cried out in lewdness; then upon the hands that
had delighted in sensual touches; and finally upon the soles of the
feet, so swift of yore, when she was running to satisfy her desires, and
that would now walk no more.

The cure wiped his fingers, threw the bit of cotton dipped in oil into
the fire, and came and sat down by the dying woman, to tell her that
she must now blend her sufferings with those of Jesus Christ and abandon
herself to the divine mercy.

Finishing his exhortations, he tried to place in her hand a blessed
candle, symbol of the celestial glory with which she was soon to be
surrounded. Emma, too weak, could not close her fingers, and the taper,
but for Monsieur Bournisien would have fallen to the ground.

However, she was not quite so pale, and her face had an expression of
serenity as if the sacrament had cured her.

The priest did not fail to point this out; he even explained to Bovary
that the Lord sometimes prolonged the life of persons when he thought it
meet for their salvation; and Charles remembered the day when, so near
death, she had received the communion. Perhaps there was no need to
despair, he thought.

In fact, she looked around her slowly, as one awakening from a dream;
then in a distinct voice she asked for her looking-glass, and remained
some time bending over it, until the big tears fell from her eyes. Then
she turned away her head with a sigh and fell back upon the pillows.

Her chest soon began panting rapidly; the whole of her tongue protruded
from her mouth; her eyes, as they rolled, grew paler, like the two
globes of a lamp that is going out, so that one might have thought
her already dead but for the fearful labouring of her ribs, shaken
by violent breathing, as if the soul were struggling to free itself.
Felicite knelt down before the crucifix, and the druggist himself
slightly bent his knees, while Monsieur Canivet looked out vaguely at
the Place. Bournisien had again begun to pray, his face bowed against
the edge of the bed, his long black cassock trailing behind him in the
room. Charles was on the other side, on his knees, his arms outstretched
towards Emma. He had taken her hands and pressed them, shuddering at
every beat of her heart, as at the shaking of a falling ruin. As the
death-rattle became stronger the priest prayed faster; his prayers
mingled with the stifled sobs of Bovary, and sometimes all seemed lost
in the muffled murmur of the Latin syllables that tolled like a passing
bell.

Suddenly on the pavement was heard a loud noise of clogs and the
clattering of a stick; and a voice rose--a raucous voice--that sang--

"Maids in the warmth of a summer day Dream of love and of love always"

Emma raised herself like a galvanised corpse, her hair undone, her eyes
fixed, staring.

"Where the sickle blades have been, Nannette, gathering ears of corn,
Passes bending down, my queen, To the earth where they were born."

"The blind man!" she cried. And Emma began to laugh, an atrocious,
frantic, despairing laugh, thinking she saw the hideous face of the poor
wretch that stood out against the eternal night like a menace.

"The wind is strong this summer day, Her petticoat has flown away."

She fell back upon the mattress in a convulsion. They all drew near. She
was dead.



\chapter{Chapter Nine}

There is always after the death of anyone a kind of stupefaction;
so difficult is it to grasp this advent of nothingness and to resign
ourselves to believe in it. But still, when he saw that she did not
move, Charles threw himself upon her, crying--

"Farewell! farewell!"

Homais and Canivet dragged him from the room.

"Restrain yourself!"

"Yes." said he, struggling, "I'll be quiet. I'll not do anything. But
leave me alone. I want to see her. She is my wife!"

And he wept.

"Cry," said the chemist; "let nature take her course; that will solace
you."

Weaker than a child, Charles let himself be led downstairs into the
sitting-room, and Monsieur Homais soon went home. On the Place he
was accosted by the blind man, who, having dragged himself as far as
Yonville, in the hope of getting the antiphlogistic pomade, was asking
every passer-by where the druggist lived.

"There now! as if I hadn't got other fish to fry. Well, so much the
worse; you must come later on."

And he entered the shop hurriedly.

He had to write two letters, to prepare a soothing potion for Bovary, to
invent some lie that would conceal the poisoning, and work it up into an
article for the "Fanal," without counting the people who were waiting to
get the news from him; and when the Yonvillers had all heard his story
of the arsenic that she had mistaken for sugar in making a vanilla
cream. Homais once more returned to Bovary's.

He found him alone (Monsieur Canivet had left), sitting in an arm-chair
near the window, staring with an idiotic look at the flags of the floor.

"Now," said the chemist, "you ought yourself to fix the hour for the
ceremony."

"Why? What ceremony?" Then, in a stammering, frightened voice, "Oh, no!
not that. No! I want to see her here."

Homais, to keep himself in countenance, took up a water-bottle on the
whatnot to water the geraniums.

"Ah! thanks," said Charles; "you are good."

But he did not finish, choking beneath the crowd of memories that this
action of the druggist recalled to him.

Then to distract him, Homais thought fit to talk a little horticulture:
plants wanted humidity. Charles bowed his head in sign of approbation.

"Besides, the fine days will soon be here again."

"Ah!" said Bovary.

The druggist, at his wit's end, began softly to draw aside the small
window-curtain.

"Hallo! there's Monsieur Tuvache passing."

Charles repeated like a machine---

"Monsieur Tuvache passing!"

Homais did not dare to speak to him again about the funeral
arrangements; it was the priest who succeeded in reconciling him to
them.

He shut himself up in his consulting-room, took a pen, and after sobbing
for some time, wrote--

"I wish her to be buried in her wedding-dress, with white shoes, and a
wreath. Her hair is to be spread out over her shoulders. Three coffins,
one of oak, one of mahogany, one of lead. Let no one say anything to me.
I shall have strength. Over all there is to be placed a large piece of
green velvet. This is my wish; see that it is done."

The two men were much surprised at Bovary's romantic ideas. The chemist
at once went to him and said--

"This velvet seems to me a superfetation. Besides, the expense--"

"What's that to you?" cried Charles. "Leave me! You did not love her.
Go!"

The priest took him by the arm for a turn in the garden. He discoursed
on the vanity of earthly things. God was very great, was very good: one
must submit to his decrees without a murmur; nay, must even thank him.

Charles burst out into blasphemies: "I hate your God!"

"The spirit of rebellion is still upon you," sighed the ecclesiastic.

Bovary was far away. He was walking with great strides along by the
wall, near the espalier, and he ground his teeth; he raised to heaven
looks of malediction, but not so much as a leaf stirred.

A fine rain was falling: Charles, whose chest was bare, at last began to
shiver; he went in and sat down in the kitchen.

At six o'clock a noise like a clatter of old iron was heard on the
Place; it was the "Hirondelle" coming in, and he remained with his
forehead against the windowpane, watching all the passengers get
out, one after the other. Felicite put down a mattress for him in the
drawing-room. He threw himself upon it and fell asleep.

Although a philosopher, Monsieur Homais respected the dead. So bearing
no grudge to poor Charles, he came back again in the evening to sit up
with the body; bringing with him three volumes and a pocket-book for
taking notes.

Monsieur Bournisien was there, and two large candles were burning at the
head of the bed, that had been taken out of the alcove. The druggist, on
whom the silence weighed, was not long before he began formulating some
regrets about this "unfortunate young woman." and the priest replied
that there was nothing to do now but pray for her.

"Yet," Homais went on, "one of two things; either she died in a state of
grace (as the Church has it), and then she has no need of our prayers;
or else she departed impertinent (that is, I believe, the ecclesiastical
expression), and then--"

Bournisien interrupted him, replying testily that it was none the less
necessary to pray.

"But," objected the chemist, "since God knows all our needs, what can be
the good of prayer?"

"What!" cried the ecclesiastic, "prayer! Why, aren't you a Christian?"

"Excuse me," said Homais; "I admire Christianity. To begin with, it
enfranchised the slaves, introduced into the world a morality--"

"That isn't the question. All the texts-"

"Oh! oh! As to texts, look at history; it, is known that all the texts
have been falsified by the Jesuits."

Charles came in, and advancing towards the bed, slowly drew the
curtains.

Emma's head was turned towards her right shoulder, the corner of her
mouth, which was open, seemed like a black hole at the lower part of her
face; her two thumbs were bent into the palms of her hands; a kind
of white dust besprinkled her lashes, and her eyes were beginning to
disappear in that viscous pallor that looks like a thin web, as if
spiders had spun it over. The sheet sunk in from her breast to her
knees, and then rose at the tips of her toes, and it seemed to Charles
that infinite masses, an enormous load, were weighing upon her.

The church clock struck two. They could hear the loud murmur of the
river flowing in the darkness at the foot of the terrace. Monsieur
Bournisien from time to time blew his nose noisily, and Homais' pen was
scratching over the paper.

"Come, my good friend," he said, "withdraw; this spectacle is tearing
you to pieces."

Charles once gone, the chemist and the cure recommenced their
discussions.

"Read Voltaire," said the one, "read D'Holbach, read the
'Encyclopaedia'!"

"Read the 'Letters of some Portuguese Jews,'" said the other; "read 'The
Meaning of Christianity,' by Nicolas, formerly a magistrate."

They grew warm, they grew red, they both talked at once without
listening to each other. Bournisien was scandalized at such audacity;
Homais marvelled at such stupidity; and they were on the point of
insulting one another when Charles suddenly reappeared. A fascination
drew him. He was continually coming upstairs.

He stood opposite her, the better to see her, and he lost himself in a
contemplation so deep that it was no longer painful.

He recalled stories of catalepsy, the marvels of magnetism, and he
said to himself that by willing it with all his force he might perhaps
succeed in reviving her. Once he even bent towards he, and cried in a
low voice, "Emma! Emma!" His strong breathing made the flames of the
candles tremble against the wall.

At daybreak Madame Bovary senior arrived. Charles as he embraced her
burst into another flood of tears. She tried, as the chemist had done,
to make some remarks to him on the expenses of the funeral. He became so
angry that she was silent, and he even commissioned her to go to town at
once and buy what was necessary.

Charles remained alone the whole afternoon; they had taken Berthe
to Madame Homais'; Felicite was in the room upstairs with Madame
Lefrancois.

In the evening he had some visitors. He rose, pressed their hands,
unable to speak. Then they sat down near one another, and formed a large
semicircle in front of the fire. With lowered faces, and swinging one
leg crossed over the other knee, they uttered deep sighs at intervals;
each one was inordinately bored, and yet none would be the first to go.

Homais, when he returned at nine o'clock (for the last two days only
Homais seemed to have been on the Place), was laden with a stock of
camphor, of benzine, and aromatic herbs. He also carried a large jar
full of chlorine water, to keep off all miasmata. Just then the servant,
Madame Lefrancois, and Madame Bovary senior were busy about Emma,
finishing dressing her, and they were drawing down the long stiff veil
that covered her to her satin shoes.

Felicite was sobbing--"Ah! my poor mistress! my poor mistress!"

"Look at her," said the landlady, sighing; "how pretty she still is!
Now, couldn't you swear she was going to get up in a minute?"

Then they bent over her to put on her wreath. They had to raise the head
a little, and a rush of black liquid issued, as if she were vomiting,
from her mouth.

"Oh, goodness! The dress; take care!" cried Madame Lefrancois. "Now,
just come and help," she said to the chemist. "Perhaps you're afraid?"

"I afraid?" replied he, shrugging his shoulders. "I dare say! I've seen
all sorts of things at the hospital when I was studying pharmacy. We
used to make punch in the dissecting room! Nothingness does not terrify
a philosopher; and, as I often say, I even intend to leave my body to
the hospitals, in order, later on, to serve science."

The cure on his arrival inquired how Monsieur Bovary was, and, on
the reply of the druggist, went on--"The blow, you see, is still too
recent."

Then Homais congratulated him on not being exposed, like other people,
to the loss of a beloved companion; whence there followed a discussion
on the celibacy of priests.

"For," said the chemist, "it is unnatural that a man should do without
women! There have been crimes--"

"But, good heaven!" cried the ecclesiastic, "how do you expect an
individual who is married to keep the secrets of the confessional, for
example?"

Homais fell foul of the confessional. Bournisien defended it; he
enlarged on the acts of restitution that it brought about. He cited
various anecdotes about thieves who had suddenly become honest. Military
men on approaching the tribunal of penitence had felt the scales fall
from their eyes. At Fribourg there was a minister--

His companion was asleep. Then he felt somewhat stifled by the
over-heavy atmosphere of the room; he opened the window; this awoke the
chemist.

"Come, take a pinch of snuff," he said to him. "Take it; it'll relieve
you."

A continual barking was heard in the distance. "Do you hear that dog
howling?" said the chemist.

"They smell the dead," replied the priest. "It's like bees; they leave
their hives on the decease of any person."

Homais made no remark upon these prejudices, for he had again dropped
asleep. Monsieur Bournisien, stronger than he, went on moving his lips
gently for some time, then insensibly his chin sank down, he let fall
his big black boot, and began to snore.

They sat opposite one another, with protruding stomachs, puffed-up
faces, and frowning looks, after so much disagreement uniting at last in
the same human weakness, and they moved no more than the corpse by their
side, that seemed to be sleeping.

Charles coming in did not wake them. It was the last time; he came to
bid her farewell.

The aromatic herbs were still smoking, and spirals of bluish vapour
blended at the window-sash with the fog that was coming in. There were
few stars, and the night was warm. The wax of the candles fell in great
drops upon the sheets of the bed. Charles watched them burn, tiring his
eyes against the glare of their yellow flame.

The watering on the satin gown shimmered white as moonlight. Emma was
lost beneath it; and it seemed to him that, spreading beyond her own
self, she blended confusedly with everything around her--the silence,
the night, the passing wind, the damp odours rising from the ground.

Then suddenly he saw her in the garden at Tostes, on a bench against the
thorn hedge, or else at Rouen in the streets, on the threshold of their
house, in the yard at Bertaux. He again heard the laughter of the happy
boys beneath the apple-trees: the room was filled with the perfume
of her hair; and her dress rustled in his arms with a noise like
electricity. The dress was still the same.

For a long while he thus recalled all his lost joys, her attitudes,
her movements, the sound of her voice. Upon one fit of despair followed
another, and even others, inexhaustible as the waves of an overflowing
sea.

A terrible curiosity seized him. Slowly, with the tips of his fingers,
palpitating, he lifted her veil. But he uttered a cry of horror that
awoke the other two.

They dragged him down into the sitting-room. Then Felicite came up to
say that he wanted some of her hair.

"Cut some off," replied the druggist.

And as she did not dare to, he himself stepped forward, scissors in
hand. He trembled so that he pierced the skin of the temple in several
places. At last, stiffening himself against emotion, Homais gave two
or three great cuts at random that left white patches amongst that
beautiful black hair.

The chemist and the cure plunged anew into their occupations, not
without sleeping from time to time, of which they accused each other
reciprocally at each fresh awakening. Then Monsieur Bournisien sprinkled
the room with holy water and Homais threw a little chlorine water on the
floor.

Felicite had taken care to put on the chest of drawers, for each
of them, a bottle of brandy, some cheese, and a large roll. And the
druggist, who could not hold out any longer, about four in the morning
sighed--

"My word! I should like to take some sustenance."

The priest did not need any persuading; he went out to go and say mass,
came back, and then they ate and hobnobbed, giggling a little without
knowing why, stimulated by that vague gaiety that comes upon us after
times of sadness, and at the last glass the priest said to the druggist,
as he clapped him on the shoulder--

"We shall end by understanding one another."

In the passage downstairs they met the undertaker's men, who were coming
in. Then Charles for two hours had to suffer the torture of hearing the
hammer resound against the wood. Next day they lowered her into her
oak coffin, that was fitted into the other two; but as the bier was
too large, they had to fill up the gaps with the wool of a mattress. At
last, when the three lids had been planed down, nailed, soldered, it was
placed outside in front of the door; the house was thrown open, and the
people of Yonville began to flock round.

Old Rouault arrived, and fainted on the Place when he saw the black
cloth!



\chapter{Chapter Ten}

He had only received the chemist's letter thirty-six hours after the
event; and, from consideration for his feelings, Homais had so worded it
that it was impossible to make out what it was all about.

First, the old fellow had fallen as if struck by apoplexy. Next, he
understood that she was not dead, but she might be. At last, he had put
on his blouse, taken his hat, fastened his spurs to his boots, and set
out at full speed; and the whole of the way old Rouault, panting, was
torn by anguish. Once even he was obliged to dismount. He was dizzy; he
heard voices round about him; he felt himself going mad.

Day broke. He saw three black hens asleep in a tree. He shuddered,
horrified at this omen. Then he promised the Holy Virgin three chasubles
for the church, and that he would go barefooted from the cemetery at
Bertaux to the chapel of Vassonville.

He entered Maromme shouting for the people of the inn, burst open the
door with a thrust of his shoulder, made for a sack of oats, emptied a
bottle of sweet cider into the manger, and again mounted his nag, whose
feet struck fire as it dashed along.

He said to himself that no doubt they would save her; the doctors would
discover some remedy surely. He remembered all the miraculous cures
he had been told about. Then she appeared to him dead. She was there;
before his eyes, lying on her back in the middle of the road. He reined
up, and the hallucination disappeared.

At Quincampoix, to give himself heart, he drank three cups of coffee
one after the other. He fancied they had made a mistake in the name in
writing. He looked for the letter in his pocket, felt it there, but did
not dare to open it.

At last he began to think it was all a joke; someone's spite, the jest
of some wag; and besides, if she were dead, one would have known it. But
no! There was nothing extraordinary about the country; the sky was blue,
the trees swayed; a flock of sheep passed. He saw the village; he was
seen coming bending forward upon his horse, belabouring it with great
blows, the girths dripping with blood.

When he had recovered consciousness, he fell, weeping, into Bovary's
arms: "My girl! Emma! my child! tell me--"

The other replied, sobbing, "I don't know! I don't know! It's a curse!"

The druggist separated them. "These horrible details are useless. I will
tell this gentleman all about it. Here are the people coming. Dignity!
Come now! Philosophy!"

The poor fellow tried to show himself brave, and repeated several times.
"Yes! courage!"

"Oh," cried the old man, "so I will have, by God! I'll go along o' her
to the end!"

The bell began tolling. All was ready; they had to start. And seated in
a stall of the choir, side by side, they saw pass and repass in front of
them continually the three chanting choristers.

The serpent-player was blowing with all his might. Monsieur Bournisien,
in full vestments, was singing in a shrill voice. He bowed before the
tabernacle, raising his hands, stretched out his arms. Lestiboudois
went about the church with his whalebone stick. The bier stood near the
lectern, between four rows of candles. Charles felt inclined to get up
and put them out.

Yet he tried to stir himself to a feeling of devotion, to throw himself
into the hope of a future life in which he should see her again. He
imagined to himself she had gone on a long journey, far away, for a long
time. But when he thought of her lying there, and that all was over,
that they would lay her in the earth, he was seized with a fierce,
gloomy, despairful rage. At times he thought he felt nothing more, and
he enjoyed this lull in his pain, whilst at the same time he reproached
himself for being a wretch.

The sharp noise of an iron-ferruled stick was heard on the stones,
striking them at irregular intervals. It came from the end of the
church, and stopped short at the lower aisles. A man in a coarse brown
jacket knelt down painfully. It was Hippolyte, the stable-boy at the
"Lion d'Or." He had put on his new leg.

One of the choristers went round the nave making a collection, and the
coppers chinked one after the other on the silver plate.

"Oh, make haste! I am in pain!" cried Bovary, angrily throwing him a
five-franc piece. The churchman thanked him with a deep bow.

They sang, they knelt, they stood up; it was endless! He remembered that
once, in the early times, they had been to mass together, and they had
sat down on the other side, on the right, by the wall. The bell began
again. There was a great moving of chairs; the bearers slipped their
three staves under the coffin, and everyone left the church.

Then Justin appeared at the door of the shop. He suddenly went in again,
pale, staggering.

People were at the windows to see the procession pass. Charles at the
head walked erect. He affected a brave air, and saluted with a nod those
who, coming out from the lanes or from their doors, stood amidst the
crowd.

The six men, three on either side, walked slowly, panting a little.
The priests, the choristers, and the two choirboys recited the De
profundis*, and their voices echoed over the fields, rising and falling
with their undulations. Sometimes they disappeared in the windings of
the path; but the great silver cross rose always before the trees.

     *Psalm CXXX.


The women followed in black cloaks with turned-down hoods; each of them
carried in her hands a large lighted candle, and Charles felt himself
growing weaker at this continual repetition of prayers and torches,
beneath this oppressive odour of wax and of cassocks. A fresh breeze was
blowing; the rye and colza were sprouting, little dewdrops trembled at
the roadsides and on the hawthorn hedges. All sorts of joyous sounds
filled the air; the jolting of a cart rolling afar off in the ruts, the
crowing of a cock, repeated again and again, or the gambling of a foal
running away under the apple-trees: The pure sky was fretted with rosy
clouds; a bluish haze rested upon the cots covered with iris. Charles as
he passed recognised each courtyard. He remembered mornings like this,
when, after visiting some patient, he came out from one and returned to
her.

The black cloth bestrewn with white beads blew up from time to time,
laying bare the coffin. The tired bearers walked more slowly, and it
advanced with constant jerks, like a boat that pitches with every wave.

They reached the cemetery. The men went right down to a place in the
grass where a grave was dug. They ranged themselves all round; and while
the priest spoke, the red soil thrown up at the sides kept noiselessly
slipping down at the corners.

Then when the four ropes were arranged the coffin was placed upon them.
He watched it descend; it seemed descending for ever. At last a thud was
heard; the ropes creaked as they were drawn up. Then Bournisien took
the spade handed to him by Lestiboudois; with his left hand all the
time sprinkling water, with the right he vigorously threw in a large
spadeful; and the wood of the coffin, struck by the pebbles, gave forth
that dread sound that seems to us the reverberation of eternity.

The ecclesiastic passed the holy water sprinkler to his neighbour. This
was Homais. He swung it gravely, then handed it to Charles, who sank to
his knees in the earth and threw in handfuls of it, crying, "Adieu!" He
sent her kisses; he dragged himself towards the grave, to engulf himself
with her. They led him away, and he soon grew calmer, feeling perhaps,
like the others, a vague satisfaction that it was all over.

Old Rouault on his way back began quietly smoking a pipe, which Homais
in his innermost conscience thought not quite the thing. He also noticed
that Monsieur Binet had not been present, and that Tuvache had "made
off" after mass, and that Theodore, the notary's servant wore a blue
coat, "as if one could not have got a black coat, since that is the
custom, by Jove!" And to share his observations with others he went from
group to group. They were deploring Emma's death, especially Lheureux,
who had not failed to come to the funeral.

"Poor little woman! What a trouble for her husband!"

The druggist continued, "Do you know that but for me he would have
committed some fatal attempt upon himself?"

"Such a good woman! To think that I saw her only last Saturday in my
shop."

"I haven't had leisure," said Homais, "to prepare a few words that I
would have cast upon her tomb."

Charles on getting home undressed, and old Rouault put on his blue
blouse. It was a new one, and as he had often during the journey wiped
his eyes on the sleeves, the dye had stained his face, and the traces of
tears made lines in the layer of dust that covered it.

Madame Bovary senior was with them. All three were silent. At last the
old fellow sighed--

"Do you remember, my friend, that I went to Tostes once when you had
just lost your first deceased? I consoled you at that time. I thought of
something to say then, but now--" Then, with a loud groan that shook his
whole chest, "Ah! this is the end for me, do you see! I saw my wife go,
then my son, and now to-day it's my daughter."

He wanted to go back at once to Bertaux, saying that he could not sleep
in this house. He even refused to see his granddaughter.

"No, no! It would grieve me too much. Only you'll kiss her many times
for me. Good-bye! you're a good fellow! And then I shall never forget
that," he said, slapping his thigh. "Never fear, you shall always have
your turkey."

But when he reached the top of the hill he turned back, as he had turned
once before on the road of Saint-Victor when he had parted from her. The
windows of the village were all on fire beneath the slanting rays of the
sun sinking behind the field. He put his hand over his eyes, and saw
in the horizon an enclosure of walls, where trees here and there formed
black clusters between white stones; then he went on his way at a gentle
trot, for his nag had gone lame.

Despite their fatigue, Charles and his mother stayed very long that
evening talking together. They spoke of the days of the past and of the
future. She would come to live at Yonville; she would keep house for
him; they would never part again. She was ingenious and caressing,
rejoicing in her heart at gaining once more an affection that had
wandered from her for so many years. Midnight struck. The village as
usual was silent, and Charles, awake, thought always of her.

Rodolphe, who, to distract himself, had been rambling about the wood all
day, was sleeping quietly in his chateau, and Leon, down yonder, always
slept.

There was another who at that hour was not asleep.

On the grave between the pine-trees a child was on his knees weeping,
and his heart, rent by sobs, was beating in the shadow beneath the load
of an immense regret, sweeter than the moon and fathomless as the night.
The gate suddenly grated. It was Lestiboudois; he came to fetch his
spade, that he had forgotten. He recognised Justin climbing over the
wall, and at last knew who was the culprit who stole his potatoes.



\chapter{Chapter Eleven}

The next day Charles had the child brought back. She asked for her
mamma. They told her she was away; that she would bring her back some
playthings. Berthe spoke of her again several times, then at last
thought no more of her. The child's gaiety broke Bovary's heart, and he
had to bear besides the intolerable consolations of the chemist.

Money troubles soon began again, Monsieur Lheureux urging on anew his
friend Vincart, and Charles pledged himself for exorbitant sums; for he
would never consent to let the smallest of the things that had belonged
to HER be sold. His mother was exasperated with him; he grew even more
angry than she did. He had altogether changed. She left the house.

Then everyone began "taking advantage" of him. Mademoiselle Lempereur
presented a bill for six months' teaching, although Emma had never taken
a lesson (despite the receipted bill she had shown Bovary); it was an
arrangement between the two women. The man at the circulating library
demanded three years' subscriptions; Mere Rollet claimed the postage due
for some twenty letters, and when Charles asked for an explanation, she
had the delicacy to reply--

"Oh, I don't know. It was for her business affairs."

With every debt he paid Charles thought he had come to the end of them.
But others followed ceaselessly. He sent in accounts for professional
attendance. He was shown the letters his wife had written. Then he had
to apologise.

Felicite now wore Madame Bovary's gowns; not all, for he had kept some
of them, and he went to look at them in her dressing-room, locking
himself up there; she was about her height, and often Charles, seeing
her from behind, was seized with an illusion, and cried out--

"Oh, stay, stay!"

But at Whitsuntide she ran away from Yonville, carried off by Theodore,
stealing all that was left of the wardrobe.

It was about this time that the widow Dupuis had the honour to inform
him of the "marriage of Monsieur Leon Dupuis her son, notary at Yvetot,
to Mademoiselle Leocadie Leboeuf of Bondeville." Charles, among the
other congratulations he sent him, wrote this sentence--

"How glad my poor wife would have been!"

One day when, wandering aimlessly about the house, he had gone up to the
attic, he felt a pellet of fine paper under his slipper. He opened it
and read: "Courage, Emma, courage. I would not bring misery into your
life." It was Rodolphe's letter, fallen to the ground between the boxes,
where it had remained, and that the wind from the dormer window had just
blown towards the door. And Charles stood, motionless and staring, in
the very same place where, long ago, Emma, in despair, and paler even
than he, had thought of dying. At last he discovered a small R at the
bottom of the second page. What did this mean? He remembered Rodolphe's
attentions, his sudden, disappearance, his constrained air when they
had met two or three times since. But the respectful tone of the letter
deceived him.

"Perhaps they loved one another platonically," he said to himself.

Besides, Charles was not of those who go to the bottom of things; he
shrank from the proofs, and his vague jealousy was lost in the immensity
of his woe.

Everyone, he thought, must have adored her; all men assuredly must have
coveted her. She seemed but the more beautiful to him for this; he
was seized with a lasting, furious desire for her, that inflamed his
despair, and that was boundless, because it was now unrealisable.

To please her, as if she were still living, he adopted her
predilections, her ideas; he bought patent leather boots and took to
wearing white cravats. He put cosmetics on his moustache, and, like her,
signed notes of hand. She corrupted him from beyond the grave.

He was obliged to sell his silver piece by piece; next he sold the
drawing-room furniture. All the rooms were stripped; but the bedroom,
her own room, remained as before. After his dinner Charles went up
there. He pushed the round table in front of the fire, and drew up her
armchair. He sat down opposite it. A candle burnt in one of the gilt
candlesticks. Berthe by his side was painting prints.

He suffered, poor man, at seeing her so badly dressed, with laceless
boots, and the arm-holes of her pinafore torn down to the hips; for the
charwoman took no care of her. But she was so sweet, so pretty, and her
little head bent forward so gracefully, letting the dear fair hair fall
over her rosy cheeks, that an infinite joy came upon him, a happiness
mingled with bitterness, like those ill-made wines that taste of
resin. He mended her toys, made her puppets from cardboard, or sewed up
half-torn dolls. Then, if his eyes fell upon the workbox, a ribbon lying
about, or even a pin left in a crack of the table, he began to dream,
and looked so sad that she became as sad as he.

No one now came to see them, for Justin had run away to Rouen, where he
was a grocer's assistant, and the druggist's children saw less and less
of the child, Monsieur Homais not caring, seeing the difference of their
social position, to continue the intimacy.

The blind man, whom he had not been able to cure with the pomade, had
gone back to the hill of Bois-Guillaume, where he told the travellers of
the vain attempt of the druggist, to such an extent, that Homais when
he went to town hid himself behind the curtains of the "Hirondelle" to
avoid meeting him. He detested him, and wishing, in the interests of his
own reputation, to get rid of him at all costs, he directed against
him a secret battery, that betrayed the depth of his intellect and the
baseness of his vanity. Thus, for six consecutive months, one could read
in the "Fanal de Rouen" editorials such as these--

"All who bend their steps towards the fertile plains of Picardy have, no
doubt, remarked, by the Bois-Guillaume hill, a wretch suffering from
a horrible facial wound. He importunes, persecutes one, and levies a
regular tax on all travellers. Are we still living in the monstrous
times of the Middle Ages, when vagabonds were permitted to display in
our public places leprosy and scrofulas they had brought back from the
Crusades?"

Or--

"In spite of the laws against vagabondage, the approaches to our great
towns continue to be infected by bands of beggars. Some are seen going
about alone, and these are not, perhaps, the least dangerous. What are
our ediles about?"

Then Homais invented anecdotes--

"Yesterday, by the Bois-Guillaume hill, a skittish horse--" And then
followed the story of an accident caused by the presence of the blind
man.

He managed so well that the fellow was locked up. But he was released.
He began again, and Homais began again. It was a struggle. Homais won
it, for his foe was condemned to life-long confinement in an asylum.

This success emboldened him, and henceforth there was no longer a dog
run over, a barn burnt down, a woman beaten in the parish, of which
he did not immediately inform the public, guided always by the love of
progress and the hate of priests. He instituted comparisons between the
elementary and clerical schools to the detriment of the latter; called
to mind the massacre of St. Bartholomew a propos of a grant of one
hundred francs to the church, and denounced abuses, aired new views.
That was his phrase. Homais was digging and delving; he was becoming
dangerous.

However, he was stifling in the narrow limits of journalism, and soon a
book, a work was necessary to him. Then he composed "General Statistics
of the Canton of Yonville, followed by Climatological Remarks." The
statistics drove him to philosophy. He busied himself with great
questions: the social problem, moralisation of the poorer classes,
pisciculture, caoutchouc, railways, etc. He even began to blush at being
a bourgeois. He affected the artistic style, he smoked. He bought two
chic Pompadour statuettes to adorn his drawing-room.

He by no means gave up his shop. On the contrary, he kept well abreast
of new discoveries. He followed the great movement of chocolates; he
was the first to introduce "cocoa" and "revalenta" into the
Seine-Inferieure. He was enthusiastic about the hydro-electric
Pulvermacher chains; he wore one himself, and when at night he took off
his flannel vest, Madame Homais stood quite dazzled before the golden
spiral beneath which he was hidden, and felt her ardour redouble for
this man more bandaged than a Scythian, and splendid as one of the Magi.

He had fine ideas about Emma's tomb. First he proposed a broken column
with some drapery, next a pyramid, then a Temple of Vesta, a sort of
rotunda, or else a "mass of ruins." And in all his plans Homais always
stuck to the weeping willow, which he looked upon as the indispensable
symbol of sorrow.

Charles and he made a journey to Rouen together to look at some tombs
at a funeral furnisher's, accompanied by an artist, one Vaufrylard, a
friend of Bridoux's, who made puns all the time. At last, after having
examined some hundred designs, having ordered an estimate and made
another journey to Rouen, Charles decided in favour of a mausoleum,
which on the two principal sides was to have a "spirit bearing an
extinguished torch."

As to the inscription, Homais could think of nothing so fine as Sta
viator*, and he got no further; he racked his brain, he constantly
repeated Sta viator. At last he hit upon Amabilen conjugem calcas**,
which was adopted.

     * Rest traveler.

     ** Tread upon a loving wife.

A strange thing was that Bovary, while continually thinking of Emma, was
forgetting her. He grew desperate as he felt this image fading from his
memory in spite of all efforts to retain it. Yet every night he dreamt
of her; it was always the same dream. He drew near her, but when he was
about to clasp her she fell into decay in his arms.

For a week he was seen going to church in the evening. Monsieur
Bournisien even paid him two or three visits, then gave him up.
Moreover, the old fellow was growing intolerant, fanatic, said Homais.
He thundered against the spirit of the age, and never failed, every
other week, in his sermon, to recount the death agony of Voltaire, who
died devouring his excrements, as everyone knows.

In spite of the economy with which Bovary lived, he was far from being
able to pay off his old debts. Lheureux refused to renew any more
bills. A distraint became imminent. Then he appealed to his mother, who
consented to let him take a mortgage on her property, but with a great
many recriminations against Emma; and in return for her sacrifice she
asked for a shawl that had escaped the depredations of Felicite. Charles
refused to give it her; they quarrelled.

She made the first overtures of reconciliation by offering to have the
little girl, who could help her in the house, to live with her. Charles
consented to this, but when the time for parting came, all his courage
failed him. Then there was a final, complete rupture.

As his affections vanished, he clung more closely to the love of his
child. She made him anxious, however, for she coughed sometimes, and had
red spots on her cheeks.

Opposite his house, flourishing and merry, was the family of the
chemist, with whom everything was prospering. Napoleon helped him in the
laboratory, Athalie embroidered him a skullcap, Irma cut out rounds of
paper to cover the preserves, and Franklin recited Pythagoras' table in
a breath. He was the happiest of fathers, the most fortunate of men.

Not so! A secret ambition devoured him. Homais hankered after the cross
of the Legion of Honour. He had plenty of claims to it.

"First, having at the time of the cholera distinguished myself by a
boundless devotion; second, by having published, at my expense,
various works of public utility, such as" (and he recalled his pamphlet
entitled, "Cider, its manufacture and effects," besides observation
on the lanigerous plant-louse, sent to the Academy; his volume of
statistics, and down to his pharmaceutical thesis); "without counting
that I am a member of several learned societies" (he was member of a
single one).

"In short!" he cried, making a pirouette, "if it were only for
distinguishing myself at fires!"

Then Homais inclined towards the Government. He secretly did the
prefect great service during the elections. He sold himself--in a word,
prostituted himself. He even addressed a petition to the sovereign
in which he implored him to "do him justice"; he called him "our good
king," and compared him to Henri IV.

And every morning the druggist rushed for the paper to see if his
nomination were in it. It was never there. At last, unable to bear it
any longer, he had a grass plot in his garden designed to represent the
Star of the Cross of Honour with two little strips of grass running from
the top to imitate the ribband. He walked round it with folded arms,
meditating on the folly of the Government and the ingratitude of men.

From respect, or from a sort of sensuality that made him carry on his
investigations slowly, Charles had not yet opened the secret drawer of
a rosewood desk which Emma had generally used. One day, however, he
sat down before it, turned the key, and pressed the spring. All Leon's
letters were there. There could be no doubt this time. He devoured them
to the very last, ransacked every corner, all the furniture, all the
drawers, behind the walls, sobbing, crying aloud, distraught, mad. He
found a box and broke it open with a kick. Rodolphe's portrait flew full
in his face in the midst of the overturned love-letters.

People wondered at his despondency. He never went out, saw no one,
refused even to visit his patients. Then they said "he shut himself up
to drink."

Sometimes, however, some curious person climbed on to the garden hedge,
and saw with amazement this long-bearded, shabbily clothed, wild man,
who wept aloud as he walked up and down.

In the evening in summer he took his little girl with him and led her to
the cemetery. They came back at nightfall, when the only light left in
the Place was that in Binet's window.

The voluptuousness of his grief was, however, incomplete, for he had no
one near him to share it, and he paid visits to Madame Lefrancois to be
able to speak of her.

But the landlady only listened with half an ear, having troubles
like himself. For Lheureux had at last established the "Favorites du
Commerce," and Hivert, who enjoyed a great reputation for doing errands,
insisted on a rise of wages, and was threatening to go over "to the
opposition shop."

One day when he had gone to the market at Argueil to sell his horse--his
last resource--he met Rodolphe.

They both turned pale when they caught sight of one another. Rodolphe,
who had only sent his card, first stammered some apologies, then grew
bolder, and even pushed his assurance (it was in the month of August and
very hot) to the length of inviting him to have a bottle of beer at the
public-house.

Leaning on the table opposite him, he chewed his cigar as he talked, and
Charles was lost in reverie at this face that she had loved. He seemed
to see again something of her in it. It was a marvel to him. He would
have liked to have been this man.

The other went on talking agriculture, cattle, pasturage, filling out
with banal phrases all the gaps where an allusion might slip in. Charles
was not listening to him; Rodolphe noticed it, and he followed the
succession of memories that crossed his face. This gradually grew
redder; the nostrils throbbed fast, the lips quivered. There was at
last a moment when Charles, full of a sombre fury, fixed his eyes on
Rodolphe, who, in something of fear, stopped talking. But soon the same
look of weary lassitude came back to his face.

"I don't blame you," he said.

Rodolphe was dumb. And Charles, his head in his hands, went on in a
broken voice, and with the resigned accent of infinite sorrow--

"No, I don't blame you now."

He even added a fine phrase, the only one he ever made--

"It is the fault of fatality!"

Rodolphe, who had managed the fatality, thought the remark very offhand
from a man in his position, comic even, and a little mean.

The next day Charles went to sit down on the seat in the arbour. Rays
of light were straying through the trellis, the vine leaves threw their
shadows on the sand, the jasmines perfumed the air, the heavens were
blue, Spanish flies buzzed round the lilies in bloom, and Charles was
suffocating like a youth beneath the vague love influences that filled
his aching heart.

At seven o'clock little Berthe, who had not seen him all the afternoon,
went to fetch him to dinner.

His head was thrown back against the wall, his eyes closed, his mouth
open, and in his hand was a long tress of black hair.

"Come along, papa," she said.

And thinking he wanted to play; she pushed him gently. He fell to the
ground. He was dead.

Thirty-six hours after, at the druggist's request, Monsieur Canivet came
thither. He made a post-mortem and found nothing.

When everything had been sold, twelve francs seventy-five centimes
remained, that served to pay for Mademoiselle Bovary's going to
her grandmother. The good woman died the same year; old Rouault was
paralysed, and it was an aunt who took charge of her. She is poor, and
sends her to a cotton-factory to earn a living.

Since Bovary's death three doctors have followed one another at Yonville
without any success, so severely did Homais attack them. He has an
enormous practice; the authorities treat him with consideration, and
public opinion protects him.

He has just received the cross of the Legion of Honour.


\end{document}


End of the Project Gutenberg EBook of Madame Bovary, by Gustave Flaubert

*** END OF THIS PROJECT GUTENBERG EBOOK MADAME BOVARY ***

***** This file should be named 2413.txt or 2413.zip *****
This and all associated files of various formats will be found in:
        http://www.gutenberg.org/2/4/1/2413/

Produced by An Anonymous Volunteer, Noah Adams and David Widger

Updated editions will replace the previous one--the old editions
will be renamed.

Creating the works from public domain print editions means that no
one owns a United States copyright in these works, so the Foundation
(and you!) can copy and distribute it in the United States without
permission and without paying copyright royalties.  Special rules,
set forth in the General Terms of Use part of this license, apply to
copying and distributing Project Gutenberg-tm electronic works to
protect the PROJECT GUTENBERG-tm concept and trademark.  Project
Gutenberg is a registered trademark, and may not be used if you
charge for the eBooks, unless you receive specific permission.  If you
do not charge anything for copies of this eBook, complying with the
rules is very easy.  You may use this eBook for nearly any purpose
such as creation of derivative works, reports, performances and
research.  They may be modified and printed and given away--you may do
practically ANYTHING with public domain eBooks.  Redistribution is
subject to the trademark license, especially commercial
redistribution.



*** START: FULL LICENSE ***

THE FULL PROJECT GUTENBERG LICENSE
PLEASE READ THIS BEFORE YOU DISTRIBUTE OR USE THIS WORK

To protect the Project Gutenberg-tm mission of promoting the free
distribution of electronic works, by using or distributing this work
(or any other work associated in any way with the phrase "Project
Gutenberg"), you agree to comply with all the terms of the Full Project
Gutenberg-tm License (available with this file or online at
http://gutenberg.org/license).


Section 1.  General Terms of Use and Redistributing Project Gutenberg-tm
electronic works

1.A.  By reading or using any part of this Project Gutenberg-tm
electronic work, you indicate that you have read, understand, agree to
and accept all the terms of this license and intellectual property
(trademark/copyright) agreement.  If you do not agree to abide by all
the terms of this agreement, you must cease using and return or destroy
all copies of Project Gutenberg-tm electronic works in your possession.
If you paid a fee for obtaining a copy of or access to a Project
Gutenberg-tm electronic work and you do not agree to be bound by the
terms of this agreement, you may obtain a refund from the person or
entity to whom you paid the fee as set forth in paragraph 1.E.8.

1.B.  "Project Gutenberg" is a registered trademark.  It may only be
used on or associated in any way with an electronic work by people who
agree to be bound by the terms of this agreement.  There are a few
things that you can do with most Project Gutenberg-tm electronic works
even without complying with the full terms of this agreement.  See
paragraph 1.C below.  There are a lot of things you can do with Project
Gutenberg-tm electronic works if you follow the terms of this agreement
and help preserve free future access to Project Gutenberg-tm electronic
works.  See paragraph 1.E below.

1.C.  The Project Gutenberg Literary Archive Foundation ("the Foundation"
or PGLAF), owns a compilation copyright in the collection of Project
Gutenberg-tm electronic works.  Nearly all the individual works in the
collection are in the public domain in the United States.  If an
individual work is in the public domain in the United States and you are
located in the United States, we do not claim a right to prevent you from
copying, distributing, performing, displaying or creating derivative
works based on the work as long as all references to Project Gutenberg
are removed.  Of course, we hope that you will support the Project
Gutenberg-tm mission of promoting free access to electronic works by
freely sharing Project Gutenberg-tm works in compliance with the terms of
this agreement for keeping the Project Gutenberg-tm name associated with
the work.  You can easily comply with the terms of this agreement by
keeping this work in the same format with its attached full Project
Gutenberg-tm License when you share it without charge with others.

1.D.  The copyright laws of the place where you are located also govern
what you can do with this work.  Copyright laws in most countries are in
a constant state of change.  If you are outside the United States, check
the laws of your country in addition to the terms of this agreement
before downloading, copying, displaying, performing, distributing or
creating derivative works based on this work or any other Project
Gutenberg-tm work.  The Foundation makes no representations concerning
the copyright status of any work in any country outside the United
States.

1.E.  Unless you have removed all references to Project Gutenberg:

1.E.1.  The following sentence, with active links to, or other immediate
access to, the full Project Gutenberg-tm License must appear prominently
whenever any copy of a Project Gutenberg-tm work (any work on which the
phrase "Project Gutenberg" appears, or with which the phrase "Project
Gutenberg" is associated) is accessed, displayed, performed, viewed,
copied or distributed:

This eBook is for the use of anyone anywhere at no cost and with
almost no restrictions whatsoever.  You may copy it, give it away or
re-use it under the terms of the Project Gutenberg License included
with this eBook or online at www.gutenberg.org

1.E.2.  If an individual Project Gutenberg-tm electronic work is derived
from the public domain (does not contain a notice indicating that it is
posted with permission of the copyright holder), the work can be copied
and distributed to anyone in the United States without paying any fees
or charges.  If you are redistributing or providing access to a work
with the phrase "Project Gutenberg" associated with or appearing on the
work, you must comply either with the requirements of paragraphs 1.E.1
through 1.E.7 or obtain permission for the use of the work and the
Project Gutenberg-tm trademark as set forth in paragraphs 1.E.8 or
1.E.9.

1.E.3.  If an individual Project Gutenberg-tm electronic work is posted
with the permission of the copyright holder, your use and distribution
must comply with both paragraphs 1.E.1 through 1.E.7 and any additional
terms imposed by the copyright holder.  Additional terms will be linked
to the Project Gutenberg-tm License for all works posted with the
permission of the copyright holder found at the beginning of this work.

1.E.4.  Do not unlink or detach or remove the full Project Gutenberg-tm
License terms from this work, or any files containing a part of this
work or any other work associated with Project Gutenberg-tm.

1.E.5.  Do not copy, display, perform, distribute or redistribute this
electronic work, or any part of this electronic work, without
prominently displaying the sentence set forth in paragraph 1.E.1 with
active links or immediate access to the full terms of the Project
Gutenberg-tm License.

1.E.6.  You may convert to and distribute this work in any binary,
compressed, marked up, nonproprietary or proprietary form, including any
word processing or hypertext form.  However, if you provide access to or
distribute copies of a Project Gutenberg-tm work in a format other than
"Plain Vanilla ASCII" or other format used in the official version
posted on the official Project Gutenberg-tm web site (www.gutenberg.org),
you must, at no additional cost, fee or expense to the user, provide a
copy, a means of exporting a copy, or a means of obtaining a copy upon
request, of the work in its original "Plain Vanilla ASCII" or other
form.  Any alternate format must include the full Project Gutenberg-tm
License as specified in paragraph 1.E.1.

1.E.7.  Do not charge a fee for access to, viewing, displaying,
performing, copying or distributing any Project Gutenberg-tm works
unless you comply with paragraph 1.E.8 or 1.E.9.

1.E.8.  You may charge a reasonable fee for copies of or providing
access to or distributing Project Gutenberg-tm electronic works provided
that

- You pay a royalty fee of 20% of the gross profits you derive from
     the use of Project Gutenberg-tm works calculated using the method
     you already use to calculate your applicable taxes.  The fee is
     owed to the owner of the Project Gutenberg-tm trademark, but he
     has agreed to donate royalties under this paragraph to the
     Project Gutenberg Literary Archive Foundation.  Royalty payments
     must be paid within 60 days following each date on which you
     prepare (or are legally required to prepare) your periodic tax
     returns.  Royalty payments should be clearly marked as such and
     sent to the Project Gutenberg Literary Archive Foundation at the
     address specified in Section 4, "Information about donations to
     the Project Gutenberg Literary Archive Foundation."

- You provide a full refund of any money paid by a user who notifies
     you in writing (or by e-mail) within 30 days of receipt that s/he
     does not agree to the terms of the full Project Gutenberg-tm
     License.  You must require such a user to return or
     destroy all copies of the works possessed in a physical medium
     and discontinue all use of and all access to other copies of
     Project Gutenberg-tm works.

- You provide, in accordance with paragraph 1.F.3, a full refund of any
     money paid for a work or a replacement copy, if a defect in the
     electronic work is discovered and reported to you within 90 days
     of receipt of the work.

- You comply with all other terms of this agreement for free
     distribution of Project Gutenberg-tm works.

1.E.9.  If you wish to charge a fee or distribute a Project Gutenberg-tm
electronic work or group of works on different terms than are set
forth in this agreement, you must obtain permission in writing from
both the Project Gutenberg Literary Archive Foundation and Michael
Hart, the owner of the Project Gutenberg-tm trademark.  Contact the
Foundation as set forth in Section 3 below.

1.F.

1.F.1.  Project Gutenberg volunteers and employees expend considerable
effort to identify, do copyright research on, transcribe and proofread
public domain works in creating the Project Gutenberg-tm
collection.  Despite these efforts, Project Gutenberg-tm electronic
works, and the medium on which they may be stored, may contain
"Defects," such as, but not limited to, incomplete, inaccurate or
corrupt data, transcription errors, a copyright or other intellectual
property infringement, a defective or damaged disk or other medium, a
computer virus, or computer codes that damage or cannot be read by
your equipment.

1.F.2.  LIMITED WARRANTY, DISCLAIMER OF DAMAGES - Except for the "Right
of Replacement or Refund" described in paragraph 1.F.3, the Project
Gutenberg Literary Archive Foundation, the owner of the Project
Gutenberg-tm trademark, and any other party distributing a Project
Gutenberg-tm electronic work under this agreement, disclaim all
liability to you for damages, costs and expenses, including legal
fees.  YOU AGREE THAT YOU HAVE NO REMEDIES FOR NEGLIGENCE, STRICT
LIABILITY, BREACH OF WARRANTY OR BREACH OF CONTRACT EXCEPT THOSE
PROVIDED IN PARAGRAPH F3.  YOU AGREE THAT THE FOUNDATION, THE
TRADEMARK OWNER, AND ANY DISTRIBUTOR UNDER THIS AGREEMENT WILL NOT BE
LIABLE TO YOU FOR ACTUAL, DIRECT, INDIRECT, CONSEQUENTIAL, PUNITIVE OR
INCIDENTAL DAMAGES EVEN IF YOU GIVE NOTICE OF THE POSSIBILITY OF SUCH
DAMAGE.

1.F.3.  LIMITED RIGHT OF REPLACEMENT OR REFUND - If you discover a
defect in this electronic work within 90 days of receiving it, you can
receive a refund of the money (if any) you paid for it by sending a
written explanation to the person you received the work from.  If you
received the work on a physical medium, you must return the medium with
your written explanation.  The person or entity that provided you with
the defective work may elect to provide a replacement copy in lieu of a
refund.  If you received the work electronically, the person or entity
providing it to you may choose to give you a second opportunity to
receive the work electronically in lieu of a refund.  If the second copy
is also defective, you may demand a refund in writing without further
opportunities to fix the problem.

1.F.4.  Except for the limited right of replacement or refund set forth
in paragraph 1.F.3, this work is provided to you 'AS-IS' WITH NO OTHER
WARRANTIES OF ANY KIND, EXPRESS OR IMPLIED, INCLUDING BUT NOT LIMITED TO
WARRANTIES OF MERCHANTIBILITY OR FITNESS FOR ANY PURPOSE.

1.F.5.  Some states do not allow disclaimers of certain implied
warranties or the exclusion or limitation of certain types of damages.
If any disclaimer or limitation set forth in this agreement violates the
law of the state applicable to this agreement, the agreement shall be
interpreted to make the maximum disclaimer or limitation permitted by
the applicable state law.  The invalidity or unenforceability of any
provision of this agreement shall not void the remaining provisions.

1.F.6.  INDEMNITY - You agree to indemnify and hold the Foundation, the
trademark owner, any agent or employee of the Foundation, anyone
providing copies of Project Gutenberg-tm electronic works in accordance
with this agreement, and any volunteers associated with the production,
promotion and distribution of Project Gutenberg-tm electronic works,
harmless from all liability, costs and expenses, including legal fees,
that arise directly or indirectly from any of the following which you do
or cause to occur: (a) distribution of this or any Project Gutenberg-tm
work, (b) alteration, modification, or additions or deletions to any
Project Gutenberg-tm work, and (c) any Defect you cause.


Section  2.  Information about the Mission of Project Gutenberg-tm

Project Gutenberg-tm is synonymous with the free distribution of
electronic works in formats readable by the widest variety of computers
including obsolete, old, middle-aged and new computers.  It exists
because of the efforts of hundreds of volunteers and donations from
people in all walks of life.

Volunteers and financial support to provide volunteers with the
assistance they need, is critical to reaching Project Gutenberg-tm's
goals and ensuring that the Project Gutenberg-tm collection will
remain freely available for generations to come.  In 2001, the Project
Gutenberg Literary Archive Foundation was created to provide a secure
and permanent future for Project Gutenberg-tm and future generations.
To learn more about the Project Gutenberg Literary Archive Foundation
and how your efforts and donations can help, see Sections 3 and 4
and the Foundation web page at http://www.pglaf.org.


Section 3.  Information about the Project Gutenberg Literary Archive
Foundation

The Project Gutenberg Literary Archive Foundation is a non profit
501(c)(3) educational corporation organized under the laws of the
state of Mississippi and granted tax exempt status by the Internal
Revenue Service.  The Foundation's EIN or federal tax identification
number is 64-6221541.  Its 501(c)(3) letter is posted at
http://pglaf.org/fundraising.  Contributions to the Project Gutenberg
Literary Archive Foundation are tax deductible to the full extent
permitted by U.S. federal laws and your state's laws.

The Foundation's principal office is located at 4557 Melan Dr. S.
Fairbanks, AK, 99712., but its volunteers and employees are scattered
throughout numerous locations.  Its business office is located at
809 North 1500 West, Salt Lake City, UT 84116, (801) 596-1887, email
business@pglaf.org.  Email contact links and up to date contact
information can be found at the Foundation's web site and official
page at http://pglaf.org

For additional contact information:
     Dr. Gregory B. Newby
     Chief Executive and Director
     gbnewby@pglaf.org


Section 4.  Information about Donations to the Project Gutenberg
Literary Archive Foundation

Project Gutenberg-tm depends upon and cannot survive without wide
spread public support and donations to carry out its mission of
increasing the number of public domain and licensed works that can be
freely distributed in machine readable form accessible by the widest
array of equipment including outdated equipment.  Many small donations
($1 to $5,000) are particularly important to maintaining tax exempt
status with the IRS.

The Foundation is committed to complying with the laws regulating
charities and charitable donations in all 50 states of the United
States.  Compliance requirements are not uniform and it takes a
considerable effort, much paperwork and many fees to meet and keep up
with these requirements.  We do not solicit donations in locations
where we have not received written confirmation of compliance.  To
SEND DONATIONS or determine the status of compliance for any
particular state visit http://pglaf.org

While we cannot and do not solicit contributions from states where we
have not met the solicitation requirements, we know of no prohibition
against accepting unsolicited donations from donors in such states who
approach us with offers to donate.

International donations are gratefully accepted, but we cannot make
any statements concerning tax treatment of donations received from
outside the United States.  U.S. laws alone swamp our small staff.

Please check the Project Gutenberg Web pages for current donation
methods and addresses.  Donations are accepted in a number of other
ways including checks, online payments and credit card donations.
To donate, please visit: http://pglaf.org/donate


Section 5.  General Information About Project Gutenberg-tm electronic
works.

Professor Michael S. Hart is the originator of the Project Gutenberg-tm
concept of a library of electronic works that could be freely shared
with anyone.  For thirty years, he produced and distributed Project
Gutenberg-tm eBooks with only a loose network of volunteer support.


Project Gutenberg-tm eBooks are often created from several printed
editions, all of which are confirmed as Public Domain in the U.S.
unless a copyright notice is included.  Thus, we do not necessarily
keep eBooks in compliance with any particular paper edition.


Most people start at our Web site which has the main PG search facility:

     http://www.gutenberg.org

This Web site includes information about Project Gutenberg-tm,
including how to make donations to the Project Gutenberg Literary
Archive Foundation, how to help produce our new eBooks, and how to
subscribe to our email newsletter to hear about new eBooks.


The Project Gutenberg EBook of Madame Bovary, by Gustave Flaubert

This eBook is for the use of anyone anywhere at no cost and with
almost no restrictions whatsoever.  You may copy it, give it away or
re-use it under the terms of the Project Gutenberg License included
with this eBook or online at www.gutenberg.org


Title: 

Author: Gustave Flaubert

Translator: Eleanor Marx-Aveling

Release Date: February 25, 2006 [EBook #2413]

Language: English

Character set encoding: ASCII

*** START OF THIS PROJECT GUTENBERG EBOOK MADAME BOVARY ***




Produced by An Anonymous Volunteer, Noah Adams and David Widger












