\documentclass{tufte-book}
\usepackage{phd}
\def\hlred{{\color{Magenta}}}
\usepackage{doccommands}
\usepackage{morelogos}
\usepackage{hyperref}
\begin{document}
\chapter{DEFINING CLASSES}

\epigraph{First there was one user and I took a lot of time to satisfy myself. Then I had 10 users, and a whole new level of difficulties arose. Then I had a hundred users and another level of things happened. I had a thousand users, I had ten thousand each of those were special phases in the development, important. I couldn't have gone with ten thousand until I'd done
it with a thousand. But each time a new wave of
changes came along, the idea was to have \tex  get
better, and not get more diverse as it needed to handle
new things.}{Donald Knuth}

\index{Classes@\textbf{Classes}}
\index{Classes!\textbackslash ProvideClass}

\section{Introduction}

To \emph{make} a book is an interesting and somewhat involved process\cite{town}. The text is set in type and printed on pages, the pages are gathered and folded into signatures and then bound and covered. Many of the aspects of this process that has passed down to us by previous generations is discussed extensively in other sections of this book.  Class authors have to distill this knowledge in a set of typographical rules to be described in a class file. The first thing such an author must do is to describe the \emph{rationale} of developing such a class. The \doccls{octavo}\cite{octavo} class was developed to enable printing books in dimensions that follow traditional styles. The \doccls{memoir}\cite{memoir}  class to offer a flexible system on which other classes could be based. The \doccls{tufte-book} and \doccls{tufte-handout} classes to provide a style that resembles those found in Tufte books. Many Universities offer \emph{Thesis} classes to standardize the way these are produced. Many of these Universities, translated the styles previously typed and the results are a typographical disaster, only mitigated by the ability to display beautiful mathematics. As these are printed on standard \emph{photocopy paper} one cannot do much with the layout. 

%\begin{figure}[htp]
%\includegraphics[width=0.9\textwidth]{./graphics/children-book}
%\caption{A class developed for children's book.}
%\end{figure}
%\index{classes!octavo}

So before you write your class, think who your users are going to be
and what type of publications will they be producing. 
\url{http://www.youtube.com/watch?feature=player_embedded&v=XzChtmC-4-Y#!}

\section{The Technical Side of Things}
A class is simply a file of macros with the extension \cls. It is probably
better to offer a simple example of a class first. Our first class will
load the standard book class and will do nothing other than load the
listings and a few other packages.
Use your editor
and type the following:

\emphasis{ProvidesClass,LoadClass}
\begin{teX}
\NeedsTeXFormat{LaTeX2e}[1994/06/01]
\ProvidesClass{myclass}[2010/12/11 v3.5.0 myclass]
\LoadClass{book}
\end{teX}

This is the simplest way to overload your changes over an existing class.

It is also at this stage to have handy a small test file \docfile{myclass-test.tex} to see that everything was generated correctly and is working.

\begin{teX}
\documentclass{myclass}
\usepackage{lipsum}
\begin{document}
\lipsum[1-2]
\end{document}

\end{teX}


We now ready to make our class a bit more useful, by loading a few packages
that we commonly use.

\emphasis{RequirePackage}
\begin{teX}
\NeedsTeXFormat{LaTeX2e}[1994/06/01]
\ProvidesClass{myclass}[2010/12/11 v3.5.0 myclass]
\LoadClass{book}
\RequirePackage{listings}
\RequirePackage{phd}
\end{teX}

As you probably noticed, in a class or package file the recommended way to load a
package is to use \cs{RequirePackage} rather than \cs{usepackage}. You also do
not need to enclose any code within \cs{makeatletter}\ldots\cs{makeatother}.


\subsection{Identifying your class}

The first thing a class must do is to identify any other formats it needs and to announce
its name.\cs{NeedsTeXFormat}\cs{ProvidesClass}

The following example, delares the version of \LaTeXe\ that it requires and then
gives the class name. It can be found in the preable of most well written classes. You should also put some remarks to identify you as the author, the version number and other similar details. These are discussed in more detail in the next Chapter, where you will see how to automate documentation for your class.

\begin{teX}
 \NeedsTeXFormat{LaTeX2e}[1994/06/01]
 \ProvidesClass{myclass-book}[2010/12/11 v3.5.0 myclass-book]
\end{teX}

The above syntax must be followed exactly so that this information can be
used by \texttt{LoadClass} or \cs{documentclass} (for classes) or \cs{RequirePackage}
 or\cs{usepackage} (for packages) to test that the release is not too old.
The whole of this $<release-info>$ information is displayed by \cs{listfiles} and
should therefore not be too long.

\begin{teX}
 %%
 % Load the common style elements
 \input{myclass-common.def}
\end{teX}



Another command that can be used is \doccmd{ProvidesFile}. 
This is similar to the two previous commands except that here the fullname,
including the extension, must be given. It is used for declaring any files other
than main class and package files.

This is useful, if you decide to have your main definitions in a separate file.



\section{Class Options}
Before we see in detail how to add Options to a class, we need to review a package called
\texttt{xkeyval}. Unless you are in the business of re-discovering wheels, this is an absolute must
for developing, readable and maintenable code.You can download xkeyval at
 \url{http://ctan.org/tex-archive/macros/latex/contrib/xkeyval/}


\verb+\usepackage[textcolor=red,font=times]{mypack}+


Class options are best set by using booleans\cs{newboolean}.

We first set a new boolean that we name@myclass@afourpaper. This is used using the package
\texttt{ifthen}\sidenote{The ifthen package was developed by 
David Carlisle, can be downloaded at \url{ http://www.ifi.uio.no/it/latex-links/ifthen.pdf }} 
Then we can |DecalareOptionX| and we set the boolean to default to true. If the user then types

myclass[a4paper]

The a4paper options will be set. This is a much better and concise way of defining options.
\cs{newboolean}


\begin{teX}
\newboolean{@myclass@afourpaper}
\DeclareOptionX[myclass]<common>{a4paper}
  {
   \setboolean{@myclass@afourpaper}
   {true}
  }
\end{teX}

Note that the command provide by \texttt{ifthen} \doccmd{setboolean} takes true or false, as \#2, and sets \#1 accordingly. In the above code we set the option as true. \margindoc{setboolean}


It is much easier and most programmers use the \texttt{ifthen} package to check
for option booleans

\begin{teX}
\ifthenelse{\boolean{@myclass@afourpaper}}
  {\geometry{
        a4paper,
        left=24.8mm,
        top=27.4mm,
        headsep=2\baselineskip,
        textwidth=107mm,
        marginparsep=8.2mm,
        marginparwidth=49.4mm,
        textheight=49\baselineskip,
        headheight=\baselineskip
    }
  }
 {}
\end{teX}

\section{Set-up the fount sizes}
LaTeX does not provide definitions of all the font-sizes. Unless you are
extending an existing class, this is one of the first tasks you need to 
do in your new class.

Normally class authors will define all the commonly defined size commands,
such as  \cs{small}, \cs{normalsize} and other similar commands.

In the example shown below, we first start by defining the \doccmd{normalsize} font
size. In this book the \doccmd{normalsize}  is defined as 14pt. We also define the vertical
spaces that we need to have abovedisplay and belowdisplayskip. These are all very difficult to
remember and once you have something you are happy with, just copy from class to class
or even define a samll definition file to keep them all together.


{\fontfamily{phv}\selectfont Helvetica looks like this}
and {\fontencoding{OT1}\fontfamily{ppl} Palatino looks like this}.


 The user has access to a number of commands which change the size of
 the fount, relative to the `main' size used for the bulk of the text.


 These \cs{size} commands issue a \cs{@setfontsize}\index{Latex kernel!@setfontsize} 
 command.

\begin{teX}
  \@setfontsize\size\font-size{baselineskip} where:
\end{teX}



  \begin{enumerate}
    \item font-size The absolute size of the fount to use from
        now on.
    \item baselineskip The normal value of \cs{baselineskip}
        for the size of the fount selected. (The actual value will be
       % |\baselinestretch| * \meta{baselineskip}.)
    \end{enumerate}

A number of commands, defined in the \LaTeX  kernel, shorten the
following  definitions and are used throughout. These are:

    \begin{center}
    \begin{tabular}{ll@{\qquad}ll@{\qquad}ll}
    \verb=\@vpt= & 5 & \verb=\@vipt= & 6 & \verb=\@viipt= & 7 \\
    \verb=\@viiipt= & 8 & \verb=\@ixpt= & 9 & \verb=\@xpt= & 10 \\
    \verb=\@xipt= & 10.95 & \verb=\@xiipt= & 12 & \verb=\@xivpt= & 14.4\\
    \ldots
    \end{tabular}
    \end{center}


\subsection{Setting up the normalsize}
 The user command to obtain the `main' size is \cs{normalsize}. \LaTeX\
 uses \cs{@normalsize} \index{Latex kernel!@normalsize} when referring to the main size and maintains this
 value even if \doccmd{normalsize} is redefined. The \doccmd{normalsize} macro also
  sets values for \cs{abovedisplayskip}, \cs{abovedisplayshortskip} and 
\cs{belowdisplayshortskip}.



\begin{teX}
%%
% Set the font sizes and baselines to match Tufte's books
% normalsize
%%
\renewcommand\normalsize{%
   \@setfontsize\normalsize\@xpt{14}%
   \abovedisplayskip 10\p@ \@plus2\p@ \@minus5\p@
   \abovedisplayshortskip \z@ \@plus3\p@
   \belowdisplayshortskip 6\p@ \@plus3\p@ \@minus3\p@
   \belowdisplayskip \abovedisplayskip
   \let\@listi\@listI}

\normalbaselineskip=14pt
\normalsize
\end{lstlisting}

\begin{lstlisting}[language={[common]TeX},% 
                           alsolanguage={[LaTeX]TeX},% 
                           alsolanguage={[primitive]TeX},%
                           alsolanguage={Verse}]
\renewcommand\small{%
   \@setfontsize\small\@ixpt{12}%
   \abovedisplayskip 8.5\p@ \@plus3\p@ \@minus4\p@
   \abovedisplayshortskip \z@ \@plus2\p@
   \belowdisplayshortskip 4\p@ \@plus2\p@ \@minus2\p@
   \def\@listi{\leftmargin\leftmargini
               \topsep 4\p@ \@plus2\p@ \@minus2\p@
               \parsep 2\p@ \@plus\p@ \@minus\p@
               \itemsep \parsep}%
   \belowdisplayskip \abovedisplayskip
}
\renewcommand\footnotesize{%
   \@setfontsize\footnotesize\@viiipt{10}%
   \abovedisplayskip 6\p@ \@plus2\p@ \@minus4\p@
   \abovedisplayshortskip \z@ \@plus\p@
   \belowdisplayshortskip 3\p@ \@plus\p@ \@minus2\p@
   \def\@listi{\leftmargin\leftmargini
               \topsep 3\p@ \@plus\p@ \@minus\p@
               \parsep 2\p@ \@plus\p@ \@minus\p@
               \itemsep \parsep}%
   \belowdisplayskip \abovedisplayskip
}
\renewcommand\scriptsize{\@setfontsize\scriptsize\@viipt\@viiipt}
\renewcommand\tiny{\@setfontsize\tiny\@vpt\@vipt}
\renewcommand\large{\@setfontsize\large\@xipt{15}}
\renewcommand\Large{\@setfontsize\Large\@xiipt{16}}
\renewcommand\LARGE{\@setfontsize\LARGE\@xivpt{18}}
\renewcommand\huge{\@setfontsize\huge\@xxpt{30}}
\renewcommand\Huge{\@setfontsize\Huge{24}{36}}

%% Define a HUGE for fun
\newcommand\HUGE{\@setfontsize\Huge{38}{47}}  
\end{teX}

\section{Adjusting paragraph parameters}

 The parameters which control \tex 's behaviour when typesetting
 paragraphs receive a bit of a tweak here. Contrary to the usual
 behaviour of modifying the grid with glue when difficulties are
 encountered with vertical space, here we shall try to counteract
 these tendencies and enforce as much as possible uniformity of the 
 grid of lines.

A good value for paragraph indentation is \texttt{parindent 0.5pt}, for vertical spacing between
paragraphs that are indented use 0pt. At this point if you are using any marginals it is a good idea
to allow hyphenation with the \docpkg{ragged2e} package. Since marginals use very narrow paragraphs you may
get a very funny looking marginal text. Using the package, adjustments can be made to hyphenate
the marginal text.

\begin{teX}
%%
% \RaggedRight allows hyphenation

\RequirePackage{ragged2e}
\setlength{\RaggedRightRightskip}{\z@ plus 0.08\hsize}
\setlength{\RaggedRightParindent}{1pc}

% Paragraph indentation and separation for normal text
\newcommand{\@tufte@reset@par}{%
  \setlength{\RaggedRightParindent}{1.0pc}%
  \setlength{\parindent}{1pc}%
  \setlength{\parskip}{0pt}%
}
\@tufte@reset@par

% Paragraph indentation and separation for marginal text
\newcommand{\@tufte@margin@par}{%
  \setlength{\RaggedRightParindent}{0.5pc}%
  \setlength{\parindent}{0.5pc}%
  \setlength{\parskip}{0pt}%
}
\end{teX}


\section{Formatting Chapters and Sections}
The section on Chapters etc, has more on this, but we will touch on it briefly.
I normally use the \docpkg{titlesec} and \docpkg{titletoc} package to handle the complexity of these commands:

\begin{teX}
\titleformat{\subsection}%
  [hang]% shape
  {\normalfont\large}% format applied to label+text removed \itshape
  {\thesubsection}% label
  {1em}% horizontal separation between label and title body
  {}% before the title body
  []% after the title body
\end{teX}

These are normally followed by the "titlespacing" commands to define the space around these sections.

\begin{teX}
%% We set the titlespacing using the package titlesec and titletoc
%
\titlespacing*{\chapter}{0pt}{20pt}{40pt}
\titlespacing*{\section}{0pt}{3.5ex plus 1ex minus .2ex}{2.3ex plus .2ex}
\titlespacing*{\subsection}{0pt}{3.25ex plus 1ex minus .2ex}{1.5ex plus.2ex}
\end{teX}

\section{Adjusting the Index}
For classes representing books, the index is treated like a chapter whereas for others it is normally
treated like a section. Whatever your document ends up like, indices are best done in a multi-column environment.
One possibility is shown below, using the package "multcol". \margindoc{RequirePackage\{multico\l\}}

\begin{teX}
\RequirePackage{multicol}
\renewenvironment{theindex}
  {\begin{fullwidth}%
    \small%
    \ifthenelse{\equal{\@tufte@class}{book}}%
      {\chapter{\indexname}}%
      {\section*{\indexname}}%
    \parskip0pt%
    \parindent0pt%
    \let\item\@idxitem%
    \begin{multicols}{3}%
  }
  {\end{multicols}%
    
\renewcommand\@idxitem{\par\hangindent 2em}
\renewcommand\subitem{\par\hangindent 3em\hspace*{1em}}
\renewcommand\subsubitem{
    \par\hangindent 4em\hspace*{2em}
}
\renewcommand\indexspace{
    \par\addvspace{
       1.0\baselineskip plus 0.5ex minus 0.2ex}\relax
    }%
%we now  swallow the letter heading in the index
\newcommand{\lettergroup}[1]{}

\end{teX}

The code, renews the "theindex" environment, with minor tweaks and defines it as a three column
layout at "fullwidth".

\section{Provide some hooks}
It is useful at the end of the class to allow for localization of the class
by importing a local file. This is easily achieved by checking if the file exists
and then loading it.  If there is a |myclass-book-local.sty|  file, load it.

\begin{teX}
\IfFileExists{myclass-book-local.tex}
  {\input{myclass-book-local}
   \MyClassInfoNL{Loading myclass-book-local.tex}}
  {}
\end{teX}

If you intent to publish your class, you may also want to consider adding a hook for a patch-file.


\section{The final act of kindness to your users}
Many common classes, such as the |memeoir| use such a tactic to avoid breaking old code.\index{\string IfFileExists}

\begin{teX}
 \IfFileExists{mypatch.sty}{%
 \RequirePackage{mempatch}}{}
\end{teX}



\chapter{Packaging Code}

In the previous chapter we have outlined what a class or package is and the numerous details you have to watch out for.

Once you start contemplating writing either a class or a package, you should
consider packaging the code. What is meant by packaging the code is
simply to put all your macros and logic in a file with a \dtx extension and
written in a specific manner. This method of writing code, in my opinion
is one of the reasons of the success and longevity of \latex. It provides
a consistent---if somewhat difficult at first to follow--way of documenting
a program and at the same time producin the documentation. Most package
or class authors make their code available through CTAN. You can even 
include in the \dtx file code for producing  the xml file required by CTAN
to produce the html page describing your project, and any readme files. 
Personally I also add a readme file in \texttt{markdown style}, to handle
a redme file at the Github repository I use and even to produce a \texttt{.gitignore} file.


Motivation The important parts of a package are the code, the documentation
of the code, and the user documentation. Using the \docpkg{Doc}  and
DocStrip programs, it’s possible to combine all three of these into a single,
documented \latex (.dtx) file. The primary advantage of a .dtx file is that
it enables you to use arbitrary LATEX constructs to comment your code.
Hence, macros, environments, code stanzas, variables, and so forth can be
explained using tables, figures, mathematics, and font changes. Code can
be organized into sections using LATEX’s sectioning commands. Doc even
facilitates generating a unified index that indexes both macro definitions (in
the LATEX code) and macro descriptions (in the user documentation). 

This emphasis on writing verbose, nicely typeset comments for code---essentially
treating a program as a book that describes a set of algorithms---is known
as literate programming \cite{literate} and has been in use since the early days of \tex\ .

This tutorial shows how to write a single file that serves as both documentation
and driver file, which is a more typical usage of the \texttt{Doc} system than
using separate files.

\section{Conventions}

A dtx file is written using the specific conventions of the \docpkg{doc} and 
\docpkg{docstrip} program. It is generally written in sections made up of what
are termed as guards. These guards are similar to xml tags. You can have as
many as you want. For example if you open and examine any .dtx file it is
likely that it will have a section guarded by the tags \verb+ <*driver>...</driver>+. When the file is processed it produces a driver file.


\begin{teX}
%<*driver>
\NeedsTeXFormat{LaTeX2e}
\ProvidesFile{MWEdtx.drv}%
  [2013/01/13 v1.0 ]%
\documentclass{ltxdoc}
... code
\begin{document}
  \DocInput{\jobname.dtx}%
\end{document}
%</driver>
\end{teX}

Most packages and classes have as a minimum the following sections typed in
between appropriate tags.
\medskip

\begin{tabular}{lp{4.5cm}}
\toprule
Guard & Description\\
\midrule
\verb+ %<*driver>...%</driver>+ & driver file \\
\verb+ %<*install>...%</install>+ & installation batch file \\
\verb+ %<*readme>...%</readme>+ & readme file \\
\verb+ %<*internal>...%</internal>+ & internall code not outputted to any file \\
\verb+ %<*package>...%</package>+ & package code \\
\verb+ %<*class>...%</class>+ & class code \\
\bottomrule
\end{tabular}
\medskip

The naming of these guards is immaterial, you can use any convention you
like, but my recommendation is to limit them to the above words, as it easier
for someone to understand your code. You can have as many of these guards as possible and even combination tags, where a piece of code can be written to both
sections or excluded. More about this later.

\begin{fullwidth}
\emphasis{from,preamble,postamble,endpostamble,endpreamble,
nopreamble, nopostamble,generate,usedir,DocInput,file}
\begin{teX}
% \iffalse meta-comment
%<*internal>
\iffalse
%</internal>
%<*readme>
template --- short description
%</readme>
%<*internal>
\fi
\def\nameofplainTeX{plain}
\ifx\fmtname\nameofplainTeX\else
  \expandafter\begingroup
\fi
%</internal>
%<*install>
\input docstrip.tex
\keepsilent
\askforoverwritefalse
\preamble
template --- short description.
\endpreamble
\postamble
 Copyright (C) 2011 by Dr. Yiannis Lazarides 
                       <yannislaz@gmail.com>
\endpostamble
\usedir{tex/latex/\jobname}
\generate{
  \file{\jobname.sty}{\from{\jobname.dtx}{package}}
 }
%</install>
%<install>\endbatchfile

%<*internal>
\usedir{source/latex/\jobname}
\generate{
  \file{\jobname.ins}{\from{\jobname.dtx}{install}}
}
\nopreamble\nopostamble
\usedir{doc/latex/demopkg}
\generate{
  \file{README.txt}{\from{\jobname.dtx}{readme}}
  \file{test-01.tex}{\from{\jobname.dtx}{test-01}}
  \file{TODO.tex}{\from{\jobname.dtx}{TODO}}
}
\ifx\fmtname\nameofplainTeX
  \expandafter\endbatchfile
\else
  \expandafter\endgroup
\fi
%</internal>

%<*driver>
\NeedsTeXFormat{LaTeX2e}
\ProvidesFile{MWEdtx.drv}%
  [2013/01/13 v1.0 ]%
\documentclass{ltxdoc}
\usepackage[T1]{fontenc}
\usepackage{lmodern}
\usepackage{\jobname}
% Heiko's package to provide
% hyperlink facilities
\usepackage[numbered]{hypdoc}
\EnableCrossrefs
\CodelineIndex
\RecordChanges
\begin{document}
  \DocInput{\jobname.dtx}%
\end{document}
%</driver>
% \fi
%
% \CheckSum{12}
% \CharacterTable
%  {Upper-case    \A\B\C\D\E\F\G\H\I\J\K\L\M\N}
% \changes{1.0}{2011/05/03}{Converted to DTX file}
%
% \DoNotIndex{\newcommand,\newenvironment}
%
% \title{The \textsf{\jobname} package.
% \author{Dr. Yiannis Lazarides \\ \url{yannislaz@gmail.com}}
% \thanks{This
%        file (\texttt{\jobname.dtx}) has version number 
%        \fileversion, last revised
%        \filedate.}
% }
% 
% \date{\filedate}
%
%
% \maketitle
%
% \abstract{It is  part} 
% \section{Introduction}
%  This manual is typeset according to the conventions of 
% 
%
%
% \StopEventually{}
%
% \section{Implementation}
%
%    \begin{macrocode}
%<*package>
\NeedsTeXFormat{LaTeX2e}
\ProvidesPackage{MWEdtx}%
  [2013/13/01 v1.0 Example Class]%
%    \end{macrocode}
%
% \begin{macro}{\test}
% Here you start describing the package macros |\test|
%    \begin{macrocode}
  \newcommand\test{This is a test}
%    \end{macrocode}
% \end{macro}
% \begin{macro}{\testi}
% Here you start describing the package macros |\testi|
%    \begin{macrocode}
  \newcommand\testi{This is a test}
%    \end{macrocode}
% \end{macro}
%
% \iffalse
%</package>
% \fi
%
% \section{Test macros}
%
%<*test-01>
%
%   \begin{macrocode}
\documentclass{article}
\usepackage{graphicx}
...
\end{letter}
%    \end{macrocode}
%
%</test-01>

% This is the end of the implementation part.
%
% \bibliographystyle{alpha}
% \begingroup
% \raggedright
%
% \begin{thebibliography}{GMSN94A}
%
% \bibitem[GMS94]{GOOSSENS94}
% Michel Goossens, Frank Mittelbach, and Alexander Samarin.
% \newblock {\em The LaTeX Companion}.
% \newblock Addison-Wesley Publishing Company, 1994.
%
% \end{thebibliography}
% \endgroup
%
% \PrintIndex
% \Finale
\endinput
\end{teX}
\end{fullwidth}



\subsection{The \protect\texttt{.ins} file}

When preparing a package for distribution the first step is to write an installer
(\ins) file. An installer file extracts the code from a \dtx file, using \cs{DocStrip}
to strip off the comments and documentation, and outputs a \sty file and other files, if necessary. A \ins file is typically fairly short and does not change
significantly from one package to another. The \ins file in most newer packages
is created by the class itself. In older packages it was normal to just type a separate file. (See for example the \docfile{tugboat.dtx}).

\noindent\ins files usually start with comments specifying the copyright and license information:

\begin{teX}
%%
%% Copyright (C) year by your name %%
%% This file may be distributed and/or modified under the
%% conditions of the LaTeX Project Public License, either
%% version 1.2 of this license or (at your option) any later
%% version. The latest version of this license is in:
%%
%% http://www.latex-project.org/lppl.txt
%%
%% and version 1.2 or later is part of all distributions of
%% LaTeX version 1999/12/01 or later.
%%

\end{teX}

The \latex Project Public License (LPPL) is the license under which most
packages—and \latex itself—are distributed. Of course, you can release your
package under any license you want; the \texttt{PPL} is merely the most common
license for \latex packages. The LPPL specifies that a user can do whatever
he wants with your package—including sell it and give you nothing in return.
The only restrictions are that he must give you credit for your work, and
he must change the name of the package if he modifies anything to avoid
versioning confusion.

The next step is to load \docpkg{DocStrip}:

\begin{teX}
\input docstrip.tex
\keepsilent
\end{teX}

By default, DocStrip gives a line-by-line account of its activity. These messages
aren’t terribly useful, so most people turn them off:

\begin{teX}
\keepsilent
\end{teX}

A system administrator can specify the base directory under which all
TEX-related files should be installed, e.g., \texttt{/usr/share/texmf}. (See
\cs{BaseDirectory} in the DocStrip manual.) The \ins file specifies where
its files should be installed relative to that. The following is typical:

\begin{teX}
\usedir{tex/latex/hpackagei}
\preamble
htexti \endpreamble
\end{teX}

The next step is to specify a preamble, which is a block of commentary that
will be written to the top of every generated file:

\begin{teX}
\preamble

This is a generated file.
Copyright (C) <year> by <your name>
This file may be distributed and/or modified under the
conditions of the LaTeX Project Public License, either
version 1.2 of this license or (at your option) any later
version. The latest version of this license is in:
http://www.latex-project.org/lppl.txt
and version 1.2 or later is part of all distributions of
LaTeX version 1999/12/01 or later.

\endpreamble
\end{teX}


The preceding preamble would cause \verb+ <package>.sty+  to begin as follows:

\begin{teX}
%%
%% This is file ‘hpackagei.sty’,
%% generated with the docstrip utility.
%%
%% The original source files were:
%%
%% hpackagei.dtx (with options: ‘package’)
%%
%% This is a generated file.
%%
%% Copyright (C) hyeari by hyour namei %%
%% This file may be distributed and/or modified under the
%% conditions of the LaTeX Project Public License, either
%% version 1.2 of this license or (at your option) any later
%% version. The latest version of this license is in:
%%
%% http://www.latex-project.org/lppl.txt
%%
%% and version 1.2 or later is part of all distributions of
%% LaTeX version 1999/12/01 or later.
\end{teX}


\begin{teX}
\generate {\file {style-file} {\from {dtx-file} {tag}}}
\end{teX}

We now reach the most important part of a \ins{ins} file: the specification of
what files to generate from the .dtx file. The following tells DocStrip to
generate package.sty from package.dtx by extracting only those parts
marked as `package'  in the .dtx file. (Marking parts of a \dtx file is
described in Section 3.)

\begin{teX}
\generate{
  \file{<package>.sty}{\from{<package>.dtx}{package}}
}
\end{teX}

\doccmd{generate} can extract any number of files from a given .dtx file. It can
even extract a single file from multiple .dtx files. See the \texttt{DocStrip} manual for details.

\subsection{Generating messages} %\Msg {<text>}

The next part of a \ins file consists of commands to output a message to
the user, telling him what files need to be installed and reminding him how
to produce the user documentation. The following set of \cs{Msg} commands is
typical:

{\footnotesize
\begin{teX}
\obeyspaces
\Msg{****************************************************}
\Msg{* *}
\Msg{* To finish the installation you have to move the *}
\Msg{* following file into a directory searched by TeX: *}
\Msg{* *}
\Msg{* hpackagei.sty *}
\Msg{* *}
\Msg{* To produce the documentation run the file *}
\Msg{* hpackagei.dtx through LaTeX. *}
\Msg{* *}
\Msg{* Happy TeXing! *}
\Msg{* *}
\Msg{****************************************************}
Note the use of \obeyspaces to inhibit TEX from collapsing multiple spaces
into one.
\endbatchfile
Finally, we tell DocStrip that we’ve reached the end of the .ins file:
\endbatchfile
\end{teX}

}
Appendix A.1 lists a complete, skeleton .ins file. Appendix A.2 is similar
but contains slight modifications intended to produce a class (\cls) file
instead of a style (\sty) file

\section{The .dtx file}
A \dtx\  file contains both the commented source code and the user documentation
for the package. Running a \dtx\  file through latex typesets the
user documentation, which usually also includes a nicely typeset version of
the commented source code.

Due to some Doc trickery, a \dtx\  file is actually evaluated twice. The first
time, only a small piece of \latex\  driver code is evaluated. The second time,
comments in the \dtx\  file are evaluated, as if there were no `%'  preceding
them. This can lead to a good deal of confusion when writing \dtx\  files
and occasionally leads to some awkward constructions. Fortunately, once
the basic structure of a \dtx\  file is in place, filling in the code is fairly
straightforward.



\begin{teX}
\CharacterTable {<text>}
\end{teX}

The second mechanism that Doc uses to ensure that a \dtx  file is uncorrupted
is a character table. If you put the following command teX into
your \dtx\  file, then Doc will ensure that no unexpected character translation
took place in transport:

\begin{teX}
% \CharacterTable
% {Upper-case \A\B\C\D\E\F\G\H\I\J\K\L\M\N\O\P\Q\R\S\T\U\V\W\X\Y\Z
...
% Exclamation \! Double quote \" Hash (number) \#
% Dollar \$ Percent \% Ampersand \&
% Acute accent \’ Left paren \( Right paren \)
\end{teX}

It is highly unlikely one would pick-up an error because of any corruption
of the character table. 

\begin{teX}
A success message looks like this:
***************************
* Character table correct *
***************************

and an error message looks like this:
! Package doc Error: Character table corrupted.
\end{teX}

ALthough one tends to think that it is not necessary any longer to resort
to such checks, at least in 
\url{http://tex.stackexchange.com/questions/9638/polyglossia-doesnt-work-with-the-ltxdoc-document-class} it caught a problem with polyglossia.

\subsection{DoNotIndex}



When producing an index, Doc normally indexes every control sequence
(i.e., backslashed word or symbol) in the code. The problem with this level
of automation is that many control sequences are uninteresting from the
perspective of understanding the code. For example, a reader probably
doesn’t want to see every location where \cs{if} is used—or \cs{the} or \cs{let} or
\cs{begin} or any of numerous other control sequences.

As its name implies, the \cs{DoNotIndex} command gives Doc a list of control
sequences that should not be indexed. \cs{DoNotIndex} can be used any
number of times, and it accepts any number of control sequence names per
invocation:

\begin{teX}
%\DoNotIndex{\#,\$,\%,\&,\@,\\,\{,\}}
%\DoNotIndex{\closeout,\day,\def,\edef}
\end{teX}

\subsection{User documentation}

We can finally start writing the user documentation. A typical beginning
looks like this:

\begin{teX}
% \title{The \textsf{hpackagei} package\thanks{This document
% corresponds to \textsf{hpackagei}~\fileversion,
% dated~\filedate.}}
% \author{hyour namei \\ \texttt{hyour e-mail addressi}}
%
% \maketitle
\end{teX}

The title can certainly be more creative, but note that it’s common for
package names to be typeset with \cs{textsf} and for \cs{thanks} to be used to
specify the package version and date. This yields one of the advantages
of literate programming: Whenever you change the package version (the
optional second argument to \cs{ProvidesPackage}), the user documentation
is updated accordingly. Of course, you still have to ensure manually that
the user documentation accurately describes the updated package.

Write the user documentation as you would any \latex document, except
that you have to precede each line with a ``\%''. Note that the \docpkg{ltxdoc} document
class is derived from article, so the top-level sectioning command is
\cs{section}, not \cs{chapter}.

\section{General tips}

When you first start writing a package or class using the \docpkg{doc} and
\docpkg{docstrip} programs, you may get disorientated with all the guards
and percentage marks. Common errors can be introduced with wrong spacing of 
the \cs{macrocode} macros, so a good editor is a must.

\begin{enumerate}
\item Read the related manuals and guides \docfile{doc}, \docfile{docstrip},
      and the \docfile{clsguide}.
\item Start from a template.
\item Examine other people's code.
\item Compile often.
\item Always create a couple of test files to test the package or the class. These can also be generated by the \dtx file when processed and can also be used as 
examples for users of the package.
\item Preferably keep track of changes using a tracking system such as Git and
Github.
\item Upload to CTAN.
\end{enumerate}



\input{./manual/book.cls.tex}

\end{document}
\end{document}

You should also be
familiar with \latex  for Class and Package Writers”, which is available
from CTAN (\url{http://www.ctan.org}) and comes with most LATEX2" distributions
in a file called \texttt{clsguide.dvi}. Finally, you should know how to
install packages that are shipped as a \texttt{.dtx} file plus a \texttt{.ins} file.

style (.sty) file is primarily a collection of macro and
environment definitions. One or more style files (e.g., a main style file that
\cs{input}  or \cs{RequirePackages} multiple helper files) is called a package.
Packages are loaded into a document with \cs{usepackage}\meta{main.sty}.
In the rest of this document, we use the notation texttt{<package>} to represent
the name of your package.







































