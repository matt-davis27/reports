\documentclass[11pt,twocolumn]{ltugboat}
\usepackage[utf8]{inputenc}
\usepackage[T1]{fontenc}
\usepackage{phd}
\usepackage{marginnote}
\def\hlred{{\color{Magenta}}}
\usepackage{doccommands}
\usepackage{morelogos}
\usepackage{hyperref}
\hypersetup{pdftex,
  bookmarks,
  raiselinks,
  pageanchor,
  hyperindex,
  colorlinks,
  allcolors=theblue, 
  %anchorcolor= blue,
  %filecolor=blue,
  urlcolor=theblue,
  linkcolor= theblue,
  pdftitle={My Title},
 }
\title{An Answer to the Question: "What is Enlightenment?"}
\author{Immanuel Kant}
\date{Konigsberg in Prussia, 30th September, 1784.}
\begin{document}
\maketitle

\epigraph{As Douglas Adams said:
"Man has always assumed that he was more intelligent than dolphins because he had achieved so much...the wheel, New York, wars and so on\ldots while all the dolphins had ever done was muck about in the water having a good time. But conversely, the dolphins had always believed that they were far more intelligent than man\ldots for precisely the same reason."}

%An Answer to the Question: "What is Enlightenment?"
%
%IMMANUEL KANT
%
%Konigsberg in Prussia, 30th September, 1784.

\lettrine{E}{nlightenment is man's emergence} from his self-incurred immaturity. Immaturity is the inability to use one's own understanding without the guidance of another. This immaturity is self-incurred if its cause is not lack of understanding, but lack of resolution and courage to use it without the guidance of another. The motto of enlightenment is therefore: \textit{Sapere aude}!\footnote{Sapere aude is a Latin phrase meaning "dare to be wise", or more precisely "dare to know". Originally used by Horace, after becoming closely associated with The Enlightenment by Immanuel Kant in his seminal essay, What is Enlightenment?. Kant claimed it was the motto for the entire period, and used it to explore his theories of reason in the public sphere. Later, Michel Foucault took up Kant's formulation in an attempt to find a place for the individual in his post-structuralist philosophy and to come to terms with the problematic legacy of the Enlightenment.} Have courage to use your own understanding!

Laziness and cowardice are the reasons why such a large proportion of men, even when nature has long emancipated them from alien guidance (\textit{naturaliter maiorennes}), nevertheless gladly remain immature for life. For the same reasons, it is all too easy for others to set themselves up as their guardians. It is so convenient to be immature! If I have a book to have understanding in place of me, a spiritual adviser to have a conscience for me, a doctor to judge my diet for me, and so on, I need not make any efforts at all. I need not think, so long as I can pay; others will soon enough take the tiresome job over for me. The guardians who have kindly taken upon themselves the work of supervision will soon see to it that by far the largest part of mankind (including the entire fair sex) should consider the step forward to maturity not only as difficult but also as highly dangerous. Having first infatuated their domesticated animals, and carefully prevented the docile creatures from daring to take a single step without the leading-strings to which they are tied, they next show them the danger which threatens them if they try to walk unaided. Now this danger is not in fact so very great, for they would certainly learn to walk eventually after a few falls. But an example of this kind is intimidating, and usually frightens them off from further attempts.

Thus it is difficult for each separate individual to work his way out of the immaturity which has become almost second nature to him. He has even grown fond of it and is really incapable for the time being of using his own understanding, because he was never allowed to make the attempt. Dogmas and formulas, those mechanical instruments for rational use (or rather misuse) of his natural endowments, are the ball and chain of his permanent immaturity. And if anyone did throw them off, he would still be uncertain about jumping over even the narrowest of trenches, for he would be unaccustomed to free movement of this kind. Thus only a few, by cultivating their own minds, have succeeded in freeing themselves from immaturity and in continuing boldly on their way.

There is more chance of an entire public enlightening itself. This is indeed almost inevitable, if only the public concerned is left in freedom. For there will always be a few who think for themselves, even among those appointed as guardians of the common mass. Such guardians, once they have themselves thrown off the yoke of immaturity, will disseminate the spirit of rational respect for personal value and for the duty of all men to think for themselves. The remarkable thing about this is that if the public, which was previously put under this yoke by the guardians, is suitably stirred up by some of the latter who are incapable of enlightenment, it may subsequently compel the guardians themselves to remain under the yoke. For it is very harmful to propagate prejudices, because they finally avenge themselves on the very people who first encouraged them (or whose predecessors did so). Thus a public can only achieve enlightenment slowly. A revolution may well put an end to autocratic despotism and to rapacious or power-seeking oppression, but it will never produce a true reform in ways of thinking. Instead, new prejudices, like the ones they replaced, will serve as a leash to control the great unthinking mass.

For enlightenment of this kind, all that is needed is freedom. And the freedom in question is the most innocuous form of all—freedom to make public use of one's reason in all matters. But I hear on all sides the cry: Don't argue! The officer says: Don't argue, get on parade! The tax-official: Don't argue, pay! The clergyman: Don't argue, believe! (Only one ruler in the world says: Argue as much as you like and about whatever you like, but obey!). . All this means restrictions on freedom everywhere. But which sort of restriction prevents enlightenment, and which, instead of hindering it, can actually promote it ? I reply: The public use of man's reason must always be free, and it alone can bring about enlightenment among men; the private use of reason may quite often be very narrowly restricted, however, without undue hindrance to the progress of enlightenment. But by the public use of one's own reason I mean that use which anyone may make of it as a man of learning addressing the entire reading public. What I term the private use of reason is that which a person may make of it in a particular civil post or office with which he is entrusted.

Now in some affairs which affect the interests of the commonwealth, we require a certain mechanism whereby some members of the commonwealth must behave purely passively, so that they may, by an artificial common agreement, be employed by the government for public ends (or at least deterred from vitiating them). It is, of course,impermissible to argue in such cases; obedience is imperative. But in so far as this or that individual who acts as part of the machine also considers himself as a member of a complete commonwealth or even of cosmopolitan society, and thence as a man of learning who may through his writings address a public in the truest sense of the word, he may 'indeed argue without harming the affairs in which he is employed for some of the time in a passive capacity. Thus it would be very harmful if an officer receiving an order from his superiors were to quibble openly, while on duty, about the appropriateness or usefulness of the order in question. He must simply obey. But he cannot reasonably be banned from making observations as a man of learning on the errors in the military service, and from submitting these to his public for judgement. The citizen cannot refuse to pay the taxes imposed upon him; presumptuous criticisms of such taxes, where someone is called upon to pay them, may be punished as an outrage which could lead to general insubordination. Nonetheless, the same citizen does not contravene his civil obligations if, as a learned individual, he publicly voices his thoughts on the impropriety or even injustice of such fiscal measures. In the same way, a clergyman is bound to instruct his pupils and his congregation in accordance with the doctrines of the church he serves, for he was employed by it on that condition. But as a scholar, he is completely free as well as obliged to impart to the public all his carefully considered, well-intentioned thoughts on the mistaken aspects of those doctrines, and to offer suggestions for a better arrangement of religious and ecclesiastical affairs. And there is nothing in this which need trouble the conscience. I;or what he teaches in pursuit of his duties as an active servant of the church is presented by him as something which he is not empowered to teach at his own discretion, but which he is employed to expound in a prescribed manner and in someone else's name. He will say: Our church teaches this or that, and these are the arguments it uses. He then extracts as much practical value as possible for his congregation from precepts to which he would not himself subscribe with full conviction, but which he can nevertheless undertake to expound, since it is not in fact wholly impossible that they may contain truth. At all events, nothing opposed to the essence of religion is present in such doctrines. For if the clergyman thought he could find anything of this sort in them, he would not be able to carry out his official duties in good conscience, and would have to resign. Thus the use which someone employed as a teacher makes of his reason in the presence of his congregation is purely private, since a congregation, however large it is, is never any more than a domestic gathering. In view of this, he is not and cannot be free as a priest, sinÏ he is acting on a commission imposed from outside. Conversely, as a scholar addressing the real public (i.e. the world at large) through his writings, the clergyman making public use of his reason enjoys unlimited freedom to use his own reason and to speak in his own person. For to maintain that the guardians of the people in spiritual matters should themselves be immature, is an absurdity which amounts to making absurdities permanent.


But should not a society of clergymen\footnote{\protect\footnotesize Isn't what this is happening now in a lot of religions. Consider Law as well as Moslem religions, the Orthodox synods and the like.}, for example an ecclesiastical synod or a venerable presbytery (as the Dutch call it), be entitled to commit itself by oath to a certain unalterable set of doctrines, in order to secure for all time a constant guardianship over each of its members, and through them over the people ? I reply that this is quite impossible. A contract of this kind,concluded with a view to preventing all further enlightenment of mankind for ever, is absolutely null and void, even if it is ratified by the supreme power, by Imperial Diets\footnote{In politics, a diet is a formal deliberative assembly. The term is mainly used historically for the Imperial Diet, the general assembly of the Imperial Estates of the Holy Roman Empire, and for the legislative bodies of certain countries. Modern usage mainly relates to the Japanese Parliament, called "Diet" in English.}
and the most solemn peace treaties. One age cannot enter into an alliance on oath to put the next age in a position where it would be impossible for it to extend and correct its knowledge, particularly on such important matters, or to make any progress whatsoever in enlightenment. This would be a crime against human nature, whose original destiny lies precisely in such progress. Later generations are thus perfectly entitled to dismiss these agreements as unauthorised and criminal. To test whether any particular measure can be agreed upon as a law for a people, we need only ask whether a people could well impose such a law upon itself. This might well be possible for a specified short period as a means of introducing a certain order, pending, as it were, a better solution. This would also mean that each citizen, particularly the clergyman, would be given a free hand as a scholar to comment publicly, i.e. in his writings, on the inadequacies of current institutions. Meanwhile, the newly established order would continue to exist, until public insight into the nature of such matters had progressed and proved itself to the point where, by general consent (if not unanimously), a proposal could be submitted to the crown. This would seek to protect the congregations who had, for instance, agreed to alter their religious establishment in accordance with their own notions of what higher insight is, but it would not try to obstruct those who wanted to let things remain as before. But it is absolutely impermissible to agree, even for a single lifetime, to a permanent religious constitution which no-one might publicly question. For this would virtually nullify a phase in man's upward progress, thus making it fruitless and even detrimental to subsequent generations. A man may for his own person, and even then only for a limited period, postpone enlightening himself in matters he ought to know about. But to renounce such enlightenment completely, whether for his own person or even more so for later generations, means violating and trampling underfoot the sacred rights of mankind. But something which a people may not even impose upon itself can still less be imposed upon it by a monarch; for his legislative authority depends precisely upon his uniting the collective will of the people in his own. So long as he sees to it that all true or imagined improvements are compatible with the civil order, he can otherwise leave his subjects to do whatever they find necessary for their salvation, which is none of his business. But it is his business to stop anyone forcibly hindering others from working as best they can to define and promote their salvation. It indeed detracts from his majesty if he interferes in these affairs by subjecting the writings in which his subjects attempt to clarify their religious ideas to governmental supervision. This applies if he does so acting upon his own exalted opinions— in which case he exposes himself to the reproach: Caesar non est supra Grammaticos—but much more so if he demeans his high authority so far as to support the spiritual despotism of a few tyrants within his state against the rest of his subjects.

If it is now asked whether we at present live in an enlightened age, the answer is: No, but we do live in an age of enlightenment. As things are at present, we still have a long way to go before men as a whole can be in a position (or can ever be put into a position) of using their own understanding confidently and well in religious matters, without outside guidance. But we do have distinct indications that the way is now being cleared for them to work freely in this direction, and that the obstacles to universal enlightenment, to man's emergence from his self-incurred immaturity, are gradually becoming fewer. In this respect our age is the age of enlightenment, the century of Frederick.

A prince who does not regard it as beneath him to say that he considers it his duty, in religious matters, not to prescribe anything to his people, but to allow them complete freedom, a prince who thus even declines to accept the presumptuous title of tolerant, is himself enlightened. He deserves to be praised by a grateful present and posterity as the man who first liberated mankind from immaturity (as far as government is concerned), and who left all men free to use their own reason in all matters of conscience. Under his rule, ecclesiastical dignitaries, notwithstanding their official duties, may in their capacity as scholars freely and publicly submit to the judgement of the world their verdicts and opinions, even if these deviate here Ind there from orthodox doctrine. This applies even more to all others who are not restricted by any official duties. This spirit of freedom is also spreading abroad, even where it has to struggle with outward obstacles imposed by governments which misunderstand their own function. For such governments an now witness a shining example of how freedom may exist without in the least jeopardising public concord and the unity of the commonwealth. Men will of their own accord gradually work their way out of barbarism so long as artificial measures are not deliberately adopted to keep them in it.

I have portrayed matters of religion as the focal point of enlightenment, i.e. of man's emergence from his self-incurred immaturity. This is firstly because our rulers have no interest in assuming the role of guardians over their subjects so far as the arts and sciences are concerned, and secondly, because religious immaturity is the most pernicious and dishonourable variety of all. But the attitude of mind of a head of state who favours freedom in the arts and sciences extends even further, for he realises that there is no danger even to his legislation if he allows his subjects to make public use of their own reason and to put before the public their thoughts on better ways of drawing up laws, even if this entails forthright criticism of the current legislation. We have before us a brilliant example of this kind, in which no monarch has yet surpassed the one to whom we now pay tribute.

But only a ruler\footnote{See  \href{http://en.wikipedia.org/wiki/Frederick_the_Great}{Frederick the Great}} who is himself enlightened and has no far of phantoms, yet who likewise has at hand a well-disciplined and numerous army to guarantee public security, may say what no republic would dare to say: Argue as much as you like and about whatever you like, but obey! This reveals to us a strange and unexpected pattern in human affairs (such as we shall always find if we consider them in the widest sense, in which nearly everything is paradoxical). A high degree of civil freedom seems advantageous to a people's intellectual freedom, yet it also sets up insuperable barriers to it. Conversely, a lesser degree of civil freedom gives intellectual freedom enough room to expand to its fullest extent. Thus once the germ on which nature has lavished most care—man's inclination and vocation to think freely—has developed within this hard shell, it gradually reacts upon the mentality of the people, who thus gradually become increasingly able to act freely Eventually, it even influences the principles of governments, which find that they can themselves profit by treating man, who is more than a machine, in a manner appropriate to his dignity.



\end{document}

The End.

 
From http://ebooks.gutenberg.us/WorldeBookLibrary.com/whatenli.htm
 

World Public Library and Project Gutenberg Consortia Center, 
bringing the world's eBook Collections together.


So the future isn't a boot stamping on a human face, forever. It's a person in a beige business outfit advocating beige policies that nobody wants (but nobody can quite articulate a coherent alternative to) with a false mandate obtained by performing rituals of representative democracy that offer as much actual choice as a Stalinist one-party state. And resistance is futile, because if you succeed in overthrowing the beige dictatorship, you will become that which you opposed.

Thoughts?
http://www.antipope.org/charlie/blog-static/2013/02/political-failure-modes-and-th.html

http://news.ycombinator.com/item?id=5187236













