\documentclass[a4paper,11pt,oneside]{article}
\usepackage[left=2.5cm,right=2.0cm,top=2cm,bottom=2.0cm]{geometry}
\usepackage{HS,lipsum}
\usepackage{graphicx}
\usepackage{lastpage}
\parindent0pt
\def\signature{%
 \YL\encl}
\def\defaultfont{\sffamily}
%% headings
% we default to plain - left justify
\makeatletter
\def\ps@plain{%
 \let\@oddhead\@empty
 \def\@oddfoot{\hfill \footnotesize \sffamily Page \textsf{\thepage} of \pageref{LastPage}}%
 \def\@evenfoot{\hfill\footnotesize \sffamily Page \textsf{thepage} of \pageref{LastPage} }}
\pagestyle{plain}
\makeatother
\begin{document}
\sffamily
\mbox{}
\includegraphics[width=.98\textwidth]{JVlogo.PNG}


\Date 

\OurRef{HS-118/HLG/YL/mr/5025}

\TO

\RE{WIR P3A-MEP-SS-881, WIR P3A-MEP-SS-882, P3A-MEP-SS-884,\\ 
    P3B-MEP-SS-340, P3B-MEP-SS-341,\\
    Fire Pump Alignment Inspection\\
 }

\setlength{\parskip}{\baselineskip}


We refer to the above WIRs relating to the recent witnessing of the aligning of the Fire Pumps. These were returned by the Engineer with status resubmit and the comment: `resubmit for ``A'' status, comply TPI comments nos. 2, 3 \&4. We comment as follows:

\begin{enumerate}

\item In comment number two the TPI Inspector reported as follows:

\begin{quote}\sffamily\itshape
As the allowable limits are not indicated on the manufacturer's manual of operation (only minimum given), however the Pumps Angular alignment is under the Contractor's responsibility.
\end{quote}

The Contractor confirms that the angular alignment was measured and witnessed by all parties. The values for the angular alignment, were measured and found satisfactory. The inspection was signed by the witnessing parties. The Contractor notes that the couplings were aligned to a tolerance that is less than an order of magnitude to that specified by the manufacturer. 

The \textit{maximum allowable misalignment limits} were shown on the Field Inspection Record as well as the manufacturer's literature. The maximum allowable limit for the pumps under discussion is 0.015\,in (0.38\,mm). As you one can discern from the measurements the parallel and angular settings were well within the limits specified by the manufacturer.

It is customary during pump alignment to also measure the \textit{separation distance} between  the coupling halves. This must not be less than $1/4\,\textup{in}$ (6.35\,mm). As the measurements clearly indicate, the couplings were set properly at a separation distance exceeding this limit.

The Contractor is unable to understand the TPI Inspector's comment: `As the allowable limits are not indicated on the manufacturer's manual of operation (only minimum given)'. We presume the inspector is referring to the \textit{separation distance}, as explained above and which bears no relationship to the alignment measurements. Manufacturers don't specify an upper limit as it is unnecessary.

The last part of the TPI's comment: `\ldots the Pumps Angular alignment is under the Contractor's responsibility' is noted by the Contractor. The Contractor also notes that the comment was unnecessary and unrelated to the inspection. The Contractor's responsibilities are well covered within the clauses of the Contract document and it is  preferable that statements of a contractual nature be avoided in inspection comments, which by their nature must remain focused on QA/QC procedures.  


\item In comment number three the TPI Inspector noted:

\begin{quote}
\itshape Proceed with pump tag no. prior to T\&C.
\end{quote}

The Contractor as per the QA/QC procedures will offer all tagging at the completion of all inspections as a single WIR. This comment is unrelated to a witnessing inspection for \textit{pump alignment}. 

\item In comment number four, the TPI Inspector noted:

\begin{quote}
\itshape
The Jockey pumps parallel and angular alignment is not done.
\end{quote}

Jockey pumps were not offered as part of this inspection. 

\end{enumerate}

%The Contractor  reminds the Engineer of the costs being incurred by the necessity of having to keep a large complement of staff post MOU, due to the large number of EIs and RFIs issued and by the continuous addition of new comments on drawings, many of which constitute instructions. Further complicating these issues by 
%the return of WIRs with comments such as those described in this letter and by adopting the TPIs comments without any examining the facts further adds to Engineering costs.


We request the Engineer to reconsider and upgrade the status of these WIRs. In the meantime we are proceeding with pressurizing the systems and balance activities Testing and Commissioning to avoid any delays.


\signature

\end{document}