%  Converting from one set of units 
%  to another. F->C etc.
% Use fp  package for calculations
% version 1.0
% Dr. Yiannis  Lazarides
\documentclass{article}
\usepackage{fp}
\usepackage{siunitx}
% Default three decimal places
\gdef\numdec{3}

\begin{document}
%\SetConversion{C}{K}{273.15}
\def\CtoK#1{\FPadd\result{#1}{273.15}%
\FPround\result{\result}{2}%
\sisetup{
  fixed-exponent = 2,
 scientific-notation = false}
 \num{\result}}


%% Convert Centigrate to Fahreneit
\def\CtoF#1{\FPdiv\resulta{9}{5}%
\FPmul\resultb{\resulta}{#1}%
\FPadd\resultc{\resultb}{32}%
\FPround\resultc{\resultc}{3}%
 \num{\resultc}%
\sisetup{%
fixed-exponent = 0,
scientific-notation = false}
}

%% Convert Fahreneit to Centigrade
\def\FtoC#1{\FPdiv\resulta{5}{9}%
\FPsub\resultb{#1}{32}%
\FPmul\resultc{\resultb}{\resulta}%
\FPround\resultc{\resultc}{\numdec}%
\sisetup{%
fixed-exponent = 0,
scientific-notation = false}
\num{\resultc}%
}

%% Convert Fahreneit to Rankine
\def\FtoRa#1{\FPmul\result{#1}{9}%
\FPdiv\result{\result}{5}%
\FPround\result{\result}{\numdec}%
\result%
}

%% Convert Kelvin to Rankine
\def\KtoRa#1{\FPadd\result{#1}{459.67}%
\FPround\result{\result}{\numdec}\result}

%% Convert Rankine to Celcius
\def\RtoC#1{\FPsub\result{#1}{459.67}% to fahreneit
\FtoC{\result}}


%% Testing a few numbers
\FtoC{32}

\FtoC{212}

\CtoF{0}


\end{document}