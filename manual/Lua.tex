\documentclass{article}
\usepackage[utf8]{luainputenc}
\usepackage{url}
\begin{document}


\def\A{A}
\def\Z#1{#1}
\def\name{Yiannis}


%% Function to find a file ...
\gdef\FindFile#1{%
\directlua{local function find_file(...)
    return kpse.find_file(...) or ""
end 
tex.sprint(find_file("#1"))}}

\FindFile{plain.tex}

\input{\FindFile{plain.tex}}
\end{document}

\begin{luacode}

texconfig.shell_escape = 't'

%tex.sprint(os.getenv("HOME"))

%function createDir (dirname)
%      os.execute("mkdir " .. dirname)
%    end
%
%createDir("c:test")



a = {n=10}

tex.sprint("Table "..table.getn(a).."  ")


tex.sprint("Table "..table.getn(a).."  ")


local function find_file(...)
    return kpse.find_file(...) or ""
end


tex.print(find_file("plain.tex"))



tex.sprint("Time "..os.time().."\par")
  tex.print(math.random())
  tex.print(2+2)
  \A,B=1,2
  tex.print("B".." Jane")
N, F = 1, 1
while N < 1000 do
 N = N + 1
% F = F * N
 tex.print(N.."!  "..F.."\par")
end


function Test(name)
  tex.print{name}
end

Test("\name")


% Function average
function Average(Num1, Num2)
 return (Num1 + Num2) / 2
end

Average(0, 10)

tex.print("\par")

tex.print("Average= "..Average(0, 10).."\par")

tex.print(string.lower("HELLO THEREere!").."\par")


tex.print("Length of string = "..string.len("ABCDE").."\par")

tex.print(os.date())

tex.print(os.time())


%///////////////////--
%// Curve Fitting //--
%--///////////////////--
%
%-- v 0.2
%
%-- Lua 5.1 compatible
%
%-- little add-on to the matrix module, to show some curve fitting
%
%-- http://luaforge.net/projects/LuaMatrix
%-- http://lua-users.org/wiki/SimpleFit
%
%-- Licensed under the same terms as Lua itself.
%
%-- requires matrix module
%local matrix = require "matrix"

%-- The Fit Table
% local fit = {}
%
%-- Note all these Algos use the Gauss-Jordan Method to caculate equation systems
%
%-- function to get the results
%local function getresults( mtx )
%   assert( #mtx+1 == #mtx[1], "Cannot calculate Results" )
%   mtx:dogauss()
%   -- tresults
%   local cols = #mtx[1]
%   local tres = {}
%   for i = 1,#mtx do
%      tres[i] = mtx[i][cols]
%   end
%   return unpack( tres )
%end
%
%-- fit.linear ( x_values, y_values )
%-- fit a straight line
%-- model (  y = a + b * x  )
%-- returns a, b
%function fit.linear( x_values,y_values )
%   -- x_values = { x1,x2,x3,...,xn }
%   -- y_values = { y1,y2,y3,...,yn }
%   
%   -- values for A matrix
%   local a_vals = {}
%   -- values for Y vector
%   local y_vals = {}
%
%   for i,v in ipairs( x_values ) do
%      a_vals[i] = { 1, v }
%      y_vals[i] = { y_values[i] }
%   end
%
%   -- create both Matrixes
%   local A = matrix:new( a_vals )
%   local Y = matrix:new( y_vals )
%
%   local ATA = matrix.mul( matrix.transpose(A), A )
%   local ATY = matrix.mul( matrix.transpose(A), Y )
%
%   local ATAATY = matrix.concath(ATA,ATY)
%
%   return getresults( ATAATY )
%end
%
%-- fit.parabola ( x_values, y_values )
%-- Fit a parabola
%-- model (  y = a + b * x + c * x² )
%-- returns a, b, c
%function fit.parabola( x_values,y_values )
%   -- x_values = { x1,x2,x3,...,xn }
%   -- y_values = { y1,y2,y3,...,yn }
%
%   -- values for A matrix
%   local a_vals = {}
%   -- values for Y vector
%   local y_vals = {}
%
%   for i,v in ipairs( x_values ) do
%      a_vals[i] = { 1, v, v*v }
%      y_vals[i] = { y_values[i] }
%   end
%
%   -- create both Matrixes
%   local A = matrix:new( a_vals )
%   local Y = matrix:new( y_vals )
%
%   local ATA = matrix.mul( matrix.transpose(A), A )
%   local ATY = matrix.mul( matrix.transpose(A), Y )
%
%   local ATAATY = matrix.concath(ATA,ATY)
%
%   return getresults( ATAATY )
%end
%
%-- fit.exponential ( x_values, y_values )
%-- Fit exponential
%-- model (  y = a * x^b )
%-- returns a, b
%function fit.exponential( x_values,y_values )
%   -- convert to linear problem
%   -- ln(y) = ln(a) + b * ln(x)
%   for i,v in ipairs( x_values ) do
%      x_values[i] = math.log( v )
%      y_values[i] = math.log( y_values[i] )
%   end
%
%   local a,b = fit.linear( x_values,y_values )
%
%   return math.exp(a), b
%end
%
%return fit


\end{luacode}


the standard approximation $\pi = \directlua{tex.sprint(math.pi)}$




\section{Executing functions via luacode}
\begin{luacode}
%A=13/3.1413
%
%
%tex.print(A)
%
%
%tex.print("Page goal ".." "..tex.hfuzz)
%
%

function foo(angle)
  return math.cos(angle), math.sin(angle)
end

a,b=foo(2)
tex.print("angle : "..a,b,"\par")   

tex.print("angle : "..foo(1.7890))


\end{luacode}


%\X
%
%\the\hfuzz

%\FPsub\temp{\X}{\Z}

\begin{luacode}
t={1,1,"mary",5,8,13}

tex.print(t[3])
\end{luacode}

\def\z#1{%
   \luadirect{%
    t={1,1,"#1",5,8,13}
    tex.print(t[3])}
}

\z{Johnny}


\begin{luacode}
%-- account.lua
%-- from PiL 1, Chapter 16

Account = {balance = 0}

function Account:new (o, name)
  o = o or {name=name}
  setmetatable(o, self)
  self.__index = self
  return o
end

function Account:deposit (v)
  self.balance = self.balance + v
end

function Account:withdraw (v)
  if v > self.balance then error("insufficient funds on account "..self.name) end
  self.balance = self.balance - v
end

function Account:show (title)
  tex.print(title or "", self.name, self.balance)
  tex.print("\par")
end

a = Account:new(nil,"demo")
a:show("after creation")
a:deposit(1000.00)
a:show("after deposit")
a:withdraw(100.00)
a:show("after withdraw")



\end{luacode}


Ini files are normally used as configuration files, although I have used them for all sort of data capturing. If you wish you can use a Yaml format.

;general
languages=en,ngerman

;This is a comment
[article]
  pagewidth=15cm 
  pageheight=20cm
  paper=a4

[book]
  includeparts=true

;defines chapter
  [chapter]

;packages
[graphicx]
  keys="keya,keyb,keyc"

























\end{document}