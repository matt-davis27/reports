\chapter{Structural Forms \& Devices}

 \begin{marginfigure}%
   \includegraphics[width=\linewidth]{./graphics/wells}
  \captionof{figure}{Title page by  F W Goudy}
  \label{fig:marginfig1}
\end{marginfigure}
\large\normalsize
\newthought{Make the title page a symbol of dignity and presence of the text}. The title page makes
a statement and is normally specifically composed for a book. Although a canned solution as that offered
by \latex\ is easy, adequate thought and design must be given to this. Typographically, poise is made of white space.
Many fine title pages consists of a modest line or two near the top, and a line or two near the bottom, with littel
more than taut, balanced white space in between.
\begin{marginfigure}%
  \includegraphics[width=\linewidth]{./graphics/musicbook}
  \captionof{figure}{The title page states that this is the complete edition of Skinner’s strathspeys and reels and "… is respectfully dedicated to all lovers of Scotch music", containing over one hundred strathspeys, reels, highland schottisches, slow airs, hornpipes and jigs, all composed by Skinner himself. Unlike future publications such as The Scottish Violinist (1900), this piece also contains lyrics for which Skinner had written music. The song words have been written by R. Grant, The Baird o’ Ugie, Peterhead, and La Teste of The Palace, Elgin. Lyrics contained in this work range from the title piece of the work, 'The Miller’ o Hirn', a schottische or strathspey to 'Our Highland Queen', a solo strathspey written for Queen Victoria. Certainly, this collection could be described as Skinner's first ever major published work' although he had had minor works published from an early age with the Highland Polka being printed when he was only seventeen years old. This was followed up the next year with the Ettrick Valley Quadrilles. Twelve New Strathspeys and Reels were published in 1865 being followed in 1866 with Thirty New Strathspeys and Reels, with both of these pieces being incorporated into The Miller o’ Hirn giving us Skinner’s first ever major published works.}
  \label{fig:marginfig1}
\end{marginfigure}

Earlier books had long and very descriptive titles and sub-titles. In the sidemargin, we show a book written by Skinner on Scottish music. 
\section{Headings}

headings can have many forms, but the most commonly found in books is flushed left. The example in Table shows eading 
placed symmetrically and with no numbers.


\begin{figure}[htbp]
 
  \fbox{
    \begin{minipage}{10 cm}
   \bigskip
 
   \begin {center}
     
     {\huge \bf Main Section Title} 
    \end{center}
    \lipsum[1]

\begin{center}
{\LARGE \bf MAIN CROSSHEAD} 
 \end{center}
\lorem

\begin{center}
{\large \bf Heavy Crosshead} 
 \end{center}
\lorem

\begin{center}
{\large MEDIUM CROSSHEAD} 
 \end{center}
\lorem

\begin{center}
{\large Light Crosshead} 
S{ch}u{tz}vorri{ch}tung
 \end{center}
\lorem
\caption{Headings can have any form. The Light Crosshead has been spaced out using the \texttt{\hlred{soul}} package} \index{packages!soul}
\label{fig:headings}
\bigskip
  \end{minipage}
   
  }


\end{figure}


One of the most common requirement when defining a new class or document style
is the ability to change \LaTeX\'s common way of printing chapters, section and subsections.
Unfortunately this is not as simple as it sounds. Serious thought has been given in how to internationalize \latex\:

\begin{quotation}
One of the first problems non-English users of LATEX run into, is that of English terms (‘Abstract’,
‘Contents’) contained in the document styles. The resourceful user, or the TEXnician consulted, will
probably take out a text editor and hunt through the style file for the offending string, replacing it by
its equivalent in his/her native language. This process will most likely result in new styles, called (for
Dutch) ‘artikel’, ‘rapport’ and so on.

At this point it may occur to the conscientious user that Leslie Lamport must have foreseen this situation,
and probably have made provisions for it, so why not see if there is a suggested approach for this.
A cursory perusal of the table of contents of the LATEX book will then lead our user to section 5.1.4
‘Customizing the document style’. There Lamport takes half a page to explain how the word ‘Chapter’
is really just the value of a control sequence \doccmd{@chapapp}. This implies that changing this text to
‘Hoofdstuk’ does not require editing of document styles at all, it merely needs a one-line option file.
Charming, one would say. However, immediately after that there is a strange sentence ‘You may also
want to redefine the \cmd{appendix} command, replacing Appendix [:]’. Curiouser and curiouser!
Can’t I just redefine some control sequence that yields the word ‘Appendix’? Well, as it turns out,
‘Chapter’ is the only text that has been parametrized, the rest is hard-wired into the document styles.
\end{quotation}

\section{The Latex Page}

\begin{figure}%
  \includegraphics[width=1.03\linewidth]{./graphics/pagedefinition}
  \caption{Page elements. The values shown are those in effect in the current document, not the
defaults.} 
  \begin{tabular}{rlrl}
  1 &one inch +|\hoffset|  = \the\hoffset  &2 &one inch + |\voffset| = \the\voffset\\
  3 &|\oddsidemargin|  = \the\oddsidemargin  &4 & |\topmargin| = \the\topmargin\\
  5 &|\headheight|  = \the\headheight  &6 & |\headsep| = \the\headsep\\
  7 &|\textheight|  = \the\textheight  &8 & |\textwidth| = \the\textwidth\\
  9 &|\marginparsep|  = \the\marginparsep  &10 & |\marginparwidth| = \the\marginparwidth\\
 11&|\footskip|  = \the\footskip  &  & |\marginparpush| = \the\marginparpush\\
~  &|\hoffset|  = \the\hoffset  & & |\voffset| = \the\voffset\\
~  &|\paperwidth|  = \the\paperwidth  & & |\paperheight| = \the\paperheight\\
~  &|\parskip|  = {\the\parskip\par}  & & |\parindent| = {\par \the\parindent\par}\\
 \end{tabular}
  \label{fig:marginfig1}
\end{figure}



\LaTeX\ provides a number of common commands that are shown in Table \ref{tbl:latexsec}

\begin{table}[htbp]
\begin{center}
\begin{tabular}{ll}
\toprule
Command & Result\\
\midrule
part  &\\
chapter &\\
section &\\
subsection &\\
subsubsection &\\
paragraph &\\
subparagraph &\\
\bottomrule
\end{tabular}
\caption{\LaTeX sectioning commands}
\label{tbl:latexsec}
\end{center}
\end{table}


\section{What is a page style?}

In \latex\ the page style refers to the part of the page building mechanism that
attaches the headers and footers to the actual page. Technically, in a two-page
setup, the headers and footers are inserted via four macros \cmd{@evenhead}, \cmd{@oddhead},
\cmd{@evenfoot} and \cmd{@oddfoot}. Activating a particular page style will redefine
these four macros to do whatever the style is designed to.

The actual vertical placement of the headers and footers is not a
matter of the page style. This depends on the dimensioning of the page layout.

\section{Activating a page style}
A page style can be activated in two ways: globally and locally (here locally
means \emph{on this page only}).

{\obeylines
|\pagestyle{<style>}|
|\thispagestyle{<style>}|
}

The |\pagestyle{style}| activates hstylei from the next page and onwards, while \cmd{thispagestyle} enables |style| on this page only.

The page style used on a particular page is decided as follows. If any |\thispagestyle|
macros have been issued on this page, we use the last one. If no
|\thispagestyle| macros have appeared, we use the global page style (i.e. the style
recorded to have been set via a |\pagestyle| at the previous page break/start).
\medskip


\latex\ provides four standard pagestyles:

\begin{tabular}{lp{5.0cm}}
\toprule
Style & Effect\\
\midrule
|empty| &no headers or footers\\
|plain|    &no header, footer contains page number centered\\
|headings| &no footer, header contains name of chapter/section and/or subsection and page number\\
|myheadings| &no footer, header contains page number and user supplied information\\
\bottomrule
\end{tabular}
\medskip

Although these are useful styles, they are quite limited. Additional page styles can be defined by
defining commands of the form \cmd{ps@xxx}. This command is executed when a \cmd{pagestyle\{xxx\}}
is given in the document. The \cmd{ps@xxx} command should define the following commands for the
contents of the headers and footers:
\medskip

\begin{tabular}{lp{5.0cm}}
\toprule
Style & Effect\\
\midrule
|\@oddhead| & header on odd numbered pages in two-sided documents (on all pages in one-sided)\\
|\@evenhead| &header on even numbered pages in two-sided documents\\
|\@oddfoot| &footer on odd numbered pages in two-sided documents (on all pages in one-sided)\\
|\@evenfoot| &footer on even numbered pages in two-sided documents\\
\bottomrule
\end{tabular}
\medskip



These are not user commands, but rather \emph{variables} that are used by LATEX’s output routine.
As the command names contain the character ’@’, they should be defined in a package file, or
otherwise be sandwiched between the commands \cmd{makeatletter} and \cmd{makeatother}.

So, to define a new page style called, say, thesis, you first need to define a command called |\ps@thesis|. For example, suppose you want the new page style to have empty headers, and the footers to contain the page number with a dash on either side (e.g. -10- ) centred, then you could do:

\begin{teXXX}
\newcommand{\ps@thesis}{
   \renewcommand{\@oddhead}{}%     header blank
   \renewcommand{\@evenhead}{}%    header blank
   \renewcommand{\@oddfoot}{\hfill-\thepage-\hfill}%     
   \renewcommand{\@evenfoot}{\hfill-\thepage-\hfill}%  
\end{teXXX}


The \cmd{pagenumbering} command defines the layout of the page number. It has a parameter from
the following list:
\medskip

\begin{tabular}{lp{5.0cm}}
\toprule
Style & Effect\\
\midrule
|arabic| & arabic numerals\\
|roman| & lower case roman numerals\\
|Roman| & upper case roman numerals\\
|alph| & lower case letter\\
|Alph| & upper case letter\\
\bottomrule
\end{tabular}
\medskip

The command is used as  |\pagenumbering{Alph}|. 

The \cmd{pagenumbering}{xxx} defines the command \cmd{thepage} to be the expansion of the page number
in the given notation xxx. The pagestyle command then would include |\thepage| in the
appropriate place. Additionally the |\pagenumbering| command resets the page number to 1. The
|\pagestyle| and |\pagenumbering| apply to the page that is being constructed, so they should be
used at a location where it is clear to what page they apply

\section{Changing the way headers and footer are made}

Most classes will treat the header and footers as three parts: a left a center and a right part. This can be done by using the package \docpkg{fancyheader} which we will examine a bit later on, or you can define your own by defining these 12 portions of the running headers and footers. See for example page 90 of the memoir class.



\section{Modifying sectioning commands}
The chapter info  is made up of three parts:

\begin{itemize}
     \item the number (say, 2), displayed by the macro \cmd{thechapter}, the {\thechapter} 
     \item the name (in English, Chapter), displayed by the macro \cmd{chaptername}, the \chaptername
     \item the title, contained in the argument of \cmd{chaptermark}, the |\chaptermark{A Test}|  in this chapter. 
\end{itemize}


\begin{teXXX}
\makeatletter 
  \@chapapp
\makeatother
\end{teXXX}

An example:

\begin{Verbatim}
\pagestyle{fancy}
\fancyheadoffset[LE,RO]{\marginparsep+\marginparwidth}
\renewcommand{\chaptermark}[1]{\markboth{#1}{}}
\renewcommand{\sectionmark}[1]{\markright{\thesection\ #1}}
\fancyhf{}
\fancyhead[LE,RO]{\bfseries\thepage}
\fancyhead[LO]{\bfseries\rightmark}
\fancyhead[RE]{\bfseries\leftmark}
\fancypagestyle{plain}{%
\fancyhead{} % get rid of headers
\renewcommand{\headrulewidth}{0pt} % and the line
}
\end{Verbatim}

\end{document}

\begin{Verbatim}
\renewcommand{\chaptermark}[1]{%
\markboth{\MakeUppercase{%
\chaptername}\ \thechapter.%
\ #1}{}}
\renewcommand{\chaptermark}[1]{%
\markboth{\MakeUppercase{%
\chaptername\ \thechapter.%
\ #1}}{}}
\renewcommand{\chaptermark}[1]{%
\markboth{#1}{}}
\renewcommand{\chaptermark}[1]{%
\markboth{\thechapter.\ #1}{}}
\renewcommand{\chaptermark}[1]{%
\markboth{\thechapter.%
\ \chaptername.\ #1}{}}
\end{Verbatim}

\section{Using the \texttt{titlesec} package}

    The \texttt{titlesec} and \texttt{titletoc} packages by Javier Bezos, provide an easy way 
to change the internal \LaTeXe\ commands when creating a new style. It provides a number of easy
commands and defaults.

\begin{Verbatim}
\renewcommand{\chaptertitlename}{The Chapter}
\titleformat{\chapter}%
  [display]% shape
  {\relax\ifthenelse{\NOT\boolean{@tufte@symmetric}}
              {\begin{fullwidth}}{}}% format applied to label+text
  {\Large \textsf{\chaptertitlename} \Large \textsf{\thechapter}}% label sans-serif number 
  {0pt}% horizontal separation between label and title body
  {\huge\textsf}% before the title body \itshape in tufte
  [\ifthenelse{\NOT\boolean{@tufte@symmetric}}{\end{fullwidth}}{}]% after the title body
\end{Verbatim}


You can just get away with one or two commands the most useful one is
\doccmd{titleformat}

\begin{Verbatim}
  \titleformat{command}[shape]{format}{label}{sep}{before}[after]
\end{Verbatim}

This should be followed by the command \doccmd{titlespacing} to define
the spacing before and after the section or chapter.

\begin{Verbatim}
\titlespacing*{\chapter}{0pt}{20pt}{40pt}
\titlespacing*{\section}{0pt}{3.5ex plus 1ex minus .2ex}{2.3ex plus .2ex}
\titlespacing*{\subsection}{0pt}{3.25ex plus 1ex minus .2ex}{1.5ex plus.2ex}
\end{Verbatim}

The \doccmd{titlespacing} has the following format
\begin{verbatim}
  \titlespacing{command}{left}{beforesep}{aftersep}hright]
\end{verbatim}

The starred version kills the indentation of the paragraph following the title, except in drop, wrap and
runin where this possibility does not make sense.

 left increases the left margin, except in the ...margin, and drop shape, where this parameter
sets the title width, in wrap, the maximum width, and in runin, the indentation just before the
title. With negative value the title overhangs.6

 beforesep is the vertical space before the title.

 aftersep is the separation between title and text—vertical with hang, block, and display, and
horizontal with runin, drop, wrap and ...margin. By making the value negative, you may
define an effective space of less than \parskip 7pt

 The hang, block and display shapes have the possibility of increasing the hrighti margin with
this optional argument.

\section{Functional TeX}

Functional TeX  is my attempt to bring some common programming paradigms into
text using a Javascript type environment and json notation.

\begin{Verbatim}
defines the \chapter command and its friends
chapter = {
          chaptermark : chapter, 
          thechapter  : thechapter,
          location      : ,
          spacebefore        : ,
          spaceafter          :,
          font           : fontobject,
          fontfamily  :  ,
          etc...,
          spacing :{shape: 'hang'
           }                     
}     
\end{Verbatim}


















\subsection{test}LATEX uses a
marker mechanism to remember the chapter and section (section and subsection) information for
a page; this is discussed in detail in the LATEX Companion, Section 4.3.1.


Usually, for documents of class book and report, you may want to use chapter and section
information in the headings (chapter only for one-sided printing), and for documents of class
article, section and subsection information (section only for one-sided printing).



There are two ways you can use and change the higher- and lower-level sectioning information
available to you. The macros: \doccmd{leftmark} (higher-level) and rightmark (lower-level) contain
the information processed by LATEX, and you can use them directly as shown in Section 8.
\begin{verbatim}
The \leftmark contains the Left argument of the Last \markboth on the page, the \rightmark
contains the Right argument of the fiRst \markboth or the only argument of the fiRst \markright
on the page. If no marks are present on a page they are “inherited” from the previous page.

You can influence how chapter, section, and subsection information (only two of them!) is displayed
by redefining the \chaptermark, \sectionmark, and \subsectionmark commands4. You must
put the redefinition after the first call of \pagestyle{fancy} as this sets up the defaults.



Figure 3 shows some variants for “Chapter 2. Do it now” (the last example is appropriate in
some non-English languages). The % signs at the end of the lines are to prevent unwanted space.
Normally you would continue the lines and remove these % signs5.
For the lower-level sectioning information, do the same with \markright.
So if “Section 2.2. First steps” is the current section, then
\renewcommand{\sectionmark}[1]{\markright{\thesection.\ #1}}
will give “2.2. First steps”
Redefining the \chaptermark and \sectionmark commands may not eliminate all uppercaseness.
E.g. the bibliography will have a title of bibliography in the header, as the \MakeUppercase is
explicitly given in the definition of \thebibliography. Similar for index etc. If you don’t want to
redefine these commands, you can use the \nouppercase command that fancyhdr makes available
in the header and footer fields. Note that this may screw other things, like uppercase roman
numerals in your headers, so it should be used with care. Essentially this command typesets its
argument in an environment where \MakeUppercase and \uppercase are changed into do-nothing
operations.





It should be noted that the LATEX marking mechanism works fine with chapters (which always
start on a new page) and sections (which are reasonably long). It does not work quite as well with
short sections and subsections. This is a problem with LATEX, not with fancyhdr.
As an example let’s take a page layout where the leftmarks are generated by the sections and the
rightmarks by the subsections (as is default in the article class). Take a page with some short
sections, e.g.
Section 1.
subsection 1.1
subsection 1.2
Section 2.
As the leftmark contains the last mark of the page it will be “Section 2.”, and the rightmark will
be “subsection 1.1” as it will be the first mark of the page. So the page header info will combine
section 2 with subsection 1.1 which isn’t very nice. The best you can do in these cases is use only
the \rightmarks and redefine \sectionmark accordingly. A LATEX command \firstleftmark
would also be a nice addition (see the extramarks package in section 19).
Another problem with the marks in the standard LATEX classes is that the higher level sectioning
commands (e.g. \chapter) call \markboth with an empty right argument. This means that on
the first page of a chapter (or a section in article style) the \rightmark will be empty. If this is
a problem you must manually insert extra \markright commands or redefine the \chaptermark
(\sectionmark) commands to issue a \markboth command with two decent parameters.
As a final remark you should also note that the * forms of the \chapter etc. commands do not

call the mark commands. So if you want your preface to set the header info but not be numbered
nor be put in the table of contents, you must issue the \markboth command yourself, e.g.
\chapter*{Preface\markboth{Preface}{}}
Entering the \markboth command inside the \chapter* insures that the mark will not be separated
from the title by a page break. Of course with \chapter* this wouldn’t be a problem if you
put the mark command after the chapter title, as the \chapter* command starts at a new page.
However with a \section* it could be dangerous to say:
\section*{Preface}
\markboth{Preface}{}
as a page break may occur between the two commands.

\end{verbatim}














