\section{Drawing circuits with TikZ and circuitikz}
Massimo A. Redaelli's circuitikz package
provides macros for typesetting electrical and electronic schematics.
It has a lean syntax, native to LATEX, and supports directly the
PDF output format. \ctan{circuitikz}{circuitikz}. Full examples can be found at \href{http://www.texample.net/tikz/examples/circuitikz/}{TeXample}.

\emphasis{begin,end,circuitikz}
\begin{teXXX}
\begin{circuitikz} \draw 
 (0,0) node[and port] (myand) {}
 (myand.in 1) node[anchor=east] {1}
 (myand.in 2) node[anchor=east] {2}
 (myand.out) node[anchor=west] {3};
;\end{circuitikz}
\end{teXXX}


\begin{circuitikz} 
%diode
\draw (0,0) node[and port] (myand) {}
 (myand.in 1) node[anchor=east] {1}
 (myand.in 2) node[anchor=east] {2}
 (myand.out) node[anchor=west] {3};
%Transformer
\end{circuitikz}
\medskip

A transformer can be drawn using similar commands, except now the node is called
|node[transformer]|. Anchors are provided as shown in the example.


\medskip

\emphasis{begin,end,circuitikz,transformer}
\begin{circuitikz}
\draw
 (0,-2) node[transformer] (T) {}
 (T.A1) node[anchor=east] {A1}
 (T.A2) node[anchor=east] {A2}
 (T.B1) node[anchor=west] {B1}
 (T.B2) node[anchor=west] {B2}
 (T.base) node{K};
\end{circuitikz}


\begin{teXXX}
\begin{circuitikz} \draw 
  (0,0) node[transformer] (T) {}
  (T.A1) node[anchor=east] {A1}
  (T.A2) node[anchor=east] {A2}
  (T.B1) node[anchor=west] {B1}
  (T.B2) node[anchor=west] {B2}
  (T.base) node{K}
;\end{circuitikz}
\end{teXXX}

