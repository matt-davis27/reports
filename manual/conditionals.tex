\documentclass{article}
\usepackage{lipsum,fancyvrb}
\DefineShortVerb{\|}
\begin{document}

The ability of a computer language to provide branching commands is important. TeX provides a number of conditionals.

There  are  17 control 
sequences  that   compare  various  quantities,  and  they are  used  to   make decisions  and  to  
implement  loops. Most  are  easy  to   use  even  for 
beginners, but  the  three commands  |\ifx|,   |\if|   and 
|\ifcat|   are different.  They are harder to  learn and are 
executed in different ways, are intended for different 
applications, and are confusing. 

\def\qwe#1{#1}   
\def\rty#1{#1}

\expandafter\expandafter\expandafter\ifx\qwe{1}\rty{1} 
{yes}\else {no}
\fi 

This   suggests  one use  for  |\ifx|  
namely,  comparison  of  strings.  To  compare  two 
strings,  place  them  in  macros,  and  compare  the  
macros.  |\ifx|   is, in  f a c t ,  heavily used  in  ref.  1 for 
this  purpose.  However


The |if| conditional.

The |if|conditional expands its arguments and then compares them. 

\def\star#1{#1}

\if\star{*}* True is star 
   \else false is not start
\fi

\if\qwe{1}\rty{1} 
{yes}\else {no}
\fi 

\def\test#1{
\def\zero{0}
\if\zero#1 true \else no\fi}


\test1
\end{document}

\def\sorttext#1{\setbox0\vbox{{\language255\hsize=0pt\hfuzz\maxdimen
\parfillskip0pt\noindent#1\par}\sortvlist\unpack}\unvbox0 }
\def\sortvlist{{\unskip\unpenalty \setbox0\lastbox
\ifvoid0\noindent\else\setbox0\hbox{\unhbox0\ }\sortvlist\sortin\fi}}
\def\sortin{\setbox2\lastbox\ifdim\wd2>\wd0{\sortin}\fi\box2\box0}
\def\unpack{{\setbox0\lastbox\ifvoid0\indent\else\unpack\unhbox0\fi}}
\sorttext{riverrun past Eve and Adam's linsfirst loved livvy.}
\sorttext{\lipsum[1]}