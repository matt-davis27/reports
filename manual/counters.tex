\chapter{Counters}

A counter is a \tex variable taking integer values. You can choose
to define independent counters or counters subordinate to some
other counter. In the last case the counter example is reset to zero
whenever the counter other is increased. \index{counters}
\index{counters!LaTex counters}

\subsection{Creation and initialization}
\index{counters!counters}

\latex defines a few macros and allocate registers automatically for them. You can create a new counter with the command \cmd{newcounter}.

\begin{teX}
\newcounter{example}
\newcounter{example}[other]
\end{teX}

Usually you place this command in the preamble of your
document. If you forget to initialize a counter before you refer to
it, you will get the \latex\  \texttt{Error No counter ‘example’ defined}.

The value of a counter can be displayed by using:

\begin{teX}
  \theexample
\end{teX}

Every time \latex creates a variable to hold the name for a counter it also defines |the+countername|, to conveniently provide means to display get the contents of the counter.


\section*{Example}

\begin{teX}
\newcounter{acount}
\setcounter{acount}{0}
\stepcounter{acount}
\addtocounter{acount}{14}

\theacount

\ifnum\theacount>10 number is greater
        than ten \else number is less than ten\fi
\end{teX}



\section{Modifying counters}

There are four ways to change the value of your counter, excluding
the possibility to reset it using dependancies. These are:

\begin{teX}
\setcounter{example}{<value>}
\addtocounter{example}{<value>}
\stepcounter{example}
\refstepcounter{example}
\end{teX}

The first two set respectively increase the counter example by the
specified value. The other two increase the value of the counter by
one and reset the counters subordinate to example to zero. If you
use the reference step, references to labels placed after this
command will refer with respect to this counter until further notice.

\section{Examples of counters}

\tex\ and \latex\ make heavy use of counters, for example:

\begin{description}
\item{section, subsection} To enumerate sections, subsections (etc.);
page To keep track of the page number;
\item{enumi, enumii} The counter of (nested) enumerate environments;
\item{tocdepth} Table of contents depth;
\item{secnumdepth} The section numbering depth;
\item{footnote} To mark footnotes;
\item{figure, table} To label the figures and tables;
\item{equation} To label equations.
\end{description}

Notice that the labels subsection up to subparagraph are
subordinate to the next highest sectioning command.

How much?

\begin{teX}
\newcounter{exercise}
\newcounter{subex}[exercise]
\setcounter{subex}{-2}
\refstepcounter{exercise}
\stepcounter{subex}
\addtocounter{subex}{2}
\addtocounter{exercise}{1}
\label{test}
\end{teX}

\section{Displaaying counters in your text}

At the moment you define a new counter example, LATEX creates a
new command \doccmd{theexample}, which displays the counter in Arabic
numbers. Standard ways to display a counter are:
\marginnote{
\break\doccmd{alph}\break
 \doccmd{Alph}\break 
\doccmd{arabic}\break 
\doccmd{fnsymbol}\break 
\doccmd{roman}\break
\doccmd{Roman}\break}

alph a,b,c,d,e,f,g,h,i,. . .

Alph A,B,C,D,E,F,G,H,I,. . .

arabic 1,2,3,4,5,6,7,8,9,. . .

fnsymbol .

roman i,ii,iii,iv,v,vi,vii,viii,ix,. . .

Roman I,II,III,IV,V,VI,VII,VIII,IX,. . .

\section{Changing the display type of a counter}

\begin{teX}
\newcounter{RomanYear}
\setcounter{RomanYear}{\year+1}

The year \theRomanYear\  in Roman numerals is \Roman{RomanYear}
\end{teX}
\medskip

This will output:

\newcounter{RomanYear}
\setcounter{RomanYear}{\year+1}
\texttt{The year \theRomanYear\  in Roman numerals is \Roman{RomanYear}}

In the above example \cmd{year} is a \tex\ command that gets the year from the
system automatically. I have also used the `+1' to increment the year. This is not a standard command but is provided
with the package \docpkg{calc}, which is part of the standard \latex\ distribution.
\TODO{add about text counters}

\section{calculations}

You can use the package \docpkg{calc} to carry out calculations while setting counters, but remember that it is all integer arithmetic:

\begin{teX}
\newcounter{X}
\setcounter{X}{(7/2)+1}
\theX\relax
\end{teX}


\section{latex counters and allocations}

\latex has a class specifically defining various counters and variables called |ltalloc.dtx|

The following are from plain TEX:

\begin{tabular}{lp{4.0cm}}
\toprule
Counter & Descripttion\\
\midrule
|\z@|    &A zero dimen or number. It's more efficient to write |\parindent\z@| than |\parindent 0pt|.\\
|\@ne|   &The number 1.\\
|\m@ne| &The number -1.\\
|\tw@|    &The number 2\\
|\sixt@@n| &The number 16\\
|\@m|         &The number 1000\\
|\@MM|      &The number 20000\\
|\@xxxii|    &The constant 32\\
|\chardef\@xxxii=32|   &\\
|\@Mi|  &Constant 10001\\
|\@Mii|  &Constant 10002\\
|\@Miii|  &Constant 10003\\
|\@miv|  &Constant 10004\\
\bottomrule
\end{tabular}
\bigskip

{Large\obeylines
W. Smith
W{\kern -0.1em}. Struckmann
WA
}













