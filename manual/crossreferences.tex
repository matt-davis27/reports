\chapter{Cross Referencing}
\label{ch:crossref}
This chapter discusses \latex's commands and methods for cross-referencing.
The basic user commands provided are the \cmd{label} and \cmd{ref} that are used to define  cross-references.

\newthought{counters}

In order to store the values a counter |\ref{<foo>}| is used. This stores the most recently incremented referencable counter. in the
current environment. (Chapter, section, theorem and enumeration counters counters
are referencable, footnote counters are not.)


|\pageref{foo}|: page number at which |\label{foo}| command appeared.
where foo can be any string of characters not containing |`\'|, |`{'| or |`}'|.

It is important to note that the scope of the |\label| command is delimited by environments, so

\begin{teXXX}
\begin{theorem} 
  \label{foo} ... 
\end{theorem} \label{bar}
\end{teXXX}

defines |\ref{foo}| to be the theorem number and |\ref{bar}| to be the current
section number.

Note: |\label| does the right thing in terms of spacing -- i.e., leaving a space
on both sides of it is equivalent to leaving a space on either side.

A refence counter \texttt{CNT} is incremented by the command \cmd{refstepcounter}, which sets \label{refcount}

\begin{teXXX}
  \@currentlabel == {cnt}{eval(\p@count\theCNT)} 
\end{teXXX}

The command
|\label{FOO}| then writes the following to the auxiliary file |\indexat{auxout}|.

\emphasis{@currentlabel,thepage,newlabel}
\begin{teXXX}
  \newlabel{FOO}{{eval(\@currentlabel)}{eval(\thepage)}}
\end{teXXX}


\begin{Verbatim}
10 \def\@setref#1#2#3{%
11  \ifx#1\relax
12   \protect\G@refundefinedtrue
13   \nfss@text{\reset@font\bfseries ??}%
14   \@latex@warning{Reference `#3' on page \thepage \space
15          undefined}%
16   \else
17    \expandafter#2#1\null
18  \fi}
19 \def\ref#1{\expandafter\@setref\csname r@#1\endcsname\@firstoftwo{#1}}
20 \def\pageref#1{\expandafter\@setref\csname r@#1\endcsname
21                \@secondoftwo{#1}}
\end{Verbatim}

The definition of \cmd{label} will cause a |\label{bar}| command to define |\ref{bar}| to expand to something
like |\foo{4.d}|

\begin{teXXX}
32 \def\label#1{\@bsphack
33    \protected@write\@auxout{}%
34       {\string\newlabel{#1}{{\@currentlabel}{\thepage}}}%
35   \@esphack}
36 \def\refstepcounter#1{\stepcounter{#1}%
37   \protected@edef\@currentlabel
38 {\csname p@#1\endcsname\csname the#1\endcsname}%
39 }
\end{teXXX}

It is important to note that lines [33-34] write a string to the auxiliary file, that is the \cmd{newlabel} is not expanded but is merely written in the auxiliary file:

\begin{teX}
\newlabel{ch:crossreferences}{{1}{9}{Cross Referencing\relax }{chapter.1}{}}
\end{teX}

The basic mechanism is that every \cmd{label} stores its information in the auxiliary file in the form of a command, when the file is compiled the auxiliary file is read and the commands generated are then able to be executed and the |\ref| printed out.

\ref{ch:crossref}

The basic form of the command is an |r@| prefix plus the label |r@ch:crossref|.

%\begin{figure}
%\caption{Test}
%\label{fig:test}
%\end{figure}

So, now we have references and labels, which is perhaps the strongest points of \alltex.






