\chapter{Working with tabular data}

This \docpkg{PgfplotsTable} package reads tab-separated numerical tables from input and generates code for pretty-printed
\latex -tabulars. It rounds to the desired precision and prints it in different number formatting styles.

|PgfplotsTable| works with plain text  tables in which entries \emph{cells} are separated by a separation
character. The initial separation character is white space which means at least one space or tab"(see
option col sep below). Those tables can have a header line which contains column names and most other
columns typically contain numerical data.

The following listing shows pgfplotstable.example.dat and is used often throughout this documentation.

\medskip
\begin{fullwidth}
   \pgfplotstabletypeset{pgfplotstable.example.dat}
\end{fullwidth}
\medskip

In the sections that follow you will see how to format the tables. First, we will see how we can select only particular
columns

\begin{teX}
\pgfplotstabletypeset[columns={dof,level,[index]4}]{pgfplotstable.example1.dat}
\end{teX}

\medskip

\pgfplotstabletypeset[columns={dof,level,[index]4}]{pgfplotstable.example1.dat}

\medskip

\pgfplotstabletypeset[
columns={dof,error1,info},
column type/.add={|}{}  % results in '|c'
]
{pgfplotstable.example1.dat}

\medskip


You can use \doccmd{toprule}, \doccmd{midrule} and \doccmd{bottomrule} in the |.style| is |booktabs| is loaded. This will
produce more good-looking tables
\medskip
\begin{teX}
\pgfplotstabletypeset[
    every head row/.style={before row=\toprule,after row=\midrule},
    every last row/.style={after row=\bottomrule}
  ]
{pgfplotstable.example1.dat}
\end{teX}
\medskip




\begin{fullwidth}
\begin{table*}[htbp]
\pgfplotstabletypeset[
    every head row/.style={before row=\toprule,after row=\midrule},
    every last row/.style={after row=\bottomrule}
  ]
{pgfplotstable.example1.dat}

\caption{Tabular data imported from a file}{~} %second bracket will push marginpar down

\end{table*}

\end{fullwidth}


\bigskip


The conversion from an unprocessed input table to a final typesetted tabular code uses four stages for every
cell,

\begin{enumerate}
  \item  Loading the table,
  \item  Preprocessing,
  \item Typesetting,
  \item  Postprocessing.
\end{enumerate}

The main idea is to select one typesetting algorithm (for example format my numbers with the configured
number style). This algorithm usually doesn't need to be changed. Fine tuning can then be done using
zero, one or more preprocessors and postprocessors. Preprocessing can mean to select only particular rows
or to apply some sort of operation before the typesetting algorithm sees the content. Postprocessing means
to apply fine-tuning to the resulting TEX output { for example to deal with empty cells or to insert unit
suffixes or modify fonts for single cells.




\section{Writing the data to a file}
\begin{teX}
\section{Writing to a file directly}

%% Define a new command for activity key-dates
%% this can be saved for shipout later
\newcommand{\keydate}[2]{#1  #2 \\}
\newcommand{\out}[2]{\immediate\write\tempfile{\noexpand\keydate{ #1}{#2}}}

%% We open the file to  to write the key-dates
%% we will use it later to import
\newwrite\tempfile
\immediate\openout\tempfile=keydates.tex


%% Example
%% 
   
   \out{Close Ceilings Rotana Floor 10}{Line 2}
   \out{Rotana,  power-on}{12$^{th}$ Sept 2010}
   \out{Rotana Qatar Cool switch-on}{12th September}

%% finally we close the file
%%
\immediate\closeout\tempfile

%% we now read the file and dostuff with it
%% 
\section{Summary of  Key Dates}
\keydate { Naffco Controllers}{30 January 2013}
\keydate { Rotana kitchen extract}{25 Feb 2013}
\keydate { Shangri-la kitchen extract}{15 Mar 2013}
\keydate { Merweb kitchen extract}{30 Mar 2013}

\time
\end{teX}


\section{Bonus Part}
\normalsize

After reading the table from a file, you can export the file as HTML:

\begin{teX}
\pgfplotstabletypeset[
    begin table={<table>}, end table={</table>},
    typeset cell/.style={
      /pgfplots/table/@cell content={<td>#1</td>}
    },
   before row=<tr>,after row=</tr>,
   skip coltypes, typeset=false, string type,
   TeX comment=,
   columns={level,dof},
   outfile=pgfplotstable.example1.out.html,
]{pgfplotstable.example1.dat}

\lstinputlisting
    [basicstyle=\ttfamily\small]
    {pgfplotstable.example1.out.html}
\end{teX}

This will produce the following HTML table. 


\pgfplotstabletypeset[
begin table={<table>}, end table={</table>},
typeset cell/.style={
/pgfplots/table/@cell content={<td>#1</td>}
},
before row=<tr>,after row=</tr>,
skip coltypes, typeset=false, string type,
TeX comment=,
columns={level,dof},
outfile=pgfplotstable.example1.out.html,
]{pgfplotstable.example1.dat}
\lstinputlisting
[basicstyle=\ttfamily\small]
{pgfplotstable.example1.out.html}



\section{The datatools package}
\normalsize
\index{datatools!introduction}
\index{datatools!database}
You can use the \docpkg{datatools} package to get a bit closer to a real database. The datatool package consists of the \docpkg{datatool}, \docpkg{datapie} and \docpkg{dataplot} packages. 

You can use the datatool package to
\begin{enumerate}
\item create or load databases
\item sort rows of a database (either numerically or alphabetically, ascending or descending)
\item perform repetitive operations on each row of a database
\item Check if an argument is an integer, a real number, currency or string.
\item Names can be converted to initials
\item strings can be tested to determine if they are upper or lower case.
\item can compare strings
\end{enumerate}

The datatool package author suggest. 

\begin{quote}
Once
a database has been created (or loaded), it is possible to iterate through each row
of data, to make it easier to perform repetitive actions, such as mail merging.

Whilst \tex is an excellent typesetting language, it is not designed as a
database management system, and attempting to use it as such is like
trying to fasten a screw with a knife instead of a screwdriver: it can be done, but
requires great care and is more time consuming. Version 2.0 of the datatool package
uses a completely different method of storing the data to previous versions.1 As
a result, the code is much more efficient, however, large databases and complex
operations will still slow the time taken to process your document. Therefore, if
you can, it is better to do the complex operations using whatever system created
the data in the first place.
\end{quote}




