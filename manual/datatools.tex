% !TEX TS-program = pdflatex
% !TEX encoding = UTF-8 Unicode

% This is a simple template for a LaTeX document using the "article" class.
% See "book", "report", "letter" for other types of document.

\documentclass[11pt]{article} % use larger type; default would be 10pt

\usepackage[utf8]{inputenc} % set input encoding (not needed with XeLaTeX)

%%% Examples of Article customizations
% These packages are optional, depending whether you want the features they provide.
% See the LaTeX Companion or other references for full information.

%%% PAGE DIMENSIONS
\usepackage{geometry} % to change the page dimensions
\geometry{a4paper} % or letterpaper (US) or a5paper or....
% \geometry{margins=2in} % for example, change the margins to 2 inches all round
% \geometry{landscape} % set up the page for landscape
%   read geometry.pdf for detailed page layout information

\usepackage{graphicx} % support the \includegraphics command and options

% \usepackage[parfill]{parskip} % Activate to begin paragraphs with an empty line rather than an indent

%%% PACKAGES
\usepackage{booktabs} % for much better looking tables
\usepackage{array} % for better arrays (eg matrices) in maths
\usepackage{paralist} % very flexible & customisable lists (eg. enumerate/itemize, etc.)
\usepackage{verbatim} % adds environment for commenting out blocks of text & for better verbatim
\usepackage{subfig} % make it possible to include more than one captioned figure/table in a single float
% These packages are all incorporated in the memoir class to one degree or another...

%%% HEADERS & FOOTERS
\usepackage{fancyhdr} % This should be set AFTER setting up the page geometry
\pagestyle{fancy} % options: empty , plain , fancy
\renewcommand{\headrulewidth}{0pt} % customise the layout...
\lhead{}\chead{}\rhead{}
\lfoot{}\cfoot{\thepage}\rfoot{}

%%% SECTION TITLE APPEARANCE
\usepackage{sectsty}
\allsectionsfont{\sffamily\mdseries\upshape} % (See the fntguide.pdf for font help)
% (This matches ConTeXt defaults)

%%% ToC (table of contents) APPEARANCE
\usepackage[nottoc,notlof,notlot]{tocbibind} % Put the bibliography in the ToC
\usepackage[titles,subfigure]{tocloft} % Alter the style of the Table of Contents
\renewcommand{\cftsecfont}{\rmfamily\mdseries\upshape}
\renewcommand{\cftsecpagefont}{\rmfamily\mdseries\upshape} % No bold!

%%% END Article customizations

%%
\usepackage{etex}
\usepackage{pgfplotstable}
% recommended:
\usepackage{booktabs}
\usepackage{array}
\usepackage{colortbl}
\usepackage{lipsum}
\usepackage{datatool}


%%% The "real" document content comes below...

\title{Brief Article}
\author{The Author}
%\date{} % Activate to display a given date or no date (if empty),
         % otherwise the current date is printed 

\begin{document}

\section{Tools to Manipulate Data}

In Chapter we discussed how information can be written to a file and retrieved later. This basic operation makes it 
possible to create databases using flat files. Two great packages make the job even easier
pgfplotstable and datatool. 

\section{The datatool package}

The datatool package consists of the \docpkg{datatool}, datapie and dataplot packages. Personally I use it only for data manipulation
that requires sorting.

You can use the datatool package to
\begin{enumerate}
\item create or load databases
\item sort rows of a database (either numerically or alphabetically, ascending or descending)
\item perform repetitive operations on each row of a database
\item Check if an argument is an integer, a real number, currency or string.
\item Names can be converted to initials
\item strings can be tested to determine if they are upper or lower case.
\item can compare strings
\end{enumerate}


\pgfplotstableset{% global config, for example in the preamble
columns/dof/.style={int detect,column type=r,column name=\textsc{Dof}},
columns/error1/.style={
sci,sci zerofill,sci sep align,precision=1,sci superscript,
column name=$e_1$,
},
columns/error2/.style={
sci,sci zerofill,sci sep align,precision=2,sci 10e,
column name=$e_2$,
},
columns/{grad(log(dof),log(error2))}/.style={
string replace={0}{}, % erase '0'
column name={$\nabla e_2$},
dec sep align,
},
columns/{quot(error1)}/.style={
string replace={0}{}, % erase '0'
column name={$\frac{e_1^{(n)}}{e_1^{(n-1)}}$}
},
empty cells with={--}, % replace empty cells with '--'
every head row/.style={before row=\toprule,after row=\midrule},
every last row/.style={after row=\bottomrule}
}
\pgfplotstabletypeset[ % local config, applies only for this table
1000 sep={\,},
columns/info/.style={
fixed,fixed zerofill,precision=1,showpos,
column type=r,
}
]
{example.dat}

\end{document}