\newgeometry{left=25mm,right=10mm,top=25mm,bottom=15mm}
\chapter{Document Level Javascript}
\index{JavaScript}
\label{JavaScript}

\begin{multicols}{2}
Most documents produced by \alltex end being in one of the pdf formats. If I would take a guess, most users todat either use pdfLaTeX or XeLaTeX. The PDF file format is an open standard similar to the DVI and has since sometime ago, become an ISO standard.

To enable annotations and interactivity, one can embed into a PDF document Javascript. In this section, we will cover some of the ways you can achieve this, as well as describe some of the limitations.



\section{What is Document Level Javacript?}
\index{PDF}
In a PDF document scripts are stored in a special location. When the PDF document is opened, 
the document level functions are scanned, and any "exposed script" is executed.

Normally, the type of scripts you would place at the document level are general purpose.


\section{The \protect\texttt{execJS} Environment}
 
One package that can assist you with Document Level Javascript is the bundle in |acroread|. These series of packages is the sole work of , but they do not appear to be supported and maintained well. 

\begin{teXXX}
 \usepackage[pdftex]{insdljs}
\end{teXXX}

The |execJS| is used primarily for post-creation processing. The |execJS| environment can be used, for example, to automatically import named icons into the document, which can in turn, be used for an animation.

The |execJS| is an environment in which you can write verbatim JavaScript code. It writes a couple of auxiliary files to disk; in particular, the environment creates an \texttt{.fdf} file generated by the |execJS| environment. This JavaScript is not saved with the document.



\section{Forms essentials}
%\begin{marginfigure}%
%  \includegraphics[width=\linewidth]{./graphics/amato}
%  \caption{Boxes in \protect\TeX}
%  \label{fig:boxes}
%\end{marginfigure}

You can extend the capability of your forms by using JavaScript to automate formatting, calculations, and data validation. In addition, you can develop customized actions assigned to user events. Finally, it is possible for your forms to interact with databases and web services

There are two types of PDF forms: Acrobat forms and Adobe LiveCycle Designer forms (XML form object model).

Acrobat forms present information using form fields. They are useful for providing the user with a structured format within which to view or print information. Forms permit the user to fill in information, select choices, and digitally sign the document. Once the user has entered data, the information within the PDF form can be sent to the next step in the workflow for extraction or, in the case of browser-based forms, immediately transferred to a database. If you are creating a new form, the recommended type is LiveCycle Designer forms since its format readily allows for web service interactions and compatibility with document processing needs within enterprise-wide infrastructures.

The XML form object model uses a document object model (DOM) architecture to manage the components that comprise a form. These include the base template, the form itself, and the data contained within the form fields. In addition, all calculations, validations, and formatting are specified and managed within the DOM and XML processes.

Static XML forms were supported in Acrobat 6.0, and dynamic XML forms are now supported in Acrobat 7.0. Both types are created using LiveCycle Designer. A static XML form presents a fixed set of text, graphics, and field areas at all times. Dynamic XML forms are created by dividing a form into a series of subforms and repeating subforms. They support dynamically changing fields that can grow or shrink based on content, variable-size rows and tables, and intelligent data import/export features. \lstinline!var i:integer;!
\end{multicols}

\emphasis[orange]{function,if}
\begin{teXXX}
\documentclass[a4paper]{article}
\usepackage[latin1]{inputenc}
\usepackage[T1]{fontenc}
\usepackage{textcomp}
\usepackage{mathptmx}
\usepackage[scaled=.92]{helvet}
\renewcommand{\familydefault}{phv}
\usepackage[left =25mm, top=25mm, bottom=10mm, right=10mm]{geometry}
\usepackage{fancyhdr}
\lhead{Camel Press, Inc.} \chead {} \rhead {XYZ Dept.}
\lfoot{} \cfoot{} \rfoot {}
\pagestyle{fancy}
\usepackage{graphicx}
\usepackage{color}
\usepackage[pdftex, a4paper= true, colorlinks=true,
   pdftitle={Key form}, pdfsubject={Key},
   pdfauthor={Dr Y Lazarides}, pdfpagemode=UseNone, 
   pdfstartview =FitH, pagebackref, pdfhighlight={/N}
]{hyperref}
\usepackage[pdftex]{insdljs}
\end{teXXX}
\begin{multicols}{2}
\textbf{An alert box.}\quad All Javascript is preferable to be typed in the document preamble, and even better in 
a separate file. We use the |insDLJS| environment to delimit the JavaScript code, within 
the normal \latex code. We provide a simple function, that will pop and alert box, with
a ``Hello! World!'' message. 
\end{multicols}
\begin{teXXX}
\begin{insDLJS}[exaaaa]{exaaaa}{Document Level JavaScript}
//indicate function has not as yet run
var done=0;
//function to run when opening the document
function myFirstJavaScriptFunction()
{
  if(!done){
    done=1;
    app.alert("Hello! World!");
  } 
}
\end{insDLJS}
\end{teXXX}
\begin{multicols}{2}
\index{OpenAction}
\textbf{Activate the function.}\quad The Javascript function itself, will not be activated, unless we tell Acrobat to activate it. This is done, via the
|\OpenAction| command provided by the package:
\end{multicols}
\begin{teXXX}
\OpenAction{/S /JavaScript /JS (myFirstJavaScriptFunction();)}
\begin{document}
\section*{Request for akey}
 \begin{Form}
    {form}(*@\label{form}@*)
 \textbf{Employee} \\
%% Note in JavaScript backgroundcolor = fillColor 
\long\def\textbox#1#2{\parbox{2cm}{First name }\TextField[name={#1},value={#2}, bordercolor={0.65 0.79 0.94}, backgroundcolor={0.25 0625 0.75}]{ }
\smallskip}
\textbox{Yiannis}{Test}
 \parbox{2cm}{Name       }\TextField[name=name]{ } 
 \smallskip
 \parbox[t]{2cm}{Department :} \ChoiceMenu[name= abt ]{}{%
 Sales=v, Production=f ,Service=s } \ \
 \textbf{Time} \ \
 Time : \ChoiceMenu[name=zeit, combo=true, bordercolor={0.65 0.79 0.94}]{}{%
 limited=b, unlimited=u}\\
 from :\TextField[name=from]{} \\
 until: \TextField[name=until]{} \\
 \textbf{Doors} \\
 Frontdoor : \CheckBox[name=ht]{ } \\
 Ground floor : \CheckBox [name=e1]{ } \\
 Firstfloor: \CheckBox[name=e2]{} \\
\end{Form}
\end{document}
\end{teXXX}
\begin{multicols}{2}
\section{Acrobat form field types}
There are seven types of Acrobat form fields, each associated with a field type value as shown in the following table.
\medskip
{\centering
\begin{tabular}{ll}
\toprule
Form field &Field type value\\
\midrule
Button &button\\
Check box &checkbox\\
Combo box &combobox\\
List box &listbox\\
Radio button &radiobutton\\
Text field &text\\
Digital signature &signature\\
\bottomrule
\end{tabular}
\par
}
\bigskip

Note that line  \ref{openaction} and line \ref{form} is the line number in the code above.

The \cmd{textField} can be used to provide a textfield of a form:

\begin{teXXX}
\textField 
#1 optional, used ot enter any modification of 
     the appearance or actions
#2 the title of the text field
#3 the width of the bounding rectangle
#4 the height of the bounding rectangle
\TextField[name=radius,width=10em, bordercolor={0.650 .790 .94}]{}%
\end{teXXX}

\clearpage

\section{pdfliteral}

Besides JavaScript pdfliterals can be injected striaght into the source,
a naive use woulbe be as follows:

\emphasis{pdfsave,pdfliteral,pdfrestore}
\begin{teXXX}
Hamburgensis
\pdfsave\pdfliteral page {0.1 w 1 Tr}Hamburgensis\pdfrestore
Hamburgensis
\end{teXXX}


{\printf{>\Huge \pdfsave\pdfliteral page {0.1 w 1 Tr}Hamburgensis\pdfrestore}}

\normalsize

\end{multicols}

\restoregeometry

\def\pdf{\texttt{PDF}\xspace}
\let\PDF\pdf

\section{What's in a PDF?}

A typical \PDF file contains objects, multiple compression mechanisms, different font formats, and a mixture of vector and raster graphics together with a varity of metadata and ancillary content. Create a small hello world program with pdfLaTeX and then open it with your text editor.

\begin{Code}
\documentclass{minimal}
\begin{document}
Hello World!
\end{document}
\end{Code}

I opened it using a common editor \texttt{Notepad2} and got:

\begin{figure}[htbp]
\includegraphics{./graphics/hello-pdf}
\end{figure}


\section{Building a Simple PDF}

Before we start experimenting with building a \pdf from scratch, we will need some tools that do not come necessarily with the \tex distribution (although we do not really need them). Download |pdftk| from pdflabs
\footnote{\protect\url{http://www.pdflabs.com/docs/install-pdftk/}}

Follow the instructions on the web page. Download and unzip the following file. Move the resulting pdftk.exe and libiconv2.dll to a convenient location, such as  |C:\windows\system32|. Open a command prompt and type |pdftk --help|. You should see the |pdftk| documentation appear in the terminal.  The toolkit allows you to merge and split \pdf documents, rotate \pdf pages, decrypt and encrypt documents, fill \pdf forms with data, apply watermarks, attach files and print and change \pdf metadata.

\section{File layout}

A simple valid \pdf has four parts, in order:

\begin{enumerate}[1.]
\item The \textit{header}, which defines the \pdf version number.
\item The \textit{body}, containing the pages, graphics and all the ancillary information, all encoded as a series of \textit{objects}.
\end{enumerate}

The header contains infomation about the version of \pdf available. In this case it is shown as |%PDF-1.5|. 
Most files are backwards and forward compatible and especially Adobe Reader will read broken pdfs. As pdf files always contain binary data, they can easily get corrupted if line endings are changes, if for example they are transferred over ftp or as email. To allow legacy file transfer programs to determine that the file is binary, it is ususal to include some btes with character codes higher than 127 in the header.

|%ÐÔÅØ|

The percent sign indicates another header line, the other few bytes are arbitrary character codes in excess of 127. The whole header in our hello-pdf example is:


|%PDF-1.5|

|%ÐÔÅØ|



























