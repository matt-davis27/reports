

\chapter{fun}
Let us now have some fun, for a change. Let us obscure some Christmas Carols!

\bgroup\RaggedRight\let\Lit=\let\Lit\r=\char\def\cj{\r32}\def\fff{f}
\def\ra{\r97}\def\I{Jingle\cj}
\Lit\fj=\def 
\def\Ii{jingle }  
\def\ii{bells } 
\def\e{and }
\def\t{the }
\def\ti{The }
\def\sh{sleigh}
\def\Ij{In a one horse open \sh\jjj }
\def\ig{ing }
\def\fn{fun }
\def\ih{Oh, }

\fj\ght{th}
\def\wi{w\ra y }
\fj\wg{we got }
\fj\wG{We got }
\fj\wt{wh\ra t }
\fj\wT{Wh\ra t }
\fj\hh{it is to }
\fj\si{\smallskip}
\fj\sn{Dashing through the snow }
\fj\ll{soon Miss F\ra nny Bright }
\fj\lll{Mis\fff ortune seemed  his lot }
\fj\upi{upsot }
\fj\rd{ride }
\fj\ws{w\ra s }
\fj\wS{W\ra s }
\fj\ug{ugh}
\fj\L{L\ra \ug}
\fj\l{l\ra \ug }
\let\jjj=^^M  



\let\zi=\catcode 
\zi`\.=1 \zi`\;=2  %redefine opening and closing braces
\fj\ja.\sn\jjj%def
\Ij\jjj O'er \t fields we go\jjj \L\ig all \t \wi\jjj \ii on bob tails r\ig\jjj {M}{a}{k}\ig spirits bright\jjj \wT\fn \hh \l \e  s\ig\jjj A \sh\ig song tonight\jjj\si;

\let\v=\zi   \v `\!=0 !fj!jb.!ih !Ii !ii, !Ii !ii !jjj !I all !t !wi !jjj !ih !wt !fn!hh !rd !jjj !Ij !I !ii, !Ii !ii !jjj !I all !t !wi!jjj 
!ih !wt !fn !hh !rd !jjj; !si!fj!jbi.!Ij!jjj!si;  

\v `\?=0
\fj\jc.A day or two ago?jjj I thought I'd take a ride?jjj \uppercase{\e}\ll\jjj ?wS seated by my side?jjj 
\ti horse ?ws lean ?e lank?jjj !lll?jjj \wG into a drifted bank\jjj \e then \wg \upi\jjj\si;
\fj\jbii.\jb\jbi\Ij  yeah\jjj; \let\v=\catcode \v`\.=1 \v`\;=2
!fj!jbiii.!jb!jbi;
\ja
\jb
\jbi
\jc\jbii
\v `\|=0  |jbiii




\egroup


\def\rvs#1#2#3#4#5#6#7 {#7#6#5#4#3#2#1}
\rvs thguoht   
\rvs eth........     


\begin{teX}
 \makeatletter

 % The recursion macro used (from David Carlisle)
\def\Recursion#1{%
 #1\relax
 \expandafter\@firstoftwo
 \else
 \expandafter\@secondoftwo
 \fi}

 \newcount\IndexRecursion
 \IndexRecursion=\z@

 \def\PiFrac#1{{%
 \Recursion
 {\ifnum#1>\@ne\relax}
 {\@tempcnta=#1
   \advance\@tempcnta\m@ne
   \advance\IndexRecursion\@ne
   \PiValues(\IndexRecursion)
   +\frac{\strut\displaystyle 1}{\strut\displaystyle\PiFrac{\@tempcnta}}}
 {\advance\IndexRecursion\@ne
 \PiValues(\IndexRecursion)}}}



\newarray\PiValues
\readarray{PiValues}{3&7&15&1&292&1&1&1&2&1&3&1&14}
 \(\pi\approx\PiFrac{4}\approx\PiFrac{13}\)
 \textbf{\(\pi\) as a continued fraction}
\makeatother

\relax

\end{teX}




