\part{Data Structures}

\chapter{Data Structures}

\chapter{Lamda Calculus}
\normalsize
\makeatletter
% Filename: Lambda.sty
% Author: Alan Jeffrey
% Last modified: 12 Feb 1990
%
% This style is (c) 1990 Alan Jeffrey.  Permission is granted to make
% copies, as long as
%

\def\Identity#1{#1}

\def\Error%
   {\errmessage{Abandon verification all 
                ye who enter here}}

\def\First#1#2{#1}
\def\Second#1#2{#2}

\def\Compose#1#2#3{#1{#2{#3}}}

\def\Twiddle#1#2#3{#1{#3}{#2}}

\let\True=\First
\let\False=\Second
\let\Not=\Twiddle

\def\And#1#2{#1{#2}\False}
\def\Or#1#2{#1\True{#2}}

\def\Lift#1#2#3#4{#1{#4}{#2}{#3}{#4}}

\def\Lessthan#1#2{\TeXif{\ifnum#1<#2 }}

\def\gobblefalse\else\gobbletrue\fi#1#2%
   {\fi#1}
\def\gobbletrue\fi#1#2%
   {\fi#2}
\def\TeXif#1%
   {#1\gobblefalse\else\gobbletrue\fi}

\def\Nil#1#2{#2}
\def\Cons#1#2#3#4{#3{#1}{#2}}
\def\Stream#1{\Cons{#1}{\Stream{#1}}}
\def\Singleton#1{\Cons{#1}\Nil}

\def\Head#1{#1\First\Error}
\def\Tail#1{#1\Second\Error}

\def\Foldl#1#2#3%
   {#3{\Foldl@{#1}{#2}}{#2}}
\def\Foldl@#1#2#3#4%
   {\Foldl{#1}{#1{#2}{#3}}{#4}}
\def\Foldr#1#2#3%
   {#3{\Foldr@{#1}{#2}}{#2}}
\def\Foldr@#1#2#3#4%
   {#1{#3}{\Foldr{#1}{#2}{#4}}}

\def\Cat#1#2{\Foldr\Cons{#2}{#1}}

\def\Reverse{\Foldl{\Twiddle\Cons}\Nil}

\def\All#1{\Foldr{\Compose\And{#1}}\True}
\def\Some#1{\Foldr{\Compose\Or{#1}}\False}
\def\Isempty{\All{\First\False}}

\def\Filter#1%
   {\Foldr{\Lift{#1}\Cons\Second}\Nil}

\def\Map#1{\Foldr{\Compose\Cons{#1}}\Nil}

\def\Insert#1#2#3%
   {#3{\Insert@{#1}{#2}}{\Singleton{#2}}}
\def\Insert@#1#2#3#4%
   {#1{#2}{#3}%
      {\Cons{#2}{\Cons{#3}{#4}}}%
      {\Cons{#3}{\Insert{#1}{#2}{#4}}}}
\def\Insertsort#1{\Foldr{\Insert{#1}}\Nil}

\def\Unlistize#1{[#1\Unlistize@{}]}
\def\Unlistize@#1{#1\Foldr\Commaize{}}
\def\Commaize#1#2{, #1#2}

\def\Listize[#1]%
   {\Listize@#1,\relax]}
\def\Listize@#1,#2]%
   {\TeXif{\ifx\relax#2}%
        {\Singleton{#1}}%
        {\Cons{#1}{\Listize@#2]}}}

\def\Show#1[#2]%
   {\Unlistize{#1{\Listize[#2]}}}

% The mathcodes for the letters A, ..., Z, a, ..., z are changed to
% generate text italic rather than math italic by default. This makes
% multi-letter names look neater. The mathcode for character 'c'
% is set to "7000 (variable family) + "400 (text italic) + c.
% 
% This neat bit of code is due to Mike Spivey.

\def\@setmcodes#1#2#3{{\count0=#1 \count1=#3
        \loop \global\mathcode\count0=\count1 \ifnum \count0<#2
        \advance\count0 by1 \advance\count1 by1 \repeat}}

\@setmcodes{`A}{`Z}{"7441}
\@setmcodes{`a}{`z}{"7461}

\def\Number#1{\csname Number-#1\endcsname}
\def\Label#1{\csname Label-#1\endcsname}

\newcount\Lastnum

\def\Forward#1%
   {\global\advance\Lastnum by 1
    \csnameafter\xdef{Number-#1}%
       {\the\Lastnum}%
    \csnameafter\xdef{Label-\the\Lastnum}%
       {\@currentlabel}}

\def\csnameafter#1#2%
   {\expandafter#1\csname#2\endcsname}

\def\Bylist#1%
   {\Map\Label
      {\Insertsort\Lessthan
         {\Map\Number{#1}}}}

\def\By{\Show\Bylist}

\let\bindspace=~
\def~{\ifmmode \, \else \bindspace \fi}

\def\start#1{\lefteqn{#1}\quad\\}

\def\nil{[\,\,]}


\newtheorem{fact}{Fact}
\def\thefact{\@roman\c@fact}

\def\cstok#1{\leavevmode\thinspace\hbox{\vrule\vtop{\vbox{\hrule\kern1pt
        \hbox{\vphantom{\tt/}\thinspace{\tt#1}\thinspace}}%
      \kern1pt\hrule}\vrule}\thinspace}

\begingroup \catcode `|=0 \catcode `[= 1
\catcode`]=2 \catcode `\{=12 \catcode `\}=12
\catcode`\\=12 |gdef|@xTeXcode#1\end{TeXcode}[#1|end[TeXcode]]
|endgroup

\def\TeXcode
   {\@verbatim \smallskip\hrule\medskip \frenchspacing\@vobeyspaces \@xTeXcode}
\def\endTeXcode
   {\medskip\hrule\smallskip\endtrivlist}

\makeatother








\section{intro}
Originally, I wanted lists in \TeX\ for
a paper I was writing which contained a lot of facts.

\begin{fact}
\Forward{Fac-first}
   First factual fact.
\end{fact}
\begin{fact}
\Forward{Fac-cows}
   Cows have four legs.
\end{fact}
\begin{fact}
\Forward{Fac-people}
   People have two legs.
\end{fact}
\begin{fact}
\Forward{Fac-yawn}
   Lots of facts in a row can be dull.
\end{fact}

These are generated with commands like
\begin{teXX}
\begin{fact}
\Forward{Fac-yawn}
   Lots of facts in a row can be dull.
\end{fact}
\end{teXX}
The facts could then be referred by saying
\begin{teXX}
\By[Fac-yawn,Fac-people,Fac-first]
\end{teXX}
to get 
\By[Fac-yawn,Fac-people,Fac-first]. \By[Fac-people,Fac-cows] 
And as if by magic, the facts come out sorted, rather than in
the jumbled order I typed them.  This is very useful, as I can
reorganize my document to my heart's content, and not have to worry
about getting my facts straight 
\sidenote{The facts were defined using a theorem environment as follows:\\ 
\noindent\obeylines
\texttt{\textbackslash newtheorem\{fact\}\{Fact\}}
\texttt{\textbackslash def\textbackslash thefact{\textbackslash @roman\textbackslash c@fact}}}. 

Originally I tried programming this sorting routine in \TeX's
list macros, from Appendix~D of \TB, but I soon ran into trouble.
The problem is that all the Appendix~D macros work by assigning
values to macros.  For example:
\begin{verbatim}
\concatenate\foo=\bar\baz
\end{verbatim}
expands out to
\begin{verbatim}
\ta=\expandafter{\bar}
\tb=\expandafter{\baz}
\edef\foo{\the\ta\the\tb}
\end{verbatim}
\section{Functions}
\normalsize

A {\em function\/ } is a mathematical object that takes in an argument
(which could well be another function) and returns some other mathematical
object.  For example the function $Not$ takes in a boolean and returns
its complement.  I'll write function application without brackets,
so $Not~b$ is the boolean complement of $b$.  

Function application
binds to the left, so $f~a~b$ is $(f~a)~b$ rather than $f~(a~b)$.
For example, $Or~a~b$ is the boolean or of $a$ and $b$, and
$Or~True$ is a perfectly good function that takes in a boolean
and returns $True$.

The obvious equivalents of functions in \TeX\ are macros ---
if we define a function $foo$ to be:
\begin{eqnarray*}
   foo~x  &  =  &  True
\end{eqnarray*}
then it can be translated into \TeX\ as:

\begin{teX}
\def\Foo#1{\True}
\end{teX}

%\def\Foo#1{\True}
%\noindent Typing |\Foo{a}| we get \Foo{a} 


So where $Foo$ is a function that takes in one argument, |\Foo|
is a macro that takes in one parameter.  Nothing has changed except
the jargon and the font.  \TeX\ macros can even be partially applied,
for example if we defined:

\begin{eqnarray*}
   Baz  &  =  &  Or~True
\end{eqnarray*}

then the \TeX\ equivalent would be

\begin{teX}
\def\Baz{\Or\True}
\end{teX}
Once |\Baz| is expanded, it will expect to be given a parameter,
but when we are defining things, we can go around partially applying
them all we like.

Here, I'm using $=$ without formally defining it, which is rather
naughty.  If I say $x = y$, this means 
``given enough parameters, $x$ and $y$ will eventually
expand out to the same thing.''  For example $Foo = Baz$, because
for any $x$,

\begin{eqnarray*}
   \textbackslash start{Foo~x}
   &  =  &  True  \\
   &  =  &  Or~True~x  \\
   &  =  &  Baz~x
\end{eqnarray*}

Normally, functions have to {\em respect equality} which means that:


\begin{teX}
\item if $x = y$ then $f~x = f~y$, and
\item if $x$ respects equality, then $f~x$ respects equality.
\end{teX}

However, some \TeX\ control sequences don't obey this.  For example,
|\string\Foo| and |\string\Baz| are different, even though
$Foo = Baz$.  Hence $string$ doesn't respect equality.

Unless otherwise stated, we won't assume functions respect equality,
although all the functions defined here do.

All of our functions have capital letters, so that their \TeX\ equivalents
(|\Not|, |\Or| and so on) don't clash with standard \TeX\ or
\LaTeX\ macros.

\subsection{Identity}

The simplest function is the {\em identity\/} function, called
$Identity$ funnily enough, which is defined:
\begin{eqnarray*}
   Identity~x  &  =  &  \Identity{x}
\end{eqnarray*}

This, it must be admitted, is a pretty dull function, but
it's a useful basic combinator.  It can be implemented
in \TeX\ quite simply.

\begin{teX}
\def\Identity#1{#1}
\end{teX}

The rules around this definition mean that it is actually part of
|Lambda.sty| and not just another example.

\subsection{Error}

Whereas $Identity$ does nothing in a fairly pleasant sort of way,
$Error$ does nothing in a particularly brutal and harsh fashion.
Mathematically, $Error$ is the function that destroys everything
else in front of it.  It is often written as $\perp$.

\begin{eqnarray*}
   Error~x  &  =  &  Error
\end{eqnarray*}

In practice, destroying the entire document when we hit one error
is a bit much, so we'll just print out an error message.
The user can carry on past an error at their own risk, as the code
will no longer be formally verified.

\begin{teX}
\def\Error
   {\errmessage{Abandon verification all 
                ye who enter here}}
\end{teX}
Maybe this function ought to return a more useful error message \ldots

\subsection{First and Second}

Two other basic functions are $First$ and $Second$, both of which
take in two arguments, and do the obvious thing.  They are defined:
\begin{eqnarray*}
    First~x~y  &  =  &  x  \\
   Second~x~y  &  =  &  y
\end{eqnarray*}
We could, in
fact, define $Second$ in terms of $Identity$ and $First$.  
For any $x$ and $y$,


\def\start#1{\lefteqn{#1}\quad\\}

\begin{eqnarray*}
   \start{First~Identity~x~y}  
   &  =  &  Identity~y  \\
   &  =  &  y  \\
   &  =  &  Second~x~y
\end{eqnarray*}
So $First~Identity = Second$.  This means that anywhere in our \TeX\ code
we have |\First\Identity| we could replace it by |\Second|.
This is perhaps not the most astonishing \TeX\ fact known to humanity,
but this sort of proof did enable more complex bits of \TeX\ to be
verified before they were run.

The \TeX\ definitions of |\First| and |\Second| are pretty
obvious.

\begin{teX}
\def\First#1#2{#1}
\def\Second#1#2{#2}
\end{teX}

\def\foo#1{#1}
\def\bar#1{#1}

And |\First\foo\bar{Foo Bar}| typesets \texttt{\First\foo\bar{Foo Bar}}


Note that in \TeX\, |\First\foo\bar| expands out to
|\foo| {\em without\/} expanding out |\bar|.
This is very useful, as we can write macros that would take
forever and a day to run if they expanded all their arguments,
but which actually terminate quite quickly.  This is called
{\em lazy evaluation\/} by the functional programming community.

\subsection{Compose}

Given two functions $f$ and $g$ we would like to be able to {\em compose\/}
them to produce a function that first applies $g$ then applies $f$.
Normally, this is written as $f \circ g$, but unfortunately \TeX\ doesn't
have infix functions, so we'll have to write it $Compose~f~g$.
\begin{eqnarray*}
   Compose~f~g~x  &  =  &  f~(g~x)
\end{eqnarray*}
>From this definition, we can deduce that $Compose$ is associative:
\begin{eqnarray*}
   \start{Compose~(Compose~f~g)~h}  
   &  =  &  Compose~f~(Compose~g~h)
\end{eqnarray*}
and $Identity$ is the left unit of $Compose$:
\begin{eqnarray*}
        Compose~Identity~f  &  =  &  f 
\end{eqnarray*}
The reader may wonder why $Identity$ is called a {\em left\/} unit
even though it occurs on the right of the $Compose$ --- this is a side-effect
of using prefix notations where infix is more normal.  The infix version
of this equation is:
\begin{eqnarray*}
   Identity \circ f  &  =  &  f
\end{eqnarray*}
so $Identity$ is indeed on the left of the composition.

$Compose$ can be implemented in \TeX\ as
\begin{teX}
\def\Compose#1#2#3{#1{#2{#3}}}
\end{teX}

\noindent Defining,

\begin{teX}
\def\goo#1{#1}
\def\foo#1{#1$^2$}
\end{teX}
\def\goo#1{#1}
\def\foo#1{#1$^2$}


and typing
\begin{teX}
\texttt{\Compose\foo\goo{x}}

\texttt{\Compose\goo\foo{x}}
\end{teX}

\noindent we get 

\texttt{\Compose\foo\goo{x}}\\
\texttt{\Compose\goo\foo{x}}\\

\def\goo#1{#1}
\def\foo#1{#1+#1}

\texttt{\Compose\foo\goo{x}}\\

As an example, suppose that an airplane's elevation at time $t$ is given by the function $h(t)$ and that the oxygen concentration at elevation $x$ is given by the function $c(x)$. Then $(c \circ h)(t)$ describes the oxygen concentration around the plane at time $t$.


\subsection{Twiddle}

Yet another useful little function is $Twiddle$, which takes in 
a function and reverses the order that function takes its (first two)
arguments.

\begin{eqnarray*}
   Twiddle~f~x~y  &  =  &  f~y~x
\end{eqnarray*}

Again, there aren't many immediate uses for such a function, but it'll
come in handy later on.  It satisfies the properties

\begin{eqnarray*}
              Twiddle~First  &  =  &  Second  \\
             Twiddle~Second  &  =  &  First  \\
    Compose~Twiddle~Twiddle  &  =  &  Identity
\end{eqnarray*}

Its \TeX\ equivalent is
\begin{teX}
\def\Twiddle#1#2#3{#1{#3}{#2}}
\end{teX}

This function is called ``twiddle'' because it is sometimes written
$\widetilde f$ (and $\sim$ is pronounced ``twiddle'').  

\def\foo#1#2{#1:#2}

\def\Twiddle#1#2#3{#1{#3}{#2}}
\Twiddle\foo{a}{b}


\section{Booleans}

As we're trying to program a sorting routine, it would be nice to
be able to define orderings on things, and to do this we need some
representation of boolean variables.  Unfortunately \TeX\ doesn't have a type
for booleans, so we'll have to invent our own.  We'll
implement a boolean as a function $b$ of the form

\begin{eqnarray*}
   b~x~y  &  
   =  &  
   \left\{
      \begin{array}{ll}
         x  &  \mbox{if $b$ is true}  \\
         y  &  \mbox{otherwise}
      \end{array}
   \right.
\end{eqnarray*}

More formally, a 
boolean $b$ is a function which respects equality,
such that for all $f$, $g$ and $z$:
\begin{eqnarray*}
   b~f~g~z  &  =  &  b~(f~z)~(g~z)
\end{eqnarray*}
and for all $f$ and $g$ which respect equality,
\begin{eqnarray*}
   b~(f~b)~(g~b)  &  =  &  b~(f~First)~(g~Second)
\end{eqnarray*}

All the functions in this section satisfy these properties.  Surprisingly
enough, so does $Error$, which is quite useful, as it allows us to 
reason about booleans which ``go wrong''.  

\subsection{True, False and Not}

Since we are implementing booleans as functions, we already have the
definitions of $True$, $False$ and $Not$.
\begin{eqnarray*}
    True  &  =  &  First  \\
   False  &  =  &  Second  \\
     Not  &  =  &  Twiddle
\end{eqnarray*}
So for free we get the following results:
\begin{eqnarray*}
          Not~True  &  =  &  False  \\
         Not~False  &  =  &  True  \\
   Compose~Not~Not  &  =  &  Identity  
\end{eqnarray*}
The \TeX\ implementation is not exactly difficult:

\begin{teX}
\let\True=\First
\let\False=\Second
\let\Not=\Twiddle
\end{teX}

\def\Foo#1#2{#1 #2}

\True\Foo{a}{b} \\

Test 

\False\Foo{a}{b} \\

This needs checking!


\subsection{And and Or}

The definitions of $And$ and $Or$ are:
\begin{eqnarray*}
   And~a~b  &  
   =  &  
   \left\{
      \begin{array}{ll}
         b      &  \mbox{if $a$ is true}  \\
         False  &  \mbox{otherwise}
      \end{array}
   \right.
   \\
   Or~a~b  &  
   =  &  
   \left\{
      \begin{array}{ll}
         True   &  \mbox{if $a$ is true}  \\
         b      &  \mbox{otherwise}
      \end{array}
   \right.
\end{eqnarray*}
With our definition of what a boolean is, this is just the same as
\begin{eqnarray*}
   And~a~b  &  =  &  a~b~False  \\
    Or~a~b  &  =  &  a~True~b
\end{eqnarray*}
>From these conditions, we can show that $And$ is associative, and 
has left unit $True$ and left zeros $False$ and $Error$:
\begin{eqnarray*}
   And~(And~a~b)~c  &  =  &  And~a~(And~b~c)  \\
        And~True~b  &  =  &  b  \\
       And~False~b  &  =  &  False  \\
       And~Error~b  &  =  &  Error
\end{eqnarray*}
$Or$ is associative, has left unit $False$ and left zeros $True$ and $Error$:
\begin{eqnarray*}
   Or~(Or~a~b)~c  &  =  &  Or~a~(Or~b~c)  \\
      Or~False~b  &  =  &  b  \\
       Or~True~b  &  =  &  True  \\
      Or~Error~b  &  =  &  Error
\end{eqnarray*}
De~Morgan's laws hold:
\begin{eqnarray*}
   Not~(And~a~b)  &  =  &  Or~(Not~a)~(Not~b)  \\
    Not~(Or~a~b)  &  =  &  And~(Not~a)~(Not~b)  
\end{eqnarray*}
and $And$ and $Or$ left-distribute through one another:
\begin{eqnarray*}
   Or~a~(And~b~c)  &  =  &  And~(Or~a~b)~(Or~a~c)  \\
   And~a~(Or~b~c)  &  =  &  Or~(And~a~b)~(And~a~c)
\end{eqnarray*}
$And$ and $Or$ are {\em not\/} commutative, though.  For example,
\begin{eqnarray*}
   start{Or~True~Error}  
   &  =  &  True~True~Error  \\
   &  =  &  True
\end{eqnarray*}
but
\begin{eqnarray*}
   start{Or~Error~True}  
   &  =  &  Error~True~True \\
   &  =  &  Error
\end{eqnarray*}
This is actually quite useful since there are some booleans that
need to return an error occasionally.  If $a$ is $True$ when $b$ 
is safe (i.e.\ doesn't become $Error$) and is $False$ otherwise, we can
say $Or~a~b$ and know we're not going to get an error.  This is handy
for things like checking for division by zero, or trying to get the
first element of an empty list.

Similarly, because of the possibility of $Error$,
$And$ and $Or$ don't right-distribute through each other,
as
\begin{eqnarray*}
    {Or~(And~False~Error)~True}\\
   &  \ne  &  And~(Or~False~True)~(Or~Error~True)
\end{eqnarray*}
As errors shouldn't crop up, this needn't worry us too much.
\begin{teX}
\def\And#1#2{#1{#2}\False}
\def\Or#1#2{#1\True{#2}}
\end{teX}



\subsection{Lift}

Quite a lot of the time we won't be dealing with booleans, but with
{\em predicates}, which are just functions that return a boolean.
For example, the predicate $Lessthan$ is defined below so that
$Lessthan~i~j$ is true whenever $i<j$.
Given a predicate $p$ we would like to be able to 
{\em lift\/} it to $Lift~p$, defined:
\begin{eqnarray*}
   Lift~p~f~g~x  &  =  &  p~x~f~g~x
\end{eqnarray*}

For example, $Lift~(Lessthan~0)~f~g$ takes in a number and applies
$f$ to it if it is positive and $g$ to it otherwise.  This is quite
useful for defining functions.

\begin{teX}
\def\Lift#1#2#3#4{#1{#4}{#2}{#3}{#4}}
\end{teX}




\subsection{Lessthan and \TeX if}

Finally, we would like to be able to use \TeX's built-in booleans
as well as our own.  For example, we would like a predicate
$Lessthan$ such that:
\begin{eqnarray*}
   Lessthan~i~j  &  
   =  &
   \left\{
      \begin{array}{ll}
         True   &  \mbox{if } i < j    \\
         False  &  \mbox{if } i \ge j  \\
         Error  &  \mbox{otherwise}  
      \end{array}
   \right.
\end{eqnarray*}


The $Error$ condition happens if we try applying $Lessthan$ to something
that isn't a number --- $Lessthan~True~False$ is $Error$

{Actually, that's a little white lie --- trying to persuade \TeX\ to
    do run-time type checking isn't much fun.  So the \TeX\ implementation
    of this is actually a {\em refinement\/} where the $Error$ condition
    has been replaced by whatever it is \TeX\ does if you try doing
    {\tt\string\ifnum $x$ < $y$} when $x$ and $y$ aren't numbers}.

This is fine as a mathematical definition, but how will
we implement it?  If we assume we have a macro |\TeXif|,
which converts \TeX\ if-statements into booleans, we could just
define:
\begin{teX}
\def\Lessthan#1#2{\TeXif{\ifnum#1<#2 }}
\end{teX}
So the question is just how to define |\TeXif|.
Unfortunately, the ``obvious'' code does not work:
\begin{teX}
\def\TeXif#1#2#3{#1#2\else#3\fi}
\end{teX}
For example, |\TeXif\iftrue\True\True| doesn't expand out to
|\True|.  Instead, it expands as:



\begin{teX}
\TeXif\iftrue\True\True
   = \iftrue\True\else\True\fi
   = \True\else\True\fi
   = \else\fi
   =
\end{teX}

Another common \TeX nique is to use a macro \cmd{next} to
be the expansion text:
\begin{teX}
\def\TeXif#1#2#3%
   {#1\def\next{#2}\else\def\next{#3}\fi
    \next}
\end{teX}
However, this uses \TeX's stomach to do the \cmd{def}, and we are
trying to do this using only the mouth.  One (slightly tricky) solution
is to use pattern-matching to gobble up the offending \cmd{else} and/or
\cmd{fi}.



\begin{teX}
\def\gobblefalse\else\gobbletrue\fi#1#2%
   {\fi#1}
\def\gobbletrue\fi#1#2%
   {\fi#2}
\def\TeXif#1%
   {#1\gobblefalse\else\gobbletrue\fi}
\end{teX}

So if the \TeX\ if-statement is true, |\gobblefalse| gobbles
up the false-text, otherwise |\gobbletrue| gobbles up the 
true-text.  For example,

\begin{teX}
\TeXif\iftrue\True\True
   = \iftrue\gobblefalse\else
         \gobbletrue\fi\True\True
   = \gobblefalse\else
         \gobbletrue\fi\True\True
   = \fi\True
   = \True
\end{teX}

Phew.  And so we have booleans.


\section{Lists}
\def\nil{[~]\xspace}

A list is a (possibly infinite) sequence of values.  For example,
the list $[1,2,3]$ contains three numbers, the list \nil contains
none, and the list $[1,2,3,\ldots]$ contains infinitely many.
A list is either {\em empty\/} (written \nil) or is comprised
of a {\em head\/}
$x$ and a {\em tail\/} $xs$ (in which case it's written $x:xs$).
For example, 1:2:3:\nil is $[1,2,3]$.  

In a similar fashion to the implementation of booleans,
a list $xs$ is implemented as a function of the form
\begin{eqnarray*}
   xs~f~e  &
   =  &
   \left\{
      \begin{array}{ll}
         e       &  \mbox{if $xs$ is empty}  \\
         f~y~ys  &  \mbox{if $xs$ has head $y$ and tail $ys$}
      \end{array}
   \right.
\end{eqnarray*}
Again, we are implementing a datatype as a function, a quite powerful
trick, just not one usually seen in \TeX.  We will assume that
whenever a list $x:xs$ is applied to $f$ and $e$, $f~x$ respects equality.
This allows us to assume that if $xs = ys$ then $x:xs = x:ys$,
which is handy.  

\subsection{Nil, Cons, Stream and Singleton}

The simplest list is $Nil$, the empty list which we have been writing
as~\nil.  

\begin{eqnarray*}
   Nil  &  =  &  Second
\end{eqnarray*}
The other possible list is $Cons~x~xs$, which has head $x$ and tail $xs$.
\begin{eqnarray*}
   Cons~x~xs~f~e  &  =  &  f~x~xs
\end{eqnarray*}
Every list can be constructed using these functions.
The list $[1,2,3]$ is $Cons~1~(Cons~2~(Cons~3~Nil))$, and the
list $[a,a,a,\ldots]$ is $Stream~a$ where $Stream$ is defined:
\begin{eqnarray*}
   Stream~a  &  =  &  Cons~a~(Stream~a)
\end{eqnarray*}

There's even at least one application for infinite lists,
as we'll see in Section~\ref{outputroutines}.

The singleton list $[a]$ is $Singleton~a$, defined as:
\begin{eqnarray*}
   Singleton~a  &  =  &  Cons~a~Nil
\end{eqnarray*}
These all have straightforward \TeX\ definitions.

\begin{teX}
\let\Nil=\Second
\def\Cons#1#2#3#4{#3{#1}{#2}}
\def\Stream#1{\Cons{#1}{\Stream{#1}}}
\def\Singleton#1{\Cons{#1}\Nil}
\end{teX}




\subsection{Head and Tail}

So, we can construct any list we like, but we still can't get any information
out of it.  To begin with, we'd like to be able to get the head
and tail of a list.

\begin{eqnarray*}
   Head~xs  &  =  &  xs~First~Error  \\
   Tail~xs  &  =  &  xs~Second~Error
\end{eqnarray*}

For example, the tail of $x:xs$ is
\begin{eqnarray*}
   \ start{Tail~(Cons~x~xs)}  
   &  =  &  Cons~x~xs~Second~Error  \\
   &  =  &  Second~x~xs  \\
   &  =  &  \ Tail{\ Cons{x}{xs}}
\end{eqnarray*}
The tail of |\nil| is, as one would expect,
\begin{eqnarray*}
   \ start{Tail~Nil}  
   &  =  &  Nil~Second~Error  \\
   &  =  &  Error
\end{eqnarray*}
And the head of $Stream~a$ is


\begin{eqnarray*}
   \ start{Head~(Stream~a)}  
   &  =  &  Stream~a~First~Error  \\
   &  =  &  Cons~a~(Stream~a)~First~Error  \\
   &  =  &  First~a~(Stream~a)  \\
   &  =  &  \textbackslash Head{\textbackslash Stream{a}}
\end{eqnarray*}


So we can get the head of an infinite list in finite time.  This is
fortunate, as otherwise there wouldn't be much point in allowing
infinite objects. 


\begin{teX}
\def\Head#1{#1\First\Error}
\def\Tail#1{#1\Second\Error}
\end{teX}



\subsection{Foldl and Foldr}

Using $Head$ and $Tail$ we can get at the beginning of any non-empty list,
but in general we need more information than that.  Rather than write
a whole bunch of recursive functions on lists, I'll implement two
fairly general functions, with which we can implement (almost) everything
else.

$Foldl$ and $Foldr$ both take in functions and apply them recursively
to a list.  $Foldl$ starts at the left of the list, and $Foldr$
starts at the right.  For example,

\begin{eqnarray*}
   Foldl~f~e~[1,2,3]  &  =  &  f~(f~(f~e~1)~2)~3  \\
   Foldr~f~e~[1,2,3]  &  =  &  f~1~(f~2~(f~3~e))
\end{eqnarray*}

These functions will be used a lot later on.  $Foldl$ can be defined:

\begin{eqnarray*}
      Foldl~f~e~xs  &  =  &  xs~(Foldl'~f~e)~e  \\
   Foldl'~f~e~x~xs  &  =  &  Foldl~f~(f~e~x)~xs
\end{eqnarray*}

So $Foldl~f~e~\ nil$ is

\begin{eqnarray*}
   \ start{Foldl~f~e~Nil}  
   &  =  &  Nil~(Foldl'~f~e)~e  \\
   &  =  &  \ Foldl{f}{e}\ Nil
\end{eqnarray*}
And $Foldl~f~e~(x:xs)$ is
\begin{eqnarray*}
   \ start{Foldl~f~e~(Cons~x~xs)}  
   &  =  &  Cons~x~xs~(Foldl'~f~e)~e  \\
   &  =  &  Foldl'~f~e~x~xs  \\
   &  =  &  Foldl~f~(f~e~x)~xs
\end{eqnarray*}
For example, $Foldl~f~e~[1,2,3]$ is
\begin{eqnarray*}
   \ start{Foldl~f~e~[1,2,3]}  
   &  =  &  Foldl~f~(f~e~1)~[2,3]  \\
   &  =  &  Foldl~f~(f~(f~e~1)~2)~[3]  \\
   &  =  &  Foldl~f~(f~(f~(f~e~1)~2)~3)~\ nil  \\
   &  =  &  f~(f~(f~e~1)~2)~3
\end{eqnarray*}
as promised.  Similarly, we can define $Foldr$ as
\begin{eqnarray*}
      Foldr~f~e~xs  &  =  &  xs~(Foldr'~f~e)~e  \\
   Foldr'~f~e~x~xs  &  =  &  f~x~(Foldr~f~e~xs)
\end{eqnarray*}
For $Foldr~f$ to respect equality, $f~x$ should respect equality.

When we do the unfolding, we discover that
\begin{eqnarray*}
       Foldr~f~e~\ nil  &  =  &  e  \\
   Foldr~f~e~(x:xs)  &  =  &  f~e~(Foldr~f~e~xs)
\end{eqnarray*}
$Foldr$ tends to be more efficient than $Foldl$, because $Foldl$
has to run along the entire list before it can start applying $f$,
whereas $Foldr$ can apply $f$ straight away.  If $f$ is a lazy function,
this can make quite a difference.  $Foldl$ on infinite lists, anyone?

\begin{teX}
\def\Foldl#1#2#3%
   {#3{\Foldl@{#1}{#2}}{#2}}
\def\Foldl@#1#2#3#4%
   {\Foldl{#1}{#1{#2}{#3}}{#4}}
\def\Foldr#1#2#3%
   {#3{\Foldr@{#1}{#2}}{#2}}
\def\Foldr@#1#2#3#4%
   {#1{#3}{\Foldr{#1}{#2}{#4}}}
\end{teX}

\subsection{Cat}

Given two lists, we would like to be able to stick them together,
which is what $Cat$ (short for ``concatenate'')
does.  For example, $Cat~[1,2]~[3,4]$ is
$[1,2,3,4]$.  It can be defined using $Foldr$:
\begin{eqnarray*}
   Cat~xs~ys  &  =  &  Foldr~Cons~ys~xs
\end{eqnarray*}
So
\begin{eqnarray*}
   &  =  &  Foldr~Cons~[3,4]~[1,2]  \\
   &  =  &  Cons~1~(Foldr~Cons~[3,4]~[2])  \\
   &  =  &  Cons~1~(Cons~2~(Foldr~Cons~[3,4]~\ nil))  \\
   &  =  &  Cons~1~(Cons~2~[3,4])  \\
   &  =  &  \ Unlistize{\ Cat{\ Listize[1,2]}{\ Listize[3,4]}}
\end{eqnarray*}
The \TeX\ code for |\Cat| is suspiciously similar to its mathematical
definition.
\begin{teX}
\def\Cat#1#2{\Foldr\Cons{#2}{#1}}
\end{teX}




\subsection{Reverse}

We can reverse any list with the function $Reverse$, defined using
$Foldl$:
\begin{eqnarray*}
   Reverse  &  =  &  Foldl~(Twiddle~Cons)~Nil
\end{eqnarray*}
For example, $Reverse~[1,2,3]$ can be calculated:
\begin{eqnarray*}
   \ start{Reverse~[1,2,3]}  
   &  =  &  Foldl~(Twiddle~Cons)~Nil~[1,2,3]  \\
   &  =  &  Twiddle~Cons  \\
   &     &  \ quad (Twiddle~Cons~(Twiddle~Cons~Nil~1)~2)~3  \\
   &  =  &  Cons~3~(Cons~2~(Cons~1~Nil))  \\
   &  =  &  \ Show\ Reverse[1,2,3]
\end{eqnarray*}

The \TeX\ code for|\Reverse| doesn't even take in any parameters.

\begin{teX}
\def\Reverse{\Foldl{\Twiddle\Cons}\Nil}
\end{teX}



\subsection{All, Some and Isempty}

Given a predicate $p$, we can find out if all the elements of
a list satisfy $p$ with $All~p$.  Similarly we can find if something
in the list satisfies $p$ with $Some~p$.  For example,
\begin{eqnarray*}
   All~(Lessthan~1)~[1,2,3]  
   &  =  &  \ All{\ Lessthan 1}{\ Listize[1,2,3]}{True}{False}  \\
   Some~(Lessthan~1)~[1,2,3]  
   &  =  &  \ Some{\ Lessthan 1}{\ Listize[1,2,3]}{True}{False}
\end{eqnarray*}
These can be defined
\begin{eqnarray*}
    All~p  &  =  &  Foldr~(Compose~And~p)~True  \\
   Some~p  &  =  &  Foldr~(Compose~Or~p)~False
\end{eqnarray*}
For example, $Isempty$ can be defined
\begin{eqnarray*}
   Isempty  &  =  &  All~(First~False)
\end{eqnarray*}
This is probably not the most efficient check in the world, but we
hardly ever need it --- $Foldl$ or $Foldr$ will normally do the job.
\begin{teX}
\def\All#1{\Foldr{\Compose\And{#1}}\True}
\def\Some#1{\Foldr{\Compose\Or{#1}}\False}
\def\Isempty{\All{\First\False}}
\end{teX}

\subsection{Filter}

$Filter$ takes a predicate $p$ and a list $xs$, and returns a list
containing only those elements of $xs$ that satisfy $p$.  For example,
\begin{eqnarray*}
   Filter~(Lessthan~1)~[1,2,3]  &  =  &  \ Show\ Filter{\ Lessthan 1}[1,2,3]
\end{eqnarray*}
$Filter$ can be defined as a $Foldr$:
\begin{eqnarray*}
   Filter~p  &  =  &  Foldr~(Lift~p~Cons~Second)~Nil
\end{eqnarray*}
Another easy bit of \TeX:
\begin{teX}
\def\Filter#1%
   {\ Foldr{\ Lift{#1}\ Cons\ Second}\ Nil}
\end{teX}



\subsection{Map}

$Map$ takes a function $f$ and a list $xs$ and applies $f$ to every
element of $xs$.  For example,

\begin{eqnarray*}
   Map~f~[1,2,3]  &  =  &  \ Show\ Map{f~}[1,2,3]
\end{eqnarray*}
This is another job for $Foldr$.
\begin{eqnarray*}
   Map~f  &  =  &  Foldr~(Compose~Cons~f)~Nil
\end{eqnarray*}
We shall see $Map$ used later on, to convert from a list of
names such as \ Show\ Map{\ Compose\  mbox\tt}[Fac-yawn,Fac-cows], to a list of 
labels such as \ By[Fac-yawn,Fac-cows].

\begin{teX}
\def\Map#1{\Foldr{\Compose\Cons{#1}}\Nil}
\end{teX}





\subsection{Insert}

The only function we need which isn't easily defined as a reduction
is $Insert$, which inserts an element into a sorted list.  For example,
\begin{eqnarray*}
   Insert~Lessthan~3~[1,2,4,5]  &  =  &  \ Show\ Insert\ Lessthan3[1,2,4,5]
\end{eqnarray*}
$Insert$ takes in an ordering as its first parameter, so we're not stuck
with one particular order.  It is defined directly in terms of the 
definition of lists.
\begin{eqnarray*}
      Insert~o~x~xs  &  =  &  xs~(Insert'~o~x)~(Singleton~x)  \\
   Insert'~o~x~y~ys  &  =  &  o~x~y  \\
                     &     &  \quad  (Cons~x~(Cons~y~ys))  \\
                     &     &  \quad  (Cons~y~(Insert~o~x~ys))
\end{eqnarray*}
We can then define the function all this has been leading up to,
$Insertsort$ which takes an ordering and a list, and insert-sorts the
list according to the ordering.  For example,
\begin{eqnarray*}
   Insertsort~Lessthan~[2,3,1,2]  &  =  &  \ Show\ Insertsort\ Lessthan[2,3,1,2]
\end{eqnarray*}

We can implement this as a fold:
\begin{eqnarray*}
   Insertsort~o  &  =  &  Foldr~(Insert~o)~Nil
\end{eqnarray*}

And so we've got sorted lists.
\begin{teX}
\def\Insert#1#2#3%
   {#3{\Insert@{#1}{#2}}{\Singleton{#2}}}
\def\Insert@#1#2#3#4%
   {#1{#2}{#3}%
      {\Cons{#2}{\Cons{#3}{#4}}}%
      {\Cons{#3}{\Insert{#1}{#2}{#4}}}}
\def\Insertsort#1{\Foldr{\Insert{#1}}\Nil}
\end{teX}

Interestingly, as we have implemented unbounded lists in \TeX's mouth,
this means we can implement a Turing Machine\index{Turing Machine}.  So, if you believe
the Church-Turing\index{Church-Turing} thesis, \TeX's mouth is as powerful as any
computer anywhere.  Isn't that good to know?



\section{Sorting reference lists}
\label{thissection}

So, these are the macros I've got to play with --- how do we apply them to
sorting lists of references?  Well, I'm using \LaTeX, which keeps the
current reference in a macro called |\@currentlabel|, which 
is~\ref{thissection} at the moment, as this is Section~\ref{thissection}.
So I just need to store the value of |\@currentlabel| somehow.

Fortunately, I'm only ever going to be making references to facts
earlier on in the document, in order to make sure I'm not proving
any results in terms of themselves.  So I don't need to play around
with auxiliary files, and can just do everything in terms of
macros.

\makeatletter

The |\@currentlabel|  \@currentlabel 
\makeatother


\subsection{Number and Label}

Each label in the document is given a unique number, in the order
the labels were put down.  So the number of |Fac-cows| 
is |\Number{Fac-cows}|, which expands out to~|\Number{Fac-cows}|,
the number of |Fac-people| is~|\Number{Fac-people}|, and so on.

Each number has an associated label with it.  For example,
the first label is |\Label{1}|, which is~|\Label{1}|,
the second label is~|\Label{2}| and so on.  So to find the label for
|Fac-cows|, we say |\Label{\Number{Fac-cows}}| which expands 
out to~|\Label{\Number{Fac-cows}}|.

These numbers and labels are kept track of in macros.  For example,
the number of |Fac-cows| is kept in |\cstok{Number-Fac-cows}|.
Similarly, the first label is kept in |\cstok{Label-1}|.
As these macros have dashes in their names, they aren't likely to
be used already.

So the \TeX\ code for |\Number| and |\Label| is pretty
simple.

\begin{verbatim}
\def\Number#1{\csname Number-#1\endcsname}
\def\Label#1{\csname Label-#1\endcsname}
\end{verbatim}




\subsection{Lastnum and Forward}

The number of the most recent label is kept in |\Lastnum|.
\begin{verbatim}
\newcount\Lastnum
\end{verbatim}

To put down a label |Foo|, I type |\Forward{Foo}|.
\Forward{Foo}
This increments the counter
|\Lastnum|, and |\xdef|s |\cstok{Number-Foo}|
to be the value of
|\Lastnum|, which is now~|\the\Lastnum|.  So
|\Number{Foo}| now expands to~|\Number{Foo}.|

Similarly, it |\xdef|s |\cstok{Label-\Number{Foo}}| to be
|\@currentlabel|, which is currently~|\Label{\Number{Foo}}|.
So |\Label{\Number{Foo}}| now expands to~|{\Number{Foo}}|.


\begin{teX}
\def\Forward#1%
   {\global\advance\Lastnum by 1
    \csnameafter\xdef{Number-#1}%
       {\the\Lastnum}%
    \csnameafter\xdef{Label-\the\Lastnum}%
       {\@currentlabel}}
\end{teX}
This uses |\csnameafter\foo{bar}|, which expands out to
|\foo\bar|.

\begin{teX}
\def\csnameafter#1#2%
   {\expandafter#1\csname#2\endcsname}
\end{teX}




\subsection{Listize, Unlistize and Show}


At the moment, lists have to be built up using |\Cons| and 
|\Nil|, which is rather annoying.  Similarly, we can't actually
do anything with a list once we've built it.  We'd like some way
of converting lists in the form |[a,b,c]| to and from the
form $[a,b,c]$.  This is done with |\Listize| and |\Unlistize|.
So |\Listize[a,b,c]| expands to |\Show\foo[a,b,c]| 


\begin{teX}
\Cons{a}{\Cons{b}{\Cons{c}{\Nil}}}
\end{teX}

Similarly, |\Unlistize| takes the list $[a,b,c]$ and expands out
to |[a, b, c]|.  |\Unlistize| is done with a $Foldr$.

\begin{teX}
\def\Unlistize#1{[#1\Unlistize@{}]}
\def\Unlistize@#1{#1\Foldr\Commaize{}}
\def\Commaize#1#2{, #1#2}
\end{teX}


The macro |\Listize| is just a \TeX\ hack with pattern matching.
It would have been nice to use |\@ifnextchar| for this, but
that uses |\futurelet|, which doesn't expand in the mouth.  Oh well.

\begin{teX}
\def\Listize[#1]%
   {\Listize@[#1,\relax]}
\def\Listize@#1,#2]%
   {\TeXif{\ifx\relax#2}%
        {\Singleton{#1}}%
        {\Cons{#1}{\Listize@#2]}}
\end{teX}

This only works for nonempty lists --- |\Listize[]| produces the
singleton list |\Singleton{}|.  It also uses |\relax| as its
end-of-list character, so lists with |\relax| in them have to
be done by hand.  You can't win them all.  So 

\begin{teX}
$\Unlistize{\Listize[a,b,c]}$
\end{teX}

produces 

|\Unlistize{\Listize[a,b,c]}|.  This is such a common construction
that I've defined a macro |\Show| such that

|\Show\foo[a,b,c]| expands out to

\begin{teX}
\Unlistize{\foo{\Listize[a,b,c]}}
\end{teX}

For example, the equation
\begin{eqnarray*}
   Filter\,(Lessthan\,1)\,[1,2,3]  
      &=& \ Show\ Filter{\ Lessthan 1}[1,2,3]
\end{eqnarray*}



was generated with
\begin{verbatim}
\begin{eqnarray*}
   Filter\,(Lessthan\,1)\,[1,2,3]  
      &=& \Show\Filter{\Lessthan1}[1,2,3]
\end{eqnarray*}
\end{verbatim}
Many of the examples in this article were typeset this way.

\begin{teX}
\def\Show#1[#2]%
   {\Unlistize{#1{\Listize[#2]}}}
\end{teX}

\subsection{By}

Given these macros, we can now sort any list of references with $Bylist$,
defined
\begin{eqnarray*}
   Bylist~xs  &  =  &  Map~Label  \\
              &     &  \quad (Insertsort~Lessthan \\
              &     &  \quad\quad (Map~Number~xs))
\end{eqnarray*}
This takes in a list of label names like \verb|Fac-yawn|,
converts it into a list of numbers with $Map~Number$,
sorts the resulting list with $Insertsort~Lessthan$,
and finally converts all the numbers into labels like 
\Label{\Number{Fac-yawn}} with $Map~Label$.  For example,
\begin{eqnarray*}
   \start{Bylist~\Show\Map{\Compose\mbox\tt}[Fac-yawn,Fac-cows]}
   &  =  &  Map~Label~(Insertsort~Lessthan \\
   &     &  \quad (Map~Number~
               \Show\Map{\Compose\mbox\tt}[Fac-yawn,Fac-cows]))  \\
   &  =  &  Map~Label~(Insertsort~Lessthan~
               \Show\Map\Number[Fac-yawn,Fac-cows]) \\
   &  =  &  Map~Label~\Show\Compose{\Insertsort\Lessthan}{\Map\Number}
               [Fac-yawn,Fac-cows]  \\
   &  =  &  \Show\Bylist[Fac-yawn,Fac-cows]
\end{eqnarray*}




The \TeX\ code for this is
\begin{verbatim}
\def\Bylist#1%
   {\Map\Label
      {\Insertsort\Lessthan
         {\Map\Number{#1}}}}
\end{verbatim}

\def\Bylist#1%
   {\Map\Label
      {\Insertsort\Lessthan
         {\Map\Number{#1}}}}

So we can now stick all this together, and define the macro |\By|
that prints out lists of references.  It is
\begin{verbatim}
\def\By{\Show\Bylist}
\end{verbatim}

\def\By{\Show\Bylist}
So |\By[Fac-yawn,Fac-cows,Fac-first]| is~|\By[Fac-yawn,Fac-cows,fac-first]|.
Which is quite nice.


