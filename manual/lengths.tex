\chapter{Length macros}

Length manipulation macros

You can change the values of the variables defining the page layout with two commands. With this one you can set a new value:

\begin{teX}
\setlength{parameter}{length}
\end{teX}


with this other one, you can add a value to the existing one:

\begin{teX}
\addtolength{parameter}{length}
\end{teX}

new lengths can becreated using the \cmd{newlength}.

\begin{teX}
\newlength{parameter}
\end{teX}

You may also set a length from the size of a text with one of these commands:


\begin{teX}
 \settowidth{parameter}{some text}
 \settoheight{parameter}{some text}
 \settodepth{parameter}{some text}
\end{teX}

When using these commands, you may duplicate the text that you want to use as reference if you plan to also display it. But LaTeX also provides a set of commands to avoid this duplication:


In the following example ww will create a newbox doccmd{Abox}

\begin{teX}
\newsavebox{\Abox}
\savebox{\Abox}{\parbox{2in}{\lorem}}
\usebox{\Abox} \hspace{1cm} \usebox{\Abox}
\end{teX}


We can view the dimensions of the box using

\begin{teX}
\showthe\Abox
\showthe\wd\Abox
\showthe\ht\Abox
\showthe\dp\Abox
\end{teX}


\noindent Typing the following code:
\begin{teX}
\newsavebox{\Abox}
\savebox{\Abox}{\parbox{2in}{\lorem}}
\usebox{\Abox} \hspace{0.6cm} \usebox{\Abox}
\end{teX}

\begin{fullwidth}
\newsavebox{\Abox}
\savebox{\Abox}{\parbox{2in}{\lorem}}
\usebox{\Abox} \hspace{0.6cm} \usebox{\Abox}
\end{fullwidth}


\bigskip

To get the dimensions some clever manipulations must be done. We need to define the box using \cmd{wd}, \cmd{ht} and \cmd{dp}.


\newcommand*{\settoboxwidth}[2]{\setlength{#1}{\wd#2}}

\newsavebox{\mybox}
\newlength{\mylength}
\sbox{\mybox}{Hello World}
\settoboxwidth{\mylength}{\mybox}

\usebox\mybox

\the\mybox  

\the\wd\mybox

\the\ht\mybox

\medskip

We are now going to try and define macros that show the dimensions of a box on the screen or they print it out.

This can be quite useful, while we are debugging a new layout or manipulating boxes.

We first start by defining a numbers of new \emph{lengths}. We do this as we will not be using the \tex commands
|\wd|, |\ht| and |\dp|\sidenote{It is considered bad practice to combine \tex and \latex commands together, although nothing stops us from doing so.}.

\begin{teX}
\newlength{\myhh}
\newlength{\mydd}
\newlength{\myww}
\end{teX}

The next step is to set these lengths to the corresponding parameters of the box macro |\Abox|.

\begin{teX}
\settoheight{\myhh}{\usebox{\Abox}}
\settodepth{\mydd}{\usebox{\Abox}}
\settowidth{\myww}{\usebox{\Abox}} 
\end{teX}

In the last part we define macros to print these dimensions.

\begin{teX}
\newcommand{\showboxheight}[1]{
   \settoheight{\myhh}{\usebox{#1}}
   \the\myhh
}


\newcommand{\showboxdepth}[1]{
   \settodepth{\myhh}{\usebox{#1}}
   \the\mydd
}

\newcommand{\showboxwidth}[1]{
   \settodepth{\myhh}{\usebox{#1}}
   \the\myww
}
\end{teX}


\newlength{\imageh}
\newlength{\imaged}
\newlength{\imagew}

\newcommand{\setimageh}[1]{
 \settoheight{\imageh}{\usebox{#1}}
}

\newcommand{\setimagew}[1]{
 \settowidth{\imagew}{\usebox{#1}}
}

\newcommand{\setimaged}[1]{
 \settowidth{\imaged}{\usebox{#1}}
}


\newcommand{\ImageDimensions}[1]{
   % create and save the box
   \newsavebox{\Image}
   \savebox{\Image}{\includegraphics[scale=0.5]{#1}}
  {\hfill\usebox{\Image}\hfill}
  \setimageh{\Image}
  \setimagew{\Image}
  \setimaged{\Image}
   {\break\footnotesize
     {\raggedright 
      The height is of image (#1) is : \the\imageh\break
     The width of the  image (#1) is : \the\imagew\break
     The depth of the  image (#1) is : \the\imaged\break }}
}


\section*{Determine the height and width of an image}
We are now going to use use our new routines to measure the width and height of an image box. The user interface
is going to be kept as simple as possible as shown below:---

\begin{teX}
\ImageDimensions{./graphics/maths1}
\end{teX}

Given an image and a path, we will print the image and its dimensions as follows:
\medskip

\ImageDimensions{./graphics/maths1}

\bigskip

The code is fairly easy to understand and follows the same pattern as that for a box composed of
text. As a matter of fact the box can contain any combination of letters and images.

\begin{teX}
\newlength{\imageh}
\newlength{\imaged}
\newlength{\imagew}

\newcommand{\setimageh}[1]{
 \settoheight{\imageh}{\usebox{#1}}
}

\newcommand{\setimagew}[1]{
 \settowidth{\imagew}{\usebox{#1}}
}

\newcommand{\setimaged}[1]{
 \settowidth{\imaged}{\usebox{#1}}
}


\newcommand{\ImageDimensions}[1]{
   % create and save the box
   \newsavebox{\Image}
   \savebox{\Image}{\includegraphics[scale=0.5]{#1}}
  {\hss\usebox{\Image}\hss}
  \setimageh{\Image}
  \setimagew{\Image}
   {\break\footnotesize
     {\raggedright 
      The height is of image (#1) is : \the\imageh\break
     The width of the  image (#1) is : \the\imagew\break
     The depth of the  image (#1) is : \the\imaged\par}}
}
\end{teX}


\section*{Examples}

You may wish to look at the example below to see how you can use these. The command |\newsavebox| creates a placeholder for storing a text; the command |\savebox| stores the specified text in this placeholder, and does not display anything in the document; and |\usebox| recalls the content of the placeholder into the document.


Resize an image to take exactly half the text width :

\begin{teX}
\includegraphics[width=0.5\textwidth]{mygraphic}
\end{teX}



Make distance between items larger (inside an itemize environment) :

\begin{teX}
\addtolength{\itemsep}{0.5\baselineskip}
\end{teX}

Use of savebox to resize an image to the height of the text:

% Create the holders we will need for our work

\begin{teX}
\newlength{\mytitleheight}
\newsavebox{\mytitletext}
% Create the reference text for measures
\savebox{\mytitletext}{%
  \Large\bfseries This is our title%
}
\settoheight{\mytitleheight}{\usebox{\mytitletext}}
% Now creates the actual object in our document
\framebox[\textwidth][l]{%
  \includegraphics[height=\mytitleheight]{my_image}%
  \hspace{2mm}%
  \usebox{\mytitletext}%
}
\end{teX}

\makeatletter
\def\triangle#1{%
   \par
   \vbox{%
     \baselineskip4.3pt
     \m@th
     \toks@{}%
     \count@#1
     \loop\ifnum\count@>\z@
        \advance\count@\m@ne
        \toks@\expandafter{\the\toks@{\bullet}}%
        \centerline{$\the\toks@$}%
     \repeat
   }%
}
\makeatother


\triangle{15}





