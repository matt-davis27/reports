\newgeometry{left=1in,right=1in,top=1in,bottom=1in}
\parindent1em
\chapter{NUMBERS}
\epigraph{Every TEXie knows, or should know, that Donald
Knuth, the Grand Wizard of TEX and METAFONT,
designed these wonderful programs using only integer
arithmetic.}{Claudio Beccari}

\begin{multicols}{2}
In this chapter integers and their denotations will be treated, the conversions that are possible
either way, allocation and the use of \cmd{count} registers, and arithmetic with integers.Claudio Beccari TUGboat, Volume 23 (2002), No. 3/4. An important part of the grammar of \tex is the rigorous definition of a \arg{number}, the
syntactic entity that \tex expects when semantically an integer is expected.  Towards the end, |\count| registers, arithmetic, and
tests for numbers are treated.
For clarity of discussion a distinction will be made here between integers and numbers,
but note that a hnumberi can be both an ‘integer’ and a ‘number’. ‘Integer’ will be taken
to denote a mathematical number: a quantity that can be added or multiplied. ‘Number’
will be taken to refer to the printed representation of an integer: a string of digits, in other
words.
\end{multicols}

The common control \tex statements  are shown below:

\begin{enumerate}[(1)]
\item{\cmd{number}} Convert a number to decimal representation.
\item{\cmd{romannumeral}}  Convert a positive number to lowercase roman representation.
\item{\cmd{ifnum}} Test relations between numbers.
\item{\cmd{ifodd}} Test whether a number is odd.
\item{\cmd{ifcase}} Enumerated case statement.
\item{\cmd{count}} Prefix for count registers.
\item{\cmd{countdef}} Define a control sequence to be a synonym for a \cmd{count} register.
\item{\cmd{newcount}} Allocate an unused \doccmd{count} register.
\item{\cmd{advance}} Arithmetic command to add to or subtract from a numeric variable.
\item{\cmd{multiply}} Arithmetic command to multiply a hnumeric variablei.
\item{\cmd{divide}} Arithmetic command to divide a hnumeric variablei.
\end{enumerate}



\section{The number}\margindoc{number}

The number control sequence can be used in cases where you want to convert from a defined counter to
a decimal number for example \verb+\number\day+ will output the integer representing the current day of the month.
\number\day

The \doccmd{romannumeral} will convert to a lowercase roman number. 
\verb+ \romannumeral 2020+  will output {\textit{\romannumeral 2020}}


\section{The countdef}

\doccmd{countdef} defines a symbolic name for a count register (0..255). For instance, we can define a counter
to count examples 
\begin{teX}
\countdef\examplenum=0  % use count0
\examplenum=1 % first example number is 1
The example number is \advance\examplenum by 1  is \number\examplenum
\countdef\examplenum=2 \examplenum=1 % first example number is 1
\end{teX}

The example number is |\advance\examplenum by 1|  is |\number\examplenum|

\begin{teX}
\newcommand{\example}{%
   \noindent
    \textit{Example \number\examplenum\smallskip }%
   \advance\examplenum by 1%
 }
\end{teX}






\section{Counting examples}


\begin{teXXX}
\countdef\examplenum=0  % use count0
\examplenum=1 % first example number is 1
The example number is \advance\examplenum by 1  is \number\examplenum
\end{teXXX}

%\countdef\examplenum=2 \examplenum=1 % first example number is 1



\section{multiply} \marginnote{\doccmd{multiply}}

This command multiplies the $<numeric variable>$ by $<number>$ (i.e by an integer). The numeric variable can be 
a parameter of type integer, dimen, glue or muglue. It is not your average real numbers multiplication

%\example
\begin{teXXX}
\count0=200
\multiply \count0 by 2
\the\count0
\end{teXXX}

This will output \texttt{0} . Note that \TeX\ only works with integers and you need to watch that you do not overflow \TeX 's capacity.


\section{Real numbers}

The \texttt{fp} package by Michael Mehlich, provides macros for fixed point arithmetic
for TeX with numbers ranging from:\people{Michael Mehlich}\index{floating point arithmetic}, \index{real numbers}

    |-999999999999999999.999999999999999999|\\
to  |+999999999999999999.999999999999999999|\\


This is a very old package and the documentation consists of one |readme.txt| file. It is by far the best for floating point arithmetic. Most of the macros receive three arguments with the first one being a macro or a string that receives the values. For example to get the power of a number:

\begin{verbatim}
\FPpow\test{8.0000000000000001}{2.0000000000000000}
\test
\FPpow\test{8.0000000000000001}{0.0000000000000000}
\test
\end{verbatim}


\FPpow\test{8.0000000000000001}{2.0000000000000000}

\noindent\test\\
\FPpow\test{8.0000000000000001}{0.0000000000000000}
\test\\


The package provides the following four operations:

\emphasis{FPadd,FPdiv,FPmul,FPsub,FPpow}
\begin{teXXX}
%binary operations
\FPadd#1#2#3	% #1 := #2+#3
\FPdiv#1#2#3	% #1 := #2/#3
\FPmul#1#2#3	% #1 := #2*#3
\FPsub#1#2#3	% #1 := #2-#3
\end{teXXX}

All \texttt{fp}, commands are prefixed with \texttt{FP}. The commands define a macro to hold the values on the fly.

Armed with such a powerful package it is now possible to do so much more with \tex. Let us try another a bit more difficult example. Both the temperature and pressure of air fall with increasing altitude up to about 10 000 $m$. ASHRAE (1997) gives the following equation for the calculation of atmospheric pressure up to a height of 10 000 $m$:

\begin{equation}
p = 101.325 (1 - 2.25577 \cdot 10^{-5}Z)^{5.2559} 
\end{equation}

\noindent and

\begin{equation}
t = 15 - 0.0065Z \label{temp}
\end{equation}


\noindent for the calculation of temperature up to a height of 11 000 $m$, where $p$ is pressure in $kPa$, $t$ is temperature in $K$ and $Z$ is altitude in $m$ above sea level.

We can write the equation \ref{temp} in \tex as follows:

\begin{teX}
\makeatletter
%% The temperature of air at altitude is
%% given in ASHRAE as t=15-0.0065 Z
%% where t = temperature in deg C
%% and Z is altitude in m. equation is
%% valid up to 11,000 m
%% temperature@altitude will print
%% this temperature

\def\temperature@altitude#1{%
  \FPmul\termone{65}{#1}%
  \FPdiv\termtwo{\termone}{10000}%
  \FPsub\temperature{15}{\termtwo}%
  \texttt{\temperature}
}

\noindent The temperature at 11000 $m$ is 
\temperature@altitude{11000}\hfil\hfill

\makeatother
\end{teX}



%% The calculation is defined here
\makeatletter
%% The temperature of air at altitude is
%% given in ASHRAE as t=15-0.0065 Z
%% where t = temperature in deg C
%% and Z is altitude in m. equation is
%% valid up to 11,000 m
%% temperature@altitude will print
%% this temperature

\def\temperature@altitude#1{%
\FPmul\termone{65}{#1}%
\FPdiv\termtwo{\termone}{10000}%
\FPsub\temperature{15}{\termtwo}%
\texttt{\temperature}
}

\let\TemperatureAtAltitude\temperature@altitude
\makeatother
%% end of macro


\noindent The temperature at 11000 $m$ is \TemperatureAtAltitude{11000} \textcelsius \sidenote{Note that the \textcelsius\ symbol, was typest using the command \texttt{\textbackslash textcelsius of the \texttt{textcomp} package}}
The equation for pressure is a bit more complicated as it involves powers.

\begin{teX}
\def\pressure@altitude#1{%
  \FPmul\zone{2.25577}{#1}%
  \FPdiv\ztwo{\zone}{100000}%
  \FPsub\zthree{1}{\ztwo}%
  \FPpow\zfour{\zthree}{5.2559}%
  \FPmul\zfive{101.325}{\zfour}%
\zfive%
}
\end{teX}


\makeatletter
\def\pressure@altitude#1{%
\FPmul\zone{2.25577}{#1}%
\FPdiv\ztwo{\zone}{100000}%
\FPsub\zthree{1}{\ztwo}%
\FPpow\zfour{\zthree}{5.2559}%
\FPmul\zfive{101.325}{\zfour}%
\zfive%
}

\let\PressureAtAltitude\pressure@altitude
\makeatother

The pressure at 900~$m$ is\texttt{\PressureAtAltitude{900} kPa} and at\\
a 1000~$m$ is \texttt{\PressureAtAltitude{1000} kPa}.\hfill\hfill\\



The obvious question that you may want to ask now, is how to we format the number? 
In \alltex there is no formatting way of pretty printing, but the \docpkg{numprint}, developed by Harald Harders \people{Harald Harders} comes to the rescue. It provides macros for the rounding-off of numbers to as many decimal places as we may wish. It provides the macros \cmd{nprounddigits} and \cmd{numprint} to do so. The package can print numbers both in |mathmode| as well as normal |text mode|.\sidenote{The \texttt{fp} package also provides a \texttt{round} function, but it is a bit more complicated. We will discuss this later on.}




\begin{teX}
\nprounddigits{2}$\numprint{20.12345678}$
\end{teX}

{%\selectlanguage{english}
\nprounddigits{2}\numprint{12.4567890}}



\begin{teX}
\def\pressure@altitude#1{%
  \FPmul\zone{2.25577}{#1}%
  \FPdiv\ztwo{\zone}{100000}%
  \FPsub\zthree{1}{\ztwo}%
  \FPpow\zfour{\zthree}{5.2559}%
  \FPmul\zfive{101.325}{\zfour}%
  \selectlanguage{english}%
  \nprounddigits{2}\numprint{\zfive}%
}
\end{teX}




\makeatletter
\def\pressure@altitude#1{%
\FPmul\zone{2.25577}{#1}%
\FPdiv\ztwo{\zone}{100000}%
\FPsub\zthree{1}{\ztwo}%
\FPpow\zfour{\zthree}{5.2559}%
\FPmul\zfive{101.325}{\zfour}%
\selectlanguage{english}%
\nprounddigits{2}\numprint{\zfive}%
}

\let\PressureAtAltitude\pressure@altitude
\makeatother


The pressure at 900~$m$ is\texttt{\PressureAtAltitude{900} kPa} and at
a 1000~$m$ is \texttt{\PressureAtAltitude{1000} kPa}.\hfill\hfill\\


Having managed to set-out the macros we need to calculate as well as to typeset to two decimal points, the next exercise to create a table with values from 500 m to 10000 m as shown in Table \ref{pvsaltitude}




\begin{table}
\begin{center}
\begin{tabular}{cc}
\toprule
Altitude & Pressure\\
(m)   & (kPa)\\
\midrule
\Z500  &\PressureAtAltitude{1000}\\
1000 &\PressureAtAltitude{1000}\\
2000 &\PressureAtAltitude{2000}\\
3000 &\PressureAtAltitude{3000}\\
4000 &\PressureAtAltitude{3000}\\
5000 &\PressureAtAltitude{5000}\\
6000 &\PressureAtAltitude{6000}\\
7000 &\PressureAtAltitude{8000}\\
8000 &\PressureAtAltitude{6000}\\
9000 &\PressureAtAltitude{6000}\\
10000 &\PressureAtAltitude{6000}\\
\bottomrule
\end{tabular}
\caption{Pressure (kPa) vs altitude (m)}
\label{pvsaltitude}
\end{center}
\end{table}



We have managed so far, not only to produce a fairly complicated calculation, but also to typeset both the results as well as the explanation of the methodology we used to achieve it. Isn't \tex great?




We are now going to check if \tex can work out the surface area of your body, provided that you are an \textit{average} person. This is given by the Du Bois\cite{dubois1916} surface area, $A_D$, where\index{Du Bois equation}

\[
A_{D}=0.202m^{0.425}h^{0.725}
\]

\noindent in which $m$ is the body mass in |kg| and |h| its height in |m|. Using the above equation a man of 70 $kg$ (11 stone) with a height of 1.8 $m$ (5' 11") has a surface area of 1.9 $m^2$. One |met| unit
is often taken as corresponding to 100 W, approximately, representing the emission from
a resting adult.


\begin{teX}
%% given the body mass m = #1 and the height h = #2
%% using the Du Bois equation
\def\body@area#1#2{
  \FPpow\t@one{#1}{0.425}%
  \FPpow\t@two{#2}{0.725}%
  \FPmul\t@three{\t@one}{\t@two}%
  \FPmul\A@D{0.202}{\t@three}%
  \FPmul\@watts{\A@D}{58.2}%
  \nprounddigits{1}\numprint{\A@D}%
Body area is \A@D   and generates \@watts Watts%
}
\end{teX}

\makeatletter
%% Calculates the area of an average person
%% given the body mass m and the height h
%% using the Du Bois equation
\def\body@area#1#2{
  \FPpow\t@one{#1}{0.425}%
  \FPpow\t@two{#2}{0.725}%
  \FPmul\t@three{\t@one}{\t@two}%
  \FPmul\A@D{0.202}{\t@three}%
  \FPmul\@watts{\A@D}{58.2}%
  \nprounddigits{1}\numprint{\A@D}%
Body area is \A@D   and generates \@watts Watts%
}


\texttt{\body@area{70}{1.8} }

\makeatother


\section{Trigonometric and other functions}

The |fp| package also provides macros for the calculation of trigonometric and other functions. For example, if you want to calculate the $sin$ of $90 \textcelsius$ you can write:\sidenote{All values must be in \textit{radians}.}

\begin{teX}
\FPdiv\@resulta{\FPpi}{2}
\FPsin\theta@t{\@resulta}

The value of $sin{\pi/2}$ is \theta@t
\end{teX}


\makeatletter
\FPdiv\@resulta{\FPpi}{2}
\FPsin\theta@t{\@resulta}

\noindent > \texttt{The value of $sin{(\pi/2)}$ is \theta@t}

\makeatother


\bigskip
\normalsize

If we assume the outside temperature, $t_{\theta}$, varies sinusoidally with time, $\theta$, and its maximum
 value, $t_{15}$, occurs at 15.00 h sun-time we can write:

\begin{equation}
t_{\theta}=t_{15}-\frac{D}{2}\left[1-sin{\frac{(\theta\pi-9\pi)}{12}}\right] 
\label{diurnal}
\end{equation}

\noindent where $D$ is the difference between the mean daily maximum and minimum temperatures,
otherwise termed the \textit{diurnal range}.\index{diurnal range}

As before we define the equation by splitting it apart and using the |fp| macros for addition, subtraction multiplication and the trigonometric function.



\begin{teX}
%% Finding the dry-bulb temperature
%% at any time of the day, given
%% hour and diurnal range and max temp at 15h00
%% #1 t15 #2 theta #3 D
\def\ttheta#1#2#3{%
  \FPmul\ninepi{9}{\FPpi}% 9 x pi
%9 x pi = \ninepi \\
  \FPmul\thetapi{#2}{\FPpi}%theta x pi
%13xpi = \thirteenpi\\
  \FPsub\sina{\thetapi}{\ninepi}%theta x pi - 9 pi
%difference \sina \\
  \FPdiv\sinb{\sina}{12}
%sin/12 \sinb \\
  \FPsin\sinc{\sinb}% sin term
%sin must be 0.866 \sinc 
  \FPsub\ttwo{1}{\sinc}%1-sin
  \FPdiv\DD{#3}{2}% D/2
  \FPmul\tthree{\DD}{\ttwo}
  \FPsub\finall{#1}{\tthree}
  \nprounddigits{3}\numprint{\finall}%
}
\end{teX}


%% Finding the dry-bulb temperature
%% at any time of the day, given
%% hour and diurnal range and max temp at 15h00
%% #1 t15 #2 theta #3 D
\def\ttheta#1#2#3{%
\FPmul\ninepi{9}{\FPpi}% 9 x pi
%9 x pi = \ninepi \\
\FPmul\thirteenpi{#2}{\FPpi}%theta x pi
%13xpi = \thirteenpi\\
\FPsub\sina{\thirteenpi}{\ninepi}%theta x pi - 9 pi
%difference \sina \\
\FPdiv\sinb{\sina}{12}
%sin/12 \sinb \\
\FPsin\sinc{\sinb}% sin term
%sin must be 0.866 \sinc 
\FPsub\ttwo{1}{\sinc}%1-sin
\FPdiv\DD{#3}{2}% D/2
\FPmul\tthree{\DD}{\ttwo}
\FPsub\finall{#1}{\tthree}
\nprounddigits{3}\numprint{\finall}%
}




\begin{table}[htbp]
\begin{center}
\begin{tabular}{cl}
\toprule
Sun-time & temperature\\
hours & \ZZZZ\textcelsius\\
\midrule
\Z1&\ttheta{20}{1}{8.3}\\
\Z8&\ttheta{20}{8}{8.3}\\
\Z9&\ttheta{20}{9}{8.3}\\
10&\ttheta{20}{10}{8.3}\\
11&\ttheta{20}{11}{8.3}\\
12&\ttheta{20}{12}{8.3}\\
13&\ttheta{20}{13}{8.3}\\
14&\ttheta{20}{14}{8.3}\\
15&\ttheta{20}{15}{8.3}\\
16&\ttheta{20}{16}{8.3}\\
17&\ttheta{20}{17}{8.3}\\
18&\ttheta{20}{18}{8.3}\\
\bottomrule
\end{tabular}
\caption{Diurnal variations in temperature}
\label{tbl:diurnal}
\end{center}
\end{table}


To summarize all trigonometric values use radians. For example to check $sin(\pi/3)$ we can write

\begin{teXXX}
\FPdiv\tempa{\FPpi}{3}
\FPsin\tempb{\tempa}\\

\nprounddigits{3}\numprint{\tempb}
\end{teXXX}

\FPdiv\tempa{\FPpi}{3}

\FPsin\tempb{\tempa}

\FPround\temp{\tempb}{3}
\temp

Just verify all values, like in any programming language to ensure that bugs do not creep in. Personally, for any calculations I prefer to split complicated equations to smaller parts and check each and every step to ensure that all values are correct. 


\newthought{Pascal's triangle}

Peeking through the \texttt{fp} code there is a class for calculating any row of the Pascal triangle


\begin{center}
\FPpascal\test{0}\test\\
\FPpascal\test{1}\test\\
\FPpascal\test{2}\test\\
\FPpascal\test{3}\test\\
\FPpascal\test{4}\test\\
\FPpascal\test{5}\test\\
\FPpascal\test{6}\test\\
\FPpascal\test{7}\test\\
\FPpascal\test{8}\test\\
\end{center}



You just allocate a macro to receive the results and you simply type the row you want to calculate:

\begin{teX}
\begin{center}
  \FPpascal\test{0}\test\\
  \FPpascal\test{1}\test\\
  \FPpascal\test{2}\test\\
  \FPpascal\test{3}\test\\
  \FPpascal\test{4}\test\\
  \FPpascal\test{5}\test\\
  \FPpascal\test{6}\test\\
\end{center}
\end{teX}





%\FPround\test{2345.78989098789}{3}    



%Before we understand how to write macros that can cahieve what the fp package is doing we need to understand some of the programming patterns used.
%
%If we have a macro with for example three parameters, as shown below. In order to parse the argument, we may need to add a delimiter at the end automatically. The fp has been using consistently two dots for delimiting arguments.
%
%\begin{verbatim}
%\def\@calld#1{%
%  % #1 macro to call
%  % #2 macro, which gets the result
%  % #3 value and adding two dots as delimiters!
%  \edef\next{\noexpand#1..\noexpand\relax}%
%  \next%
%}
%\end{verbatim}
%
%
%\begin{comment}
%\def\@receiver#1{#1}
%
%\def\Round#1\@calld\@receiver#1 
%
%\Round{200.4444}
%\end{comment}
%
%
%
%
%
%\begin{teX}
%
%\def\Round#1#2#3{\@calld\@round#1{#2}{#3}}
%
%\def\@round#1#2.#3.#4\relax#5{
%This at @round}
%
%
%%%%private fp-area (don't use these macros)%%%
%
%%round value
%\def\@round#1#2.#3.#4\relax#5{%
%  % #1 macro, which gets the result
%  % #2 integer part of value
%  % #3 fractional part of value
%  % #4 dummy to swallow everything after the second '.'
%  % #5 round position
%  {\FP@beginmessage{ROUND}%
%   %
%   \FP@count=#5\relax%
%   %
%   \ifnum\FP@count<0\relax%
%     \FP@errmessage{Rounding value before point not allowed}%
%   \else%
%     \FP@readvalue{x}{#2}{#3}%
%     \ifnum\FP@count=0\relax%
%        \ifnum\FP@xfa<500000000\relax%
%	  \edef\FP@tmp{#2}%
%	\else%
%	  \advance\FP@xib1\relax%
%	  \ifnum\FP@xib<1000000000\relax\else%
%	    \advance\FP@xia1\relax%
%	    \advance\FP@xib-1000000000\relax%
%	  \fi%
%	  \ifnum\FP@xs<0\relax%
%	    \edef\FP@tmp{-}%
%	  \else%
%	    \edef\FP@tmp{}%
%	  \fi%
%	  \ifnum\FP@xia=0\relax%
%	    \edef\FP@tmp{\FP@tmp\the\FP@xib}%
%	  \else%
%	    \advance\FP@xib1000000000\relax%
%	    \edef\FP@tmp{\FP@tmp\the\FP@xia\expandafter\FP@ignorenext\the\FP@xib}%
%	  \fi%
%	\fi%
%     \else%
%       \ifnum\FP@count>18\relax%
%         \FP@count=18\relax%
%       \fi%
%       \advance\FP@xfa1000000000\relax%
%       \advance\FP@xfb1000000000\relax%
%       \edef\FP@tmp{\expandafter\FP@ignorenext\the\FP@xfa\expandafter\FP@ignorenext\the\FP@xfb}%
%       \edef\FP@rd{}%
%       \FP@shift=\FP@count%
%       \FP@times=1\relax%
%       \loop%
%          \ifnum\FP@count>0\relax%
%             \edef\FP@rd{\FP@rd\expandafter\FP@first\FP@tmp\noexpand\relax}%
%             \edef\FP@tmp{\expandafter\FP@ignorenext\FP@tmp}%
%	     \advance\FP@count-1\relax%
%	     \multiply\FP@times10\relax\ifnum\FP@times=1000000000\relax\FP@times=1\relax\fi%
%          \repeat%
%       \edef\FP@tmp{\expandafter\FP@first\FP@tmp\noexpand\relax}%
%       \expandafter\FP@rega\FP@tmp\relax%
%       \ifnum\FP@rega>4\relax%
%	 \ifnum\FP@shift>8\relax%
%	   \FP@ninesplit\FP@rd%
%	   \expandafter\FP@rega\expandafter0\FP@rd\relax%
%	   \advance\FP@rega1\relax%
%	   \ifnum\FP@rega<\FP@times%
%	     \advance\FP@rega\FP@times%
%	     \advance\FP@res1000000000\relax%
%	   \else%
%	     \advance\FP@res1\relax%
%	     \ifnum\FP@res<1000000000\relax%
%	       \advance\FP@res1000000000\relax%
%	     \else%
%	       \advance\FP@xib1\relax%
%	       \ifnum\FP@xib<1000000000\relax\else%
%	         \advance\FP@xia1\relax%
%	         \advance\FP@xib-1000000000\relax%
%	       \fi%
%	     \fi%
%	   \fi%
%	  
%
%\Round{\test}{9999.999999}{4}
%\end{teX}



\section{Conditionals with fp}
\index{fp!conditionals}


The |fp| package offers some conditionals for checking real numbers.
Consider the following code:

\emphasis{FPset,FPifgt}
\begin{teXXX}
\FPset\test{12000}
\FPset\limit{10000}

\FPifgt\test\limit 12000 is greater than \limit
  \else ... \fi
\end{teXXX}

%\FPset\test{12000}
%\FPset\limit{10000}
%
%\FPifgt\test\limit 12000 is greater than \limit
%  \else ... \fi
%




  
\section{e-TeX Expressions}
\index{e-Tex!Expressions}\index{expressions}

\etex introduces the notion of expressions of type |number|, |dimen|, |glue|, or
|muglue|, that can be used whenever a quantity of that type is needed. Such
expressions are evaluated by \etex's scanning mechanism; they are initiated by
one of the commands \cmd{numexpr}, \cmd{dimexpr}, \cmd{glueexpr}, or \cmd{muexpr} 
(determining the type t) and optionally terminated by one \cmd{relax} (that will be absorbed
by the scanning mechanism). 

An expression consists of one or more terms of
the same type to be added or subtracted; a term of type \textit{t} consists of a factor
of that type, optionally multiplied and/or divided by numeric factors; finally a
factor of type t is either a parenthesized subexpression or a quantity (number,
etc.) of that type. Thus, the conditional

\begin{teX}
\ifdim\dimexpr (2pt-5pt)*\numexpr 3-3*13/5\relax + 34pt/2<\wd20
\end{teX}

is true if and only if the width of box 20 exceeds 32 pt . Note the use of| \relax|
to terminate the inner (numeric) expression, the outer (dimen) expression is
terminated automatically by the token <12 that does not fit into the expression
syntax.

The arithmetic performed by "e-TEX's expressions does not do much that
could not be done by TEX's arithmetic operations \cmd{advance}, \cmd{multiply}, and
\cmd{divide}, although there are some notable differences: Each factor is checked to
be in the allowed range, numbers must be less than $2^{31}$ in absolute value, dimensions or glue components must be less than $2^{14}$pt, |mu|, |fil|, etc. respectively.


The arithmetic operations are performed individually, except for `scaling' operations
(a multiplication immediately followed by a division) which are performed
as one combined operation with a 64-bit product as intermediate value. 

The
result of each operation is again checked to be in the allowed range. Finally the
results of divisions and scalings are rounded, whereas TEX's \cmd{divide} truncates.

The important new feature is, however, that the evaluation of expressions
does not involve assignments and can therefore be performed in circumstances
where assignments are not allowed, e.g., inside an \cmd{edef} or \cmd{write}. This also
allows the definition of \textit{purely expandable loop constructions}\index{loops! purely expandable}:



\begin{teX}
\def\foo#1#2{\number#1
\ifnum#1<#2,
  \expandafter\foo
  \expandafter{\number\numexpr#1+1\expandafter}%
  \expandafter{\number#2\expandafter}%
\fi}


\foo{9}{15}
\end{teX}


\def\foo#1#2{\number#1
\ifnum#1<#2,
  \expandafter\foo
  \expandafter{\number\numexpr#1+1\expandafter}%
  \expandafter{\number#2\expandafter}%
\fi}


\texttt{> \foo{9}{15}}


The |\foo| is essentialy a recursive macro. It keeps calling itself until the conditional |if| returns true. The |\expandafter|'s are difficult to grasp, especially the one's before the curly bracket (|}|). 
The macro can happily be writen as\sidenote{Slightly modified to insert an iteration step as the third argument}:

\begin{teX}
\def\foo#1#2#3{\noindent\number#1, 
\ifnum#1<#2
  \foo{\number\numexpr#1+\number#3}{\number#2}{\number#3}%
   \else\relax
  
\noindent\TeX\par
\fi}
\end{teX}

\noindent Which gives the same output:

\def\foo#1#2#3{\noindent\number#1, 
\ifnum#1<#2
  \foo{\number\numexpr#1+\number#3}{\number#2}{\number#3}%
   \else\relax
\fi}

\foo{9}{15}{1}

It is important to have the line \texttt{\textbackslash else\textbackslash relax} inserted, otherwise you might get strange result. The macro uses \cmd{numexpr} to add the iteration step on the start value.

(See 
\url{http://www.ibm.com/developerworks/linux/library/l-recurs.html}) for more information on recursion and especially tail-end recursion.


The expresssion |{\number\numexpr22/2+23}| results in {\number\numexpr22/2+23}



\section*{The \texttt{intcalc} package}

The \docpkg{intcalc}, developed by Heiko Oberdiek\people{Heiko Oberdiek}
is mainly based on \etex. However, if \etex is not present the package provide its own routines. The package provides expandable arithmetic operations with integers.\index{integers!calculations} 
\index{integers!etex}

In \tex integers mean numbers. The same restrictions apply, the range is limited to [-2147483647, 2,147,483,647]. \index{Numbers!integers}

The operations have the form of macros that take one or two integers as parameter
and return the integer result. The macro name is a three letter operation, such as |add| prefixed by the package name, e.g. |\intcalcAdd{10}{43}| returns 53.

The macros are fully expandable, exactly two expansion steps generate the
result. Therefore the operations may be used nearly everywhere in \tex, even
inside |\number|, |\csname|, file names, or other expandable contexts.
The package contains two implementations of the operations. If etex\ is detected
then the macros are implemented using its features |(\numexpr)|. Otherwise
the slower implementation without etex's help is chosen.\index{numexpr}

Arguments can be anything that \tex interprets as |\number|". Examples:
plain numbers, count or length register, macros that expands to a number.

The package has been written in a style common of newer packages and is worth to read through it.

%% The command places the macro under the package index, prints it out and
%% puts a comment in the sidemargin



The commands \pmac{intcalc}{intcalcAdd}, \pmac{intcalc}{intcalcSub},
\pmac{intcalc}{intcalcMul} and \pmac{intcalc}{intcalcDiv} are used to provide the normal arithmetic operators for addition, subtraction, multiplication and division.

The modulo of a number can be determined using \pmac{intcalc}{intcalcMod} and 
\pmac{intcalc}{intcalcPow} can be used to calculate powers.

\begin{teXXX}
\intcalcPow{2}{12}
\end{teXXX}

\textit{\printf{> \intcalcPow{2}{12}}}


The rest of the commands are discussed later on, but the naming scheme is very intuitive.

\pmac{intcalc}{intcalcInv}

\pmac{intcalc}{intcalcAbs}

\pmac{intcalc}{intcalcSgn}
\pmac{intcalc}{intcalcMin}
\pmac{intcalc}{intcalcMax}

\pmac{intcalc}{intcalcCmp}

\pmac{intcalc}{intcalcNum}



The advantage of using the package is that numerical expressions can be
used and that they are fully expandable.

For example:

\emphasis{intcalcAdd,intcalcMin,intcalcMax}
\begin{teXXX}
\intcalcAdd{12+31+22}{11/2}
\intcalcMin{135678}{7890+545678}
\intcalcMax{135678}{7890+545678}
\end{teXXX}

\def\printf#1{\leavevmode \mbox{> \texttt{#1}}\par}

\printf{> \intcalcAdd{12+31+22}{11/2}}
\printf{> \intcalcMin{135678}{7890+545678}}
\printf{> \intcalcMax{135678}{7890+545678}}


The macro |\intcalcNum| converts its argument to a normalized integer number without unnecessary leading zeros or signs. The result matches the regular expression:

\begin{Verbatim}
  0|-?[1-9][0-9]*
\end{Verbatim}

Typing for example the following:

\begin{teX}
  \intcalcNum{-+01234567}
  \intcalcNum{-1234567}
\end{teX}


\printf{\intcalcNum{-+01234567}}
\printf{\intcalcNum{-1234567}}

As you can observe the macro will gobble everything except the first sign.
\pmac{intcalc}{intcalcFac}
\pmac{intcalc}{intcalcSqr}
\pmac{intcalc}{intcalcShl}
\pmac{intcalc}{intcalcShr}
The package uses mostly \tex macros, for example to get the maximum of a number,
it uses a delimited macro (using !) to get two numbers and then compare them.

\begin{teX}
\makeatletter
\def\InCa@Min#1!#2!{%
 \ifnum#1<#2 %
   #1%
 \else
   #2%
 \fi
}
\printf{> \InCa@Min 200!345!|}

\def\InCa@Min#1!#2!{%
 \ifnum#1<#2 %
   #1%
 \else
   #2%
 \fi
}

\printf{\InCa@Min 200!345!}
\makeatother
\end{teX}



\newthought{Factorial}

Do not expect anything spectacular as Knuth's macro for factorials, Oberdiek's definition is much more efficient at the cost of him typing it the values in his package.
As he writes in the documentation for  |\InCa@Fac|, it does not make much sense to calculate the faculty by a general algorithm. The
allowed range of arguments is too low because of the limited integer domain.

\begin{teXX}
146 \def\InCa@Fac#1!{%
147 \ifcase#1 1% 0!
148 \or 1% 1!
149 \or 2% 2!
150 \or 6% 3!
151 \or 24% 4!
152 \or 120% 5!
153 \or 720% 6!
154 \or 5040% 7!
155 \or 40320% 8!
156 \or 362880% 9!
157 \or 3628800% 10!
158 \or 39916800% 11!
159 \or 479001600% 12!
160 \else
161 \ifnum#1<\z@
162    0\IntCalcError:FacNegative%
163   \else
164     0\IntCalcError:FacOverflow%
165   \fi
166 \fi
167 }
\end{teXX}

\section*{Big Integer Number calculations}

\begin{multicols}{2}
Big number calculations can be performed by using the package \docpkg{bigintcalc}, by Heiko Oberdiek. 
Package \texttt{bigintcalc} defines arithmetic operations that deal with big integers. Big
integers can be given either as explicit integer number or as macro code that
expands to an explicit number. Big means that there is no limit on the size of
the number. Big integers may exceed TEX's range limitation of -2147483647 and
2147483647. Only memory issues will limit the usable range. To calculate the values we use, intuitive, but rather long expressions:

\begin{teXXX}
\bigintcalcPow{2}{15}
\end{teXXX}

\printf{\bigintcalcPow{2}{15}}
\end{multicols}

In mathematics, a \textit{double Mersenne number} is a Mersenne prime of the form

\[M_{M_p} = 2^{2^p-1}-1\]


where $p$ is a Mersenne prime exponent.

The smallest double Mersenne numbers 

The sequence of double Mersenne numbers begins %<ref name="Caldwell">Chris Caldwell, [http://primes.utm.edu/mersenne/index.html#unknown ''Mersenne Primes: History, Theorems and Lists''] at the [[Prime Pages]].</ref>

\[M_{M_2} = M_3 = 7 \]
\[M_{M_3} = M_7 = 127 \]
\[M_{M_5} = M_{31} = 2147483647 \]
\[M_{M_7} = M_{127} = 170141183460469231731687303715884105727 \]

\textit{OEIS id=A077586}

\begin{multicols}{2}
One interesting Mersene number is $M_{67}$, this was originally thought to be a prime number,
but let us firt calculate it.
\end{multicols}
\begin{teXXX}
\def\Mersenne#1{%
  \gdef\exponent##1{\bigintcalcPow{2}{##1}}%
  \bigintcalcSub{\exponent{#1}}{1}
}
\end{teXXX}
\def\Mersenne#1{%
  \gdef\exponent##1{\bigintcalcPow{2}{##1}}%
  \bigintcalcSub{\exponent{#1}}{1}
}


This produces the number 

$M_{67}=$\Mersenne{67}





To check if it is prime

\begin{teXXX}
\bigintcalcMul{193707721}{761838257287}
\npthousandsep{ }
\numprint{\bigintcalcSub{\exponent{257}}{1}}
\end{teXXX}

\bigintcalcMul{193707721}{761838257287} %NNNNNNNNNNNNNNNNN  

%\noindent
%\npthousandsep{ }
%\numprint{\bigintcalcSub{\exponent{2281}}{1}}{0}


and here is the $M_{2203}$ Mersenne number, quite impressive, if you think it was calculated with \tex.

%{\tt
%\noindent%
%\npthousandsep{ }%
%\numprint{\bigintcalcSub{\exponent{4253}}{1}}
%}
%
%\newthought{Calculating Prime Numbers}
%
%\def\Prime#1{%
%\def\stepone{\bigintcalcMul{#1}{#1}} 
%\def\steptwo{\bigintcalcAdd{\stepone}{#1}}
%\def\stepthree{\bigintcalcAdd{\steptwo}{41}}
%\stepone: 
%\steptwo: 
%\stepthree: 
%}
%
%
%\Prime{19}\\
%\Prime{20}\\

%where $p$ is a Mersenne prime exponent.
%
%The smallest double Mersenne numbers 
%
%The [[sequence]] of double Mersenne numbers begins <ref name="Caldwell">Chris Caldwell, [http://primes.utm.edu/mersenne/index.html#unknown ''Mersenne Primes: History, Theorems and Lists''] at the [[Prime Pages]].</ref>
%:<math>M_{M_2} = M_3 = 7 </math>
%:<math>M_{M_3} = M_7 = 127 </math>
%:<math>M_{M_5} = M_{31} = 2147483647 </math>
%:<math>M_{M_7} = M_{127} = 170141183460469231731687303715884105727 </math> {{OEIS|id=A077586}}.




%\section{Character constants}
%In formats and macro collections numeric constants are often needed. There are several ways to implement these in TEX.
%Firstly,
%
%\begin{teX}
%\newcount\SomeConstant \SomeConstant=42
%\end{teX}
%
%This is wasteful, as it uses up a \cmd{count} register.
%Secondly,
%
%\begin{teX}
%\def\SomeConstant{42}
%\end{teX}
%
%Better but accident prone: \tex has to expand to find the number – which in itself is a slight
%overhead – and may inadvertently expand some tokens that should have been left alone.
%Thirdly,
%
%\emphasis{chardef}
%\begin{teXXX}
%\chardef\SomeConstant=42
%\end{teXXX}
%
%
%This one is fine. A |chardef token| has the same status as a |\count| register: both are |internal integers|. Therefore a number defined this way can be used everywhere that a |count| register is
%feasible. For large numbers |chardef| \index{TeX primitive!chardef} can be replaced by |mathchardef| \index{TeX primitive! mathchardef}, which runs to |7FFF = 32 767|. Note that a |mathchardef| token can usually only appear in math mode, but in the context
%of a number it can appear anywhere.
%
%One can also use dimension registers and simply remove the |sp| at the end. This is a bit more difficult but can result in saving constants that include large numbers.
%
%
%
%
%
%

From the \texbook

Explain how to round |\dimen2|
to the nearest multiple of\/ |\dimen3|, assuming that |\dimen3| is positive.

\begin{Code}
\advance\dimen2 by\ifnum\dimen2<0 -\fi.5\dimen3|\par
\divide\dimen2 by\dimen3 \multiply\dimen2 by\dimen3
\end{Code}

\textbf{answer}


{
\dimen2=100pt
\dimen3=33pt

{


\advance\dimen2 by\ifnum\dimen2<0 -\fi.5\dimen3\par
\divide\dimen2 by\dimen3 \multiply\dimen2 by\dimen3

\the\dimen2
}

\the\dimen2

\advance\dimen2 by .5\dimen3

\the\dimen2
}
\bigskip

Admitedly, here the critics of \tex are justified in saying \tex's constructions are difficult to follow.
Observe how Knuth added the minus(-) sign into the character stream! At first one might think that Knuth missed the else, but the fact is that it is unecessary. At the cost of an |\else| the construction would have been much more readable:
\medskip

\emphasis{else}
\begin{Code}
\dimen2=100pt
\dimen3=33pt
\advance\dimen2 by\ifnum\dimen2<0 -.5\dimen3\else .5\dimen3\fi
\divide\dimen2 by\dimen3 \multiply\dimen2 by\dimen3
\the\dimen2
\bye
\end{Code}
\medskip

All this thrift and conservatism in writing efficient programs, will of course pale into insignificance, if you are going to have a |RequirePackage{pgf}| in your package; but then maybe the reason that I can brew a coffee cup while my windows machine is booting is that programmers abandoned thrift!


\section{Defining special sectioning}

Sometimes, authors want to highlight certain paragraphs by special symbols. One such type of text, can be found in Knuth's TeXBook, with the dangerous bends. Knuth in |manmac| defines them as follows:

\begingroup
\makeatletter
\def\hang{\hangindent3em}

\let\ninepoint\small
\font\manual=manfnt
\def\dbend{{\manual\char127}} % dangerous bend sign
\def\d@nger{\medbreak\begingroup\clubpenalty=10000
  \def\par{\endgraf\endgroup\medbreak} \noindent\hang\hangafter=-2
  \hbox to0pt{\hskip-\hangindent\dbend\hfill}\ninepoint}
\outer\def\danger{\d@nger}
\def\dd@nger{\medbreak\begingroup\clubpenalty=10000
  \def\par{\endgraf\endgroup\medbreak} \noindent\hang\hangafter=-2
  \hbox to0pt{\hskip-\hangindent\dbend\kern1pt\dbend\hfill}\ninepoint}
\outer\def\ddanger{\dd@nger}
\def\enddanger{\endgraf\endgroup} % omits the \medbreak

\makeatother

\ddanger \hangindent1em
\lipsum*[1]

\makeatother
\endgroup

\medskip
\begin{Code}
%\def\hang{\hangindent\parindent}
\def\hang{\hangindent3em}
\let\ninepoint\small
\font\manual=manfnt
% the double bend
\def\dbend{{\manual\char127}} % dangerous bend sign
% one sign
\def\d@nger{\medbreak\begingroup\clubpenalty=10000
  \def\par{\endgraf\endgroup\medbreak} \noindent\hang\hangafter=-2
  \hbox to0pt{\hskip-\hangindent\dbend\hfill}\ninepoint}
% author's comand
\outer\def\danger{\d@nger}
% internal command
\def\dd@nger{\medbreak\begingroup\clubpenalty=10000
  \def\par{\endgraf\endgroup\medbreak} \noindent\hang\hangafter=-2
  \hbox to0pt{\hskip-\hangindent\dbend\kern1pt\dbend\hfill}\ninepoint}
% author's command
\outer\def\ddanger{\dd@nger}
\def\enddanger{\endgraf\endgroup} % omits the \medbreak
\end{Code}
\medbreak

You will find a problem running this straight into \latex as |\hang| is a |plain| \tex command and has been depreciated from \LaTeXe. The command can still be found in |latex209.def| (see latex209.dtx). In the code above I have modified it slightly to suit  the text style of this book. I have also assigned the |\ninepoint| to |\small|, but have kept the |manfnt|. See if you can style it differently.


\restoregeometry











