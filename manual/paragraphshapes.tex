\chapter{SHAPING PARAGRAPHS}
The |\parshape| command is the only command available in \TeX\ that can influence the shaping of individual  paragraph lines. Its general form is |\parshape|$=n\,\, i_1l_1i_2l_2\dots i_nl_n$. It creates a paragraph whose first $n$ lines have lengths $l_1,l_2,\dots, l_n$ and are indented by $i_1,i_2\dots,i_n$ from the left margin.

Consider the figure below showing an extract from the Principi\ae. We will typeset this using |\parshape|.
\pagebreak
\bgroup
\parindent=0pt
\small

\vspace*{4\baselineskip}

\hspace*{6.7cm}\includegraphics[height=18\baselineskip]{./images/060.png}
\vspace{-24\baselineskip}

\newlength{\inn}
\setlength{\inn}{210pt}

\newlength{\innn}
\setlength{\innn}{160pt}

\newlength{\innnn}
\setlength{\innnn}{90pt}

\newlength{\zz}
\setlength{\zz}{70pt}

\bottomline
\parshape 23  \zz \the\dimexpr\textwidth-120pt
                   \zz \the\dimexpr\textwidth-120pt
                   \zz \the\dimexpr\textwidth-120pt
                   \zz \inn %4
                   \zz \inn
                   \zz \inn
    \zz  \inn
   \zz  \innn
  \zz  \innn
   \zz \innn
   \zz \innn
                 \zz \innn
                 \zz \innnn
                  \zz \innnn
   \zz \innnn
                 \zz \innnn
                  \zz \innnn
   \zz \innnn
   \zz \innnn
                 \zz \innnn
                 \zz \innnn
                  \zz \innnn
                  \zz \the\dimexpr\textwidth-120pt
  Demittatur perpendicularis $QT$. Et Hyperbol{\ae} latere recto
principali (seu $\frac{2BCq.}{AC}$) dicto $L$, erit $L \times QR$ ad $L \times Pv$~ut~$QR$
ad $Pv$, id est, ut $PE$ (seu $AC$) ad $PC$; Et $L \times Pv$~~ad~~$GvP$
ut $L$ ad $Gv$; \& $GvP$ ad $Qvq.$ ut $CPq.$
ad $CDq.$; \& (per Lem.\ VIII.) $Qvq.$ ad
$Qxq.$, punctis $Q$ \& $P$ coeuntibus
fit ratio {\ae}qualitatis; \& $Qxq.$ seu
$Qvq.$ est ad $QTq.$ ut $EPq.$ ad $PFq.$,
id est ut $CAq.$ ad $PFq.$, sive (per
Lem.\ XII.) ut $CDq.$ ad
$CBq.$: \& conjunctis his omnibus
rationibus $L \times QR$ fit
ad $QTq.$ ut $AC$
ad $PC + L$ ad
$Gv + CPq.$ ad
$CDq. + CDq.$
ad $CBq.$: id est
ut $AC \times L$ (seu
$2BCq.$) $\times PCq.$
ad $PC \times
Gv \times CB \opit{quad.}$
sive ut $2PC$
ad $Gv$, sed
punctis $Q$ \& $P$
coeuntibus
{\ae}quantur $2PC$
\& $Gv$. Ergo \& his proportionalia $L \times QR$ \& $QTq.$ {\ae}quantur.
Ducantur h{\ae}c {\ae}qualia in $\frac{SPq.}{QR}$ \& fiet $L \times SPq.$ {\ae}quale $\frac{SPq. \times QTq.}{QR}$.
Ergo (per Corol.\ Theor.\ V.) vis centripeta reciproce est ut
$L \times SPq.$ id est in ratione duplicata distanti{\ae} $SP$. \QEIit\par
\nobreak
\bottomline
\captionof{figure}{Extract from \textit{Philosophi\ae\  Naturalis Principia Mathematica}, typeset using \texttt{\textbackslash parshape} and a lot of patience.}
\egroup

\begin{figure}[p]
\centering
\includegraphics[scale=1]{images/page52}
\caption{Page 52 from Isaac Newton's \textit{Philosophi\ae\  Naturalis Principia Mathematica}. Image was obtained from Google's copy, available at Google Books.}
\label{fig:principia}
\end{figure}


\section{Shaping paragraphs with \texttt{shapepar}}

By using |parshape|, you could literally make your paragraph any shape you want.
But if you want your paragraph to be shaped a heart, there's a package, |shapepar|, that
could ease your work. The package provides a few predefined shapes that you could call
up by using |diamondpar|, |squarepar|, and |heartpar|

The size is adjusted automatically so that the entire shape is filled with text. There may not be displayed maths or \verb+\vadjust+  material (no \verb+\vspace+) in the argument of shapepar. The macros work for both LaTeX and plain TeX. shapepar works in terms of user-defined shapes, though the package does provide some predefined shapes: so you can form any paragraph into the form of a heart by putting heartpar{sometext...} inside your document. The tedium of creating these polygon definitions may be alleviated by using the shapepatch extension to transfig which will convert xfig output to shapepar polygon form.
The author is Donald Arseneau.
\bigskip

\begin{small}
\renewcommand{\baselinestretch}{.85}\selectfont
\begin{minipage}[b]{\linewidth}
\footnotesize
\def~{\,}
\heartpar{%
\mbox{ }In faith, I do not love thee with mine eyes,
~For they in thee a thousand errors note,
~But is my heart that loves what they despise,
~Who in despite of view is pleased to dote;
~Nor are mine ears with thy tongue's tune delighted,
~Nor tender feeling to base touches prone,
~Nor taste nor smell, desire to be invited
~To any sensual feast with thee alone;
~But my five wits nor my five senses can
~Dissuade one foolish heart from serving thee,
~Who leaves unswayed the likeness of a man,
~Thy proud heart's slave and vassal wretch to be;
~Only my plague thus far I count my gain,
~That she that makes me sin awards me pain.}
~~~~
\end{minipage}
\end{small}

This is  a difficult to use package. If you do need to ise it carefully read the documentation.

\begin{teXXX}
\begin{small}
\renewcommand{\baselinestretch}{.85}\selectfont
\begin{minipage}[b]{\linewidth}
\footnotesize
\def~{\,}
\heartpar{%
\mbox{ }In faith, I do not love thee with mine eyes,
~For they in thee a thousand errors note,
~But is my heart that loves what they despise,
~Who in despite of view is pleased to dote;
~Nor are mine ears with thy tongue's tune delighted,
~Nor tender feeling to base touches prone,
~Nor taste nor smell, desire to be invited
~To any sensual feast with thee alone;
~But my five wits nor my five senses can
~Dissuade one foolish heart from serving thee,
~Who leaves unswayed the likeness of a man,
~Thy proud heart's slave and vassal wretch to be;
~Only my plague thus far I count my gain,
~That she that makes me sin awards me pain.}
~~~~
\end{minipage}
\end{small}

\end{teXXX}
