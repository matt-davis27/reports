\chapter{Pattern matching}

\tex's pattern matching is quite powerful\sidenote{Alan Jeffrey, howto pattern matching tugboat}, bu tis quite infuriating. It allows you to pattern-match against an arbitrary string, but when it fails, it can do nothing but produce an error message. For example if you say:

\begin{teX}
\def\foo(#1,#2){message{(#1 and #2)}}
\foo(a,b) \foo(c) \par
\end{teX}

\noindent Then you will get a message (a and b) but you will then get an error message telling you about a |Runaway argument|. It would be nice if you could use \tex's pattern-matching ability, but provide more than one pattern, and have \tex use the latest definition that matches. Alan Jeffrey wrote such a macro and we deascribe it here with minor modifications.

\begin{teX}
\howto\oneof(#2){#2}
\howto\oneof(){}
\howto\oneof(#2,#3){#2 \vee \oneof(#3)}
$$ \oneof(a,b,c,d)$$
\end{teX}

\noindent and get the result

$$\alpha\vee\beta\vee\gamma\vee\delta$$

This has been adapted very cleverly adapted by Michael Mehlich in the \cmd{fp} package for use with square brackets. This is shown below.


\begin{teX}
\NeedsTeXFormat{LaTeX2e}
\ProvidesPackage{defpattern}[1994/10/12]

\message{%
  `\string\defpattern-macro (adapted from 
   \string\howto\space (A. Jeffrey: TeX and TUG NEWS 2(2)))'%
}

%define a macro \defpattern for defining ML-like 
%pattern matching macros
%adapted from the \howto-macro from tugboat

\def\xcs#1#2{\expandafter#1\csname#2\endcsname}
\newcount\actioncount
\def\defpattern#1[#2]#3{%
  \xcs\ifx{\expandafter\gobble\string#1}\relax%
  \def#1{\erroraction#1}\fi%
  \advance\actioncount by 1\relax%
  \expandafter\defaction\expandafter{#1}{#3}{#2}%
  \def\temp##1##2##3##4##5##6##7##8##9{\def\temp{[#2]}}%
  \temp\empty\empty\empty\empty\empty\empty\empty\empty\empty%
  \edef#1{\noexpand\doaction\xcs\noexpand%
    {action-\the\actioncount}\temp}}
\def\defaction#1#2#3{%
  \xcs\def{action-\the\actioncount}##1[#3]{\applyto{#1}{#2}}}
\def\gobble#1{}
\def\applyto#1#2#3{#3{#1}{#2}}
\def\doaction#1[#2][#3]{#1[#3]\success[#2]\failure[#3]}
\def\success#1#2[#3]\failure[#4]{#2}
\def\failure#1#2{#1}
\def\erroraction#1[#2]{%
   \errmessage{I don't know how to \string#1[#2]}}
\end{teX}


\def\xcs#1#2{\expandafter#1\csname#2\endcsname}
\newcount\actioncount
\def\defpattern#1[#2]#3{%
  \xcs\ifx{\expandafter\gobble\string#1}\relax%
  \def#1{\erroraction#1}\fi%
  \advance\actioncount by 1\relax%
  \expandafter\defaction\expandafter{#1}{#3}{#2}%
  \def\temp##1##2##3##4##5##6##7##8##9{\def\temp{[#2]}}%
  \temp\empty\empty\empty\empty\empty\empty\empty\empty\empty%
  \edef#1{\noexpand\doaction\xcs\noexpand%
    {action-\the\actioncount}\temp}}
\def\defaction#1#2#3{%
  \xcs\def{action-\the\actioncount}##1[#3]{\applyto{#1}{#2}}}
\def\gobble#1{}
\def\applyto#1#2#3{#3{#1}{#2}}
\def\doaction#1[#2][#3]{#1[#3]\success[#2]\failure[#3]}
\def\success#1#2[#3]\failure[#4]{#2}
\def\failure#1#2{#1}
\def\erroraction#1[#2]{%
   \errmessage{I don't know how to \string#1[#2]}}


%\def\oneof{}

\noindent Trying it out, by defining a |\foo|:

\begin{teX}
\defpattern\foo[#2]{#2}
\defpattern\foo[]{}
\defpattern\foo[#2,#3]{#2 \vee \foo[#3]}


$$ \foo[a,b,c,d]$$

$$ \foo[a,b,d]$$
\end{teX}

\defpattern\Foo[#2]{#2}
\defpattern\Foo[]{
  \def\tempa#1{\texttt{This is just empty [#1]}}\tempa}
\defpattern\Foo[#2,#3]{#2 \vee \Foo[#3]}


This is the empty |\Foo[]|  \Foo[]  

This is |\[\Foo[\alpha]\]|  \[\Foo[\alpha]\]

This is |\[\Foo[\beta,\gamma]\]| \[\Foo[\beta,\gamma]\]




As Alan Jeffrey put it `A wonderful incomprehensible piece of programming, but hey --- it works!

The definition says that any argument patching the pattern $[x,y]$ should produce |$x\vee\oneof[y]$| , any argument matching the pattern [] should produce nothing, and any argument matching the pattern $[x]$ should produce $x$.

This is all done using \tex's pattern-matching ability (which explains the fact that the parameters for |\oneof| have to be numbered starting at \texttt{\#2}, since a patern may fail by having something before parameter \#2, which has to be matched by \texttt{\#]}.
























