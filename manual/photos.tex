\chapter{Photos and Pictures}
\begin{marginphoto}%
  \includegraphics[width=\linewidth]{./graphics/photo2}
  \caption{During the eraly days of typography fonts were designed to emulate the looks of calligraphic texts.}
  \label{fig:marginfig1}
\end{marginphoto}

\marginnote{

The history of footnotes is as long and complicated as the history of scholarship and commentary. Hebrew scholars more than two thousand years ago used systems of glossing and annotation to work on religious texts. 

Scribes in the Christian tradition in the medieval period made use of annotations in their manuscript copying practices: surrounding the original text with glosses in small letters. After the advent of printing, similar kinds of marginal annotation appeared in printed texts of the late fifteenth century. 

Humanist scholars producing printed editions of classical learning in the sixteenth century also made use of the resources of typography to display both the surviving classical text and their commentary on the same page. References to classical sources - and to modern printed editions - became more systematic, as did the expectation that such references would be consistent with scholarly practices. Scholars increasingly marked their professionalism by using complex citational conventions, which by the seventeenth century were so well established as to be the subject of parody and satire. Scriblerian satire of the early eighteenth century, whose purpose was to mock the pedantry and folly of the works of the learned, frequently included extensive parodies of footnotes and the scholarly contests they encoded. Nonetheless, during the eighteenth century, to appear authoritative and learned an author had to adopt the scholarly machinery of the reference citation.

The footnote was born out of a desire to rationalise the relation between text and citation. 

Robert Connors argues that marginal notations fell out of favour for two practical reasons: they left too much blank paper at the side of the text; and they were difficult for typographers to set. The same notes placed at the bottom of the page were more efficient, both in paper and time[1]. 

Anthony Grafton's The Footnote: A Curious History suggests the modern footnote, inaugurated by Pierre Bayle's Dictionaire Critique et Historique in 1697, signalled an epistemic revolution in historical scholarship, indicating the end of credulous scholasticism and the emergence of analytical historical methodologies. Both scholars note the considerable impact of historians such as David Hume and Edward Gibbon on the stylistic development of the discursive and citational footnote as a location for the display of gentlemanly ease as much as scholarly acumen. In the nineteenth century, German scholars such as Leopold von Ranke and Alexander von Humboldt established a systematic basis for the footnote citation, creating a methodical methodological approach that all competing scholars had to obey. In this way, the idea of the footnote was established, yet no there was no general agreement on the form these footnotes should adopt. A systematic approach to the form of the footnote was needed.}

In this section we will discuss how lines and paragraphs are turned into pages and how elements of pages such as footnotes, headers etc are inserted. As with the other chapters we will mix TeX basic commands with the more convenient \LaTeXe commnads. We will also look at some of the packages and classes that are availble to assist us with page layouts. 

\begin{minipage}{3in}
Besides illustrations that are inserted at the top of a page, plain TEX will also
insert footnotes at the bottom of a page. The ootnote macro is provided
for use within paragraphs;  for example, the footnote in the present sentence was typed
in the following way:


There are two parameters to a footnote[ first comes the reference mark, which will
appear both in the paragraph** and in the footnote itself, and then comes the text of
the footnote.45 The latter text may be several paragraphs long, and it may contain
\end{minipage}\footnote{Sidenote: ``Where God meant footnotes to go.'' ---Tufte}

\sidenote{

The history of footnotes is as long and complicated as the history of scholarship and commentary. Hebrew scholars more than two thousand years ago used systems of glossing and annotation to work on religious texts. Scribes in the Christian tradition in the medieval period made use of annotations in their manuscript copying practices: surrounding the original text with glosses in small letters. After the advent of printing, similar kinds of marginal annotation appeared in printed texts of the late fifteenth century. Humanist scholars producing printed editions of classical learning in the sixteenth century also made use of the resources of typography to display both the surviving classical text and their commentary on the same page. References to classical sources - and to modern printed editions - became more systematic, as did the expectation that such references would be consistent with scholarly practices. Scholars increasingly marked their professionalism by using complex citational conventions, which by the seventeenth century were so well established as to be the subject of parody and satire. Scriblerian satire of the early eighteenth century, whose purpose was to mock the pedantry and folly of the works of the learned, frequently included extensive parodies of footnotes and the scholarly contests they encoded. Nonetheless, during the eighteenth century, to appear authoritative and learned an author had to adopt the scholarly machinery of the reference citation.

The footnote was born out of a desire to rationalise the relation between text and citation. Robert Connors argues that marginal notations fell out of favour for two practical reasons: they left too much blank paper at the side of the text; and they were difficult for typographers to set. The same notes placed at the bottom of the page were more efficient, both in paper and time[1]. Anthony Grafton's The Footnote: A Curious History suggests the modern footnote, inaugurated by Pierre Bayle's Dictionaire Critique et Historique in 1697, signalled an epistemic revolution in historical scholarship, indicating the end of credulous scholasticism and the emergence of analytical historical methodologies. Both scholars note the considerable impact of historians such as David Hume and Edward Gibbon on the stylistic development of the discursive and citational footnote as a location for the display of gentlemanly ease as much as scholarly acumen. In the nineteenth century, German scholars such as Leopold von Ranke and Alexander von Humboldt established a systematic basis for the footnote citation, creating a methodical methodological approach that all competing scholars had to obey. In this way, the idea of the footnote was established, yet no there was no general agreement on the form these footnotes should adopt. A systematic approach to the form of the footnote was needed.}

Further reading:

Connors, Robert J., 'The Rhetoric of Citation Systems, Part I: The Development of Annotation Structures from the Renaissance to 1900', Rhetoric Review, 17 (1998), 6-48.

Connors, Robert J., 'The Rhetoric of Citation Systems, Part II: Competing Epistemic Values in Citation', Rhetoric Review, 17 (1999), 219-245.

Grafton, Anthony, The Footnote: A Curious History (London: Faber and Faber, 1997)

Grafton, Anthony, 'The Footnote from De Thou to Ranke', History and Theory, 33 (1994), 53-76

Zerby, Chuck, The Devil's Details: A History of Footnotes (Lancaster: Gazelle, 2002)

[1] Robert J. Connors, The Rhetoric of Citation Systems, Part I: The Development of Annotation Structures from the Renaissance to 1900, Rhetoric Review, 17 (1998), 6-48 (p. 30).
