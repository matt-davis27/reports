\chapter{Programming}

\normalsize

A not so amazing fact about \TeX\  is that it is Turing complete 
\footnote{Turing completeness, named after Alan Turing, is significant in that every real-world design for a computing device can be simulated by a universal Turing machine. The Church-Turing thesis states that this is a law of nature—that a universal Turing machine can, in principle, perform any calculation that any other programmable computer is capable of. Obviously, this says nothing about the effort required to write the program, or the time it may take for the machine to perform the calculation, or any abilities the machine may possess that have nothing to do with computation.}. 
Some people have done amazing things with it, like creating full programs, just using TeX. I am not surprized, hackers tend to want to stretch both the capabilities of their computers as well as their brains. After all their is a computer language called brainfuck. \footnote{Very difficult to typeset!}

Named after Alan Turing, in practice, Turing-completeness means that the rules followed in sequence on arbitrary data can produce the result of any calculation. A device with a Turing complete instruction set is the definition of a universal computer. To be Turing complete, it is enough to have conditional branching (an "if" and "goto" statement), and the ability to change memory.

To show that something is Turing complete, it is enough to show that it can be used to simulate the most primitive computer, since even the simplest computer can be used to simulate the most complicated one. All general purpose programming languages and modern machine instruction sets are Turing complete, up to relatively trivial finite-memory issues. Turing complete machines are defined as having unlimited amounts of memory, while machine instruction sets are usually designed only to work with a certain limited amount of RAM.

\LaTeX\  lets you write new commands (or adapt old ones), create or adapt environments, and write your own packages and even classes. This document will give detailed descriptions on how to do simple things, but won't cover all the issues relating to class creation.

\section{Defining New Environments}

You can define new environments using the \doccmd{newenvironment}. For example the following snippet wil define a list that automatically uses emphasized text.\doccmd{newenvironment}

\begin{Verbatim}
\newenvironment{emlist}{\begin{itemize} \em} {\end{itemize}}
\newenvironment{\ name}[num. arguments][ default 1st (optional) argument]
{ entry code, using #1, #2 etc to denote arguments} 
{ exit code - arguments can't be used}
\end{Verbatim}



Simple use isn't as painful as it looks - the following provides a variant of the itemize command, emphasising the items.

\section{Defining new environments}
\begin{verbatim}
\newenvironment{emlist}{ \begin{itemize} \em}{\end{itemize}}
 \begin{emlist}
    \item first comment
     \item second comment
 \end{emlist}
\end{verbatim}
   
This will produce

\newenvironment{emlist}{ \begin{itemize} \em}{\end{itemize}}
 \begin{emlist}
    \item first comment
     \item second comment
 \end{emlist}




The end of the environment ends the scope of the emphasis.

Let us define a new environment, for quotes, taht will start by said that the author has said something meaningful, like quoted an extract from the Hitch-hikers Guide to the Universe.

\newenvironment{quotationX}[2][George]% environment name
               {\par \ldots #1 \list{}{% small fontsize
                           \em     % emphasize text
                            \listparindent  \parindent
                           \itemindent    \listparindent%
                           \rightmargin   \leftmargin%
                           \setlength{\parsep}{2pt}
                          \setlength{\topsep}{10pt}
                         \ldots #2}%    \parsep \z@ \@plus\p@
                \item[] 
              }% begin code
               {\endlist } % exit code

%redefine

\begin{quotationX}{YL}{said}
\item Programming TeX is not difficult if you can think in backslashes. 
You need to memorize about 600 commnads and you wil be ok
\item Programming TeX is not difficult if you can think in backslashes. 
You need to memorize about 600 commnads and you wil be ok
\end{quotationX}





