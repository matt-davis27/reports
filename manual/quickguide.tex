

\chapter{Quick guide to LaTeX}         
\begin{figure}%
  \includegraphics[width=\linewidth]{./graphics/fig4-1}
  \caption{During the eraly days of typography fonts were designed to emulate the looks of calligraphic texts.}
  \label{fig:marginfig1}
\end{figure}
\section{Uisng LaTex}
This section is a quick guide to get you started with LaTeX. LaTeX was developed by Leslie Lamport and described in his book 
{\em LATEX: A Document Preparation System}\cite{Lamport1994}

\section{References}
References are placed alongside their citations as sidenotes,
as well.  This can be accomplished using the normal \doccmddef{cite}
command.\sidenote{The first paragraph of this document includes a citation.}

The complete list of references may also be printed automatically by using
the \doccmddef{bibliography} command.  (See the end of this document for an
example.)  If you do not want to print a bibliography at the end of your
document, use the \doccmddef{nobibliography} command in its place.  

To enter multiple citations at one location,\cite[-3\baselineskip]{Tufte2006,Tufte1990} you can
provide a list of keys separated by commas and the same optional vertical
offset argument: \verb+\cite{Tufte2006,Tufte1990}+.  
\begin{docspec}
  \doccmd{cite}[\docopt{offset}]\{\docarg{bibkey1,bibkey2,\ldots}\}
\end{docspec}