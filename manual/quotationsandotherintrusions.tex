\chapter{Quotations and Other Intrusions}

\newthought{For centuries quotations were not used in books}. In the earliest printed books, a quotation was marked merely by naming the speaker.

Three forms of quotation mark are still in common use. Inverted and raised commas -- ``quote'' and
`quote' -- are generally favoured in Britain and North America. But baseline and inverted commas --
are still widely used in Germany. Many typographers prefer them to take the shape of
sloped primes ('--") instead of tailed commas. 

\guillemotleft quote\guillemotright\ and <<quote>>. \sidenote{Use the \texttt{\textbackslash{guillemotleft}} and \texttt{\textbackslash{guillemotright}} commands. You can also use the \protect\index{csquote} csquote package}Guillemets or otherwise known as duck foot quotation marks, chevrons, or angle quotes - <<quote>> and <quote> - are the normal form in France and Italy and are widely used in the rest of Europe. German typographers set their guillemets the 
>>opposite way<<. In either case, thin spaces are customary between the guillemets and the text they enclose.

When quotation marks (including guillemets) are used, the question remains, how many should be there?
The usual British practice is to use single quotes first, and doubles with singles.


\texttt{`So does ``analphabetic'' mean what I think it means?' she said suspiciously.}

\begin{quotation}
And after shee sayde, Poliphilus lette vs goe and ascende vp this mount nexte the Garden, and Thelemia remayning at the stayre foote, wee ascended vp to the playne toppe. Where shee shewed vnto mee, with a heauenly eloquence, a Garden of a large compasse, made in the forme of an intricate Laborynth allyes and wayes, not to bee troden, but sayled about, for insteade of allyes to treade vppon, there were ryuers of water.
\end{quotation}

\section{The apostrophe}

The most common error in text is to use |'s| for plurals of numbers, or for multiple letters. This is unecessary, use the \emph{2010s} or the ABCs.

It is normal to avoid the period after metric units and other self-evident abbreviations. Set 11.3 m and 520 cm but 36 in. or 36", and in bibliographical references, p 36f, or pp 306-314. You can also use the \docpkg{siunitx} to give you a consistent set of units in scientific texts.

\section{Parentheses}

I used to introduce a lot of parentheses in my writings until I was shock by the AP Manual of Style:
{\emph The perceived need for parentheses is an indication that your sentence is becoming concorted}. If you do use a parentheses, follow these guidelines:\index{style!parentheses}\index{style!brackets}

\begin{itemize}
\item If the material is inside a sentence, place the period outside the parentheses.
\item If the whole sentence is within btackets, put the full stop inside. (Please remember this.)
\end{itemize}

According to Robert Bringhurst's \people{Robert Bringhurst}Elements of Typographic Style, the details of typesetting ellipsis depend on the character and size of the font being set and the typographer's preference. Bringhurst writes that a full space between each dot is "another Victorian eccentricity." In most contexts, the Chicago ellipsis is much too wide"—he recommends using flush dots, or thin-spaced dots (up to one-fifth of an em), or the prefabricated ellipsis character (Unicode U+2026, Latin entity \&hellip;). Bringhurst suggests that normally an ellipsis should be spaced fore-and-aft to separate it from the text, but when it combines with other punctuation, the leading space disappears and the other punctuation follows. He provides the following examples:
i … j	k….	l…, l	l, … l	m…?	n…!

[\dots]\lorem

$[\ldots]$\lorem

...\lorem

\begin{teX}
\mathchardef\ldotp="613A % ldot as a punctuation mark
\mathchardef\cdotp="6201 % cdot as a punctuation mark

\def\ldots{\mathinner{\ldotp\ldotp\ldotp}}
\def\cdots{\mathinner{\cdotp\cdotp\cdotp}}
\def\vdots{\vbox{\baselineskip4\p@ \lineskiplimit\z@
    \kern6\p@\hbox{.}\hbox{.}\hbox{.}}}
\def\ddots{\mathinner{\mkern1mu\raise7\p@\vbox{\kern7\p@\hbox{.}}\mkern2mu
    \raise4\p@\hbox{.}\mkern2mu\raise\p@\hbox{.}\mkern1mu}}
\def\dots{\relax\ifmmode\ldots\else$\m@th\ldots\,$\fi}
\end{teX}


This all makes for nice-looking output, but it unfortunately adds a bit
of a burden to your job as a typist, because TEX's rule for determining the end of
a sentence doesn't always work. The problem is that a period sometimes comes
in the middle of a sentence \dots like when it is used (as here) to make an ellipsis"
of three dots.

Moreover, if you try to specify `...' by typing three periods in a row,
you get `...' the dots are too close together. One way to handle this is to go
into mathematics mode, using the |\ldots| control sequence defined in plain TEX
format. For example, if you type

Hmmm |$\ldots$| I wonder why?

the result is `Hmmm $\ldots$ I wonder why?'. This works because math formulas are
exempt from the normal text spacing rules.


\begin{teXXX}
\mathchardef\ldotp="613A % ldot as a punctuation mark
\def\ldots{\mathinner{\ldotp\ldotp\ldotp}}
\end{teXXX}

\def\elide{\textup{[\phantom{,}\dots]\xspace}}


\emph{This is some text {[\thinspace\ldots\thinspace]} and this is some more\dots .}

\emph{This is some text {\smaller{[}\ldots]} and this is some more\dots .}

\emph{This is some text} {[\ldots]} \emph{and this is some more\dots .}

\emph{This is some text \elide and this is some more\dots .}


\def\elide{\relax\ifmmode\ldots\else $[\,\ldots\,]$\fi}

\emph{This is some text \elide and this is some more\dots .}

Elision is the omission of one or more sounds. If it can be applied as a typographical term, I am not sure. However, since an ellipsis represents dot-dot-dot an ellipsis within square brackets named an elision seems like a good idea.

Both |\dots| and |\ldots| would print the same. They are both defined the same way in \TeX .

\begin{teX}
 \mathchardef\ldotp="613A % ldot as a punctuation mark
 \def\ldots{\mathinner{\ldotp\ldotp\ldotp}}
 \def\dots{\relax\ifmmode\ldots\else$\m@th\ldots\,$\fi}
\end{teX}

The reason that Math mode is used is to ensure that the normal rules for spacing are not applied here.

An ellipsis within a square bracket looks ugly! It is an intrusion within the text. Bringhurst suggests that normally an ellipsis should be spaced fore-and-aft to separate it from the text, but when it combines with other punctuation, the leading space disappears and the other punctuation follows. So to define the command fully, one needs to take all aspects into consideration.

TH's suggestion is possibly the best you can get, I would just add xspace, in case someone sticks an elision at the end of the sentence or starts a new sentence \index{elision} after it.\sidenote{See \protect\url{http://tex.stackexchange.com/questions/3389/how-to-indicate-elision-in-a-quotation} for a discussion.}

\begin{teX}
 \newcommand*\elide{\textup{[\,\dots]\xspace}
\end{teX}

\begin{description}
  \item[slant] \the\fontdimen1\font
  \item[interword space] \the\fontdimen2\font
  \item[interword stretch] \the\fontdimen3\font
  \item[interword shrink] \the\fontdimen4\font
  \item[extra space] \the\fontdimen7\font
  \item[xspaceskip] \the\xspaceskip
  \item[hyphenchar] \the\hyphenchar\font
\end{description}

\section{Foreign Words and Romanization}

Foreign words and phrases used in an English text should be italicised (no
inverted commas) and should have the appropriate accents, e.g. inter alia,
raison d’être.

Exceptions: words and phrases now in common use and/or considered part of
the English language, e.g. role, ad hoc, per capita, per se, etc.

\begin{enumerate}
\item Personal names should retain their original accents, e.g. Grybauskait\.{e},
Potočnik, Wallström. Not to forget as Smith tells us to use the diaresis `where the dividing of two vowels makes two different vowels together may be taken for a dipthong, and make the verse fall short of its measure; as might have happened to the lines underneath, had no di\ae resis been used to prevent it; viz.

{\hskip3cm \narrower\narrower\it

 The Swans that in C\"ayster's water burn.\\
 In flames C\"aicus, Peneus, Alpheus, roll'd.\\
 The Tan\"ais smokes amid the boiling wave.\\

}

\item Quotations. Place verbatim quotations in foreign languages in quotation marks
without italicising the text.

\item Latin. Avoid obscure Latin phrases if writing for a broad readership. When
faced with such phrases as a translator, check whether they have the same
currency and meaning when used in English.

\item The expression ‘per diem’ (‘daily allowance’) and many others have English
equivalents, which should be preferred e.g. ‘a year’ or ‘per year’ rather than ‘per annum’.

In general Greek, Cyrillic, Chinese or Arabic scripts should be transliterated, except in specialist texts where the author is sure that his audience has knowledge of the language.

\end{enumerate}

The European Commission Directorate-General for Translation has an English Style Guide that deals in detail with foreign words and phrases in english text and romanization systems.



















