\chapter{Rules and Leaders}


You can use leaders to fill a space with copies of a pattern,
e.g., to put repeated dots between a title and a page number in a table
of contents. A leader is a single copy of the pattern. The specification of
leaders contains three pieces of information:

1) what a single leader is

2) how much space needs to be filled

3) how the copies of the pattern should be arranged within the space

TEX has three commands for specifying leaders: \cmd{leaders}, \cmd{cleaders},
and \cmd{xleaders} (p. 174). The argument of each command specifies the
leader. The commandmust be followed by glue; the size of the glue specifies
how much space is to be filled. The choice of command determines how
the leaders are arranged within the space.

Rule leaders \textit{fill} the specified amount of space with a rule extending in the direction of the skip
specified. \index{Rules and Leaders!rule leaders}

For instance,

\verb+\hbox{g\leaders\hrule\hskip20pt f}+

gives

\hbox{g\leaders\hrule\hskip20pt f}

because a horizontal rule has a default height of .4pt. On the other hand,\index{Rules and Leaders!default value}

\verb+\hbox{g\leaders\vrule\hskip10pt f}+

gives

\hbox{g\leaders\vrule\hskip10pt f}

because the height and depth of a vertical rule by default fill the surrounding box.
Spurious rule dimensions are ignored: in horizontal mode

\verb+\leaders\hrule width 10pt \hskip 20pt+

is equivalent to

\verb+\leaders\hrule \hskip 20pt+

If the width or height-plus-depth of either the skip or the box is negative, TEX uses ordinary glue
instead of leaders.

\section{Box leaders}

Box leaders fill the available spaces with copies of a given box, instead of with a rule.
For all of the following examples, assume that a box register has been allocated:

\verb+\newbox\centerdot \setbox\centerdot=\hbox{\hskip.7em.\hskip.7em}+

Now the output of

\verb+\hbox to 8cm {here\leaders\copy\centerdot\hfil there}+

is

\hbox to 8cm {here\leaders\copy\centerdot\hfil there}

That is, copies of the box register fill up the available space.

Dot leaders, as in the above example, are often used for tables of contents. In such applications it
is desirable that dots on subsequent lines are vertically aligned. The \cmd{leaders} command does this
automatically:

\begin{teX}
\hbox to 8cm {here\leaders\copy\centerdot\hfil there}
\hbox to 8cm {over here\leaders\copy\centerdot\hfil over there}
\end{teX}

gives

\hbox to 8cm {here\leaders\copy\centerdot\hfil there}
\hbox to 8cm {over here\leaders\copy\centerdot\hfil over there}


The mechanism behind this is the following: TEX acts as if an infinite row of boxes starts (invisibly)
at the left edge of the surrounding box, and the row of copies actually placed is merely the part of
this row that is not obscured by the other contents of the box.

Stated differently, box leaders are a window on an infinite row of boxes, and the row starts at the
left edge of the surrounding box. Consider the following example:

\begin{teX}
\hbox to 8cm {\leaders\copy\centerdot\hfil}
\hbox to 8cm {word\leaders\copy\centerdot\hfil}
\end{teX}

which gives,

\hbox to 8cm {\leaders\copy\centerdot\hfil}
\hbox to 8cm {word\leaders\copy\centerdot\hfil}

The row of leaders boxes becomes visible as soon as it does not coincide with other material.
The above discussion only talked about leaders in horizontal mode. Leaders can equally well be
placed in vertical mode; for box leaders the ‘infinite row’ then starts at the top of the surrounding
box.


\section{Leaders and shifted margins}

If margins have been shifted, leaders may look different depending on how the shift has been realized.
For an illustration of how \cmd{hangindent} and \cmd{leftskip} influence the look of leaders, consider
the following examples, where

\setbox0=\hbox{R a t a t a  }
\verb+\setbox0=\hbox{R a t a t a  }+

The horizontal boxes above the leaders serve to indicate the starting point of the row of leaders.
First




gives



Then

\begin{verbatim}
\hbox{\kern1em\hbox{\leaders\copy0\hskip5cm}}
\hangindent=1em \hangafter=-1 \noindent
\leaders\copy0\hskip5cm\hbox{}\par
\end{verbatim}

gives (note the shift with respect to the previous example)
\medskip

{\hbox{\kern1em\hbox{\leaders\copy0\hskip5cm}}
\hangindent=1em \hangafter=-1 \noindent
\leaders\copy0\hskip5cm\hbox{}\par}

In the first paragraph the \cmd{leftskip} glue only obscures the first leader box; in the second paragraph
the hanging indentation actually shifts the orientation point for the row of leaders. Hanging
indentation is performed in TEX by a \cmd{moveright} of the boxes containing the lines of the
paragraph.

\vfill\eject
\section{Creating two column text}


\begin{fullwidth}
\rule{10cm}{.1pt}
\medskip
\begin{figure}
\textsf\normalsize
\hbox{\hsize 10pc \parindent 1em
\vtop{In this example, we see how to use vboxes to
produce the effect of double columns. Each vbox
contains two paragraphs, typeset according to \TeX’s
usual rules except that it’s ragged right.\par
This isn’t really the best way to get true double
columns because the columns}
\hskip 2pc
\vtop{\noindent
aren’t balanced and we haven’t done anything to choose
the column break automatically or even to fix up the
last line of the first column.\par
However, the technique of putting running text into a
vbox is very useful for placing that text where you
want it on the page.}}
\caption{Justified text\hfill}
\end{figure}

\bigskip\bigskip
\rule{10cm}{.1pt}
\end{fullwidth}


This is now the normal text
\medskip


\begin{fullwidth}
\rule{10cm}{.1pt}
\medskip

\begin{figure}
\normalsize
\hbox{\hsize = 10pc \raggedright\parindent = 1em
\vtop{In this example, we see how to use vboxes to
produce the effect of double columns. Each vbox
contains two paragraphs, typeset according to \TeX’s
usual rules except that it’s ragged right.\par
This isn’t really the best way to get true double
columns because the columns}
\hskip 2pc
\vtop{\noindent 
aren’t balanced and we haven’t done anything to choose
the column break automatically or even to fix up the
last line of the first column.\par
However, the technique of putting running text into a
vbox is very useful for placing that text where you
want it on the page.}}
\caption{Raggedright text in rwo columns}
\end{figure}

\bigskip\bigskip
\rule{10cm}{.1pt}
\end{fullwidth}








