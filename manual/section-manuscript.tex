\documentclass{tufte-book}
%\usepackage[paperwidth=4.75in,paperheight=7.25in,
%            textwidth=4.2in,textheight=6.5in,
%            headsep=0.1in, footskip=0.15in,
%            marginratio=1:1]{geometry}
\usepackage{lipsum}
\usepackage{graphicx}
\begin{document}
\chapter{Introduction}
There is an unusual package named the \texttt{manuscript} that can emulate typewritten text. The author
\sidenote{ Mat\v{e}j Cepl}  describes its purpose:
\begin{quotation}
Purpose of this package is to emulate appearance of the
 document written on classical typewriter as much as possible.
 So far, whenever backward requirement of some institutions 
 (especially universities) to provide paper or thesis in the 
 layout developed in times before personal computers emerges on 
 any TeX-related public forum, it is met with strong (and 
 sometimes even angry) resistance and many advises how to 
 overcome resistance of the institution. I totally agree with 
 the basic premise of this attempt (of course, \TeX\ was created 
 for making ``masterpieces of typography'').
\end{quotation}

\begin{figure}[h]
\includegraphics[width=8cm]{../graphics/manuscript}
\end{figure}


What makes this very interesting is that this is not easily achieved in LaTeX, which goes out of its
way to provide typographically good looking results.

To use it, you simply include the package:

\begin{verbatim}
\documentclass{book}
\usepackage{manuscript}
\begin{document}
\chapter{Introduction}
\lipsum[1-5]
\end{document}
\end{verbatim}


The package code is about 65 lines long and is very concise and clear. It can give you a lot of pointers how to play
with fonts.
\end{document}