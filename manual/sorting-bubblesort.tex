%\documentclass[11pt]{book}
%\usepackage{lstdoc,booktabs,soul}
\makeatletter
% \begin{macro}{\lst@IfLE}
% \meta{string 1}|\relax\@empty|\meta{string 2}|\relax\@empty|\marg{then}\meta{else}.
% If \meta{string 1} $\leq$ \meta{string 2}, we execute \meta{then} and
% \meta{else} otherwise.
% Note that this comparision is case insensitive.
%    \begin{macrocode}
\def\lst@IfLE#1#2\@empty#3#4\@empty{%
    \ifx #1\relax
        \let\lst@next\@firstoftwo
    \else \ifx #3\relax
        \let\lst@next\@secondoftwo
    \else
        \lowercase{\ifx#1#3}%
            \def\lst@next{\lst@IfLE#2\@empty#4\@empty}%
        \else
            \lowercase{\ifnum`#1<`#3}\relax
                \let\lst@next\@firstoftwo
            \else
                \let\lst@next\@secondoftwo
            \fi
        \fi
    \fi \fi
    \lst@next}
%    \end{macrocode}
% \end{macro}



% \begin{macro}{\lst@BubbleSort}
% is in fact a derivation of bubble sort.
%    \begin{macrocode}
\def\lst@BubbleSort#1{%
    \ifx\@empty#1\else
        \lst@false
%    \end{macrocode}
% We `bubble sort' the first, second, \ldots\ elements and \ldots
%    \begin{macrocode}
        \expandafter\lst@BubbleSort@#1\relax,\relax,%
%    \end{macrocode}
% \ldots\space then the second, third, \ldots\ elements until no elemets have
% been swapped.
%    \begin{macrocode}
        \expandafter\lst@BubbleSort@\expandafter,\lst@sorted
                                      \relax,\relax,%
        \let#1\lst@sorted
        \lst@if
            \def\lst@next{\lst@BubbleSort#1}%
            \expandafter\expandafter\expandafter\lst@next
        \fi
    \fi}
\def\lst@BubbleSort@#1,#2,{%
    \ifx\@empty#1\@empty
        \def\lst@sorted{#2,}%
        \def\lst@next{\lst@BubbleSort@@}%
    \else
        \let\lst@sorted\@empty
        \def\lst@next{\lst@BubbleSort@@#1,#2,}%
    \fi
    \lst@next}
%    \end{macrocode}
% But the bubbles rise only one step per call. Putting the elements at their
% top most place would be inefficient (since \TeX\ had to read much more
% parameters in this case).
%    \begin{macrocode}
\def\lst@BubbleSort@@#1,#2,{%
    \ifx\relax#1\else
        \ifx\relax#2%
            \lst@lAddTo\lst@sorted{#1,}%
            \expandafter\expandafter\expandafter\lst@BubbleSort@@@
        \else
            \lst@IfLE #1\relax\@empty #2\relax\@empty
                          {\lst@lAddTo\lst@sorted{#1,#2,}}%
                {\lst@true \lst@lAddTo\lst@sorted{#2,#1,}}%
            \expandafter\expandafter\expandafter\lst@BubbleSort@@
        \fi
    \fi}
\def\lst@BubbleSort@@@#1\relax,{}
%    \end{macrocode}
%    \begin{macrocode}
%</doc>
%    \end{macrocode}
% \end{macro}
%
%
\makeatother

\chapter{Sorting}


\makeatletter

AAQ`	`

\end{document}


One of the difficulties of using \TeX\ as a computer language is its lack of built-in datastructures and common routines such as list manipulation and sorting. This chapter examines some simple sorting routines.


\section{Bubble sort}

Although not the most ideal of sorting algorithms, the bubble sort routine shown in the listing in figure, can be used successfully with comma delimited lists.

The principle behind extending alphabetical order to words (lexicographical order) is that all words in a list beginning with the same letter should be grouped together; within a grouping starting with a single letter, all words beginning with the same two letters shall be grouped together; and so on, maximizing the number of common initial letters between adjacent words. The ordering principle is applied at the point where the letters differ. For instance, in the sequence:

\begin{teXXX}
Astrolabe
Astronomy
Astrophysics
\end{teXXX}

\def\somelist{Astrophysics,Astronomy,Astrolable}
Defining a list as above but unsorted,

\noindent\texttt{> \somelist}

\lst@BubbleSort{\somelist}

\noindent\texttt{> \somelist}



\gdef\alist{San, San Cristobal, San Juan, Santa Clara, Santa Cruz, Santo Domingo, San Teodoro, San Tomas, Santa Barbara, San Antonio}

The original list is given by:

\alist


\lst@BubbleSort{\alist}

The sorted list is:

\noindent\texttt{> \alist}






define  string compare function

\def\strcomp#1#2{%
\def\templist{#1,#2,}
  \lst@BubbleSort{\templist}
\templist}


\strcomp{beta}{alpha}

\makeatother
The bubble sort macro shown in the following listing, has been adapted from the |listings| package, it is quite fast for most applications.

\begin{teX}
\def\lst@BubbleSort#1{%
    \ifx\@empty#1\else
        \lst@false
        \expandafter\lst@BubbleSort@#1\relax,\relax,%
        \expandafter\lst@BubbleSort@\expandafter,\lst@sorted
                                      \relax,\relax,%
        \let#1\lst@sorted
        \lst@if
            \def\lst@next{\lst@BubbleSort#1}%
            \expandafter\expandafter\expandafter\lst@next
        \fi
    \fi}
\def\lst@BubbleSort@#1,#2,{%
    \ifx\@empty#1\@empty
        \def\lst@sorted{#2,}%
        \def\lst@next{\lst@BubbleSort@@}%
    \else
        \let\lst@sorted\@empty
        \def\lst@next{\lst@BubbleSort@@#1,#2,}%
    \fi
    \lst@next}

\def\lst@BubbleSort@@#1,#2,{%
    \ifx\relax#1\else
        \ifx\relax#2%
            \lst@lAddTo\lst@sorted{#1,}%
            \expandafter\expandafter\expandafter\lst@BubbleSort@@@
        \else
            \lst@IfLE #1\relax\@empty #2\relax\@empty
                          {\lst@lAddTo\lst@sorted{#1,#2,}}%
                {\lst@true \lst@lAddTo\lst@sorted{#2,#1,}}%
            \expandafter\expandafter\expandafter\lst@BubbleSort@@
        \fi
    \fi}
\def\lst@BubbleSort@@@#1\relax,{}
\end{teX}

The interesting part of the code is the alphabetic comparison:

\begin{teXXX}
\def\lst@IfLE#1#2\@empty#3#4\@empty{%
    \ifx #1\relax
        \let\lst@next\@firstoftwo
    \else \ifx #3\relax
        \let\lst@next\@secondoftwo
    \else
        \lowercase{\ifx#1#3}%
            \def\lst@next{\lst@IfLE#2\@empty#4\@empty}%
        \else
            \lowercase{\ifnum`#1<`#3}\relax
                \let\lst@next\@firstoftwo
            \else
                \let\lst@next\@secondoftwo
            \fi
        \fi
    \fi \fi
    \lst@next}
\end{teXXX}



We can use LaTeX control loop @for to remove the commas from the list if these are not wanted in the final output.

\begin{teXXX}
\gdef\alist{alpha, beta, country, doll, feet}
\def\remove@commas#1{
\@for \i:=#1\do{%
\textit{\i}
}}
\remove@commas{\alist}
\end{teXXX}


\makeatletter

\gdef\alist{alpha, beta, country, doll, feet,}

\meaning\@for


\@for\value:=#1\do{\value}\relax
\makeatother
\end{document}



Adding commas can also be achieved, this time using the \verb+\@tfor+ looping structure.
\def\alist{a b c d e f {alpha} {beta}  {country}  {doll}  {feet}}

\def\add@commas#1{%
 \def\z##1{
    \expandafter\@tfor\expandafter\j\expandafter:\expandafter=##1\do{%
    \j , }
 }\z#1}
\add@commas{\alist}

\subsection{Remove last comma}

When using LaTeX lists it is necessary to remove the last comma from the list,

\begin{teX}
\def\alist{a,b,c,d,e,f,}%
    \def\remove@last@comma#1{%
        \def\remove@last@comma@##1##2,\@nil{##1##2}%
            \expandafter\remove@last@comma@\alist\@nil
    }%
   \remove@last@comma{\alist}
}
\end{teX}

{\tt > % 
\def\alist{a,b,c,d,e,f,}%
     \def\remove@last@comma#1{%
           \def\remove@last@comma@##1##2,\@nil{##1##2}%
                 \expandafter\remove@last@comma@\alist\@nil
     }%
    \remove@last@comma{\alist}
}


Many programmers would suggest one escapes to the shell and sort whatever needs to be sorted using another language, Lua also offers another possibility. In my opinion, since TeX is a computer language, with a bit of perseverance one can built all the datastructures and libraries required to do the job.



\end{document}
\section{Sorting tabular data}

Everytime the line is called a list is built the list has to have a structure based
on column.

\let\alist\@empty
 
\def\addtolist#1#2{
  \lst@lAddTo\alist{#2}
}

\def\test#1&#2&#3&#4 {%
   \addtolist{\alist}{#1,}%
   \expandafter\gdef\csname#1\endcsname{#1&#2&#3&#4\cr\relax}
}

%The last line needs a bit of different treatment
%Defined a new command just to make the development easier
\def\testa#1&#2&#3&#4 {%
      \addtolist{\alist}{#1,}%
     \expandafter\gdef\csname#1\endcsname{#1&#2&#3&#4}
}

\def\blue{blue}
\def\tnote#1{\footnotesize #1}

%% adding the data now
\test pipes & test &test &test 
\test elbows & two &three &four 
\test nuts and bolts & two &three &four 
\test athena & two &three &four 
\test Chapter Alpha & two &three &beta
\test Chapter Troy &other & mary &john 
\test Chapter 00 &0 & 0 &0 
\test SG 2 & \blue &0&\tnote{test}
\test SG 1 &0&0&0
\test SG 1a &0&0&0
\test SG 1b & \blue &0&\tnote{test}

Now let us sort the list (remember it only holds the name not the command with the same name).

\lst@BubbleSort{\alist}

\alist



%% typesetting the table
%\begin{table}[htbp]
\newsavebox{\tempbox}
\savebox{\tempbox}{
\centering
\begin{tabular}{lccc}
\toprule[1pt]
Column A & Column B & Column C & Column D\\
\midrule
\@for\i:=\alist \do{\csname\i\endcsname}
\vspace{-14pt}\\\bottomrule[1pt]
\end{tabular}
%\end{table}
\medskip
}
\medskip
\usebox{\tempbox}
\medskip

An interesting observation, now that we have sorted the table, is that once you can manipulate lists and tables, the basic structure of a database is available!\footnote{The DBtools offers a lot of database manipulation techniques, far more advanced taht what we have demonstrated here.}


\def\BeginRemark#1{%
    \begin{quote}\topsep0pt\let\small\footnotesize\small#1:}
\def\EndRemark{\end{quote}}


\makeatother


\newfont\yltt{ectt1000}

{\yltt this is a very good font
and this one.}

\so{\textbackslash test}

\textcaps{test}

\makeatletter
\tt
 \meaning\lst@lAddTo

\makeatother


\gdef\temp{another}

\xdef\Test{test \noexpand\temp }


\Test

\def\temp{kkk}

\Test



\makeatother
\end{document}
%
%\begin{clevertable}
%\sorton{2}
%\addrow 
%\addrow
%\addrow
%\end{clevertable}






