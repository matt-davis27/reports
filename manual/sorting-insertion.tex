\chapter{Insertion Sort}

The insertion sort can be used to sort lists. It is more efficient than the bubblesort described earlier. 

The \docpkg{xfor} provides a number of commands that can break a loop or peek at the next element of a list. Nicola L.C. Talbot \people{Talbot} also published in the documentation some macros for sorting numeric lists.

The macro below is based on these routines, but is using \doccmd{pdfstrcmp} to lexigographically compare two elements and use this comparison for sorting, using an insertion sort.

\begin{teXXX}
\makeatletter
%\let\temp\@empty
% \insertinto{new value}{list}
\newtoks\tmptok
\newcommand{\insertinto}[2]{%
   \def\nlst{}%
   \@for\n:=#2\do{%
% store new list in \tmptok
   \expandafter\tmptok\expandafter{\nlst}%
% test current value against new value
% use pdfstrcmp for alpha sort
   \def\temp{\pdfstrcmp{\n}{#1}}
   \ifnum\temp>0\relax
     \edef\newstuff{#1,\n}%
     % end for loop at the end of this iteration
     \@endfortrue
   \else
     \edef\newstuff{\n}%
   \fi
% append new stuff to new list
   \ifx\nlst\@empty
     \edef\nlst{\newstuff}%
   \else
     \edef\nlst{\the\tmptok,\newstuff}%
   \fi
   }%
% check to see if for loop was prematurely terminated
   \if@endfor
% loop may have been terminated during final iteration, in
% which case \@forremainder is empty.
   \ifx\@forremainder\@empty
% do nothing
   \else
% loop prematurely ended, append remainder of original list
% to new list
     \expandafter\tmptok\expandafter{\nlst}%
     \edef\nlst{\the\tmptok,\@forremainder}%
   \fi
   \else
% wasn't prematurely terminated, so new value hasn't been added.
% Add now.
   \expandafter\tmptok\expandafter{\nlst}%
      \ifx\nlst\@empty
         \edef\nlst{#1}%
      \fi
   \fi
   \global\let#2=\nlst
}
\end{teXXX}


A simpler approach than the bubble sort is to use pdfstrcmp for alpha sorting. The command returns a value -1,0,1 depending on the comparison. We can also use |xfor| to simplify things.

Based on this we can develop an insertion sort.

\begin{teXXX}
\def\ships{z}
\def\addship#1#2{
  \insertinto{#1}{#2}
}

\def\Armenian{A very big ship}
\def\Montrose{The story of Montrose}
\def\Harlech{The story of Harlech Castle}
\def\Manila{Manilla Castle}
\def\asum{$a^2+1=22$}
\def\Asum{$A^2+1=22$}

\expandafter\def\csname Roslin Castle\endcsname{The SS Roslin Castle left Cape Town on the 29th July 1902.}


\addship{Armenian}{\ships}
\addship{Montrose}{\ships}
\addship{Harlech}{\ships}
\addship{Manila}{\ships}
\addship{Roslin Castle}{\ships}
\addship{asum}{\ships}
\addship{Asum}{\ships}



\@for \i:=\ships\do{%
 \xdef\X{\i} %
 \def\Y{z}%
   \i:\csname\i\endcsname
 }



\typesetlist\ships


\makeatother
\end{teXXX}


























