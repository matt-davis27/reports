\epigraph{the Athenian statesman has to be Pericles, but a modern
Greek with the same name should be transliterated as Periklis.}{European Union Style Guide}

\chapter{Tokenizers}



Tokenizers are normally understood to be routines that split text into different
tokens for example a list separated by commas. We will look at two types, one that tokenizes a comma delimited list and another one that tokenizes letters. The first one is based on a package by Sacha Herpers named \docpkg{tokenizer}

Before we delve deep into the code, it might be a good idea to review \emph{delimited} macros, \ie macros that read an argument up to a delimiter such as |#1,|. The argument of such a macro will read the first argument until it encounters a comma. Note that a |space| in an argument of a macro is significant and acts as a delimiter.

\begin{teX}
\def\GetDelimitedTokens#1,#2{
  \say{Arguments #1 and #2}
}
\end{teX}

\def\say#1{%
 \hbox{\texttt{#1}}
}

\def\GetDelimitedTokens#1 #2 {
  \say{Arguments #1 and #2}
}



\noindent This produces \GetDelimitedTokens a bcf g 3456   

The |say| macro is just a convenience macro that we will use to print the information rather than use |show| and read it on the screen and is defined as follows:

\begin{teX}
\def\say#1{%
  \hbox{\texttt{#1}}
}
\end{teX}



The ability of \tex to read delimited macros allows us to parse virtually anything we wish. Now let us look at the tokenizer package a bit more carefully.


\begin{teX}
\ProvidesPackage{tokenizer}[2003/05/26]
\RequirePackage{ifthen}
\def\SH@true{true}                       % used with trim/notrim
\def\SH@false{false}                     % used with trim/notrim
\DeclareOption{trim}{\edef\SH@trimtokens{\SH@true}}     % trim tokens
\DeclareOption{notrim}{\edef\SH@trimtokens{\SH@false}}  % do not trim tokens
\ProcessOptions
\end{teX}

\noindent We first define a \cmd{GetTokens} command
|\GetTokens| |<Token1>,<Token2>|. 
This will save |<Token1>| and |<Token2>| in two commands. The names of the commands are derived from the value of |\SH@FirstArgName| and |\SH@SecondArgName|.

\begin{teX}
\def\GetTokens#1,#2\@empty{%
   \def\FirstArgument{#1}
   \DefineCommand{\FirstArgName}{#1}
   \DefineCommand{\SecondArgName}{#2}
}
\end{teX}


One other convenient command we can use is to create a command to create a macro on the fly without typing |\csname \endcsname| etc.

\begin{teX}
\def\DefineCommand#1#2{%
  \expandafter\def\csname #1\endcsname{#2}%
}
\end{teX}

For example we can use:

\begin{teX}
\DefineCommand{Cheesei}{blue-cheese}
\say{\Cheesei}
\end{teX}


\def\DefineCommand#1#2{%
  \expandafter\def\csname #1\endcsname{#2}%
}
\DefineCommand{Cheesei}{blue-cheese}
\say{\Cheesei}

Another question that may come into your mind is |@empty|. This is a plain \tex and a \latex kernel macro which is simply defined as an empty argument.
If the user specifies a balanced text in braces when \tex expects a macro argument, that text is
used as the argument. Thus, specifying |{}| will give an argument that is an empty list of tokens;
this is called an \textit{empty argument}. \sidenote{See Chapter 11, section 11.5.4, Tex By Topic}. The definition is simply

\begin{teX}
\def\@empty{}
\end{teX}


\noindent Next we need to split the arguments if the separarator exists. This we do with the macro |\CheckTokenSep<Token1>,<Token2>|. It 
Checks the arguments for the existence of a token separator.

This Command is called with a trailing comma!
If a comma was found |\SH@TokenValid| is set to true, false otherwise.


Example 1: |<t1>,<t2>,|

This is split into |<t1>| and |<t2>,| and because the second token is
not empty |\SH@TokenValid| is set to true.

Example 2: |<t1>,|
This is split into |<t1>| and an empty second token and thus evaluates
to false.
If you the trailing comma is omitted when calling |\SH@CheckTokenSep| example 2 will fail to compile.


\begin{teX}
\def\CheckTokenSep#1,#2\@empty{%
   \def\SH@CTSArgTwo{#2}%
   \ifx\SH@CTSArgTwo\@empty
      \edef\SH@TokenValid{\SH@false}
   \else
      \edef\SH@TokenValid{\SH@true}
   \fi
}
\end{teX}


\section*{The \textbackslash GetTokens}
This is the heart of the parser and the command exposed to the user. The macro splits the argument into two segments.
The first argument, and the last token which is the balance of the argument. 

\begin{teX}
%
% \GetTokens{<Target1>}{<Target2>}{<Source>}
%     Splits <Source> into two tokens at the first encounter of a comma.
%     The first token is saved in a newly created command with the name passed
%     as <Target1> and the second token likewise.
%------------------------------------------------------------------------------
\newcommand\GetTokens[3]{%
   % save the names of the commands to receive the separated tokens
   \def\SH@FirstArgName{#1}%
   \def\SH@SecondArgName{#2}%
   % check for the token separator
   \expandafter\SH@CheckTokenSep#3,\@empty%
   \ifx\SH@TokenValid\SH@true
    % separator found, split the source string
       \expandafter\SH@GetTokens#3\@empty%
   \else
     % there was no separator, return the complete source string as
     % first token
     \def\SH@token{#3}
        \ifx\SH@trimtokens\SH@true   % strip spaces if requested
           \expandafter\TrimSpaces\SH@token
        \fi
        \SH@DefineCommand{\SH@FirstArgName}{\SH@token}
          \SH@DefineCommand{\SH@SecondArgName}{}
        \fi
}
\end{teX}



\section*{Trimming Spaces}
One other useful macro is \cmd{TrimSpaces}, which removes leading nad trailing spaces from a token. This is based on an example from the  \texbook.

This command removes leading and trailing spaces from |<Token>|. The code
is Michael Downes, who provided the code as an answer to 'around the bend' question \#15.

\begin{teX}
\catcode`\Q=3
\def\TrimSpaces#1{%
  \begingroup
  \aftergroup\toks\aftergroup0\aftergroup{%
  \expandafter\trimb\expandafter\noexpand#1Q Q}%
  \global\edef#1{\the\toks0}%
}
\def\trimb#1 Q{\trimc#1Q}
\def\trimc#1Q#2{\afterassignment\endgroup \vfuzz\the\vfuzz#1}
\catcode`\Q=11

\end{teX}


\section{Example}
The example shown in listing 1 illustrates the use of
\verb+\GetTokens+. Firstly, a source string \verb+\Source+ is
created, which contains the strings to be separated. The following
while statement loops until there are no more tokens to process.
\verb+\GetTokens+ is called and the separated tokens are stored in
two commands \verb+\TokenOne+ and \verb+\TokenTwo+, which are
created by \verb+\GetTokens+. Lastly, \verb+\Source+ is replaced
by the remainder string contained in \verb+\TokenTwo+.\\


\hrule
\marginpar{\small Listing 1: example usage of \textbackslash GetTokens}

\begin{teX}
 \def\Source{this,is, a, short, test}
 The string \emph{\Source} contains the following tokens:\\
 \whiledo{\not\equal{\Source}{}}
 {
     \GetTokens{TokenOne}{TokenTwo}{\Source}
     \hspace*{.3cm}$\bullet$ [\TokenOne]\\
     \let\Source\TokenTwo
 }
\end{teX}
\hrule

\bigskip


This is the output produced by the above example:\\
\begin{center}
   \begin{minipage}{9cm}
        \newcounter{tempcounter} 
        \setcounter{tempcounter}{0} 
        \def\numcounter{\stepcounter{tempcounter}\thetempcounter} 
        \def\Source{ file, file, file, file, file}
        The string \emph{\Source} contains the following tokens:\\
        \whiledo{\not\equal{\Source}{}}
        {
            \GetTokens{TokenOne}{TokenTwo}{\Source}
            \hspace*{.3cm}--\textit{\TokenOne{\it \numcounter}}\\
            \let\Source\TokenTwo
        }
   \end{minipage}
\end{center}

\section{Tokenizing an alphabet}
This is a much easier construction, as we just need to iterate over letters one by one

\begin{teX}
\def\acclist#1#2{\def\theacc{#1}\let\next\oneacc\next#2\relax}
\def\Relax{\relax}
\def\oneacc#1{%
  \ifx#1\relax\let\next\relax\else\theacc#1 \fi
\next}

\acclist\`{AEIOUNWYaeiounwyx}\par 



\def\mylist{abcdefgh}
\def\reverse #1%
  {\edef #1{\expandafter \reverseloop #1\relax \marker }}
\def\reverseloop #1#2\marker
  {\ifx#1\relax\reverseend\fi \reverseloop #2\marker #1}
\def\reverseend #1\marker #2{}
\reverse\mylist

\mylist
\end{teX}


\noindent This will produce,

\fboxrule0.4pt \fboxsep1pt

\def\acclist#1#2{\def\theletter{#1}\let\next\oneacc\next#2\relax}
\def\Relax{\relax}
\def\oneacc#1{%
  \ifx#1\relax\let\next\relax\else\fbox{\strut \theletter#1} \fi
\next}

\Large

\noindent\rule{10cm}{0.4pt}
\smallskip

\acclist\`{AEIOUNWYaeiounwyx}\par 

\smallskip
\noindent\rule{10cm}{0.4pt}


\normalsize 



