\newpage
\section{The wrapfig package}

\captionsetup[wrapfigure]{margin=10pt,font=small,labelfont=bf, name=Fig.} % [wrapfigure]{name=Fig.}

\begin{wrapfigure}{r}{4.3cm}
   \includegraphics[width=\linewidth]{./graphics/lotus}
   \caption{\small Cyprian limestone group of Phoenician dancers, about 6½ in. high. There is a somewhat similar group, also from Cyprus, in the British Museum.}
\end{wrapfigure}

Amongst the earliest representations that are comprehensible, we have certain Egyptian paintings, and some of these exhibit postures that evidently had even then a settled meaning, and were a phrase in the sentences of the art. Not only were they settled at such an early period (B.C. 3000, fig. 1) but they appear to have been accepted and handed down to succeeding generations (fig. 2), and what is remarkable in some countries, even to our own times. The accompanying illustrations from Egypt and Greece exhibit what was evidently a traditional attitude. The hand-in-hand dance is another of these.

\begin{wrapfigure}[20]{l}{3.0cm}
\centering\small
\includegraphics[width=\linewidth]{./graphics/egyptdance}  %fig198
\caption{\small The hieroglyphics describe the dance.}
\end{wrapfigure}


The following account of Egyptian dancing is from Sir Gardiner Wilkinson's "Ancient Egypt"[2]:—

The dance consisted mostly of a succession of figures, in which the performers endeavoured to exhibit a great variety of gesture. Men and women danced at the same time, or in separate parties, but the latter were generally preferred for their superior grace and elegance.

Some danced to slow airs, adapted to the style of their movement; the attitudes they assumed frequently partook of a grace not unworthy of the Greeks; and some credit is due to the skill of the artist who represented the subject, which excites additional interest from its being in one of the oldest tombs of Thebes (B.C. 1450, Amenophis II.).

Others preferred a lively step, regulated by an appropriate tune; and men sometimes danced with great spirit, bounding from the ground, more in the manner of Europeans than of Eastern people. On these occasions the music was not always composed of many instruments, and here we find only the cylindrical maces and a woman snapping her fingers in the time, in lieu of cymbals or castanets.

%\begin{marginfigure}
%   \includegraphics[width=\linewidth]{./graphics/patera}
%   \caption{\small Cyprian limestone group of Phoenician dancers, about 6½ in. high. There is a somewhat similar group, also from Cyprus, in the British Museum. The dress, a hooded cowl, appears to be of great antiquity.}
%\end{marginfigure}

"Graceful attitudes and gesticulations were the general style of their dance, but, as in all other countries, the taste of the performance varied according to the rank of the person by whom they were employed, or their own skill, and the dance at the house of a priest differed from that among the uncouth peasantry, etc.
