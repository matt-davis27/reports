\chapter{XeTeX}

 \captionsetup[figure]{margin=10pt,font=small,labelfont=bf}  

\begin{marginfigure}%
  \includegraphics[width=1.1\linewidth]{./graphics/visigothic}
  \captionsetup[figure]{margin=10pt,font=small,labelfont=bf}  
  \caption{Origin of the cedilla from the Visigothic z. }
  \label{fig:visigothic}
\end{marginfigure}

\section{The Problem with Fonts}

If you have read the section on fonts, you will agree that to use
the full spectrum of UTF commands with LaTeX is a daunting task (but not impossible).

To do this one needs to map all the unicode files to suitable font tables in multiples
of 256 character codes. It will also remain a problem using the upper and lowercase commands
as only two tables are provided.

But for the average user of a modern  computer
and typical GUI software, using a new font in a document
involves approximately two steps:


\begin{enumerate}

\item Drop the |.ttf| or |.otf| file into the computer’s Fonts
folder;

\item Select the font name from a menu in any application.
\end{enumerate}

Any software—especially software that relates to typography—
that requires a longer or more complex procedure
will be perceived as ``user-unfriendly'' and ``hard to use'',
and will face a barrier to wide acceptance. 

\xetex aims to bring this level of simplicity to the use
of fonts with \tex. While selecting a font from a menu of
installed fonts does not directly fit the \tex paradigm, the
use of a new font is similarly straightforward:

\begin{enumerate}
\item Drop the |.ttf| or |.otf| file into the computer’s Fonts
folder;
\item Specify the font by name in the \tex document.
In Plain \tex terms, this second step might be as simple as:
\end{enumerate}

\begin{teX}
\font\myfont="Charis SIL"  at 9pt
\myfont Hello World
\end{teX}