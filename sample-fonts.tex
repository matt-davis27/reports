\documentclass{article}
\usepackage{lipsum,url}
\usepackage{textcomp}
\usepackage{amsfonts,amsmath}
\usepackage{ifpdf}
\ifpdf
   \pdfmapfile{+mathgifg.map}
\fi

\newcounter{lipsumnum}
\setcounter{lipsumnum}{1}

\newcommand{\samplefont}[2]{{#1\selectfont #2: 
0123456789, \$20, \texteuro30, \pounds60.
Na\"ive \AE sop's \OE uvres in fran\c cais were my first reading.
\lipsum[\value{lipsumnum}]\stepcounter{lipsumnum}\par}}

\DeclareMathSymbol{\dit}{\mathord}{letters}{`d}
\DeclareMathSymbol{\dup}{\mathord}{operators}{`d}


\def\test#1{#1}

\def\testnums{%
  \test 0 \test 1 \test 2 \test 3 \test 4 \test 5 \test 6 \test 7
  \test 8 \test 9 }


\def\testupperi{%
  \test A \test B \test C \test D \test E \test F \test G \test H
  \test I \test J \test K \test L \test M }

\def\testupperii{%
  \test N \test O \test P \test Q \test R \test S \test T \test U
  \test V \test W \test X \test Y \test Z }
\def\testupper{%
  \testupperi\testupperii}

\def\testloweri{%
  \test a \test b \test c \test d \test e \test f \test g \test h
  \test i \test j \test k \test l \test m }
\def\testlowerii{%
  \test n \test o \test p \test q \test r \test s \test t \test u
  \test v \test w \test x \test y \test z 
  \test\imath \test\jmath }
\def\testlower{%
  \testloweri\testlowerii}

\def\testupgreeki{%
  \test A \test B \test\Gamma \test\Delta \test E \test Z \test H
  \test\Theta \test I \test K \test\Lambda \test M }
\def\testupgreekii{%
  \test N \test\Xi \test O \test\Pi \test P \test\Sigma \test T
  \test\Upsilon \test\Phi \test X \test\Psi \test\Omega 
  \test\nabla }
\def\testupgreek{%
  \testupgreeki\testupgreekii}

\def\testlowgreeki{%
  \test\alpha \test\beta \test\gamma \test\delta \test\epsilon
  \test\zeta \test\eta \test\theta \test\iota \test\kappa \test\lambda
  \test\mu }
\def\testlowgreekii{%
  \test\nu \test\xi \test o \test\pi \test\rho \test\sigma \test\tau
  \test\upsilon \test\phi \test\chi \test\psi \test\omega }
\def\testlowgreekiii{%
  \test\varepsilon \test\vartheta \test\varpi \test\varrho
  \test\varsigma \test\varphi}
\def\testlowgreek{%
  \testlowgreeki\testlowgreekii\testlowgreekiii}




\begin{document}

\section{Text Tests}
\label{sec:text}



\samplefont{\normalfont}{Georgia}
\samplefont{\itshape}{Georgia Italic}
\samplefont{\bfseries}{Georgia Bold}
\samplefont{\bfseries\itshape}{Georgia Bold Italic}
\samplefont{\sffamily\fontseries{k}}{Franklin Gothic Book}
\samplefont{\sffamily\fontseries{k}\itshape}{Franklin Gothic Book Italic}
\samplefont{\sffamily}{Franklin Gothic Medium}
\samplefont{\sffamily\itshape}{Franklin Gothic Medum Italic}
\samplefont{\sffamily\fontseries{mc}}{Franklin Gothic Medium Condensed}
\samplefont{\sffamily\bfseries}{Franklin Gothic Demibold}
\samplefont{\sffamily\bfseries\itshape}{Franklin Gothic Demibold
  Italic}
\samplefont{\sffamily\fontseries{dc}}{Franklin Gothic Demibold Condensed}
\samplefont{\sffamily\fontseries{h}}{Franklin Gothic Heavy}
\samplefont{\sffamily\fontseries{h}\itshape}{Franklin Gothic Heavy Italic}

\section{Math Tests}
\label{sec:mthtests}



Math test are taken from\cite{Schmidt04:PSNFSS9.2}.  Note that we do
not have \texttt{\string\jmath}, so we took one from CM.  

\parindent 0pt
%\mathindent 1em


\subsection{Math Alphabets}

Math Italic (\texttt{\string\mathnormal})
\def\test#1{\mathnormal{#1},}
\begin{eqnarray*}
  && {\testnums}\\
  && {\testupper}\\
  && {\testlower}\\ 
  && {\testupgreek}\\
  && {\testlowgreek}
\end{eqnarray*}%

Math Roman (\texttt{\string\mathrm})
\def\test#1{\mathrm{#1},}
\begin{eqnarray*}
  && {\testnums}\\
  && {\testupper}\\
  && {\testlower}\\ 
  && {\testupgreek}\\
  && {\testlowgreek}
\end{eqnarray*}%

%Math Italic Bold
%\def\test#1{\mathbm{#1},}
%\begin{eqnarray*}
%  && {\testnums}\\
%  && {\testupper}\\
%  && {\testlower}\\ 
%  && {\testupgreek}\\
%  && {\testlowgreek}
%\end{eqnarray*}%

Math Bold (\texttt{\string\mathbf})
\def\test#1{\mathbf{#1},}
\begin{eqnarray*}
  && {\testnums}\\
  && {\testupper}\\
  && {\testlower}\\ 
  && {\testupgreek}
\end{eqnarray*}%

Math Sans Serif (\texttt{\string\mathsf})
\def\test#1{\mathsf{#1},}
\begin{eqnarray*}
  && {\testnums}\\
  && {\testupper}\\
  && {\testlower}\\ 
  && {\testupgreek}
\end{eqnarray*}%



%Caligraphic (\texttt{\string\mathcal})
%\def\test#1{\mathcal{#1},}
%\begin{eqnarray*}
%  && {\testupper}
%\end{eqnarray*}%

%Script (\texttt{\string\mathscr})
%\def\test#1{\mathscr{#1},}
%\begin{eqnarray*}
%  && {\testupper}
%\end{eqnarray*}%

%Fraktur (\texttt{\string\mathfrak})
%\def\test#1{\mathfrak{#1},}
%\begin{eqnarray*}
%  && {\testupper}\\
%  && {\testlower}
%\end{eqnarray*}%

%Blackboard Bold (\texttt{\string\mathbb})
%\def\test#1{\mathbb{#1},}
%\begin{eqnarray*}
%  && {\testupper}
%\end{eqnarray*}%


\clearpage
\subsection{Character Sidebearings}

\def\test#1{|#1|+}
\begin{eqnarray*}
  && {\testupperi}\\
  && {\testupperii}\\
  && {\testloweri}\\ 
  && {\testlowerii}\\ 
  && {\testupgreeki}\\
  && {\testupgreekii}\\
  && {\testlowgreeki}\\
  && {\testlowgreekii}\\
  && {\testlowgreekiii}
\end{eqnarray*}%
%
\def\test#1{|\mathrm{#1}|+}%
\begin{eqnarray*}
  && {\testupperi}\\
  && {\testupperii}\\
  && {\testloweri}\\ 
  && {\testlowerii}\\ 
  && {\testupgreeki}\\
  && {\testupgreekii}
\end{eqnarray*}%
%
%\def\test#1{|\mathbm{#1}|+}%
%\begin{eqnarray*}
%  && {\testupperi}\\
%  && {\testupperii}\\
%  && {\testloweri}\\ 
%  && {\testlowerii}\\ 
%  && {\testupgreeki}\\
%  && {\testupgreekii}\\
%  && {\testlowgreeki}\\
%  && {\testlowgreekii}\\
%  && {\testlowgreekiii}
%\end{eqnarray*}%
%%
%\def\test#1{|\mathbf{#1}|+}%
%\begin{eqnarray*}
%  && {\testupperi}\\
%  && {\testupperii}\\
%  && {\testloweri}\\ 
%  && {\testlowerii}\\ 
%  && {\testupgreeki}\\
%  && {\testupgreekii}
%\end{eqnarray*}%
%
\def\test#1{|\mathcal{#1}|+}%
\begin{eqnarray*}
  && {\testupperi}\\
  && {\testupperii}
\end{eqnarray*}%


\clearpage
\subsection{Superscript positioning}

\def\test#1{#1^{2}+}%
\begin{eqnarray*}
  && {\testupperi}\\
  && {\testupperii}\\
  && {\testloweri}\\ 
  && {\testlowerii}\\ 
  && {\testupgreeki}\\
  && {\testupgreekii}\\
  && {\testlowgreeki}\\
  && {\testlowgreekii}\\
  && {\testlowgreekiii}
\end{eqnarray*}%
%
\def\test#1{\mathrm{#1}^{2}+}%
\begin{eqnarray*}
  && {\testupperi}\\
  && {\testupperii}\\
  && {\testloweri}\\ 
  && {\testlowerii}\\ 
  && {\testupgreeki}\\
  && {\testupgreekii}
\end{eqnarray*}%
%
%\def\test#1{\mathbm{#1}^{2}+}%
%\begin{eqnarray*}
%  && {\testupperi}\\
%  && {\testupperii}\\
%  && {\testloweri}\\ 
%  && {\testlowerii}\\ 
%  && {\testupgreeki}\\
%  && {\testupgreekii}\\
%  && {\testlowgreeki}\\
%  && {\testlowgreekii}\\
%  && {\testlowgreekiii}
%\end{eqnarray*}%
%
%\def\test#1{\mathbf{#1}^{2}+}%
%\begin{eqnarray*}
%  && {\testupperi}\\
%  && {\testupperii}\\
%  && {\testloweri}\\ 
%  && {\testlowerii}\\ 
%  && {\testupgreeki}\\
%  && {\testupgreekii}
%\end{eqnarray*}
%
\def\test#1{\mathcal{#1}^{2}+}%
\begin{eqnarray*}
  && {\testupperi}\\
  && {\testupperii}
\end{eqnarray*}%


\clearpage
\subsection{Subscript positioning}

\def\test#1{\mathnormal{#1}_{i}+}%
\begin{eqnarray*}
  && {\testupperi}\\
  && {\testupperii}\\
  && {\testloweri}\\ 
  && {\testlowerii}\\ 
  && {\testupgreeki}\\
  && {\testupgreekii}\\
  && {\testlowgreeki}\\
  && {\testlowgreekii}\\
  && {\testlowgreekiii}
\end{eqnarray*}%
%
\def\test#1{\mathrm{#1}_{i}+}%
\begin{eqnarray*}
  && {\testupperi}\\
  && {\testupperii}\\
  && {\testloweri}\\ 
  && {\testlowerii}\\ 
  && {\testupgreeki}\\
  && {\testupgreekii}
\end{eqnarray*}%
%
%\def\test#1{\mathbm{#1}_{i}+}%
%\begin{eqnarray*}
%  && {\testupperi}\\
%  && {\testupperii}\\
%  && {\testloweri}\\ 
%  && {\testlowerii}\\ 
%  && {\testupgreeki}\\
%  && {\testupgreekii}\\
%  && {\testlowgreeki}\\
%  && {\testlowgreekii}\\
%  && {\testlowgreekiii}
%\end{eqnarray*}
%%
%\def\test#1{\mathbf{#1}_{i}+}%
%\begin{eqnarray*}
%  && {\testupperi}\\
%  && {\testupperii}\\
%  && {\testloweri}\\ 
%  && {\testlowerii}\\ 
%  && {\testupgreeki}\\
%  && {\testupgreekii}
%\end{eqnarray*}%
%
\def\test#1{\mathcal{#1}_{i}+}%
\begin{eqnarray*}
  && {\testupperi}\\
  && {\testupperii}
\end{eqnarray*}%


\clearpage
\subsection{Accent positioning}

\def\test#1{\hat{#1}+}%
\begin{eqnarray*}
  && {\testupperi}\\
  && {\testupperii}\\
  && {\testloweri}\\ 
  && {\testlowerii}\\ 
  && {\testupgreeki}\\
  && {\testupgreekii}\\
  && {\testlowgreeki}\\
  && {\testlowgreekii}\\
  && {\testlowgreekiii}
\end{eqnarray*}%
%
\def\test#1{\hat{\mathrm{#1}}+}%
\begin{eqnarray*}
  && {\testupperi}\\
  && {\testupperii}\\
  && {\testloweri}\\ 
  && {\testlowerii}\\ 
  && {\testupgreeki}\\
  && {\testupgreekii}
\end{eqnarray*}%
%
%\def\test#1{\hat{\mathbm{#1}}+}%
%\begin{eqnarray*}
%  && {\testupperi}\\
%  && {\testupperii}\\
%  && {\testloweri}\\ 
%  && {\testlowerii}\\ 
%  && {\testupgreeki}\\
%  && {\testupgreekii}\\
%  && {\testlowgreeki}\\
%  && {\testlowgreekii}\\
%  && {\testlowgreekiii}
%\end{eqnarray*}%
%%
%\def\test#1{\hat{\mathbf{#1}}+}%
%\begin{eqnarray*}
%  && {\testupperi}\\
%  && {\testupperii}\\
%  && {\testloweri}\\ 
%  && {\testlowerii}\\ 
%  && {\testupgreeki}\\
%  && {\testupgreekii}
%\end{eqnarray*}
%
\def\test#1{\hat{\mathcal{#1}}+}%
\begin{eqnarray*}
  && {\testupperi}\\
  && {\testupperii}
\end{eqnarray*}%


\clearpage
\subsection{Differentials}

\begin{eqnarray*}
\gdef\test#1{\dit #1+}%
  && {\testupperi}\\
  && {\testupperii}\\
  && {\testloweri}\\ 
  && {\testlowerii}\\ 
  && {\testupgreeki}\\
  && {\testupgreekii}\\
  && {\testlowgreeki}\\
  && {\testlowgreekii}\\
  && {\testlowgreekiii}\\
\gdef\test#1{\dit \mathrm{#1}+}%
  && {\testupgreeki}\\
  && {\testupgreekii}
\end{eqnarray*}%
%
\begin{eqnarray*}
\gdef\test#1{\dup #1+}%
  && {\testupperi}\\
  && {\testupperii}\\
  && {\testloweri}\\ 
  && {\testlowerii}\\ 
  && {\testupgreeki}\\
  && {\testupgreekii}\\
  && {\testlowgreeki}\\
  && {\testlowgreekii}\\
  && {\testlowgreekiii}\\
\gdef\test#1{\dup \mathrm{#1}+}%
  && {\testupgreeki}\\
  && {\testupgreekii}
\end{eqnarray*}%
%
\begin{eqnarray*}
\gdef\test#1{\partial #1+}%
  && {\testupperi}\\
  && {\testupperii}\\
  && {\testloweri}\\ 
  && {\testlowerii}\\ 
  && {\testupgreeki}\\
  && {\testupgreekii}\\
  && {\testlowgreeki}\\
  && {\testlowgreekii}\\
  && {\testlowgreekiii}\\
\gdef\test#1{\partial \mathrm{#1}+}%
  && {\testupgreeki}\\
  && {\testupgreekii}
\end{eqnarray*}%


\clearpage
\subsection{Slash kerning}

\def\test#1{1/#1+}
\begin{eqnarray*}
  && {\testupperi}\\
  && {\testupperii}\\
  && {\testloweri}\\ 
  && {\testlowerii}\\ 
  && {\testupgreeki}\\
  && {\testupgreekii}\\
  && {\testlowgreeki}\\
  && {\testlowgreekii}\\
  && {\testlowgreekiii}
\end{eqnarray*}

\def\test#1{#1/2+}
\begin{eqnarray*}
  && {\testupperi}\\
  && {\testupperii}\\
  && {\testloweri}\\ 
  && {\testlowerii}\\ 
  && {\testupgreeki}\\
  && {\testupgreekii}\\
  && {\testlowgreeki}\\
  && {\testlowgreekii}\\
  && {\testlowgreekiii}
\end{eqnarray*}


\clearpage
\subsection{Big operators}

\def\testop#1{#1_{i=1}^{n} x^{n} \quad}
\begin{displaymath}
  \testop\sum 
  \testop\prod 
  \testop\coprod 
  \testop\int 
  \testop\oint
\end{displaymath}
\begin{displaymath}
  \testop\bigotimes 
  \testop\bigoplus
  \testop\bigodot
  \testop\bigwedge 
  \testop\bigvee 
  \testop\biguplus 
  \testop\bigcup 
  \testop\bigcap 
  \testop\bigsqcup 
% \testop\bigsqcap
\end{displaymath}


\subsection{Radicals}

\begin{displaymath}
  \sqrt{x+y} \qquad \sqrt{x^{2}+y^{2}} \qquad 
  \sqrt{x_{i}^{2}+y_{j}^{2}} \qquad
  \sqrt{\left(\frac{\cos x}{2}\right)} \qquad 
  \sqrt{\left(\frac{\sin x}{2}\right)}
\end{displaymath}
  
\begingroup
\delimitershortfall-1pt
\begin{displaymath}
  \sqrt{\sqrt{\sqrt{\sqrt{\sqrt{\sqrt{\sqrt{x+y}}}}}}}
\end{displaymath}
\endgroup % \delimitershortfall


\subsection{Over- and underbraces}

\begin{displaymath}
  \overbrace{x} \quad
  \overbrace{x+y} \quad
  \overbrace{x^{2}+y^{2}} \quad
  \overbrace{x_{i}^{2}+y_{j}^{2}} \quad
  \underbrace{x} \quad
  \underbrace{x+y} \quad
  \underbrace{x_{i}+y_{j}} \quad
  \underbrace{x_{i}^{2}+y_{j}^{2}} \quad
\end{displaymath}


\subsection{Normal and wide accents}

\begin{displaymath}
  \dot{x} \quad 
  \ddot{x} \quad 
  \vec{x} \quad 
  \bar{x} \quad
  \overline{x} \quad
  \overline{xx} \quad
  \tilde{x} \quad
  \widetilde{x} \quad
  \widetilde{xx} \quad
  \widetilde{xxx} \quad
  \hat{x} \quad 
  \widehat{x} \quad 
  \widehat{xx} \quad 
  \widehat{xxx} \quad
\end{displaymath}


\subsection{Long arrows}

\begin{displaymath}
  \leftarrow \mathrel{-} \rightarrow \quad
  \leftrightarrow \quad
  \longleftarrow  \quad
  \longrightarrow \quad
  \longleftrightarrow \quad
  \Leftarrow = \Rightarrow \quad
  \Leftrightarrow \quad
  \Longleftarrow  \quad
  \Longrightarrow \quad
  \Longleftrightarrow \quad
\end{displaymath}


\subsection{Left and right delimters}

\def\testdelim#1#2{ - #1 f #2 - }
\begin{displaymath}
  \testdelim() 
  \testdelim[] 
  \testdelim\lfloor\rfloor 
  \testdelim\lceil\rceil 
  \testdelim\langle\rangle 
  \testdelim\{\} 
\end{displaymath}

\def\testdelim#1#2{ - \left#1 f \right#2 - }
\begin{displaymath}
  \testdelim() 
  \testdelim[] 
  \testdelim\lfloor\rfloor 
  \testdelim\lceil\rceil 
  \testdelim\langle\rangle 
  \testdelim\{\} 
% \testdelim\lgroup\rgroup
% \testdelim\lmoustache\rmoustache
\end{displaymath}
\begin{displaymath}
  \testdelim)(
  \testdelim][
  \testdelim// 
  \testdelim\backslash\backslash
  \testdelim/\backslash 
  \testdelim\backslash/
\end{displaymath}


\clearpage
\subsection{Big-g-g delimters}

\def\testdelim#1#2{%
  - \left#1\left#1\left#1\left#1\left#1\left#1\left#1\left#1 - 
  \right#2\right#2\right#2\right#2\right#2\right#2\right#2\right#2 -}

\begingroup
\delimitershortfall-1pt
\begin{displaymath}
  \testdelim\lfloor\rfloor 
  \qquad 
  \testdelim()
\end{displaymath}
\begin{displaymath}
  \testdelim\lceil\rceil 
  \qquad 
  \testdelim\{\} 
\end{displaymath}
\begin{displaymath}
  \testdelim[] 
  \qquad 
  \testdelim\lgroup\rgroup
\end{displaymath}
\begin{displaymath}
  \testdelim\langle\rangle
  \qquad 
  \testdelim\lmoustache\rmoustache
\end{displaymath}
\begin{displaymath}
  \testdelim\uparrow\downarrow \quad
  \testdelim\Uparrow\Downarrow \quad
\end{displaymath}
\endgroup % \delimitershortfall

\subsection{Miscellanneous formulae}

Taken from~\cite{Downes04:amsart}

\label{sec:misc}
\begin{displaymath}
  \hbar\nu=E
\end{displaymath}

Let $\mathbf{A}=(a_{ij})$ be the adjacency matrix of graph $G$. The
corresponding Kirchhoff matrix $\mathbf{K}=(k_{ij})$ is obtained from
$\mathbf{A}$ by replacing in $-\mathbf{A}$ each diagonal entry by the
degree of its corresponding vertex; i.e., the $i$th diagonal entry is
identified with the degree of the $i$th vertex. It is well known that
\begin{equation}
\det\mathbf{K}(i|i)=\text{ the number of spanning trees of $G$},
\quad i=1,\dots,n
\end{equation}
where $\mathbf{K}(i|i)$ is the $i$th principal submatrix of
$\mathbf{K}$.

\newcommand{\abs}[1]{\left\lvert#1\right\rvert}
\newcommand{\wh}{\widehat}
Let $C_{i(j)}$ be the set of graphs obtained from $G$ by attaching edge
$(v_iv_j)$ to each spanning tree of $G$. Denote by $C_i=\bigcup_j
C_{i(j)}$. It is obvious that the collection of Hamiltonian cycles is a
subset of $C_i$. Note that the cardinality of $C_i$ is $k_{ii}\det
\mathbf{K}(i|i)$. Let $\wh X=\{\hat x_1,\dots,\hat x_n\}$.  Define multiplication for the elements of $\wh X$ by
\begin{equation}\label{multdef}
\hat x_i\hat x_j=\hat x_j\hat x_i,\quad \hat x^2_i=0,\quad
i,j=1,\dots,n.
\end{equation}
Let $\hat k_{ij}=k_{ij}\hat x_j$ and $\hat k_{ij}=-\sum_{j\not=i} \hat
k_{ij}$. Then the number of Hamiltonian cycles $H_c$ is given by the
relation
\begin{equation}\label{H-cycles}
\biggl(\prod^n_{\,j=1}\hat x_j\biggr)H_c=\frac{1}{2}\hat k_{ij}\det
\wh{\mathbf{K}}(i|i),\qquad i=1,\dots,n.
\end{equation}
The task here is to express \eqref{H-cycles}
in a form free of any $\hat x_i$,
$i=1,\dots,n$. The result also leads to the resolution of enumeration of
Hamiltonian paths in a graph.

It is well known that the enumeration of Hamiltonian cycles and paths
in a complete graph $K_n$ and in a complete bipartite graph
$K_{n_1n_2}$ can only be found from \textit{first combinatorial
  principles}. One wonders if there exists a formula which can be used
very efficiently to produce $K_n$ and $K_{n_1n_2}$. Recently, using
Lagrangian methods, Goulden and Jackson have shown that $H_c$ can be
expressed in terms of the determinant and permanent of the adjacency
matrix. However, the formula of Goulden and
Jackson determines neither $K_n$ nor $K_{n_1n_2}$ effectively. In this
paper, using an algebraic method, we parametrize the adjacency matrix.
The resulting formula also involves the determinant and permanent, but
it can easily be applied to $K_n$ and $K_{n_1n_2}$. In addition, we
eliminate the permanent from $H_c$ and show that $H_c$ can be
represented by a determinantal function of multivariables, each
variable with domain $\{0,1\}$. Furthermore, we show that $H_c$ can be
written by number of spanning trees of subgraphs. Finally, we apply
the formulas to a complete multigraph $K_{n_1\dots n_p}$.

The conditions $a_{ij}=a_{ji}$, $i,j=1,\dots,n$, are not required in
this paper. All formulas can be extended to a digraph simply by
multiplying $H_c$ by 2.

The boundedness, property of $\Phi_ 0$, then yields
\[\int_{\mathcal{D}}\abs{\overline\partial u}^2e^{\alpha\abs{z}^2}\geq c_6\alpha
\int_{\mathcal{D}}\abs{u}^2e^{\alpha\abs{z}^2}
+c_7\delta^{-2}\int_ A\abs{u}^2e^{\alpha\abs{z}^2}.\]

Let $B(X)$ be the set of blocks of $\Lambda_{X}$
and let $b(X) = \abs{B(X)}$. If $\phi \in Q_{X}$ then
$\phi$ is constant on the blocks of $\Lambda_{X}$.
\begin{equation}\label{far-d}
 P_{X} = \{ \phi \in M \mid \Lambda_{\phi} = \Lambda_{X} \},
\qquad
Q_{X} = \{\phi \in M \mid \Lambda_{\phi} \geq \Lambda_{X} \}.
\end{equation}
If $\Lambda_{\phi} \geq \Lambda_{X}$ then
$\Lambda_{\phi} = \Lambda_{Y}$ for some $Y \geq X$ so that
\[ Q_{X} = \bigcup_{Y \geq X} P_{Y}. \]
Thus by M\"obius inversion
\[ \abs{P_{Y}}= \sum_{X\geq Y} \mu (Y,X)\abs{Q_{X}}.\]
Thus there is a bijection from $Q_{X}$ to $W^{B(X)}$.
In particular $\abs{Q_{X}} = w^{b(X)}$.


\renewcommand{\arraystretch}{2.2}

\[W(\Phi)= \begin{Vmatrix}
\dfrac\varphi{(\varphi_1,\varepsilon_1)}&0&\dots&0\\
\dfrac{\varphi k_{n2}}{(\varphi_2,\varepsilon_1)}&
\dfrac\varphi{(\varphi_2,\varepsilon_2)}&\dots&0\\
\hdotsfor{5}\\
\dfrac{\varphi k_{n1}}{(\varphi_n,\varepsilon_1)}&
\dfrac{\varphi k_{n2}}{(\varphi_n,\varepsilon_2)}&\dots&
\dfrac{\varphi k_{n\,n-1}}{(\varphi_n,\varepsilon_{n-1})}&
\dfrac{\varphi}{(\varphi_n,\varepsilon_n)}
\end{Vmatrix}\]



\bibliography{mathgifg}
\bibliographystyle{unsrt}


\end{document}
