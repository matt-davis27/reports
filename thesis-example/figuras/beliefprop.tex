\begin{figure}[H] 
 \centering
 \subfloat[Nós do tipo variável para nós do tipo função\label{fig:beliefprop1}]{
    \begin{tikzpicture}[node distance = 2cm,->,>=stealth',shorten >=1pt]
	\node at (0,0) [varnode] (x1) {$\vec{x}_j$};
	\node [funnode, above right of = x1] (f1) {$\psi_c$};
	\node [funnode, above left of = x1]  (f2) {$\psi_b$};
	\node [funnode, below of = x1] (f0) {$\psi_a$};
	
	\path [draw, ->, thick, red] (f1) -> (x1);
	\path [draw, ->, thick, red] (f2) -> (x1);
	\path [draw, ->, thick, blue] (x1) -> (f0);
    \end{tikzpicture}
}\qquad
 \subfloat[Nós do tipo função para nós do tipo variável\label{fig:beliefprop2}]{

    \begin{tikzpicture}[node distance = 2cm,->,>=stealth',shorten >=1pt]
	\node at (0,0) [funnode] (f1) {$\psi_a$};
	\node [varnode, above right of = f1] (x1) {$x_k$};
	\node [varnode, above left of = f1] (x2)  {$x_l$};
	\node [varnode, below of = f1] (x0) {$x_j$};

	\path [draw, ->, thick, blue]  (x1) -> (f1);
	\path [draw, ->, thick, blue]  (x2) -> (f1);
	\path [draw, ->, thick, red] (f1) -> (x0);

    \end{tikzpicture}
}
\qquad
 \subfloat[Estimativa para a marginal\label{fig:beliefprop3}]{

    \begin{tikzpicture}[node distance = 2cm,->,>=stealth',shorten >=1pt]
	\node at (0,0) [varnode] (x1) {$\vec{x}_j$};
	\node [funnode, above right of = x1] (f1) {$\psi_c$};
	\node [funnode, above left of = x1]  (f2) {$\psi_b$};
	\node [funnode, below of = x1] (f0) {$\psi_a$};
	
	\path [draw, ->, thick, red] (f1) -> (x1);
	\path [draw, ->, thick, red] (f2) -> (x1);
	\path [draw, ->, thick, red] (f0) -> (x1);
    \end{tikzpicture}
}
 \caption[Representação gráfica da passagem de mensagens]{Representação gráfica da passagem de mensagens para o método de belief propagation. Setas vermelhas representam mensagens na iteração $t$ que são compostas em seus nó de destino para dar origem às seta azuis, que representam mensagens na iteração $t+1$.}
 \label{fig:beliefprop}
\end{figure}